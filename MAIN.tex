\documentclass[11pt,a4paper,twoside]{book}


%%%%%%%%%%%%%%%%%%%%%%%%%%%%%%%%%%%%%%%%%%%%%%%%%%%%%%%%%%%%%%%%%%%%%%%%%%%%%%%%%%%%%%%%%%%%%%%%%%%%%%%%%%%%%%%%%%%%%%
%%%  Imported Packages
%%%%%%%%%%%%%%%%%%%%%%%%%%%%%%%%%%%%%%%%%%%%%%%%%%%%%%%%%
	\usepackage[slovene]{babel}
	\usepackage[utf8]{inputenc}
	\usepackage[T1]{fontenc}
	
	\usepackage{url}
	\usepackage{ifthen}
	\usepackage{amssymb}
	\usepackage{amsmath}
	\usepackage{theorem}
	\usepackage{phonetic}
	\usepackage{tablefootnote}
	\usepackage{color}
	\usepackage{tkz-graph}
	\usepackage{xparse}
	\usepackage{mathrsfs}
	\usepackage{ulem}
	\usepackage{charter}


%%%%%%%%%%%%%%%%%%%%%%%%%%%%%%%%%%%%%%%%%%%%%%%%%%%%%%%%%%%%%%%%%%%%%%%%%%%%%%%%%%%%%%%%%%%%%%%%%%%%%%%%%%%%%%%%%%%%%%
%%%  Theorems etc.
%%%%%%%%%%%%%%%%%%%%%%%%%%%%%%%%%%%%%%%%%%%%%%%%%%%%%%%%%%%%%
	{
		\theorembodyfont{\itshape}

		\newtheorem{izrek}{Izrek}[section]
		\newtheorem{lema}[izrek]{Lema}
		\newtheorem{trditev}[izrek]{Trditev}
		\newtheorem{posledica}[izrek]{Posledica}
	}

	{
		\theorembodyfont{\rmfamily}
		\newtheorem{definicija}[izrek]{Definicija}
		\newtheorem{opomba}[izrek]{Opomba}
		\newtheorem{primer}[izrek]{Primer}
		\newtheorem{zgled}[izrek]{Zgled}
		\newtheorem{vaja}[izrek]{Vaja}
	}

%%%%%%  Proofs
%%%%%%%%%%%%%%%%%%%%%%%%%%%%%%%%%%%%%%%%%%%%%%%%%%%%%%%%%%%%%
	\newenvironment{dokaz}{
		\goodbreak\par
		\textit{Dokaz.}%
	}{%
		\nopagebreak
		\hfill{\vrule width 1ex height 1ex depth 0ex}
		\medskip
		\goodbreak
	}
%%%%%%%%%%%%%%%%%%%%%%%%%%%%%%%%%%%%%%%%%%%%%%%%%%%%%%%%%%%%%%%%%%%%%%%%%%%%%%%%%%%%%%%%%%%%%%%%%%%%%%%%%%%%%%%%%%%%%%






%%%%%%%%%%%%%%%%%%%%%%%%%%%%%%%%%%%%%%%%%%%%%%%%%%%%%%%%%%%%%%%%%%%%%%%%%%%%%%%%%%%%%%%%%%%%%%%%%%%%%%%%%%%%%%%%%%%%%%
%%%  Commands
%%%%%%%%%%%%%%%%%%%%%%%%%%%%%%%%%%%%%%%%%%%%%%%%%%%%%%%%%%%%%%%%%%%%%%%%%%%%%%%%%%%%%%%%%%%%%%%%%%%%%%%%%%%%%%%%%%%%%%


%%%%%%  Auxiliary
%%%%%%%%%%%%%%%%%%%%%%%%%%%%%%%%%%%%%%%%%%%%%%%%%%%%%%%%%%%%%
	\newcommand{\sizedescriptor}[2]
	{
		\ifthenelse{\equal{#1}{0}}{}{
		\ifthenelse{\equal{#1}{1}}{\big}{
		\ifthenelse{\equal{#1}{2}}{\Big}{
		\ifthenelse{\equal{#1}{3}}{\bigg}{
		\ifthenelse{\equal{#1}{4}}{\Bigg}{
		#2}}}}}
	}
	\newcommand{\proven}[1]{\underline{#1}\vspace{0.2em}\\}
	\newcommand{\note}[1]{{\small\textcolor{blue}{(#1)}}}
	\newcommand{\someref}{{\small\textcolor{blue}{[\textbf{ref.}]}}}


%%%%%%  Logical Quantifiers and λ-Terms
%%%%%%%%%%%%%%%%%%%%%%%%%%%%%%%%%%%%%%%%%%%%%%%%%%%%%%%%%%%%%
	
	%  x = no parenthesis
	\NewDocumentCommand{\xall}
		{m O{\empty} m}
		{\forall\, {#1} \ifthenelse{\equal{#2}{}}{}{\in{#2}} \,.\, {#3}}
	\NewDocumentCommand{\xsome}
		{m O{\empty} m}
		{\exists\, {#1} \ifthenelse{\equal{#2}{}}{}{\in{#2}} \,.\, {#3}}
	\NewDocumentCommand{\xexactlyone}
		{m O{\empty} m}
		{\exists\;\!!\, {#1} \ifthenelse{\equal{#2}{}}{}{\in{#2}} \,.\, {#3}}
	\NewDocumentCommand{\xlam}
		{m O{\empty} m O{\empty}}
		{\lambda\, {#1} \ifthenelse{\equal{#2}{}}{}{\in{#2}} \,.\, {#3} \ifthenelse{\equal{#4}{}}{}{\in{#4}}}
	
	%  with parenthesis -- the first optional argument is the size (values 0-4)
	\NewDocumentCommand{\all}
		{O{auto} m O{\empty} m}
		{\xall{#2}[#3]{\sizedescriptor{#1}{\left}( {#4} \sizedescriptor{#1}{\right})}}
	\NewDocumentCommand{\some}
		{O{auto} m O{\empty} m}
		{\xsome{#2}[#3]{\sizedescriptor{#1}{\left}( {#4} \sizedescriptor{#1}{\right})}}
	\NewDocumentCommand{\exactlyone}
		{O{auto} m O{\empty} m}
		{\xexactlyone{#2}[#3]{\sizedescriptor{#1}{\left}( {#4} \sizedescriptor{#1}{\right})}}
	\NewDocumentCommand{\lam}
		{O{auto} m O{\empty} m O{\empty}}
		{\xlam{#2}[#3]{\sizedescriptor{#1}{\left}( {#4} \sizedescriptor{#1}{\right})}[#5]}


%%%%%%  Sets
%%%%%%%%%%%%%%%%%%%%%%%%%%%%%%%%%%%%%%%%%%%%%%%%%%%%%%%%%%%%%
	%  \set{1, 2, 3}         ->  {1, 2, 3}
	%  \set{a \in X}{a < 1}  ->  {a ∈ X | a < 1}
	\NewDocumentCommand{\set}
		{O{auto} m G{\empty}}
		{\sizedescriptor{#1}{\left}\{ {#2} \ifthenelse{\equal{#3}{}}{}{ \; \sizedescriptor{#1}{\middle}| \; {#3}} \sizedescriptor{#1}{\right}\}}
	%\newcommand{\vsubset}{\Mapstochar\cap}
	%\newcommand{\finseq}[1]{{#1}^*}
	\newcommand{\pst}{\mathcal{P}}
	\renewcommand{\complement}[1]{{#1}^C}


%%%%%%  Number Sets, Intervals
%%%%%%%%%%%%%%%%%%%%%%%%%%%%%%%%%%%%%%%%%%%%%%%%%%%%%%%%%%%%%
	\newcommand{\NN}{\mathbb{N}}
	\newcommand{\ZZ}{\mathbb{Z}}
	\newcommand{\QQ}{\mathbb{Q}}
	\newcommand{\RR}{\mathbb{R}}
	\newcommand{\CC}{\mathbb{C}}
	\newcommand{\intoo}[3][\RR]{{#1}_{(#2, #3)}}
	\newcommand{\intcc}[3][\RR]{{#1}_{[#2, #3]}}
	\newcommand{\intoc}[3][\RR]{{#1}_{(#2, #3]}}
	\newcommand{\intco}[3][\RR]{{#1}_{[#2, #3)}}


%%%%%%  Misc.
%%%%%%%%%%%%%%%%%%%%%%%%%%%%%%%%%%%%%%%%%%%%%%%%%%%%%%%%%%%%%
	\newcommand{\intermission}{\bigskip\medskip}
	\newcommand{\df}[1]{\emph{\textbf{#1}}}  % defined notion
	\newcommand{\ism}{\cong}  % isomorphic
	\newcommand{\equ}{\sim}  % equivalent
	\newcommand{\dfeq}{\mathrel{\mathop:}=}  % definitional equality
	\newcommand{\dfeqrev}{=\mathrel{\mathop:}}  % reversed definitional equality
	\newcommand{\id}[1][]{\textrm{Id}_{#1}}  % identity map
	\newcommand{\impl}{\Rightarrow}  % implication sign
	\newcommand{\revimpl}{\Leftarrow}  % reverse implication sign
	\newcommand{\lequ}{\Leftrightarrow}  % equivalence sign
	\newcommand{\xor}{\mathbin{\veebar}}  % exclusive disjunction sign
	\newcommand{\shf}{\mathbin{\uparrow}}  % Sheffer connective
	\newcommand{\luk}{\mathbin{\downarrow}}  % Łukasiewicz connective
	\newcommand{\rstr}[1]{\left.{#1}\right|}  % map restriction
	\newcommand{\im}{\mathrm{im}}  % map image
	\newcommand{\parto}{\mathrel{\rightharpoonup}}  % partial mapping sign
	\newcommand{\qt}[1]{{\quotedblbase}{#1}{‘‘}}  % text in quotation marks
	\newcommand{\nls}[1]{\qt{\textit{#1}}}  % sentence in a natural language
	\newcommand{\rel}{\:\!{\mathscr{R}}\:\!}  % a relation
	\newcommand{\srel}{\:\!{\mathscr{S}}\:\!}  % a second relation
	% \mathrel ne deluje najbolje, zato raje ročno nastavljeni presledki
	\newcommand{\dom}{\mathrm{dom}}  % domain
	\newcommand{\cod}{\mathrm{cod}}  % codomain
	

%%%%%%%%%%%%%%%%%%%%%%%%%%%%%%%%%%%%%%%%%%%%%%%%%%%%%%%%%%%%%%%%%%%%%%%%%%%%%%%%%%%%%%%%%%%%%%%%%%%%%%%%%%%%%%%%%%%%%%






%%%%%%%%%%%%%%%%%%%%%%%%%%%%%%%%%%%%%%%%%%%%%%%%%%%%%%%%%%%%%%%%%%%%%%%%%%%%%%%%%%%%%%%%%%%%%%%%%%%%%%%%%%%%%%%%%%%%%%
%%  Page Style & Margins (A4 page = 210mm x 297mm)

\setlength{\textwidth}{15cm}
\setlength{\textheight}{224mm}

\setlength{\topmargin}{0cm}
\setlength{\evensidemargin}{0cm}
\setlength{\oddsidemargin}{\paperwidth}
\addtolength{\oddsidemargin}{-\textwidth}
\addtolength{\oddsidemargin}{-2in}

\renewcommand{\baselinestretch}{1.25}
\setlength{\parskip}{1.5ex}


%%%%%%%%%%%%%%%%%%%%%%%%%%%%%%%%%%%%%%%%%%%%%%%%%%%%%%%%%%%%%%%%%%%%%%%%%%%%%%%%%%%%%%%%%%%%%%%%%%%%%%%%%%%%%%%%%%%%%%






\begin{document}

   %--------------------------------------------------------------------
   %--------------------------------------------------------------------
   % TITLE PAGE
   
	
	\title{\Huge \textbf{\textsc{Logika in množice}}}
	\author{A.~Bauer, D.~Lešnik, M.~Petkovšek, M.~Pretnar}
	
	\maketitle
   
   
   %--------------------------------------------------------------------
   %--------------------------------------------------------------------
   % Foreword

   \chapter*{Predgovor}%\addcontentsline{toc}{chapter}{\numberline{}Predgovor}


	%--------------------------------------------------------------------
	%--------------------------------------------------------------------
	% TOC
	
	
	\addcontentsline{toc}{chapter}{\note{kazalo se naj začne na sodi strani, tako da lahko bralec naenkrat vidi celotno kazalo}}
	\tableofcontents
	
	
	%--------------------------------------------------------------------
	%--------------------------------------------------------------------
	% BODY
	
	
	\chapter{Matematično izražanje}\label{POGLAVJE: Matematično izražanje}

	\note{Za začetek bom vnašal oporne točke besedila. Slog bo verjetno treba še popraviti in besedilo dopolniti. --Davorin}
	
	\textcolor{red}{\small \textbf{Če je možno, prosim uporabljajte tabulatorje namesto presledkov za zamike v latex kodi in koda naj nima izrecno vnešenih prelomov vrstic, pač pa se v urejevalniku besedila uporablja avtomatski word wrap, ki se prilagaja širini okna. --Davorin}}
	
	Za matematično delo je bistveno, da se lahko zanašamo na pravilnost naših trditev. To pomeni:
	\begin{itemize}
		\item
			matematične izjave morajo imeti \emph{nedvoumen pomen},
		\item
			matematične izjave lahko \emph{dokažemo}.
	\end{itemize}
	
	Stavki v običajnih jezikih nimajo nedvoumnega pomena, zato matematične izjave raje podamo v \emph{matematičnem jeziku}. Za to potrebujemo \qt{matematično abecedo}, tj.~simbolni zapis, v katerem podamo izjave. Tega obravnavamo v naslednjem razdelku, dokazovanje matematičnih izjav pa v razdelku za tem.
	
	\section{Simbolni zapis}\label{RAZDELEK: Simbolni zapis}
	
		Za množice, s katerimi najpogosteje delamo, obstajajo standardne oznake (tabela~\ref{TABELA: standardne številske množice}).
		
		\begin{table}[!ht]
			\centering
			\begin{tabular}{|cc|}
				\hline
				\textbf{Množica} & \textbf{Oznaka} \\
				\hline
				množica naravnih števil & $\NN$ \\
				množica celih števil & $\ZZ$ \\
				množica racionalnih števil & $\QQ$ \\
				množica realnih števil & $\RR$ \\
				množica kompleksnih števil & $\CC$ \\
				\hline
			\end{tabular}
			\caption{standardne številske množice}\label{TABELA: standardne številske množice}
		\end{table}
		
		Nekateri $0$ vzamejo za naravno število, nekateri ne. To je v celoti stvar dogovora, kaj pomeni pojem \qt{naravno število}. Za nas bo prišlo bolj prav, če ničlo štejemo kot element množice naravnih števil, torej $\NN = \set{0, 1, 2, 3, \ldots}$.
		
		Interval realnih števil podamo s krajiščema intervala v oklepajih --- okrogli oklepaji ( ) označujejo odprtost intervala (krajišče ni vključeno v interval), oglati oklepaji [ ] pa zaprtost (krajišče je vključeno). Tako se npr.~interval realnih števil od $0$ do $1$, ki ne vsebuje krajišč, označi z $(0, 1)$, če jih vsebuje, pa z $[0, 1]$.
		
		Včasih pridejo prav tudi intervali na drugih množicah kot $\RR$. Zato se dogovorimo, da bomo intervale označevali tako, da podamo množico, ob kateri v indeksu zapišemo krajišči v oklepajih, npr.~$\intco[\NN]{1}{5} = \set{1, 2, 3, 4}$. Realna intervala iz prejšnjega odstavka tako zapišemo kot $\intoo{0}{1}$ in $\intcc{0}{1}$.
		
		Če interval v katero smer gre v nedogled, preprosto zapišemo množico z ustrezno relacijo urejenosti in krajiščem v indeksu. Na primer, $\RR_{> 0}$ označuje množico pozitivnih realnih števil, $\RR_{\geq 0}$ pa množico nenegativnih realnih števil.
		
		\note{To bi vsaj bil moj predlog. Na ta način se izognemo dvoumnostim (kar je namen). Na primer, kaj pomeni $\forall\, a > 0$? Če zapišemo $\forall\, a \in \NN_{> 0}$ ali $\forall\, a \in \RR_{> 0}$, je jasno. Razlog, da matematiki \qt{goljufajo} in pridejo skozi brez tega, je (napol dogovorjena in ponotranjena, ampak arbitrarna) izbira črk; vsak izkušen matematik ve, da $\forall\, \epsilon > 0$ pomeni $\forall\, \epsilon \in \RR_{> 0}$. Če se ne strinjate, popravite in pustite komentar. --Davorin}
		
		Izjavo, da je $2$ naravno število, zapišemo takole: $2 \in \NN$ (beri: $2$ pripada množici naravnih števil). Kako zapišemo, da je $a$ sodo število? Število je sodo, kadar je deljivo z $2$, torej pišemo $2 | a$ (beri: $2$ deli $a$).
		
		Če imamo več izjav, jih lahko strnemo v sestavljeno izjavo. Na primer, izjavo \nls{Če je $a$ sodo število, je tudi kvadrat števila $a$ sod.}, zapišemo kot $2 | a \implies 2 | a^2$.
		
		Seveda ta izjava velja za vsa naravna števila (znaš to dokazati?). To zapišemo takole: $\all{a}{\NN}{2 | a \implies 2 | a^2}$.
		
		Kot smo navajeni iz običajnih jezikov, posamične stavke povežemo v sestavljeno poved z \emph{vezniki}. Najpogosteje uporabljeni matematični vezniki so v tabeli~\ref{TABELA: standardni izjavni vezniki}.
		
		\begin{table}[!ht]
			\centering
			\begin{tabular}{|ccc|}
				\hline
				\textbf{Izjavni veznik} & \textbf{Oznaka} & \textbf{Kako preberemo} \\
				\hline
				negacija & $\lnot{p}$ & ne $p$ \\
				konjunkcija & $p \land q$ & $p$ in $q$ \\
				disjunkcija & $p \lor q$ & $p$ ali $q$ \\
				implikacija & $p \impl q$ & če $p$, potem $q$ \\
				ekvivalenca & $p \lequ q$ & $p$ natanko tedaj, ko $q$ \\
				\hline
			\end{tabular}
			\caption{standardni izjavni vezniki}\label{TABELA: standardni izjavni vezniki}
		\end{table}
		
		\begin{opomba}
			V matematiki se za izjavne veznike običajno uporabljajo zgoraj navedene tujke, ampak vsaka od njih seveda ima svoj pomen. Dobesedni prevodi teh tujk so:
			\begin{itemize}
				\item
					negacija $\to$ zanikanje,
				\item
					konjunkcija $\to$ vezava,
				\item
					disjunkcija $\to$ ločitev,
				\item
					implikacija $\to$ vpletenost,
				\item
					ekvivalenca $\to$ enakovrednost.
			\end{itemize}
			Za primerjavo: spomnite se vezalnega in ločnega priredja iz slovenščine!
		\end{opomba}
		
		\begin{zgled}
			Naj $p$ označuje stavek \nls{Zunaj dežuje.} in $q$ stavek \nls{Vzamem dežnik.}. Tedaj $\lnot{p}$ pomeni \nls{Zunaj ne dežuje.} in $p \impl q$ pomeni \nls{Če zunaj dežuje, potem vzamem dežnik.}.
		\end{zgled}
		
		Kose sestavljene izjave lahko veže več kot en veznik. V tem primeru se (tako kot pri računanju s števili) dogovorimo o prednosti veznikov. Po dogovoru je vrstni red veznikov tak, kot v tabeli~\ref{TABELA: standardni izjavni vezniki}, tj.~najmočneje veže negacija, nato konjunkcija, nato disjunkcija, nato implikacija, nato ekvivalenca. Kadar želimo, da se najprej izvede veznik z nižjo prednostjo, uporabimo oklepaje.
		
		\begin{zgled}
			Označimo sledeče stavke:
			\begin{quote}
				$p$ \ \ldots\ldots\ \nls{Imam čas.} \\
				$q$ \ \ldots\ldots\ \nls{Ostanem doma.}
			\end{quote}
			Tedaj $\lnot{p} \land q$ pomeni isto kot $(\lnot{p}) \land q$, to je \nls{Nimam časa in ostanem doma.}, medtem ko $\lnot(p \land q)$ pomeni \nls{Ni res, da imam čas in ostanem doma.}.
		\end{zgled}
		\note{Če komu pade na pamet primer boljših stavkov, je zaželjeno, da popravi\ldots --Davorin}
		
		Poleg zgoraj navedenih izjavnih veznikov se včasih uporabljajo še sledeči (tabela~\ref{TABELA: nadaljnji izjavni vezniki}).
		
		\begin{table}[!ht]
			\centering
			\begin{tabular}{|ccc|}
				\hline
				\textbf{Izjavni veznik} & \textbf{Oznaka} & \textbf{Kako preberemo} \\
				\hline
				stroga disjunkcija & $p \xor q$ & bodisi $p$ bodisi $q$ \\
				Shefferjev\tablefootnote{Henry Maurice Sheffer (1882 -- 1964) je bil ameriški logik.} veznik & $p \shf q$ & ne hkrati $p$ in $q$ \\
				Łukasiewiczev\tablefootnote{Jan Łukasiewicz (beri: \hill{u}ukaśj\^{e}vič) (1878 -- 1956) je bil poljski logik in filozof.} veznik & $p \luk q$ & niti $p$ niti $q$ \\
				\hline
			\end{tabular}
			\caption{nekateri nadaljnji izjavni vezniki}\label{TABELA: nadaljnji izjavni vezniki}
		\end{table}
		
		Za strogo disjunkcijo (tudi: ekskluzivna disjunkcija, izključitvena disjunkcija) se uporabljajo še druge oznake: $p \oplus q$, $p + q$. Razlika med navadno in strogo disjunkcijo je sledeča: $p \lor q$ pomeni, da \emph{vsaj eden} od $p$ in $q$ velja, medtem ko $p \xor q$ pomeni, da velja \emph{natanko eden}.
		
		\begin{zgled}
			Stavek \nls{Pisni del predmeta je potrebno opraviti s kolokviji ali pisnim izpitom.} je primer navadne disjunkcije (seveda se vam prizna pisni del predmeta tudi, če uspešno odpišete tako kolokvije kot pisni izpit), stavek \nls{Grem bodisi na morje bodisi v hribe.} pa je primer stroge disjunkcije (ne da se biti na dveh mestih hkrati).
		\end{zgled}
		
		Običajno veznike iz tabele~\ref{TABELA: nadaljnji izjavni vezniki} (in vse preostale, ki jih nismo navedli) izrazimo s standardnimi (glej tabelo~\ref{TABELA: izražava nadaljnjih izjavnih veznikov s standardnimi}), včasih pa je uporabno delati neposredno z njimi. Na primer, stroga disjunkcija služi kot seštevanje v Boolovem kolobarju (glej~\note{razdelek o Boolovih kolobarjih}), Shefferjev in Łukasiewiczev veznik pa se uporabljata pri preklopnih vezjih, saj je z vsakim od njiju možno izraziti vse izjavne veznike (glej~\note{razdelek o polnih naborih}). V računalništvu imajo ti trije vezniki standardne oznake XOR, NAND, NOR.
		
		\begin{table}[!ht]
			\centering
			\begin{tabular}{|ccc|}
				\hline
				\textbf{Izjavni veznik} & \multicolumn{2}{c|}{\textbf{Nekatere izražave s standardnimi vezniki}} \\
				\hline
				$p \xor q$ & $(p \lor q) \land \lnot(p \land q)$ & $(p \land \lnot{q}) \lor (\lnot{p} \land q)$ \\
				$p \shf q$ & $\lnot(p \land q)$ & $\lnot{p} \lor \lnot{q}$ \\
				$p \luk q$ & $\lnot(p \lor q)$ & $\lnot{p} \land \lnot{q}$ \\
				\hline
			\end{tabular}
			\caption{izražava nadaljnjih izjavnih veznikov s standardnimi}\label{TABELA: izražava nadaljnjih izjavnih veznikov s standardnimi}
		\end{table}
		
		\note{Na tem mestu povejmo, kakšno prednost damo tem trem veznikom v primerjavi s standardnimi. Kateremu dogovoru sledimo?}
		
		Včasih so izjave odvisne od kakšnih parametrov. Na primer, naj $\phi(x)$ pomeni \nls{$x$ je zelen.}; tedaj $\phi(\text{trava})$ pomeni \nls{Trava je zelena.}. Takim odvisnim izjavam rečemo \df{predikati} in izražajo lastnosti, ki jim parametri (\qt{spremenljivke}) lahko zadoščajo.
		
		Predikate lahko \emph{kvantificiramo} po njihovih spremenljivkah, tj.~povemo, \qt{kako pogosto} velja lastnost, dana s predikatom. Tabela~\ref{TABELA: kvantifikatorji} podaja najpogosteje uporabljane kvantifikatorje in njihove oznake.
		
		\begin{table}[!ht]
			\centering
			\begin{tabular}{|ccc|}
				\hline
				\textbf{Kvantifikator} & \textbf{Oznaka} & \textbf{Kako preberemo} \\
				\hline
				univerzalni kvantifikator & $\xall{x}{X}{\phi(x)}$ & za vsak $x$ iz $X$ velja lastnost $\phi$ \\
				eksistenčni kvantifikator & $\xsome{x}{X}{\phi(x)}$ & obstaja $x$ iz $X$ z lastnostjo $\phi$ \\
				\note{kako se temu reče?} & $\xexactlyone{x}{X}{\phi(x)}$ & obstaja natanko en $x$ iz $X$ z lastnostjo $\phi$ \\
				\hline
			\end{tabular}
			\caption{kvantifikatorji}\label{TABELA: kvantifikatorji}
		\end{table}
		
		Oznaki $\forall$ in $\exists$ sta narobe obrnjena A in E in izhajata iz nemščine (\textbf{a}ll, \textbf{e}xistiert).
		
		\begin{zgled}
			Vemo, da za vsako nenegativno realno število obstaja enolično določen nenegativen kvadratni koren; to izjavo lahko zapišemo na sledeči način.
			\[\xall{a}{\RR_{\geq 0}}\xexactlyone{b}{\RR_{\geq 0}}{b^2 = a}\]
			Zaradi tega lahko definiramo kvadratni koren kot funkcijo $\sqrt{\phantom{I}}\colon \RR_{\geq 0} \to \RR_{\geq 0}$.
		\end{zgled}
		
		Po dogovoru kvantifikatorji vežejo šibkeje kot izjavni vezniki. Izjavo, da je vsako celo število bodisi liho bodisi sodo, torej zapišemo takole.
		\[\all[2]{a}{\ZZ}{2 | a \xor 2 | (a-1)}\]
		
		\note{Se že na tem mestu predebatirajo vezane oz.~nevezane spremenljivke ter preimenovanje spremenljivk? Kaj je slovenski prevod za `dummy variable'?}
		
		\begin{zgled}
			Za poljubno naravno število $n \in \NN$ naj $P(n)$ označuje izjavo, da je $n$ praštevilo. Torej, $P$ definiramo takole.
			\[P(n) \dfeq \all[1]{x}{\NN}{x | n \implies x = 1 \xor x = n}\]
			(Premisli, kaj bi se zgodilo, če bi namesto stroge disjunkcije vzeli navadno. Bi še vedno dobili pravilni pojem praštevila?)
			
			Naj $S(n)$ označuje, da je $n$ sestavljeno število.
			\[S(n) \dfeq \xsome{x, y}{\intoo[\NN]{1}{n}}{x \cdot y = n}\]
			(Kadar imamo več zaporednih kvantifikatorjev iste vrste, jih po dogovoru lahko strnemo kot zgoraj. Dana formula za $S(n)$ je krajši zapis za $\xsome{x}{\intoo[\NN]{1}{n}}\xsome{y}{\intoo[\NN]{1}{n}}{x \cdot y = n}$.)
			
			Zdaj lahko na pregleden način zapišemo, da je vsako naravno število od $2$ naprej bodisi praštevilo bodisi sestavljeno.
			\[\all[1]{n}{\NN_{\geq 2}}{P(n) \xor S(n)}\]
		\end{zgled}
		
		\note{
			Ideje za VAJE:\\
				\hbox{}\qquad\qquad * Napiši te in te z besedami podane matematične izjave simbolno.\\
				\hbox{}\qquad\qquad * Za te in te \qt{življenjske} izjave vsak osnoven sestavni kos označi s črko in zapiši sestavljeno izjavo z mešanico veznikov in kvantifikatorjev.\\
				\hbox{}\qquad\qquad * ...
		}
	
	
	\section{Pravila dokazovanja}\label{RAZDELEK: Pravila dokazovanja}
	
		Matematične izsledke običajno podajamo preko jasno izraženih izjav. Med študijem matematike hitro opazite, da se takšne izjave podajajo pod imeni \quotesinglbase{izrek}', \quotesinglbase{trditev}', \quotesinglbase{lema}', \quotesinglbase{posledica}' in podobno. Kdaj uporabiti katerega teh imen ni natanko določeno, pač pa je prepuščeno presoji matematika. Približno vodilo je naslednje:
		\begin{itemize}
			\item
				\df{izrek}: osrednji, bistven rezultat,
			\item
				\df{trditev}: stranski rezultat,
			\item
				\df{lema}: rezultat, ki sam po sebi nima toliko vsebine, se pa uporabi pri dokazovanju pomembnejšega rezultata,\footnote{Sicer ni nujno, da se resnična pomembnost izjav takoj pokaže. Mnogo je primerov, ko se kak matematični članek po določenem času začne ceniti ne toliko zaradi glavnega izreka, pač pa zaradi neke leme, ki se je za dokaz glavnega izreka uporabila.}
			\item
				\df{posledica}: rezultat, ki je zanimiv sam po sebi, ki pa hitro sledi iz predhodne izjave.
		\end{itemize}
		
		Če skrbno analizirate izreke, trditve itd.~s predavanj (ali matematičnih člankov), opazite, da sestojijo iz treh delov: \note{kontekst, predpostavke, sklepi}
	
	
	\section{Definicije}
	
	\chapter{Konstrukcije množic}
		\section{Preprosti primeri}
			\note{prazna množica, enojci}
		\section{Podmnožice}
		\section{Potenčna množica}
		\section{Družine množic}
		\section{Produkt množic}
		\section{Vsota množic}
		\section{Unija in presek}
		\section{Eksponentna množica}
	
	\chapter{Funkcije}
		\note{Moramo se dogovoriti, kateremu izrazu bomo dali prednost --- `funkcija' ali `preslikava'.}
		\section{Slike in praslike}
		\section{Injektivnost in surjektivnost}
			\note{Vključno z ekvivalenco z mono- in epimorfizmi.}
		\section{Bijektivnost}
			\note{Pomemben del tega razdelka bodo inverzne funkcije. Mogoče lahko to dodamo v naslov.}
	
	\chapter{Relacije}\label{POGLAVJE: Relacije}

        \section{Splošno o relacijah}

                V matematiki pogosto želimo izraziti, da so določeni objekti v nekem odnosu, npr.~eno število je večje od drugega; temu s tujko rečemo \df{relacija}. Kako to formalno izraziti? Ideja je, da relacijo podamo z množico vseh skupin elementov, ki so v relaciji. Na primer, relacijo $\leq$ na naravnih številih podamo kot podmnožico
                \[\set[1]{(a, b) \in \NN \times \NN}{\xsome{n}[\NN]{a + n = b}}.\]
                Torej, število $a$ je v relaciji $\leq$ s številom $b$ takrat, ko par $(a, b)$ pripada tej množici.

                Splošne relacije so lahko med poljubno mnogo elementi iz poljubnih (ne nujno istih) množic. Na primer, relacija komplanarnosti štirih točk v prostoru je podmnožica produkta $\RR^3 \times \RR^3 \times \RR^3 \times \RR^3$, relacija pripadnosti $\in$ med elementi neke množice $X$ in podmnožicami množice $X$ pa je podmnožica produkta $X \times \pst(X)$.

                Splošna definicija relacije je potemtakem naslednja.
                \begin{definicija}
                        \df{Relacija} na družini množic $\mathscr{D}$ je podmnožica produkta $\prod_{X \in \mathscr{D}} X$, skupaj s podatkom, za katero družino $\mathscr{D}$ gre.
                \end{definicija}

                \begin{opomba}\label{OPOMBA: definicija relacij}
                        Kaj mislimo tu z izrazom \qt{skupaj s podatkom}? Določena podmnožica ima mnogo nadmnožic in podatek, med elementi katerih množic opazujemo odnos, je za relacijo prav tako pomemben, saj so od tega odvisne lastnosti relacije. Lastnosti relacij obravnavamo kasneje v razdelku~\ref{RAZDELEK: Lastnosti relacij}, ampak če že zdaj damo primer: $\set{(a, a)}{a \in \NN}$ je refleksivna kot relacija na naravnih številih (tj.~kot podmnožica $\NN \times \NN$), ne pa tudi kot relacija na celih številih (tj.~kot podmnožica $\ZZ \times \ZZ$).

                        Kako \qt{priložiti} podatek o družini? Ena možnost je, da relacijo podamo kot urejeni par $\rel = (R, \mathscr{D})$, kjer $R \subseteq \prod_{X \in \mathscr{D}} X$. Še ena možnost je, da relacijo podamo kot družino preslikav $(\rel \to X)_{X \in \mathscr{D}}$, ki skupaj porodijo inkluzijo $\rel \hookrightarrow \prod_{X \in \mathscr{D}} X$. Ampak načeloma je povsem vseeno, ali vzamemo katero od teh dveh možnosti ali še kaj tretjega. V tej knjigi se ne bomo omejevali na posamičen formalen zapis za relacijo, bo pa seveda v vseh primerih jasno, za katero družino gre.
                \end{opomba}

                \davorin{Verjetno bi bilo smiselno omeniti še možnost podajanja relacije kot predikat $\prod_{X \in \mathscr{D}} X \to \tvs$.}

                V praksi se povečini uporabljajo relacije med dvema elementoma.
                \begin{definicija}
                        \df{Dvomestna} (ali \df{dvojiška} ali \df{binarna}) \df{relacija} $\rel$ med elementi množic $X$ in $Y$ je podmnožica produkta $X \times Y$, skupaj s podatkom o $X$ in $Y$. Za takšno relacijo definiramo:
                        \begin{itemize}
                                \item
                                        množica $X$ je \df{začetna množica} ali \df{domena} relacije $\rel$, kar označimo $\dom(\rel)$,
                                \item
                                        množica $Y$ je \df{ciljna množica} ali \df{kodomena} relacije $\rel$, kar označimo $\cod(\rel)$,
                                \item
                                        \df{definicijsko območje} ali \df{nosilec} relacije $\rel$ je množica $\dd{\rel} \dfeq \set{x \in X}{\xsome{y}[Y]{\rel[x][y]}}$ (torej $\dd{\rel} \subseteq \dom(\rel)$),
                                \item
                                        \df{zaloga vrednosti} ali \df{slika} \note{razpon?} relacije $\rel$ je množica $\rn{\rel} \dfeq \set{y \in Y}{\xsome{x}[X]{\rel[x][y]}}$ (torej $\dd{\rel} \subseteq \cod(\rel)$).
                        \end{itemize}
                \end{definicija}

                Skoraj vse relacije, ki nas zanimajo v tej knjigi, so dvomestne. Zato se dogovorimo, da z izrazom \qt{relacija} vselej mislimo dvomestno relacijo, razen če je izrecno rečeno drugače.

                Če je $\rel \subseteq X \times Y$ relacija, potemtakem lahko zapišemo, da sta $x \in X$ in $y \in Y$ v relaciji $\rel$ takole: $(x, y) \in \rel$. Ampak to vodi do čudnih zapisov v primeru običajnih relacij, npr.~$(2, 3) \in \mathnormal{<}$. To je bolje zapisati $2 < 3$ in posledično se dogovorimo, da v primeru dvojiške relacije raje uporabljamo zapis $\rel[x][y]$.

                Povečini se še dodatno omejimo na relacije z isto domeno in kodomeno.
                \begin{definicija}
                        \df{Dvomestna} (\df{dvojiška}, \df{binarna}) \df{relacija} na množici $X$ je podmnožica produkta $X \times X$, skupaj s podatkom o $X$.
                \end{definicija}

                Takšne relacije lahko lepo ponazorimo z usmerjenimi grafi. Graf relacije $\rel \subseteq X \times X$ je definiran takole: vozlišča grafa so elementi množice $X$ in za vsaka dva elementa $a, b \in X$, za katera velja $\rel[a][b]$, narišemo puščico od $a$ do $b$.

                \GraphInit[vstyle = Normal]
                \tikzset
                {
                        EdgeStyle/.append style = {->, bend left}
                }

                \begin{zgled}\label{ZGLED: graf relacije}
                        Naj bo $X = \set{A, B, C, D, E, F}$ in naj bo
                        \[\rel \dfeq \set{...}\]
                        relacija na $X$. Njen graf izgleda takole.

                        \note{graf relacije $\rel$}
                        %\begin{center}
                                %\begin{tikzpicture}
                                        %\SetGraphUnit{3}
                                        %\Vertex[Math=true, x=0, y=0]{A}
                                        %\Vertex[Math=true, x=3, y=2]{B}
                                        %\Vertex[Math=true, x=2, y=-3]{C}
                                        %\Vertex[Math=true, x=6, y=1]{D}
                                        %\Vertex[Math=true, x=8, y=-1]{E}
                                        %\Vertex[Math=true, x=10, y=2]{F}
                                        %
                                        %\Edge(A)(B)
                                        %\Loop[dist = 5em, dir = EA](B)
                                %\end{tikzpicture}
                        %\end{center}
                        %\davorin{izgled grafa je še treba popraviti}
                \end{zgled}


        \section{Operacije z relacijami}\label{RAZDELEK: Operacije z relacijami}

                Običajno je, da iz že danih matematičnih objektov lahko skonstruiramo nove preko določenih operacij. Z relacijami ni nič drugače; v tem razdelku si bomo ogledali običajne operacije na relacijah.

                Ker so relacije podmnožice, imamo za začetek vse operacije na podmnožicah. Torej, za poljubno družino $(\rel_i)_{i \in I}$ podmnožic produkta $X \times Y$ sta tudi unija $\bigcup_{i \in I} \rel_i$ in presek $\bigcap_{i \in I} \rel_i$ relaciji. Če je $\rel \subseteq X \times Y$ relacija, je njena komplementarna relacija $\complement{\rel} = X \times Y \setminus \rel \ \subseteq \ X \times Y$.

                Posebej imamo \df{prazno relacijo} $\emptyset \subseteq X \times Y$ (nobena dva elementa nista v relaciji) in \df{polno relacijo} $X \times Y \subseteq X \times Y$ (vsaka dva elementa sta v relaciji), ki sta si medsebojno komplementarni.

                Poleg operacij, ki jih relacije podedujejo od podmnožic, imamo še operacije, ki upoštevajo produktno strukturo.

                Če so $X$, $Y$, $Z$ množice in $\rel \subseteq X \times Y$, $\srel \subseteq Y \times Z$ relaciji, tedaj je \df{sklop} (\df{kompozicija}, \df{kompozitum}) \df{relacij} definiran kot
                \[\srel \circ \rel \dfeq \set[1]{(x, z) \in X \times Z}{\some{y}[Y]{\rel[x][y] \land \srel[y][z]}}\]
                (po vzoru preslikav tudi sklop relacij pišemo v obratnem vrstnem redu; glej razdelek~\ref{RAZDELEK: Izpeljava preslikav iz relacij}). Opazimo: domena $\srel \circ \rel$ je domena $\rel$, kodomena $\srel \circ \rel$ je kodomena $\srel$. Sklapljanje je asociativna operacija, torej pri sklopu večih relacij oklepaji niso pomembni.

                \begin{vaja}
                        Dokaži, da je sklapljanje relacij asociativno!
                \end{vaja}

                Večkraten sklop relacije $\rel \subseteq X \times X$ same s sabo označimo
                \[\rel^n \dfeq \underbrace{\rel \circ \rel \circ \ldots \circ \rel}_{\text{$n$ $\rel$-jev}}\]
                za $n \in \NN_{\geq 2}$. Seveda je smiselno definirati, da je $\rel^1$ enak $\rel$ in da je $\rel^0$ relacija enakosti na množici $X$, saj je to enota za sklapljanje relacij na $X$, tj.~$=_X \circ \rel = \rel = \rel \circ =_X$ (premisli, da je to res!).

                \begin{zgled}
                        Naj bo $\rel \subseteq X \times X$ relacija. Tedaj iz grafa relacije zlahka razberemo, kaj je $\rel^n$: elementa $a, b \in X$ sta v relaciji $\rel^n$ natanko tedaj, ko imamo pot dolžine $n$ od $a$ do $b$ (to deluje tudi za $n = 1$ in $n = 0$). Naj primer, če je $\rel$ relacija iz zgleda~\ref{ZGLED: graf relacije}, tedaj graf relacije $\rel^3$ izgleda takole.

                        \note{graf $\rel^3$}
                \end{zgled}

                Za poljubno relacijo $\rel \subseteq X \times Y$ definiramo \df{obratno} (\df{inverzno}) \df{relacijo} kot
                \[\rel^{-1} \dfeq \set{(y, x) \in Y \times X}{\rel[x][y]}\]
                (torej ima obratna relacija glede na izvorno zamenjano domeno in kodomeno). Posledično lahko za poljubno relacijo $\rel \subseteq X \times X$ definiramo njeno potenco s poljubno celo stopnjo: $\rel^{-n} \dfeq (\rel^{-1})^n = (\rel^n)^{-1}$.

                \begin{vaja}
                        Preveri, da velja $(\srel \circ \rel)^{-1} = \rel^{-1} \circ \srel^{-1}$!
                \end{vaja}

                \begin{zgled}
                        Graf relacije, ki je obratna relaciji $\rel \subseteq X \times X$, dobimo tako, da v grafu relacije $\rel$ obrnemo puščice. Na primer, če je $\rel$ relacija iz zgleda~\ref{ZGLED: graf relacije}, tedaj graf relacije $\rel^{-1}$ izgleda takole.

                        \note{graf $\rel^{-1}$}
                \end{zgled}

                \begin{zgled}
                        Naj bo $L$ množica ljudi. Vpeljimo oznake za naslednje relacije na $L$:
                        \begin{itemize}
                                \item
                                        $\texttt{St}$ je relacija \qt{je starš od},
                                \item
                                        $\texttt{Oč}$ je relacija \qt{je oče od},
                                \item
                                        $\texttt{Ma}$ je relacija \qt{je mati od},
                                \item
                                        $\texttt{Si}$ je relacija \qt{je sin od},
                                \item
                                        $\texttt{Hč}$ je relacija \qt{je hči od},
                                \item
                                        $\texttt{Br}$ je relacija \qt{je brat od},
                                \item
                                        $\texttt{Se}$ je relacija \qt{je sestra od}
                        \end{itemize}

                        Na primer: Marko $\texttt{Br}$ Metka pomeni \qt{Marko je brat od Metke.} (oz.~v lepši slovenščini \qt{Marko je Metkin brat.}).

                        Velja med drugim:

                        \begin{tabular}{l}
                                $\texttt{Oč} \cup \texttt{Ma} = \texttt{St}$, \\
                                $\texttt{St} \circ \texttt{St} = \texttt{St}^2 = \text{\qt{je stari starš od}}$, \\
                                $\texttt{St} \circ \texttt{Br} = \text{\qt{je stric od}}$, \\
                                $\texttt{Br} \cup \texttt{Se} = \text{\qt{je sorojenec od}}$, \\
                                $\texttt{St}^{-1} = \text{\qt{je otrok od}}$, \\
                                $\bigcup_{n \in \NN_{\geq 1}} \texttt{St}^n = \text{\qt{je prednik od}}$, \\
                                $\bigcup_{n \in \NN_{\geq 1}} \texttt{St}^{-n} = \text{\qt{je potomec od}}$, \\
                                $\texttt{St} \circ (\texttt{Br} \cup \texttt{Se}) \circ \texttt{Hč} = \text{\qt{je sestrična od}}$.
                        \end{tabular}

                        Sklapljanje relacij ni komutativno; na primer $\texttt{Ma} \circ \texttt{Oč}$ je stari oče po materini strani, $\texttt{Oč} \circ \texttt{Ma}$ pa stara mama po očetovi strani.

                        \davorin{V tem zgledu sicer predpostavljamo, da je vsaka oseba bodisi moškega bodisi ženskega spola, kar ni čisto res. Ima kdo kakšno idejo, kako se temu izogniti (in še vedno imeti lahko razumljiv zgled)?}
                \end{zgled}

                \note{Na smiselnem mestu omenimo še zožitve relacij (tako domene kot kodomene).}


        \section{Lastnosti relacij}\label{RAZDELEK: Lastnosti relacij}

                Vemo, da so na primer racionalna števila uporabnejša od celih, saj lahko v okviru njih dodatno delimo --- z drugimi besedami, racionalna števila imajo več uporabne \emph{strukture} oz.~več uporabnih \emph{lastnosti}. Podobno za relacije obstajajo lastnosti, ki so se skozi prakso izkazale za zelo uporabne. Nekatere izmed njih si bomo ogledali v tem razdelku.

                Vse sledeče lastnosti se nanašajo na dvomestno relacijo z isto domeno in kodomeno.

                \begin{definicija}
                        Naj bo $\rel \subseteq X \times X$ relacija.
                        \begin{itemize}
                                \item
                                        Relacija $\rel$ je \df{povratna} (ali \df{refleksivna}), kadar velja
                                        \[\xall{x}[X]{\rel[x][x]},\]
                                        tj.~vsak element je v relaciji s samim sabo.
                                \item
                                        Relacija $\rel$ je \df{nepovratna} (ali \df{irefleksivna}), kadar velja
                                        \[\xall{x}[X]{\lnot(\rel[x][x])},\]
                                        tj.~noben element ni v relaciji s samim sabo.
                                \item
                                        Relacija $\rel$ je \df{somerna} (ali \df{simetrična}), kadar velja
                                        \[\all{x, y}[X]{\rel[x][y] \implies \rel[y][x]},\]
                                        tj.~če je en element v relaciji z drugim, je tudi drugi s prvim.
                                \item
                                        Relacija $\rel$ je \df{protisomerna} (ali \df{antisimetrična}), kadar velja
                                        \[\all{x, y}[X]{\rel[x][y] \land \rel[y][x] \implies x = y},\]
                                        tj.~dva elementa sta obojestransko v relaciji samo v primeru, če gre za en in isti element.

                                        \davorin{Mogoče pretiravam s slovenskimi imeni\ldots}
                                \item
                                        Relacija $\rel$ je \df{nesomerna} (ali \df{asimetrična}), kadar velja
                                        \[\xall{x, y}[X]{\lnot(\rel[x][y] \land \rel[y][x])},\]
                                        tj.~nobena dva elementa nista obojestransko v relaciji.
                                \item
                                        Relacija $\rel$ je \df{prehodna} (ali \df{tranzitivna}), kadar velja
                                        \[\all{x, y, z}[X]{\rel[x][y] \land \rel[y][z] \implies \rel[x][z]},\]
                                        tj.~če je en element v relaciji z drugim in drugi s tretjim, je tudi prvi v relaciji s tretjim.
                                \item
                                        Relacija $\rel$ je \df{neprehodna} (ali \df{intranzitivna}), kadar velja
                                        \[\xall{x, y, z}[X]{\lnot(\rel[x][y] \land \rel[y][z] \land \rel[x][z])},\]
                                        tj.~če je en element v relaciji z drugim in drugi s tretjim, prvi ne more tudi biti v relaciji s tretjim.
                                \item
                                        Relacija $\rel$ je \df{enolična}, kadar velja
                                        \[\all{x, y, z}[X]{\rel[x][y] \land \rel[x][z] \implies y = z},\]
                                        tj.~vsak element je v relaciji s kvečjemu enim elementom.
                                \item
                                        Relacija $\rel$ je \df{celovita}, kadar velja
                                        \[\xall{x}[X]{\xsome{y}[Y]{\rel[x][y]}},\]
                                        tj.~vsak element je v relaciji z vsaj enim elementom, se pravi $\dd{f} = \dom(f)$.
                                \item
                                        Relacija $\rel$ je \df{sovisna}, kadar velja
                                        \[\all{x, y}[X]{x \neq y \implies \rel[x][y] \lor \rel[y][x]},\]
                                        tj.~za vsaka dva različna elementa velja, da je eden od njiju v relaciji z drugim.
                                \item
                                        Relacija $\rel$ je \df{strogo sovisna}, kadar velja
                                        \[\all{x, y}[X]{\rel[x][y] \lor \rel[y][x]},\]
                                        tj.~za vsaka dva elementa velja, da je eden od njiju v relaciji z drugim.
                        \end{itemize}
                \end{definicija}

                \begin{zgled}
                        Za nekaj običajnih relacij si oglejmo njihove lastnosti.
                        \begin{itemize}
                                \item
                                        Relacija $\leq$ na $\NN$, $\ZZ$, $\QQ$, $\RR$ je refleksivna, antisimetrična, tranzitivna in strogo sovisna.
                                \item
                                        Relacija $<$ na $\NN$, $\ZZ$, $\QQ$, $\RR$ je irefleksivna, asimetrična, tranzitivna in sovisna.
                                \item
                                        Relacija deljivosti $|$ na $\NN_{\geq 1}$ je refleksivna, antisimetrična in tranzitivna.
                                \item
                                        Relacija $\subseteq$ na $\pst(X)$ je refleksivna, antisimetrična in tranzitivna.
                                \item
                                        Relacija enakosti $=_X$ na katerikoli množici $X$ je refleksivna, simetrična, antisimetrična, tranzitivna in enolična.
                        \end{itemize}
                \end{zgled}

                Lastnosti operacij smo podali z izjavami, ampak lahko jih na ekvivalenten način podamo z operacijami ali lastnostmi grafa --- glej tabelo~\ref{TABELA: Lastnosti relacije}.

                \davorin{Ko \LaTeX\ hoče biti neumen, zna biti precej neumen. Tabelo~\ref{TABELA: Lastnosti relacije} vrže na konec celotnega poglavja, čeprav mu je zapovedano, da jo naj da \emph{prav sem}.}

                \begin{table}[!ht]
                        \centering
                        \newcommand{\opis}[1]{\begin{minipage}{0.45\textwidth}\begin{center}{#1}\end{center}\end{minipage}}
                        \def\arraystretch{3}
                        \begin{tabular}{|ccc|}
                                \hline
                                \textbf{Lastnost relacije} & \textbf{Izražava z operacijami} & \textbf{Lastnost grafa} \\
                                \hline
                                refleksivnost & $=_X \subseteq \rel$ & \opis{Vsako vozlišče ima zanko.} \\
                                irefleksivnost & $=_X \cap \rel = \emptyset$ & \opis{Nobeno vozlišče nima zanke.} \\
                                simetričnost & $\rel = \rel^{-1}$ & \opis{Vsaka puščica ima nasprotno puščico.} \\
                                antisimetričnost & $\rel \cap \rel^{-1} \subseteq =_X$ & \opis{Edine puščice z nasprotnimi puščicami so zanke.} \\
                                asimetričnost & $\rel \cap \rel^{-1} = \emptyset$ & \opis{Nobena puščica nima nasprotne puščice.} \\
                                tranzitivnost & $\rel^2 \subseteq \rel$ & \opis{Za vsako pot pozitivne dolžine obstaja puščica, ki gre od začetka do konca poti.} \\
                                intranzitivnost & $\rel^2 \cap \rel = \emptyset$ & \opis{Za nobeno pot pozitivne dolžine ne obstaja puščica, ki gre od začetka do konca poti.} \\
                                enoličnost & $\rel \circ \rel^{-1} \subseteq =_X$ & \opis{Iz vsakega vozlišča gre kvečjemu ena puščica.} \\
                                celovitost & $=_X \subseteq \rel^{-1} \circ \rel$ & \opis{Iz vsakega vozlišča gre vsaj ena puščica.} \\
                                sovisnost & $=_X \cup \rel \cup \rel^{-1} = X$ & \opis{Vsaki dve različni vozlišči sta povezani s puščico.} \\
                                stroga sovisnost & $\rel \cup \rel^{-1} = X$ & \opis{Vsaki dve vozlišči sta povezani s puščico.} \\
                                \hline
                        \end{tabular}
                        \caption{Lastnosti relacije $\rel \subseteq X \times X$ in njihove karakterizacije}\label{TABELA: Lastnosti relacije}
                \end{table}

                \begin{vaja}
                        Dokaži, da so vse karakterizacije v vsaki vrstici tabele~\ref{TABELA: Lastnosti relacije} res ekvivalentne!
                \end{vaja}

                Marsikdaj imamo sledeči problem: za določene pare elementov $(x_i, y_i)_{i \in I}$ hočemo, da so v neki relaciji in relacija mora zadoščati predpisani lastnosti. Kako definirati takšno relacijo? Smiselna izbira je vzeti najmanjšo relacijo s predpisano lastnostjo, ki vsebuje vse $(x_i, y_i)$. V ta namen definiramo pojem ogrinjače relacij.

                \begin{definicija}
                        Naj bo $\rel \subseteq X \times X$ relacija in $\mathscr{L}$ lastnost relacij na $X$. Najmanjša relacija na $X$, ki vsebuje $\rel$ in ima lastnost $\mathscr{L}$, se imenuje \df{$\mathscr{L}$-ogrinjača} ali \df{$\mathscr{L}$-ovojnica} relacije $\rel$.
                \end{definicija}

                Ogrinjača relacije je dobro definirana (v smislu, da je enolično določena): če imamo dve relaciji $\rel$ in $\srel$, ki obe vsebujeta dano relacijo in imata lastnost $\mathscr{L}$ ter sta najmanjši taki, mora potem veljati, da sta vsebovani ena v drugi, tj.~$\rel \subseteq \srel$ in $\srel \subseteq \rel$, kar pomeni, da sta enaki.

                Ni pa nujno, da ogrinjača dane relacije za dano lastnost sploh obstaja --- na primer, irefleksivna ogrinjača ne obstaja za nobeno relacijo, ki ni že sama po sebi irefleksivna (premisli, zakaj). Seveda, če relacija je irefleksivna, tedaj je svoja lastna irefleksivna ogrinjača. To očitno velja v splošnem: če ima relacija lastnost $\mathscr{L}$, je enaka svoji $\mathscr{L}$-ogrinjači.

                Premislimo, kdaj smo lahko gotovi, da ogrinjača obstaja.

                \begin{definicija}
                        Naj bo $X$ množica in $\mathscr{L}$ lastnost relacij na $X$. Rečemo, da je $\mathscr{L}$ \df{presečno dedna}, kadar velja: poljuben presek relacij na $X$ z lastnostjo $\mathscr{L}$ prav tako ima lastnost $\mathscr{L}$.
                \end{definicija}

                \begin{vaja}\label{VAJA: presečna dednost zaprta za konjunkcije}
                        Dokaži: konjunkcija končno mnogo presečno dednih lastnosti relacij na dani množici je presečno dedna.
                \end{vaja}

                \begin{trditev}\label{TRDITEV: obstoj ogrinjače iz presečne dednosti}
                        Če je $\mathscr{L}$ presečno dedna lastnost relacij na $X$, tedaj za vsako relacijo $\rel$ na $X$ obstaja njena $\mathscr{L}$-ogrinjača, in sicer je enaka preseku vseh relacij na $X$, ki vsebujejo $\rel$ in imajo lastnost $\mathscr{L}$.
                \end{trditev}

                \begin{dokaz}
                        Naj bo $\srel$ presek vseh relacij na $X$, ki vsebujejo $\rel$ in imajo lastnost $\mathscr{L}$. Posledično je $\srel$ vsebovana v vseh relacijah na $X$ z lastnostjo $\mathscr{L}$, ki vsebujejo $\rel$. Ker je $\mathscr{L}$ presečno dedna lastnost, jo ima tudi $\srel$.
                \end{dokaz}

                Kako pa vemo, kdaj je lastnost presečno dedna? Včasih lahko to razberemo kar iz oblike logične formule, s katero je lastnost podana.

                \begin{izrek}\label{IZREK: presečna dednost iz logične oblike}
                        Naj bo $\mathscr{L}$ lastnost relacij na množici $X$, ki jo lahko za poljubno relacijo $\rel$ podamo z zapisom oblike
                        \[\all[1]{x_1, x_2, \ldots, x_n}[X]{\phi(\rel, x_1, x_2, \ldots, x_n) \implies \psi(\rel, x_1, x_2, \ldots, x_n)},\]
                        kjer sta $\phi(\rel, x_1, x_2, \ldots, x_n)$ in $\psi(\rel, x_1, x_2, \ldots, x_n)$ konjunkciji končno mnogo členov oblike $\rel[x_i][x_j]$ --- v posebnem primeru je lahko $\phi(\rel, x_1, x_2, \ldots, x_n)$ konjunkcija nič členov in potem je $\mathscr{L}$ podana z zapisom oblike
                        \[\xall{x_1, x_2, \ldots, x_n}[X]{\psi(\rel, x_1, x_2, \ldots, x_n)}.\]
                        Tedaj je $\mathscr{L}$ presečno dedna lastnost in torej ima vsaka relacija na $X$ $\mathscr{L}$-ogrinjačo.
                \end{izrek}

                \begin{dokaz}
                        Naj bo $(\rel_i)_{i \in I}$ poljubna družina relacij na $X$ z lastnostjo $\mathscr{L}$ in naj bo $\rel \dfeq \bigcap_{i \in I} \rel_I$ njen presek. Dokazujemo, da $\mathscr{L}$ velja za $\rel$.

                        Vzemimo poljubne $x_1, x_2, \ldots, x_n \in X$, za katere velja $\phi(\rel, x_1, x_2, \ldots, x_n)$. Ker je $\phi(\rel, x_1, x_2, \ldots, x_n)$ konjunkcija členov oblike $\rel[x_i][x_j]$, velja tudi $\phi(\rel_i, x_1, x_2, \ldots, x_n)$ za vsak $i \in I$. Po predpostavki torej velja $\psi(\rel_i, x_1, x_2, \ldots, x_n)$ za vsak $i \in I$.

                        Vzemimo poljuben člen $\rel[x_a][x_b]$ iz $\psi(\rel, x_1, x_2, \ldots, x_n)$. Videli smo, da velja $x_a \mathrel{\rel_i} x_b$ za vsak $i \in I$, torej velja $\rel[x_a][x_b]$.

                        Vidimo, da pod našimi predpostavkami velja $\psi(\rel, x_1, x_2, \ldots, x_n)$. Sklenemo, da velja lastnost $\mathscr{L}$ za relacijo $\rel$.
                \end{dokaz}

                \begin{posledica}\label{POSLEDICA: obstoj ogrinjač}
                        Za naslednje lastnosti relacij (in njihovo poljubno konjunkcijo) vselej obstaja ogrinjača: refleksivnost, simetričnost, tranzitivnost.
                \end{posledica}

                \begin{dokaz}
                        Vse izmed naštetih lastnosti se po definiciji dajo zapisati v obliki iz izreka~\ref{IZREK: presečna dednost iz logične oblike}. Za njihovo konjunkcijo glej še vajo~\ref{VAJA: presečna dednost zaprta za konjunkcije} in trditev~\ref{TRDITEV: obstoj ogrinjače iz presečne dednosti}.
                \end{dokaz}

                \begin{vaja}
                        Dokaži, da za poljubno relacijo $\rel$ na množici $X$ velja spodnja tabela!
                        \begin{center}
                                \begin{tabular}{|c|c|}
                                        \hline
                                        \textbf{Lastnost} & \textbf{Ogrinjača relacije $\rel$} \\
                                        \hline
                                        refleksivnost & $\rel \cup =_X$ \\
                                        simetričnost & $\rel \cup \rel^{-1}$ \\
                                        tranzitivnost & $\bigcup_{n \in \NN_{\geq 1}} \rel^n$ \\
                                        \hline
                                \end{tabular}
                        \end{center}
                \end{vaja}

                \note{ena izmed nalog: Za relacijo $\rel[n][(n+1)]$ na $\NN$ (ali $\ZZ$) preveri, da je njena tranzitivna ogrinjača $<$.}


        \section{Izpeljava preslikav iz relacij}\label{RAZDELEK: Izpeljava preslikav iz relacij}

                Ko definiramo temeljne matematične pojme, imamo določeno mero izbire, kaj vzamemo za izvoren pojem, kaj pa definiramo preko drugih pojmov. V tej knjigi smo od začetka vzeli preslikave za bolj osnoven pojem in relacije lahko definiramo s pomočjo preslikav (kot omenjeno v opombi~\ref{OPOMBA: definicija relacij}, relacijo lahko definiramo kot družino preslikav), lahko pa postopamo tudi obratno --- pojem preslikave izpeljemo iz pojma relacije. Kako to gre, si bomo pogledali v tem razdelku.

                \begin{definicija}
                        \df{Delna preslikava} (ali \df{delna funkcija} ali \df{parcialna funkcija}) je enolična dvomestna relacija.
                \end{definicija}

                Kot dvomestna relacija ima vsaka delna preslikava določeno domeno, kodomeno, definicijsko območje in zalogo vrednosti. Če je $f$ delna preslikava z domeno $X$ in kodomeno $Y$, to zapišemo kot $f\colon X \parto Y$.

                V primeru delne preslikave podmnožico produkta domene in kodomene, ki določa relacijo, označimo z $\graph{f}$ in imenujemo \df{graf} delne preslikave $f$ (ne zamešaj tega s prej definiranim pojmom grafa relacije --- prejšnji pojem je pomenil graf v smislu teorije grafov, sedanji pojem pa graf v smislu preslikav). Delna preslikava je torej v celoti podana z informacijo o domeni, kodomeni in grafu.

                Ideja je, da za delno preslikavo $f\colon X \parto Y$ za vsak $x \in \dd{f}$ obstaja natanko en $y \in Y$, s katerim je $x$ v relaciji. To potem zapišemo $f(x) = y$. Torej, če je $x$ v definicijskem območju, rečemo, da je $f(x)$ definiran, kar zapišemo $\isdefined{f(x)}$, in v tem primeru je $f(x)$ enak vrednosti, s katero je $x$ v relaciji. V nasprotnem primeru rečemo, da $f(x)$ ni definiran.

                Če imamo dve vrednosti, ki morda nista definirani, ni posebej smiselno pisati enakosti med njima. Smiselna relacija med njima je \df{Kleenejeva enakost}, kar pišemo $f(x) \kleq g(y)$, kar pomeni naslednje: leva stran $f(x)$ je definirana natanko tedaj, ko je definirana desna stran $g(y)$, in če sta obe definirani, sta enaki.

                \begin{zgled}
                        Deljenje na realnih številih lahko obravnavamo kot delno preslikavo $/\colon \RR \times \RR \parto \RR$; njeno definicijsko območje je $\dd{/} = \RR \times \RR_{\neq 0}$. Za vsak $x \in \RR$ velja $\frac{x}{x^2} \kleq \frac{1}{x}$, ne pa tudi $\frac{x^2}{x} \kleq x$ (premisli, zakaj).
                \end{zgled}

                \begin{zgled}
                        Delne preslikave so zelo uporabne v računalništvu. Za algoritme pričakujemo, da jim podamo vhodne podatke in bodo potem vrnili željene izhodne podatke. Zgodi se pa lahko, da se algoritem pri nekaterih vhodnih podatkih nikoli ne ustavi (ali javi napako), se pravi, ne dobimo rezultata. Če je $P$ množica možnih podatkov, lahko poljuben algoritem obravnavamo kot delno preslikavo $P \parto P$.\footnote{Natančneje, to velja za deterministične algoritme (takšne, ki se pri enakih vhodnih podatkih vedno enako obnašajo). V primeru nedeterminističnih algoritmov dobimo splošno relacijo na $P$.}

                        Izkaže se, da za nekatere probleme ne obstaja računski postopek, ki bi pri vseh možnih vnosih vrnil pravilen odgovor. Primer tega je \df{problem zaustavitve}: želimo algoritem, ki kot vhodna podatka sprejme poljuben algoritem in poljuben vnos ter se odloči, ali se dani algoritem pri danem vnosu ustavi. Kakršenkoli program, ki sprejme takšna podatka in nikoli ne vrne napačnega rezultata, nujno določa delno preslikavo, ki ni povsod definirana. \davorin{Verjetno bomo nekje hoteli imeti razdelek o diagonalizaciji; morda lahko tja dodamo dokaz te trditve.}
                \end{zgled}

                \begin{definicija}
                        \df{Preslikava} (ali \df{funkcija}) je celovita (z drugimi besedami, povsod definirana) delna preslikava. Če je domena preslikave $f$ množica $X$ in kodomena množica $Y$, to zapišemo kot $f\colon X \to Y$.
                \end{definicija}

                Seveda lahko vsako delno preslikavo zožimo do preslikave: delna preslikava $f\colon X \parto Y$ porodi preslikavo $\rstr{f}_{\dd{f}}\colon \dd{f} \to Y$.

                \begin{vaja}
                        Operacijo sklapljanja $\circ$ smo definirali za splošne relacije (razdelek~\ref{RAZDELEK: Operacije z relacijami}). Preveri, da se ta definicija ujema z običajno definicijo sklapljanja preslikav. Premisli še, kaj je sklop delnih preslikav.
                \end{vaja}


        \section{Relacije urejenosti}\label{RAZDELEK: Relacije urejenosti}

                Že od začetka tega poglavja kot klasične primere relacij podajamo razne urejenosti, kot $\leq$ in $<$. V tem razdelku si bomo ogledali, kakšne lastnosti morajo imeti relacije, da na določen način \qt{urejajo} množico.

                Sledeča definicija povzame štiri tipične primere relacij urejenosti.

                \begin{definicija}
                        Naj bo $X$ množica in $\preceq$ relacija na $X$. Tedaj:
                        \begin{itemize}
                                \item
                                        relacija $\preceq$ je \df{šibka urejenost}, kadar je refleksivna in tranzitivna,
                                \item
                                        relacija $\preceq$ je \df{delna urejenost}, kadar je antisimetrična šibka urejenost (tj.~refleksivna, tranzitivna, antisimetrična),
                                \item
                                        relacija $\preceq$ je \df{linearna urejenost}, kadar je strogo sovisna delna urejenost (tj.~refleksivna, tranzitivna, antisimetrična, strogo sovisna),
                                \item
                                        relacija $\preceq$ je \df{stroga linearna urejenost}, kadar je irefleksivna, tranzitivna in sovisna.
                        \end{itemize}
                        \davorin{Poimenovanja v zvezi s sovisnostjo in strogostjo sem povzel po trenutnih predavanjih iz Logike in množic, ampak mislim, da bi se strogost lahko naredila bolj konsistentna.}
                \end{definicija}

                V tej definiciji smo uporabili znak $\preceq$ za relacijo. Pogosto uporabimo kakšen takšen znak, če hočemo sugerirati, da gre za relacijo urejenosti.

                Tipična primera linearne oz.~stroge linearne urejenosti sta relaciji $\leq$ in $<$ na številskih množicah $\NN$, $\ZZ$, $\QQ$, $\RR$. Tipičen primer delne urejenosti je relacija inkluzije $\subseteq$ na katerikoli potenčni množici $\pst(X)$ (če ima $X$ vsaj dva elementa, ta relacija ne bo linearna).

                Primere šibkih urejenosti pogosto dobimo na sledeči način. Naj bo $f\colon X \to Y$ preslikava in $\preceq_Y$ neka relacija urejenosti na $Y$. Za poljubna $a, b \in X$ definirajmo
                \[a \preceq_X b \dfeq f(a) \preceq_Y f(b).\]
                Tudi če je $\preceq_Y$ močnejše vrste relacija --- delna ali linearna urejenost --- je relacija $\preceq_X$ v splošnem zgolj šibka urejenost na $X$.

                \note{še več primerov}

                \note{razlaga imen relacij}

                \note{najmanjši/največji, minimalni/maksimalni elementi, natančne meje}


        \section{Ekvivalenčne relacije in kvocientne množice}

                Ena temeljnih matematičnih dejavnosti je \df{abstrakcija} \davorin{pojmovanje?}, tj.~iz posamičnih primerov izluščimo njihovo temeljno, bistveno lastnost in potem delamo s to lastnostjo. \davorin{To je pomembna stvar. Dajmo to razlago čimbolj izboljšati.} Na primer, vemo, kaj pomeni \qt{pet rac}, \qt{pet avtov}, \qt{pet sekund}, ampak kaj pomeni \qt{pet}?

                V tem razdelku si bomo ogledali, kako lahko formalno abstrahiramo pojme s posamičnih primerov s pomočjo ekvivalenčnih relacij.

                \begin{definicija}
                        \df{Ekvivalenčna relacija} je relacija, ki je refleksivna, simetrična in tranzitivna.
                \end{definicija}

                Ekvivalenčne relacije tipično označimo z $\equ$ (obstaja več načinov, kako to preberemo: vijuga, tilda, kača\ldots) ali čim podobnih.

                \begin{zgled}
                        Vsaka množica $X$ ima najmanjšo ekvivalenčno relacijo --- enakost $=_X$ --- in največjo --- polno relacijo $X \times X$.
                \end{zgled}

                \begin{zgled}
                        Za poljubni celi števili $a, b \in \ZZ$ definiramo: $a$ je v relaciji z $b$, kadar sta $a$ in $b$ iste parnosti. To določa ekvivalenčno relacijo na $\ZZ$.
                \end{zgled}

                Za poljubno relacijo $\rel \subseteq X \times X$ in poljuben $a \in X$ lahko definiramo
                \[\ec[\rel]{a} \dfeq \set{x \in X}{\rel[a][x]}.\]
                Torej, $\ec[\rel]{a}$ sestoji iz vseh elementov, s katerimi je $a$ v relaciji. V primeru, da imamo ekvivalenčno relacijo $\equ$, imenujemo množico $\ec[\equ]{a}$ \df{ekvivalenčni razred} elementa $a$. Kadar je jasno, za katero ekvivalenčno relacijo gre, pogosto ekvivalenčne razrede krajše označujemo kar z $\ec{a}$.

                Bistvo ekvivalenčne relacije je, da ekvivalenčni razredi tvorijo razbitje množice.

                \davorin{Razbitje množice bomo verjetno definirali že prej, najbrž pri vsotah množic. Če ne, potem na tem mestu pride še definicija razbitja.}

                \davorin{Marko raje uporablja izraz `razdelitev množice', ker se mu `razbitje množice' zdi preveč \qt{nasilno}. ;) Kakšna so mnenja drugih? Kateri izraz bi uporabljali?}

                \begin{izrek}[ekvivalenčne relacije natanko ustrezajo razbitjem]
                        Naj bo $X$ poljubna množica.
                        \begin{enumerate}
                                \item
                                        Naslednji trditvi sta ekvivalentni za vsako relacijo $\rel$ na $X$.
                                        \begin{enumerate}
                                                \item
                                                        $\rel$ je ekvivalenčna relacija.
                                                \item
                                                        $\set[1]{\ec[\rel]{a}}{a \in X}$ je razbitje množice $X$.
                                        \end{enumerate}
                                \item
                                        Za vsako razbitje množice $X$ obstaja enolično določena ekvivalenčna relacija $\equ$ na $X$, tako da je razbitje enako $\set[1]{\ec[\equ]{a}}{a \in X}$.
                        \end{enumerate}
                \end{izrek}

                \begin{dokaz}
                        \begin{enumerate}
                                \item
                                        \begin{itemize}
                                                \item\proven{$(\text{\textit{a}} \impl \text{\textit{b}})$}
                                                \item\proven{$(\text{\textit{b}} \impl \text{\textit{a}})$}
                                        \end{itemize}
                                \item
                        \end{enumerate}
                        \note{dokončaj dokaz}
                \end{dokaz}

                Če je $\equ$ ekvivalenčna relacija na množici $X$, tedaj množico vseh njenih ekvivalenčnih razredov označimo z
                \[X/_\equ \dfeq \set[1]{\ec{a}}{a \in X}\]
                in imenujemo \df{kvocientna množica} množice $X$ po relaciji $\equ$.

                \note{kvocientna množica kot množica abstrahiranih pojmov}

                \begin{vaja}
                        Iz posledice~\ref{POSLEDICA: obstoj ogrinjač} sklepamo, da za vsako relacijo na katerikoli množici obstaja njena ekvivalenčna ogrinjača. Dokaži: če je $\rel$ relacija na množici $X$, tedaj je njena ekvivalenčna ogrinjača enaka
                        \[\bigcup_{n \in \NN} (\rel \cup \rel^{-1})^n.\]
                \end{vaja}

                \begin{vaja}
                        Naj bo $(X, \preceq)$ šibka urejenost. Za poljubna $a, b \in X$ definiramo
                        \[a \approx b \dfeq a \preceq b \land b \preceq a.\]
                        \begin{enumerate}
                                \item
                                        Preveri, da je $\approx$ ekvivalenčna relacija na množici $X$.
                                \item
                                        Na kvocientni množici $X/_\approx$ definiramo relacijo $\leq$ na sledeči način: za poljubna $a, b \in X$ naj velja
                                        \[\ec{a} \leq \ec{b} \dfeq a \preceq b.\]
                                        Dokaži, da ta predpis podaja dobro definirano relacijo na $X/_\approx$.
                                \item
                                        Dokaži: $(X/_\approx, \leq)$ je delna urejenost.
                        \end{enumerate}
                        To je kanoničen način, kako šibko urejenost okrepimo do delne urejenosti.
                \end{vaja}

                \note{Kakšen zanimiv zgled uporabe te vaje?}

                Ko smo obravnavali bijekcije v razdelku~\ref{RAZDELEK: Bijektivnost in obratne preslikave}, smo omenili, zakaj je uporabno imeti obrate preslikav. Težava je seveda, da imajo samo bijekcije obrate (v smislu, da so tudi obrati preslikave --- kot relacije seveda imajo obrate), medtem ko včasih želimo obrniti tudi druge preslikave.

                Vzemimo na primer eksponentno funkcijo $\xlam{x}{e^x}$. Če jo obravnavamo kot preslikavo $\RR \to \RR$, seveda nima obrata, saj ni surjektivna. Ideja je, da zožimo kodomeno do zaloge vrednosti --- preslikava $\xlam{x}{e^x}\colon \RR \to \RR_{> 0}$ je bijektivna in posledično lahko definiramo njen obrat (naravni logaritem) $\ln\colon \RR_{> 0} \to \RR$.

                To je standarden trik, če preslikava ni surjektivna. Kaj pa, če ni injektivna? Pogosto v tem primeru zožimo še domeno na območje, na katerem je preslikava injektivna. Na primer, preslikavo $\xlam{x}{x^2}\colon \RR \to \RR$ zožimo do bijekcije $\xlam{x}{x^2}\colon \RR_{\geq 0} \to \RR_{\geq 0}$, kjer imamo obrat $\xlam{x}{\sqrt{x}}\colon \RR_{\geq 0} \to \RR_{\geq 0}$.

                Ima pa ta prostop težave. Prvič, v nasprotju z zožanje kodomene pri zožitvi domene izgubimo določeno količino informacije o preslikavi (kam so se preslikale vrednosti, ki so prej bile v domeni, zdaj pa niso več?). Drugič, izbira zožene domene ni kanonična. Preslikavo $\xlam{x}{x^2}$ bi ravno tako lahko zožili na $\RR_{\leq 0} \to \RR_{\geq 0}$ ali na $\QQ_{\geq 0} \cup (\RR \setminus \QQ)_{\leq 0} \to \RR_{\geq 0}$ ali celo do $\emptyset \to \emptyset$ ali še neskončno drugih možnosti, pri katerih dobimo bijekcijo.

                S pomočjo kvocientov lahko rešimo te probleme in najdemo kanoničen način, kako preslikavo popraviti do injektivne (in če dodamo še zožitev kodomene, do surjektivne in torej v celoti do bijektivne). Vemo že, da sta injektivnost in surjektivnost dualni (razdelek~\ref{RAZDELEK: Injektivnost in surjektivnost}). Kaj je dualno zožitvi kodomene? Odgovor: kvocient domene. Namreč, če zožimo množico, je tako, kot da jo zdaj gledamo od precej bliže --- vidimo samo manjše območje okoli sebe. Kvocienti počnejo obratno --- tako je, kot če bi množico pogledali od precej daleč. Ne vidimo več posamičnih potez, pač pa se te združijo v bolj splošne oblike. (Seveda se ta dualnost, tako kot pri injektivnosti in surjektivnosti, da utemeljiti tudi formalno matematično. \davorin{Bomo govorili o zožkih in kozožkih?})

                \begin{izrek}[naravna razčlenitev preslikave]
                        Za vsako preslikavo $f\colon X \to Y$ obstaja (kanonična) razčlenitev
                        \[f = i \circ \tilde{f} \circ q,\]
                        kjer je $q$ surjekcija, $\tilde{f}$ bijekcija in $i$ injekcija. Konkretno, $q\colon X \to X/_\equ$ je naravna kvocientna preslikava $q(x) = \ec{x}$, pri čemer je ekvivalenčna relacija $\equ$ na $X$ definirana kot
                        \[a \equ b \dfeq f(a) = f(b),\]
                        preslikava $i\colon \rn{f} \hookrightarrow Y$ je vključitev zaloge vrednosti v kodomeno, preslikava $\tilde{f}\colon X/_\equ \to \rn{f}$ pa je v celoti določena s pogojem
                        \[\tilde{f}([x]) = f(x)\]
                        (med drugim to pomeni, da sta množici $X/_\equ$ in $\im(f)$ v bijektivni korespondenci). \davorin{To je vir raznih izrekov o izomorfizmih v algebri. A povemo kaj na to temo?}

                        Za ponazoritev, imamo spodnji diagram.

                        \note{diagram s tikz}
                \end{izrek}

                \begin{dokaz}
                        \note{napiši dokaz}
                \end{dokaz}
	
	\chapter{Aksiomatska teorija množic}
		\section{Zermelo-Fraenklovi aksiomi}
		\section{Aksiom izbire}
		\section{Kumulativna hierarhija}
	
	\chapter{Kardinalna števila}
		\section{Končnost in neskončnost}
		\section{Števnost}
		\section{Kardinalnost množice}
	
	\chapter{Ordinalna števila}
	
	\chapter{\note{možne dodatne teme}}
		\begin{itemize}
			\item
				Več o ZFC
			\item
				Strukturirane množice in njihovi morfizmi
			\item
				Kategorije
			\item
				Številske množice (med drugim aksiom o neskončnosti, Peanovi aksiomi in debata, kako definiramo strukturirano množico preko njene karakterizacije, če obenem dokažemo obstoj in enoličnost (do izomorfizma))
		\end{itemize}
	
	
\end{document}