\documentclass[11pt,a4paper,twoside]{book}


%%%%%%%%%%%%%%%%%%%%%%%%%%%%%%%%%%%%%%%%%%%%%%%%%%%%%%%%%%%%%%%%%%%%%%%%%%%%%%%%%%%%%%%%%%%%%%%%%%%%%%%%%%%%%%%%%%%%%%
%%%  Imported Packages
%%%%%%%%%%%%%%%%%%%%%%%%%%%%%%%%%%%%%%%%%%%%%%%%%%%%%%%%%
	\usepackage[slovene]{babel}
	\usepackage[utf8]{inputenc}
	\usepackage[T1]{fontenc}
	
	\usepackage{url}
	\usepackage{ifthen}
	\usepackage{amssymb}
	\usepackage{amsmath}
	\usepackage{theorem}
	\usepackage{phonetic}
	\usepackage{tablefootnote}
	\usepackage{color}
	\usepackage{xparse}
	\usepackage{ulem}
	\usepackage{charter}


%%%%%%%%%%%%%%%%%%%%%%%%%%%%%%%%%%%%%%%%%%%%%%%%%%%%%%%%%%%%%%%%%%%%%%%%%%%%%%%%%%%%%%%%%%%%%%%%%%%%%%%%%%%%%%%%%%%%%%
%%%  Theorems etc.
%%%%%%%%%%%%%%%%%%%%%%%%%%%%%%%%%%%%%%%%%%%%%%%%%%%%%%%%%%%%%
	{
		\theorembodyfont{\itshape}

		\newtheorem{izrek}{Izrek}[section]
		\newtheorem{lema}[izrek]{Lema}
		\newtheorem{trditev}[izrek]{Trditev}
		\newtheorem{posledica}[izrek]{Posledica}
	}

	{
		\theorembodyfont{\rmfamily}
		\newtheorem{definicija}[izrek]{Definicija}
		\newtheorem{opomba}[izrek]{Opomba}
		\newtheorem{primer}[izrek]{Primer}
		\newtheorem{zgled}[izrek]{Zgled}
	}

%%%%%%  Proofs
%%%%%%%%%%%%%%%%%%%%%%%%%%%%%%%%%%%%%%%%%%%%%%%%%%%%%%%%%%%%%
	\newenvironment{dokaz}{
		\goodbreak\par
		\textit{Dokaz.}%
	}{%
		\nopagebreak
		\hfill{\vrule width 1ex height 1ex depth 0ex}
		\medskip
		\goodbreak
	}
%%%%%%%%%%%%%%%%%%%%%%%%%%%%%%%%%%%%%%%%%%%%%%%%%%%%%%%%%%%%%%%%%%%%%%%%%%%%%%%%%%%%%%%%%%%%%%%%%%%%%%%%%%%%%%%%%%%%%%






%%%%%%%%%%%%%%%%%%%%%%%%%%%%%%%%%%%%%%%%%%%%%%%%%%%%%%%%%%%%%%%%%%%%%%%%%%%%%%%%%%%%%%%%%%%%%%%%%%%%%%%%%%%%%%%%%%%%%%
%%%  Commands
%%%%%%%%%%%%%%%%%%%%%%%%%%%%%%%%%%%%%%%%%%%%%%%%%%%%%%%%%%%%%%%%%%%%%%%%%%%%%%%%%%%%%%%%%%%%%%%%%%%%%%%%%%%%%%%%%%%%%%


%%%%%%  Auxiliary
%%%%%%%%%%%%%%%%%%%%%%%%%%%%%%%%%%%%%%%%%%%%%%%%%%%%%%%%%%%%%
	\newcommand{\sizedescriptor}[2]
	{
		\ifthenelse{\equal{#1}{0}}{}{
		\ifthenelse{\equal{#1}{1}}{\big}{
		\ifthenelse{\equal{#1}{2}}{\Big}{
		\ifthenelse{\equal{#1}{3}}{\bigg}{
		\ifthenelse{\equal{#1}{4}}{\Bigg}{
		#2}}}}}
	}
	\newcommand{\proven}[1]{\underline{#1}\vspace{0.2em}\\}
	\newcommand{\note}[1]{{\small\textcolor{blue}{(#1)}}}
	\newcommand{\someref}{{\small\textcolor{blue}{[\textbf{ref.}]}}}


%%%%%%  Logical Quantifiers and λ-Terms (x = no parenthesis, u = untyped)
%%%%%%%%%%%%%%%%%%%%%%%%%%%%%%%%%%%%%%%%%%%%%%%%%%%%%%%%%%%%%
	\newcommand{\all}[4][auto]{\forall\, #2 \,{\in}\, #3\,.\sizedescriptor{#1}{\left}({#4}\sizedescriptor{#1}{\right})}
	\newcommand{\some}[4][auto]{\exists\, #2 \,{\in}\, #3\,.\sizedescriptor{#1}{\left}({#4}\sizedescriptor{#1}{\right})}
	\newcommand{\exactlyone}[4][auto]{\exists\;\!!\, #2 \,{\in}\, #3\,.\sizedescriptor{#1}{\left}({#4}\sizedescriptor{#1}{\right})}
	
	\newcommand{\uall}[3][auto]{\forall\, #2\,.\sizedescriptor{#1}{\left}({#3}\sizedescriptor{#1}{\right})}
	\newcommand{\usome}[3][auto]{\exists\, #2\,.\sizedescriptor{#1}{\left}({#3}\sizedescriptor{#1}{\right})}
	\newcommand{\uexactlyone}[3][auto]{\exists\;\!!\, #2\,.\sizedescriptor{#1}{\left}({#3}\sizedescriptor{#1}{\right})}
	
	\newcommand{\xall}[3]{\forall\, #1 \,{\in}\, #2\,.\,#3}
	\newcommand{\xsome}[3]{\exists\, #1 \,{\in}\, #2\,.\,#3}
	\newcommand{\xexactlyone}[3]{\exists\;\!!\, #1 \,{\in}\, #2\,.\,#3}
	
	\newcommand{\xuall}[2]{\forall\, #1\,.\,#2}
	\newcommand{\xusome}[2]{\exists\, #1\,.\,#2}
	\newcommand{\xuexactlyone}[2]{\exists\;\!!\, #1,.\,#2}

	\newcommand{\ldm}[4]{\lambda{#1} \,{\in}\, #2\,.\,#3 \,{\in}\, #4}


%%%%%%  Sets
%%%%%%%%%%%%%%%%%%%%%%%%%%%%%%%%%%%%%%%%%%%%%%%%%%%%%%%%%%%%%
	%  \set{1, 2, 3}         ->  {1, 2, 3}
	%  \set{a \in X}{a < 1}  ->  {a ∈ X | a < 1}
	\DeclareDocumentCommand{\set}{O{auto} m G{\empty}}{ \sizedescriptor{#1}{\left} \{{#2} \ifthenelse{\equal{#3}{}}{}{ \; \sizedescriptor{#1}{\middle}| \; {#3}} \sizedescriptor{#1}{\right}\}}
	\newcommand{\vsubset}{\Mapstochar\cap}
	\newcommand{\finseq}[1]{{#1}^*}
	\newcommand{\pst}{\mathcal{P}}


%%%%%%  Number Sets, Intervals
%%%%%%%%%%%%%%%%%%%%%%%%%%%%%%%%%%%%%%%%%%%%%%%%%%%%%%%%%%%%%
	\newcommand{\NN}{\mathbb{N}}
	\newcommand{\ZZ}{\mathbb{Z}}
	\newcommand{\QQ}{\mathbb{Q}}
	\newcommand{\RR}{\mathbb{R}}
	\newcommand{\CC}{\mathbb{C}}
	\newcommand{\intoo}[3][\RR]{{#1}_{(#2, #3)}}
	\newcommand{\intcc}[3][\RR]{{#1}_{[#2, #3]}}
	\newcommand{\intoc}[3][\RR]{{#1}_{(#2, #3]}}
	\newcommand{\intco}[3][\RR]{{#1}_{[#2, #3)}}


%%%%%%  Misc.
%%%%%%%%%%%%%%%%%%%%%%%%%%%%%%%%%%%%%%%%%%%%%%%%%%%%%%%%%%%%%
	\newcommand{\intermission}{\bigskip\medskip}
	\newcommand{\df}[1]{\emph{\textbf{#1}}}  % defined notion
	\newcommand{\ism}{\cong}  % isomorphic
	\newcommand{\equ}{\sim}  % equivalent
	\newcommand{\dfeq}{\mathrel{\mathop:}=}  % definitional equality
	\newcommand{\dfeqrev}{=\mathrel{\mathop:}}  % reversed definitional equality
	\newcommand{\id}[1][]{\textrm{Id}_{#1}}  % identity map
	\newcommand{\impl}{\Rightarrow}  % implication sign
	\newcommand{\revimpl}{\Leftarrow}  % reverse implication sign
	\newcommand{\lequ}{\Leftrightarrow}  % equivalence sign
	\newcommand{\xor}{\mathbin{\veebar}}  % exclusive disjunction sign
	\newcommand{\shf}{\mathbin{\uparrow}}  % Sheffer connective
	\newcommand{\luk}{\mathbin{\downarrow}}  % Łukasiewicz connective
	\newcommand{\rstr}[1]{\left.{#1}\right|}  % map restriction
	\newcommand{\im}{\mathrm{im}}  % map image
	\newcommand{\parto}{\mathrel{\rightharpoonup}}  % partial mapping sign
	\newcommand{\qt}[1]{{\quotedblbase}{#1}{‘‘}}  % text in quotation marks
	\newcommand{\nls}[1]{\qt{\textit{#1}}}  % sentence in a natural language
	

%%%%%%%%%%%%%%%%%%%%%%%%%%%%%%%%%%%%%%%%%%%%%%%%%%%%%%%%%%%%%%%%%%%%%%%%%%%%%%%%%%%%%%%%%%%%%%%%%%%%%%%%%%%%%%%%%%%%%%






%%%%%%%%%%%%%%%%%%%%%%%%%%%%%%%%%%%%%%%%%%%%%%%%%%%%%%%%%%%%%%%%%%%%%%%%%%%%%%%%%%%%%%%%%%%%%%%%%%%%%%%%%%%%%%%%%%%%%%
%%  Page Style & Margins (A4 page = 210mm x 297mm)

\setlength{\textwidth}{15cm}
\setlength{\textheight}{224mm}

\setlength{\topmargin}{0cm}
\setlength{\evensidemargin}{0cm}
\setlength{\oddsidemargin}{\paperwidth}
\addtolength{\oddsidemargin}{-\textwidth}
\addtolength{\oddsidemargin}{-2in}

\renewcommand{\baselinestretch}{1.25}
\setlength{\parskip}{1.5ex}


%%%%%%%%%%%%%%%%%%%%%%%%%%%%%%%%%%%%%%%%%%%%%%%%%%%%%%%%%%%%%%%%%%%%%%%%%%%%%%%%%%%%%%%%%%%%%%%%%%%%%%%%%%%%%%%%%%%%%%






\begin{document}

   %--------------------------------------------------------------------
   %--------------------------------------------------------------------
   % TITLE PAGE
   
	
	\title{\Huge \textbf{\textsc{Logika in množice}}}
	\author{A.~Bauer, D.~Lešnik, M.~Petkovšek, M.~Pretnar}
	
	\maketitle
   
   
   %--------------------------------------------------------------------
   %--------------------------------------------------------------------
   % Foreword

   \chapter*{Predgovor}%\addcontentsline{toc}{chapter}{\numberline{}Predgovor}


	%--------------------------------------------------------------------
	%--------------------------------------------------------------------
	% TOC
	
	
	\addcontentsline{toc}{chapter}{\note{kazalo se naj začne na sodi strani, tako da lahko bralec naenkrat vidi celotno kazalo}}
	\tableofcontents
	
	
	%--------------------------------------------------------------------
	%--------------------------------------------------------------------
	% BODY
	
	
	\chapter{Matematično izražanje}\label{POGLAVJE: Matematično izražanje}

	\note{Za začetek bom vnašal oporne točke besedila. Slog bo verjetno treba še popraviti in besedilo dopolniti. --Davorin}
	
	\textcolor{red}{\small \textbf{Če je možno, prosim uporabljajte tabulatorje namesto presledkov za zamike v latex kodi in koda naj nima izrecno vnešenih prelomov vrstic, pač pa se v urejevalniku besedila uporablja avtomatski word wrap, ki se prilagaja širini okna. --Davorin}}
	
	Za matematično delo je bistveno, da se lahko zanašamo na pravilnost naših trditev. To pomeni:
	\begin{itemize}
		\item
			matematične izjave morajo imeti \emph{nedvoumen pomen},
		\item
			matematične izjave lahko \emph{dokažemo}.
	\end{itemize}
	
	Stavki v običajnih jezikih nimajo nedvoumnega pomena, zato matematične izjave raje podamo v \emph{matematičnem jeziku}. Za to potrebujemo \qt{matematično abecedo}, tj.~simbolni zapis, v katerem podamo izjave. Tega obravnavamo v naslednjem razdelku, dokazovanje matematičnih izjav pa v razdelku za tem.
	
	\section{Simbolni zapis}\label{RAZDELEK: Simbolni zapis}
	
		Za množice, s katerimi najpogosteje delamo, obstajajo standardne oznake (tabela~\ref{TABELA: standardne številske množice}).
		
		\begin{table}[!ht]
			\centering
			\begin{tabular}{|cc|}
				\hline
				\textbf{Množica} & \textbf{Oznaka} \\
				\hline
				množica naravnih števil & $\NN$ \\
				množica celih števil & $\ZZ$ \\
				množica racionalnih števil & $\QQ$ \\
				množica realnih števil & $\RR$ \\
				množica kompleksnih števil & $\CC$ \\
				\hline
			\end{tabular}
			\caption{standardne številske množice}\label{TABELA: standardne številske množice}
		\end{table}
		
		Nekateri $0$ vzamejo za naravno število, nekateri ne. To je v celoti stvar dogovora, kaj pomeni pojem \qt{naravno število}. Za nas bo prišlo bolj prav, če ničlo štejemo kot element množice naravnih števil, torej $\NN = \set{0, 1, 2, 3, \ldots}$.
		
		Interval realnih števil podamo s krajiščema intervala v oklepajih --- okrogli oklepaji ( ) označujejo odprtost intervala (krajišče ni vključeno v interval), oglati oklepaji [ ] pa zaprtost (krajišče je vključeno). Tako se npr.~interval realnih števil od $0$ do $1$, ki ne vsebuje krajišč, označi z $(0, 1)$, če jih vsebuje, pa z $[0, 1]$.
		
		Včasih pridejo prav tudi intervali na drugih množicah kot $\RR$. Zato se dogovorimo, da bomo intervale označevali tako, da podamo množico, ob kateri v indeksu zapišemo krajišči v oklepajih, npr.~$\intco[\NN]{1}{5} = \set{1, 2, 3, 4}$. Realna intervala iz prejšnjega odstavka tako zapišemo kot $\intoo{0}{1}$ in $\intcc{0}{1}$.
		
		Če interval v katero smer gre v nedogled, preprosto zapišemo množico z ustrezno relacijo urejenosti in krajiščem v indeksu. Na primer, $\RR_{> 0}$ označuje množico pozitivnih realnih števil, $\RR_{\geq 0}$ pa množico nenegativnih realnih števil.
		
		\note{To bi vsaj bil moj predlog. Na ta način se izognemo dvoumnostim (kar je namen). Na primer, kaj pomeni $\forall\, a > 0$? Če zapišemo $\forall\, a \in \NN_{> 0}$ ali $\forall\, a \in \RR_{> 0}$, je jasno. Razlog, da matematiki \qt{goljufajo} in pridejo skozi brez tega, je (napol dogovorjena in ponotranjena, ampak arbitrarna) izbira črk; vsak izkušen matematik ve, da $\forall\, \epsilon > 0$ pomeni $\forall\, \epsilon \in \RR_{> 0}$. Če se ne strinjate, popravite in pustite komentar. --Davorin}
		
		Izjavo, da je $2$ naravno število, zapišemo takole: $2 \in \NN$ (beri: $2$ pripada množici naravnih števil). Kako zapišemo, da je $a$ sodo število? Število je sodo, kadar je deljivo z $2$, torej pišemo $2 | a$ (beri: $2$ deli $a$).
		
		Če imamo več izjav, jih lahko strnemo v sestavljeno izjavo. Na primer, izjavo \nls{Če je $a$ sodo število, je tudi kvadrat števila $a$ sod.}, zapišemo kot $2 | a \implies 2 | a^2$.
		
		Seveda ta izjava velja za vsa naravna števila (znaš to dokazati?). To zapišemo takole: $\all{a}{\NN}{2 | a \implies 2 | a^2}$.
		
		Kot smo navajeni iz običajnih jezikov, posamične stavke povežemo v sestavljeno poved z \emph{vezniki}. Najpogosteje uporabljeni matematični vezniki so v tabeli~\ref{TABELA: standardni izjavni vezniki}.
		
		\begin{table}[!ht]
			\centering
			\begin{tabular}{|ccc|}
				\hline
				\textbf{Izjavni veznik} & \textbf{Oznaka} & \textbf{Kako preberemo} \\
				\hline
				negacija & $\lnot{p}$ & ne $p$ \\
				konjunkcija & $p \land q$ & $p$ in $q$ \\
				disjunkcija & $p \lor q$ & $p$ ali $q$ \\
				implikacija & $p \impl q$ & če $p$, potem $q$ \\
				ekvivalenca & $p \lequ q$ & $p$ natanko tedaj, ko $q$ \\
				\hline
			\end{tabular}
			\caption{standardni izjavni vezniki}\label{TABELA: standardni izjavni vezniki}
		\end{table}
		
		\begin{opomba}
			V matematiki se za izjavne veznike običajno uporabljajo zgoraj navedene tujke, ampak vsaka od njih seveda ima svoj pomen. Dobesedni prevodi teh tujk so:
			\begin{itemize}
				\item
					negacija $\to$ zanikanje,
				\item
					konjunkcija $\to$ vezava,
				\item
					disjunkcija $\to$ ločitev,
				\item
					implikacija $\to$ vpletenost,
				\item
					ekvivalenca $\to$ enakovrednost.
			\end{itemize}
			Za primerjavo: spomnite se vezalnega in ločnega priredja iz slovenščine!
		\end{opomba}
		
		\begin{zgled}
			Naj $p$ označuje stavek \nls{Zunaj dežuje.} in $q$ stavek \nls{Vzamem dežnik.}. Tedaj $\lnot{p}$ pomeni \nls{Zunaj ne dežuje.} in $p \impl q$ pomeni \nls{Če zunaj dežuje, potem vzamem dežnik.}.
		\end{zgled}
		
		Kose sestavljene izjave lahko veže več kot en veznik. V tem primeru se (tako kot pri računanju s števili) dogovorimo o prednosti veznikov. Po dogovoru je vrstni red veznikov tak, kot v tabeli~\ref{TABELA: standardni izjavni vezniki}, tj.~najmočneje veže negacija, nato konjunkcija, nato disjunkcija, nato implikacija, nato ekvivalenca. Kadar želimo, da se najprej izvede veznik z nižjo prednostjo, uporabimo oklepaje.
		
		\begin{zgled}
			Označimo sledeče stavke:
			\begin{quote}
				$p$ \ \ldots\ldots\ \nls{Imam čas.} \\
				$q$ \ \ldots\ldots\ \nls{Ostanem doma.}
			\end{quote}
			Tedaj $\lnot{p} \land q$ pomeni isto kot $(\lnot{p}) \land q$, to je \nls{Nimam časa in ostanem doma.}, medtem ko $\lnot(p \land q)$ pomeni \nls{Ni res, da imam čas in ostanem doma.}.
		\end{zgled}
		\note{Če komu pade na pamet primer boljših stavkov, je zaželjeno, da popravi\ldots --Davorin}
		
		Poleg zgoraj navedenih izjavnih veznikov se včasih uporabljajo še sledeči (tabela~\ref{TABELA: nadaljnji izjavni vezniki}).
		
		\begin{table}[!ht]
			\centering
			\begin{tabular}{|ccc|}
				\hline
				\textbf{Izjavni veznik} & \textbf{Oznaka} & \textbf{Kako preberemo} \\
				\hline
				stroga disjunkcija & $p \xor q$ & bodisi $p$ bodisi $q$ \\
				Shefferjev\tablefootnote{Henry Maurice Sheffer (1882 -- 1964) je bil ameriški logik.} veznik & $p \shf q$ & ne hkrati $p$ in $q$ \\
				Łukasiewiczev\tablefootnote{Jan Łukasiewicz (beri: \hill{u}ukaśj\^{e}vič) (1878 -- 1956) je bil poljski logik in filozof.} veznik & $p \luk q$ & niti $p$ niti $q$ \\
				\hline
			\end{tabular}
			\caption{nekateri nadaljnji izjavni vezniki}\label{TABELA: nadaljnji izjavni vezniki}
		\end{table}
		
		Za strogo disjunkcijo (tudi: ekskluzivna disjunkcija, izključitvena disjunkcija) se uporabljajo še druge oznake: $p \oplus q$, $p + q$. Razlika med navadno in strogo disjunkcijo je sledeča: $p \lor q$ pomeni, da \emph{vsaj eden} od $p$ in $q$ velja, medtem ko $p \xor q$ pomeni, da velja \emph{natanko eden}.
		
		\begin{zgled}
			Stavek \nls{Pisni del predmeta je potrebno opraviti s kolokviji ali pisnim izpitom.} je primer navadne disjunkcije (seveda se vam prizna pisni del predmeta tudi, če uspešno odpišete tako kolokvije kot pisni izpit), stavek \nls{Grem bodisi na morje bodisi v hribe.} pa je primer stroge disjunkcije (ne da se biti na dveh mestih hkrati).
		\end{zgled}
		
		Običajno veznike iz tabele~\ref{TABELA: nadaljnji izjavni vezniki} (in vse preostale, ki jih nismo navedli) izrazimo s standardnimi (glej tabelo~\ref{TABELA: izražava nadaljnjih izjavnih veznikov s standardnimi}), včasih pa je uporabno delati neposredno z njimi. Na primer, stroga disjunkcija služi kot seštevanje v Boolovem kolobarju (glej~\note{razdelek o Boolovih kolobarjih}), Shefferjev in Łukasiewiczev veznik pa se uporabljata pri preklopnih vezjih, saj je z vsakim od njiju možno izraziti vse izjavne veznike (glej~\note{razdelek o polnih naborih}). V računalništvu imajo ti trije vezniki standardne oznake XOR, NAND, NOR.
		
		\begin{table}[!ht]
			\centering
			\begin{tabular}{|ccc|}
				\hline
				\textbf{Izjavni veznik} & \multicolumn{2}{c|}{\textbf{Nekatere izražave s standardnimi vezniki}} \\
				\hline
				$p \xor q$ & $(p \lor q) \land \lnot(p \land q)$ & $(p \land \lnot{q}) \lor (\lnot{p} \land q)$ \\
				$p \shf q$ & $\lnot(p \land q)$ & $\lnot{p} \lor \lnot{q}$ \\
				$p \luk q$ & $\lnot(p \lor q)$ & $\lnot{p} \land \lnot{q}$ \\
				\hline
			\end{tabular}
			\caption{izražava nadaljnjih izjavnih veznikov s standardnimi}\label{TABELA: izražava nadaljnjih izjavnih veznikov s standardnimi}
		\end{table}
		
		\note{Na tem mestu povejmo, kakšno prednost damo tem trem veznikom v primerjavi s standardnimi. Kateremu dogovoru sledimo?}
		
		Včasih so izjave odvisne od kakšnih parametrov. Na primer, naj $\phi(x)$ pomeni \nls{$x$ je zelen.}; tedaj $\phi(\text{trava})$ pomeni \nls{Trava je zelena.}. Takim odvisnim izjavam rečemo \df{predikati} in izražajo lastnosti, ki jim parametri (\qt{spremenljivke}) lahko zadoščajo.
		
		Predikate lahko \emph{kvantificiramo} po njihovih spremenljivkah, tj.~povemo, \qt{kako pogosto} velja lastnost, dana s predikatom. Tabela~\ref{TABELA: kvantifikatorji} podaja najpogosteje uporabljane kvantifikatorje in njihove oznake.
		
		\begin{table}[!ht]
			\centering
			\begin{tabular}{|ccc|}
				\hline
				\textbf{Kvantifikator} & \textbf{Oznaka} & \textbf{Kako preberemo} \\
				\hline
				univerzalni kvantifikator & $\xall{x}{X}{\phi(x)}$ & za vsak $x$ iz $X$ velja lastnost $\phi$ \\
				eksistenčni kvantifikator & $\xsome{x}{X}{\phi(x)}$ & obstaja $x$ iz $X$ z lastnostjo $\phi$ \\
				\note{kako se temu reče?} & $\xexactlyone{x}{X}{\phi(x)}$ & obstaja natanko en $x$ iz $X$ z lastnostjo $\phi$ \\
				\hline
			\end{tabular}
			\caption{kvantifikatorji}\label{TABELA: kvantifikatorji}
		\end{table}
		
		Oznaki $\forall$ in $\exists$ sta narobe obrnjena A in E in izhajata iz nemščine (\textbf{a}ll, \textbf{e}xistiert).
		
		\begin{zgled}
			Vemo, da za vsako nenegativno realno število obstaja enolično določen nenegativen kvadratni koren; to izjavo lahko zapišemo na sledeči način.
			\[\xall{a}{\RR_{\geq 0}}\xexactlyone{b}{\RR_{\geq 0}}{b^2 = a}\]
			Zaradi tega lahko definiramo kvadratni koren kot funkcijo $\sqrt{\phantom{I}}\colon \RR_{\geq 0} \to \RR_{\geq 0}$.
		\end{zgled}
		
		Po dogovoru kvantifikatorji vežejo šibkeje kot izjavni vezniki. Izjavo, da je vsako celo število bodisi liho bodisi sodo, torej zapišemo takole.
		\[\all[2]{a}{\ZZ}{2 | a \xor 2 | (a-1)}\]
		
		\note{Se že na tem mestu predebatirajo vezane oz.~nevezane spremenljivke ter preimenovanje spremenljivk? Kaj je slovenski prevod za `dummy variable'?}
		
		\begin{zgled}
			Za poljubno naravno število $n \in \NN$ naj $P(n)$ označuje izjavo, da je $n$ praštevilo. Torej, $P$ definiramo takole.
			\[P(n) \dfeq \all[1]{x}{\NN}{x | n \implies x = 1 \xor x = n}\]
			(Premisli, kaj bi se zgodilo, če bi namesto stroge disjunkcije vzeli navadno. Bi še vedno dobili pravilni pojem praštevila?)
			
			Naj $S(n)$ označuje, da je $n$ sestavljeno število.
			\[S(n) \dfeq \xsome{x, y}{\intoo[\NN]{1}{n}}{x \cdot y = n}\]
			(Kadar imamo več zaporednih kvantifikatorjev iste vrste, jih po dogovoru lahko strnemo kot zgoraj. Dana formula za $S(n)$ je krajši zapis za $\xsome{x}{\intoo[\NN]{1}{n}}\xsome{y}{\intoo[\NN]{1}{n}}{x \cdot y = n}$.)
			
			Zdaj lahko na pregleden način zapišemo, da je vsako naravno število od $2$ naprej bodisi praštevilo bodisi sestavljeno.
			\[\all[1]{n}{\NN_{\geq 2}}{P(n) \xor S(n)}\]
		\end{zgled}
		
		\note{
			Ideje za VAJE:\\
				\hbox{}\qquad\qquad * Napiši te in te z besedami podane matematične izjave simbolno.\\
				\hbox{}\qquad\qquad * Za te in te \qt{življenjske} izjave vsak osnoven sestavni kos označi s črko in zapiši sestavljeno izjavo z mešanico veznikov in kvantifikatorjev.\\
				\hbox{}\qquad\qquad * ...
		}
	
	
	\section{Pravila dokazovanja}\label{RAZDELEK: Pravila dokazovanja}
	
		Matematične izsledke običajno podajamo preko jasno izraženih izjav. Med študijem matematike hitro opazite, da se takšne izjave podajajo pod imeni \quotesinglbase{izrek}', \quotesinglbase{trditev}', \quotesinglbase{lema}', \quotesinglbase{posledica}' in podobno. Kdaj uporabiti katerega teh imen ni natanko določeno, pač pa je prepuščeno presoji matematika. Približno vodilo je naslednje:
		\begin{itemize}
			\item
				\df{izrek}: osrednji, bistven rezultat,
			\item
				\df{trditev}: stranski rezultat,
			\item
				\df{lema}: rezultat, ki sam po sebi nima toliko vsebine, se pa uporabi pri dokazovanju pomembnejšega rezultata,\footnote{Sicer ni nujno, da se resnična pomembnost izjav takoj pokaže. Mnogo je primerov, ko se kak matematični članek po določenem času začne ceniti ne toliko zaradi glavnega izreka, pač pa zaradi neke leme, ki se je za dokaz glavnega izreka uporabila.}
			\item
				\df{posledica}: rezultat, ki je zanimiv sam po sebi, ki pa hitro sledi iz predhodne izjave.
		\end{itemize}
		
		Če skrbno analizirate izreke, trditve itd.~s predavanj (ali matematičnih člankov), opazite, da sestojijo iz treh delov: \note{kontekst, predpostavke, sklepi}
	
	
	\section{Definicije}
	
	\chapter{Konstrukcije množic}
		\section{Preprosti primeri}
			\note{prazna množica, enojci}
		\section{Podmnožice}
		\section{Produkt množic}
		\section{Vsota množic}
		\section{Unija in presek}
		\section{Eksponentna množica}
		\section{Potenčna množica}
	
	\chapter{Funkcije}
		\note{Moramo se dogovoriti, kateremu izrazu bomo dali prednost --- `funkcija' ali `preslikava'.}
		\section{Slike in praslike}
		\section{Injektivnost in surjektivnost}
			\note{Vključno z ekvivalenco z mono- in epimorfizmi.}
		\section{Bijektivnost}
			\note{Pomemben del tega razdelka bodo inverzne funkcije. Mogoče lahko to dodamo v naslov.}
	
	\chapter{Relacije}
		\section{Operacije z relacijami}
		\section{Lastnosti relacij}
			\note{Med drugim lastnosti relacij, izražene z operacijami. Mogoče združimo ta dva razdelka?}
		\section{Relacije urejenosti}
			\note{Vključno z urejenostnimi strukturami. Vključno z morfizmi?}
		\section{Ekvivalenčne relacije}
		\section{Kvocientne množice}
			\note{Sem dodajmo kanonični razcep funkcije (na surjekcijo/kvocient, bijekcijo, injekcijo/vložitev).}
	
	\chapter{Funkcije kot funkcijske relacije}
		\note{Si to zasluži lastno poglavje?}
	
	\chapter{Družine množic}
		\note{Pride to res šele tu? To pomeni, da bomo v poglavju `Konstrukcije množic' delali samo s *končnimi* produkti, vsotami itd. in na tem mestu dodali splošne?}
	
	\chapter{Kardinalna števila}
		\section{Končnost in neskončnost}
		\section{Števnost}
		\section{Kardinalnost množice}
	
	\chapter{Ordinalna števila}
	
	\chapter{Aksiom izbire}
		\note{Nekje moramo pojasniti, kaj je sploh poanta aksiomov.}
	
	\chapter{Kumulativna hierarhija}
	
	\chapter{\note{možne dodatne teme}}
		\begin{itemize}
			\item
				Več o ZFC
			\item
				Strukturirane množice in njihovi morfizmi
			\item
				Kategorije
		\end{itemize}
	
	
\end{document}