\chapter{Indukcija}\label{poglavje:indukcija}

\note{Razlaga imena `indukcija'.}

\section{Indukcija na $\NN$}
\section{Indukcija na $\ZZ_{\geq n}$}
\note{za začetek na $\NN_{\geq n}$}
\section{Indukcija na $\ZZ$}
\section{Gnezdena indukcija}
\note{za začetek dvojna indukcija na $\NN \times \NN$, nato splošnejša gnezdena na $\NN^k$}
\section{Indukcija s parametrom}
\note{indukcija na $\NN$ se posploši na $\NN \times X$; podobno z ostalimi primeri}
\section{Krepka indukcija}\label{razdelek:krepka-indukcija}
\note{tj.~indukcija, kjer v indukcijskem koraku sklepamo z vseh manjših elementov, ne le predhodnika (pomembno kasneje za posplošitev na dobro osnovano urejene množice)}
\section{Strukturna indukcija}
\note{tj.~indukcija po kompleksnosti izrazov, generiranih iz signature strukture}


\section{Dobro osnovane urejenosti}

\davorin{Zaenkrat bom ``well-founded'' prevajal kot ``dobro osnovan'', ker je ta izraz uporabljal Marko pri LMN. Ampak dobro bi bilo, da bi ta prevod še predebatirali.}

V prejšnjih razdelkih smo si ogledali mnogo različic indukcij. Čas je, da vse te različice pripeljemo pod isto streho: da najdemo splošni pojem indukcije, ki zajema prejšnje kot posebne primere. To nam omogočajo tako imenovane dobro osnovano urejene množice.

Na kratko rečeno, dobro osnovano urejene množice so množice s toliko strukture, da lahko na njih izvajamo indukcijo --- konkretno, indukcijo v krepkem smislu, kot podano v razdelku~\ref{razdelek:krepka-indukcija}. Natančna definicija je sledeča.

\begin{definicija}
        Naj bo $X$ množica in $\wf$ relacija na njej.
        \begin{itemize}
                \item
                        Za predikat $\phi$ na $X$ rečemo, da je \df{$\wf$-induktiven}, kadar za vsak $a \in X$ velja: če $\phi$ velja za vse $x \in X_{\wf a}$, tedaj velja tudi za $a$. Simbolno zapisano:
                        \[\all{a \in X} (\all{x} X_{\wf a} \phi(x)) \implies \phi(a).\]
                \item
                        Množica $X$, skupaj z relacijo $\wf$, je \df{dobro osnovano urejena}, kadar je posod resničen predikat edini $\wf$-induktiven predikat na $X$.
        \end{itemize}
\end{definicija}

Torej: če želimo dokazati, da neka lastnost $\phi$ velja za vse elemente dobro osnovano urejene množice, dokažemo indukcijski korak za $\phi$ (v smislu: če lastnost velja za vse elemente, ``manjše'' od $a$, potem velja tudi za $a$).

Zgornja definicija dobro osnovane urejenosti je neposredno naravnana na indukcijo. To je njena prednost, je pa tudi njena slabost: kako vemo, da neka relacija $\wf$ dejansko dobro osnovano ureja dano množico? Neposredno preveriti definicijo je lahko težje, kot pa neposredno preveriti željeno univerzalno kvantificirano lastnost; v tem primeru nismo nič pridobili. Zato je dobro imeti alternativne karakterizacije dobro osnovane urejenosti.

\begin{izrek}
        Naslednje izjave so ekvivalentne za poljubno množico $X$ in relacijo $\wf$ na njej.
        \begin{itemize}
                \item
                        $\wf$ dobro osnovano ureja $X$.
                \item
                        Ne obstaja neskončna padajoča veriga v $X$. Natančneje: ne obstaja zaporedje $a\colon \NN \to X$, za katerega velja $a_{n+1} \wf a_n$ za vse $n \in \NN$.
                \item
                        Vsaka nahajajoča podmnožica $S \subseteq X$ ima minimalni element v naslednjem smislu: obstaja $a \in S$, tako da za noben $x \in S$ ne velja $x \wf a$.
        \end{itemize}
\end{izrek}

\begin{dokaz}
\end{dokaz}

Splošne dobro osnovane urejenosti so lahko precej razvejane (kot bomo kasneje videli iz primerov). Včasih se zato želimo omejiti na tako imenovane dobre urejenosti

\begin{definicija}
        Množica $X$, opremljena z relacijo $\wf$, je \df{dobra urejena}, kadar je dobro osnovano urejena in relacija $\wf$ je stroga linearna urejenost.
\end{definicija}

Tudi za dobre urejenosti imamo karakterizacijo.

\begin{izrek}
        Naslednji izjavi sta ekvivalentni za poljubno množico $X$ in relacijo $\wf$ na njej.
        \begin{itemize}
                \item
                        $\wf$ dobro ureja $X$.
                \item
                        Vsaka nahajajoča podmnožica $S \subseteq X$ ima najmanjši element v naslednjem smislu: obstaja $a \in S$, tako da za vsak $x \in S \setminus \set{a}$ velja $a \wf x$.
        \end{itemize}
\end{izrek}

\begin{dokaz}
\end{dokaz}

\note{Razlaga, v kakšnem smislu so različice indukcij iz prejšnjih razdelkov posebni primeri indukcije na dobro (osnovano) urejenih množicah. Primeri indukcije na dobro (osnovano) urejenih množicah, ki niso oblike, kot podane v prejšnjih razdelkih. Omemba, da bomo kasneje (pri ordinalnih številih) spoznali še pojem transfinitne indukcije.}


%%% Local Variables:
%%% mode: latex
%%% TeX-master: "ucbenik-lmn"
%%% End:
