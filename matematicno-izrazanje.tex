\chapter{Matematično izražanje}
\label{cha:matematicno-izrazanje}

Tako kot vsaka stroka ima tudi matematika svoj strokovni jezik, ki obsega matematične
simbole in izraze ter svojevrsten način izražanja. Matematiki stremimo k popolni
natančnosti in nedvoumnosti matematične misli. To je seveda le ideal, ki se mu bolj ali
manj približamo, dejanska matematična besedila pa pišemo ljudje za ljudi, zato ni nič
nenavadnega, da so prežeta s tradicijo in nepisanimi družbenimi dogovori, ki matematiko
oddaljijo od formalnega ideala, a jo tudi naredijo humano.
%
Pred študentom matematike je torej težka naloga, saj se mora hkrati z novo matematiko
učiti še nekoliko nenavadnega jezika. V pomoč se zato najprej posvetimo samo formi
matematičnega izražanja. In ne zamerite nam, če vam dobrohotno ponudimo še kak nasvet o
študiju matematike.

Matematično komuniciranje je raznoliko, saj je namenjeno različnim publikam in zato
posredovano na različne načine. Tako v raziskovalnem matematičnem članku ne bomo našli
pojasnil in izračunov, ki jih profesor matematike zahteva od svojih študentov. In verjetno
ni dveh matematikov, ki bi uporabljala povsem usklajen matematični zapis in izrazoslovje.
Kljub temu je matematični jezik skupen vsem matematikom in v večji meri poenoten.
Nesporazume, ki nastopijo zaradi različnih navad, pa lahko rešimo s pogovorom. Vsi
izkušeni matematiki vedo, da vedo zelo malo in zato vprašajo, ko česa ne vedo. To naj bo
torej prvi nasvet: vprašajte in če ne dobite odgovora, vprašajte še enkrat.

Ker je namen tega učbenika postaviti dobre osnove matematičnega izražanja in mišljenja,
bomo bolj natančni kot večina matematikov v praksi. Začetnik namreč potrebuje oporo v
natančnosti, kasneje, ko razume stvari bolje, pa lahko ubere bližnjice, ki jih bolj
izkušeni kolegi uporabljajo, ne da bi to sploh opazili. Sproti bomo opozarjali nanje,
kakor tudi na manjše nedoslednosti v matematični praksi, ki izhajajo iz zgodovinskega
razvoja matematike.

% * Sestavni deli besedila, brez podrobnih razlag, morda primeri, tu samo opozorimo na
%   raznovrstnost konceptov.
%    * spremno besedilo
%    * konstrukcije
%    * računi
%    * izjave (sinonimi, kako jih številčimo)
%    * dokazi (kako so označeni)
%    * definicije
%    * zgledi
%    * naloge in rešitve (namigi)
%    * formule in izrazi (kako jih številčimo in kako se nanje sklicujemo)
%    * citati in reference


% PRIMERI DRUŽBENIH DOGOVOROV

% $\vec{a}$ uporabljamo za oznako vektorja

% Pri algebri v prvem letniku na FMF je običaj, da se linearno preslikavo označi z
% $\mathcal{A}$, matriko, ki tej linearni preslikavi pripada, pa z $A$.

\section{Pisave in simboli}
\label{sec:pisave-in-simboli}

Matematična abeceda vsebuje precej več simbolov, kot zgolj običajne črke in števke. Nekatere že poznamo, na primer~$=$, $<$, $+$, $\emptyset$, $\cup$, $\cap$, $\int$ in tako naprej, precej jih še bomo spoznali. Poleg tega matematiki uporabljamo različne pisave, kot je prikazano v tabeli~\ref{tabela:oblike-crk}. Na tabli in v zvezku sicer težko ločimo med pokončno, odebeljeno in ležečo pisavo, ali med kaligrafsko in rokopisno, zato nabor pisav omejimo. V tiskanem besedilu se vedno držimo nekaterih pravil glede izbire pisav. Tako posamezne črke $a$, $b$, $c$, \ldots, $x$, $y$, $z$ pišemo v ležeči pisavi, imena elementarnih funkcij pa pokončno: $\sin$, $\cos$, $\log$, \ldots Šumnikov običajno ne uporabljamo. Včasih z uporabo znakov nakažemo povezavo med dvema objektoma: $f$ je funkcija in $F$ njen integral, $\mathcal{A}$ je linearna preslikava in $A$ njej pripadajoča matrika itd.

\begin{table}[ht]
\centering
\begin{tabular}{c|c}
\textbf{Pisava} & \textbf{Črke} \\
\hline
pokončna & $\mathrm{ABCDEFGHIJKLMNOPQRSTUVWXYZ}$ \\
odebeljena & $\mathbf{ABCDEFGHIJKLMNOPQRSTUVWXYZ}$ \\
ležeča & $ABCDEFGHIJKLMNOPQRSTUVWXYZ$ \\
kaligrafska & $\mathcal{ABCDEFGHIJKLMNOPQRSTUVWXYZ}$ \\
rokopisna & $\mathscr{ABCDEFGHIJKLMNOPQRSTUVWXYZ}$ \\
frakturna & $\mathfrak{ABCDEFGHIJKLMNOPQRSTUVWXYZ}$ \\
dvopoudarjena & $\mathbb{ABCDEFGHIJKLMNOPQRSTUVWXYZ}$
\end{tabular}
\caption{Pisave}\label{tabela:oblike-crk}
\end{table}

Črke lahko dodatno opremimo s črticami, vijugami, vektorskimi znaki, strešicami in podobno:
%
\begin{equation*}
 a \quad
 a' \quad
 \dot{a} \quad
 \bar{a} \quad
 \vec{a} \quad
 \tilde{a} \quad
 \hat{a} \quad
 \check{a}.
\end{equation*}
%
\andrej{Kako prevedemo subscript in supscript?} \davorin{Sodeč po SSKJ sta \sqt{podpis} in \sqt{nadpis} ustrezna prevoda.} Uporabimo lahko tudi \emph{podpis} ali
\emph{nadpis}, ki je lahko črka, številka, ali kak drug simbol, na primer
%
\begin{equation*}
  a_i \quad
  a^i \quad
  a_1 \quad
  a_{\star} \quad
  a^{\dagger}.
\end{equation*}
%
\davorin{\sqt{Podpis} in \sqt{nadpis} sta sicer mnogo bolj pravilna izraza, ampak dobro bi bilo omeniti tudi, kako ljudje temu rečejo v praksi: reče se, da je znak v \emph{indeksu} oz.~\emph{eksponentu}. To sicer ni prav (indeksi niso vedno podpisani in nadpis ne predstavlja vedno potenciranja), ampak morda vseeno to omenimo? Mogoče šele, ko definiramo indekse (verjetno pri funkcijah kot alternativen zapis argumenta, zlasti ko gre za zaporedje)?}. Kljub temu obilju črk in oznak posežemo še po drugih abecedah, še posebej grški, zato
se jo čimprej naučite! Grške črke skupaj z njihovo izgovorjavo najdete v
tabeli~\ref{tabela:grska-abeceda}. Prostoročni zapis grških črk se boste naučili v
razredu.
%
Pa tudi to matematikom še ni dovolj! V teoriji množic uporabljamo še hebrejske črke
alef~$\aleph$, bet~$\beth$ in gimel~$\gimel$.

\begin{table}[ht]
\begin{center}
\begin{tabular}{cc|cc}
\multicolumn{2}{c|}{\textbf{Grška črka}} & \multicolumn{2}{c}{\textbf{Izgovorjava}} \\
\textit{velika} & \textit{mala} & \textit{v slovenščini} & \textit{v grščini} \\
\hline
A & $\alpha$ & alfa & alfa \\
B & $\beta$ & beta & vita \\
$\Gamma$ & $\gamma$ & gama & {\textgamma}ama \\
$\Delta$ & $\delta$ & delta & delta \\
E & $\epsilon$, $\varepsilon$ & epsilon & epsilon \\
Z & $\zeta$ & zeta & zita \\
H & $\eta$ & eta & ita \\
$\Theta$ & $\theta$, $\vartheta$ & {\scriptsize\textTheta}eta & {\scriptsize\textTheta}ita \\
I & $\iota$ & jota & jota \\
K & $\kappa$ & kapa & kapa \\
$\Lambda$ & $\lambda$ & lambda & lamda \\
M & $\mu$ & mi & mi \\
N & $\nu$ & ni & ni \\
$\Xi$ & $\xi$ & ksi & ksi \\
O & $\omicron$ & omikron & omikron \\
$\Pi$ & $\pi$, $\varpi$ & pi & pi \\
P & $\rho$, $\varrho$ & ro & ro \\
$\Sigma$ & $\sigma$, $\varsigma$ & sigma & si{\textgamma}ma \\
T & $\tau$ & ta\hill{u} & taf \\
$\Upsilon$ & $\upsilon$ & ipsilon & ipsilon \\
$\Phi$ & $\phi$, $\varphi$ & fi & fi \\
X & $\chi$ & hi & {\textchi}i \\
$\Psi$ & $\psi$ & psi & psi \\
$\Omega$ & $\omega$ & omega & ome{\textgamma}a \\
\end{tabular}
\end{center}
\par\medskip
\footnotesize{
Izgovorjava: \hill{u} je ustnični u (kot v besedi \sqt{pav});
{\textgamma} je cerkljanski \sqt{g} (nekaj med \sqt{g} in \sqt{h} --- vprašajte sošolce s tega območja);
{\scriptsize\textTheta} je angleški nezveneči \sqt{th} (kot v besedi \sqt{thing});
{\textchi} je nemški \sqt{ch} (kot v besedi \sqt{ich}).}
\caption{Grška abeceda.}
\label{tabela:grska-abeceda}
\end{table}

In zakaj pravzaprav potrebujemo tako veliko število črk? Verjetno zato, ker je v
matematiki krajši zapis bolj učinkovit, saj zasede manj prostora na papirju, pa še hitreje
ga zapišemo in preberemo. Računalničarji imajo drugačne navade, saj pri njih velja, da naj
se uporablja opisna imena, ki razkrijejo pomen: kjer bi matematik in fizik uporabila~$m$
in~$a$, bi računalničar zapisal $\mathtt{masa\_delca}$ in $\texttt{pospesek}$.

\section{Izrazi}
\label{sec:irazi}

Matematično besedilo je mešanica naravnega jezika in simbolnega zapisa. Delom besedila, ki
so napisani s simboli, pravimo \emph{simbolni izrazi} ali krajše kar \emph{izrazi}. Vsi
ste jih že videli, denimo
%
\begin{equation*}
  (3 + 4) \cdot 6 \qquad\quad
  \int_0^1 \frac{x}{1 + x^2} \, dx \qquad\quad
  a x^2 + b x + c = 0 \qquad\quad
  x > 0 \lor x \leq 0
\end{equation*}
%
Ste se kdaj vprašali, zakaj pravzaprav pišemo ulomke z vodoravno črto, integral z znakom
$\int$, zakaj ima množenje prednost pred seštevanjem in zakaj seštevamo od leve proti
desni, čeprav bi lahko tudi v drugi smeri? Odgovor je vedno isti: to so splošno sprejete
navade, ki so se izoblikovale v razvoju matematike. To niso matematične resnice, ampak
\emph{dogovori} med ljudmi. \davorin{Ki pa imajo svoje razloge. Na primer, integralski znak $\int$ je Leibniz izpeljal iz črke S, ker je na integral gledal kot na določene vrste vsoto.}

Poglejmo na primer \qt{množenje ima prednost pred seštevanjem}. Če tega dogovora ne bi
imeli, bi bil zapis $3 + 4 \cdot 6$ dvoumen: ali naj najprej seštejemo~$3$ in~$4$ ter
vsoto pomnožimo s~$6$, ali pa naj najprej zmnožimo~$4$ in~$6$ ter zmnožku prištejemo~$3$?
Da se izognemo nesporazumom, moramo zapisati bodisi $(3 + 4) \cdot 6$ bodisi
$3 + (4 \cdot 6)$, a ker so se pred mnogimi leti matematiki dogovorili, da ima množenje
prednost pred seštevanjem, smemo v enem od obeh primerov oklepaje izpustiti in prihraniti
nekaj črnila. Prav lahko si predstavljamo svet, v katerem bi obveljal drugačen dogovor in
bi $3 + 4 \cdot 6$ pomenilo $(3 + 4) \cdot 6$.

\andrej{Ta odstavek je precej \qt{oster}, če ga zna kdo omiliti, se priporočam.}
%
Kar smo ravnokar povedali, je bolj pomembno, kot se zdi na prvi pogled. V šoli so vam
namreč v glavo hkrati vlivali matematična dejstva in dogovore o matematičnem zapisu, kot
da med enimi in drugimi ni nobene razlike. Morda ni bilo časa za poglobljene pogovore o
pisanju oklepajev, ali pa so učitelji vaših učiteljev ocenili, da mladi umi ne bi razumeli
razlike. Tako marsikdo dobi vtis, da je matematika skupek predpisov, ki jih učencem
vsiljujejo učitelji, učbeniki in šolski sistem. Ker ste kljub temu prišli študirat
matematiko, je čas, da se naučite ločiti seme od kašče. A da ne bomo preveč filozofirali,
si poglejmo nekaj dogovorov in navad v zvezi s pisanjem matematičnih izrazov.

Aritmetične operacije $+$, $-$, $\cdot$ in $/$ pišemo \emph{infiksno}, tako da operacija
stoji med obema operandoma, na primer $x + y$. Kadar zapišemo operator za operand,
pravimo, da je \emph{postfiksni}, na primer faktoriela~$x!$. Zapis operatorja je
\emph{prefiksni} če stoji pred operandom, na primer nasprotna vrednost~$-x$. Poleg teh
poznamo tudi druge zapise: potenciranje pišemo z eksponentom $x^y$, deljenje z ulomkom
$\frac{x}{y}$, kvadratni koren s posebnim simbolom $\sqrt{x}$ itn. Skrajni primer je zapis
množenja brez simbola, ko namesto $x \cdot y$ zapišemo kar $x y$. \davorin{prefiks = predpona, infiks = medpona, postfiks oz.\ sufiks = pripona}

Infiksne operacije imajo \emph{precedenco} in \emph{asociativnost}. Precedenca pove,
katera operacija ima prednost: potenciranje ima prednost pred množenjem in množenje pred
seštevanjem. Asociativnost pove, ali računamo z leve ali z desne. Na primer, seštevanje
$+$ ima levo asociativnost, zato je $5 + 2 + 1$ enako $(5 + 2) + 1$. Pri seštevanju to
sicer ni pomembno, pri odštevanju pa moramo upoštevati levo asociativnost: $5 - 2 - 1$ je
enako $(5 - 2) - 1$ in ne $5 - (2 - 1)$. \andrej{Ali imamo kak primer desno asociativne
  operacije?} \davorin{Potenciranje: $a^{b^c} = a^{(b^c)}$.}

Z vidika vsebine raznolikost matematičnega zapisa ni potrebna, saj bi lahko vse izraze
pisali na isti način. Namesto simbolov, kot so $+$, $-$ in $\sqrt{\ }$, bi lahko uporabljali
besede $\mathtt{plus}$, $\mathtt{minus}$, $\mathtt{sqrt}$ in izraze pisali prefiksno. Tak
zapis je preprost in enoten, saj se nam ni treba ukvarjati z infiksnimi in postfiksnimi
operacijami, s precedenco ter z levo in desno asociativnostjo. Uporablja se v
računalništvu, a kdo bi želel na tablo namesto $3 + \sqrt{5 - 4}$ zapisati
$\mathtt{plus(3, \mathtt{sqrt}(\mathtt{minus}(5, 4)))}$?

\andrej{Ali bi pojasnili, da so izrazi drevesa?}

% * Izrazi, oklepaji
% * Izrazi kot sintaktična drevesa
% * Razni simboli, kateri se uporabljajo za kaj (omenimo LaTeX)
% * Infiksni operatorji: oklepaji, precedenca, asociativnost
% * Ostali zapisi (ulomki, množice)

% Kako je z izrazi, ki vsebujejo ...

\section{Slike in diagrami}
\label{sec:slike-in-diagrami}

% * Diagrami in slike
% * Ali se lahko zanašamo na slike?

% * Razlika med natančnim in nenatančnim zapisom
%   * ne moremo se zanašati na `...`
%   * napačna uporaba vezanih spremenljivk, recimo `f(x) ≡ 0` namesto `f ≡ 0` ali `∀ x . f(x) ≡ 0`

% Primer zavajujoče slike: dokaz, da je vsak trikotnik enakokrak.

% Omenimo knjigo Mateje Jamnik o vizualnih dokazih, morda vzamemo en primer iz nje.


%%% Local Variables:
%%% mode: latex
%%% TeX-master: "ucbenik-lmn"
%%% End:
