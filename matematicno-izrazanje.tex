\chapter{Matematično izražanje}\label{POGLAVJE: Matematično izražanje}


	\alert{Vsak ima svoj ukaz za pisanje pripomb: \textcolor{andrejcolor}{\texttt{$\backslash$andrej\{\ldots\}}}, \textcolor{davorincolor}{\texttt{$\backslash$davorin\{\ldots\}}}, \textcolor{markocolor}{\texttt{$\backslash$marko\{\ldots\}}}, \textcolor{matijacolor}{\texttt{$\backslash$matija\{\ldots\}}}, kar bo izpisalo pripombo v ustrezni barvi. Če vam vaša barva ni všeč, popravite. ;)}
	
	\davorin{Za začetek bom vnašal oporne točke besedila. Slog bo treba še popraviti in besedilo dopolniti.}
	
	\alert{Če je možno, prosim uporabljajte tabulatorje namesto presledkov za zamike v latex kodi in koda naj nima izrecno vnešenih prelomov vrstic, pač pa se v urejevalniku besedila uporablja avtomatski word wrap, ki se prilagaja širini okna. \ --Davorin}
	
	\davorin{Trenutno preurejam zadeve, tako da je nekoliko zmešnjava. Ne preveč resno vzeti trenutnega stanja.}
	
	Za matematično delo je bistveno, da se lahko zanašamo na pravilnost naših trditev. To pomeni:
	\begin{itemize}
		\item
			matematične izjave morajo imeti \emph{nedvoumen pomen},
		\item
			matematične izjave lahko \emph{dokažemo}.
	\end{itemize}
	
	Stavki v običajnih jezikih nimajo nedvoumnega pomena, zato matematične izjave raje podamo v \emph{matematičnem jeziku}. Za to potrebujemo \qt{matematično abecedo}, tj.~simbolni zapis, v katerem podamo izjave. V tem poglavju si bomo izoblikovali intuitivno predstavo temeljnih matematičnih pojmov ter spoznali simbolni zapis za delo z njimi. Kasneje bomo tem pojmom in simbolom dali natančnejši, formalni pomen. Kar se pisanja dokazov tiče, bomo pa mu namenili lastno poglavje (poglavje~\ref{POGLAVJE: Dokazovanje}).
	
	
	\section{Črke}
	
		\note{Matematična abeceda vsebuje precej več simbolov, kot zgolj običajne črke, na primer $=$ in $<$. Pa tudi črk je več, tako po obliki (A, $A$, $\mathcal{A}$, $\mathscr{A}$, $\mathfrak{A}$,\ldots) kot po vrsti. Pregled grške abecede. Morda tudi hebrejske, vsaj $\aleph$.}
	
	
	\section{Množice}\label{RAZDELEK: Množice}
	
		Pojem množice nam omogoča, da več elementov obravnavamo kot eno celoto. To naredi obravnavo bolj obvladljivo in omogoča lažji simbolni zapis.
		
		Množico z majhnim številom elementov lahko zapišemo tako, da naštejemo vse elemente in jih obdamo z zavitimi oklepaji. Torej, če zapišemo
		\[A = \set[1]{\text{jabolko}, \text{hruška}, \text{breskev}},\]
		s tem trdimo, da je $A$ množica, katere elementi so natanko jabolko, hruška in breskev.
		
		Za vsak element, ki ga zapišemo med zavitimi oklepaji, preprosto trdimo, da pripada množici, in nič več. To pomeni, da vrstni red elementov ni pomemben --- množica
		\[\set[1]{\text{breskev}, \text{hruška}, \text{jabolko}}\]
		je ista množica kot $A$. V obeh primerih smo preprosto za tri dane elemente rekli, da pripadajo množici. Nadalje, vseeno je, kolikokrat za neki dani element rečemo, da je v množici; po prvi razglasitvi imamo ravno toliko informacije, kot po deseti. Potemtakem so množice $\set{a}$, $\set{a, a}$ in $\set{a, a, a}$ vse medsebojno enake in vsebujejo natanko en element (četudi pri zadnjih dveh morda na prvi pogled izgleda drugače).
		
		Množico, ki ne vsebuje nobenega elementa, imenujemo \df{prazna množica} in jo lahko zapišemo kot $\set{}$. Ker se ta množica tako pogosto pojavlja, ima še dodatno oznako: $\emptyset$.
		
		Če ima množica mnogo elementov, je nepraktično (ali včasih celo nemogoče) vse našteti. Zapišemo lahko na primer
		\[\set{1, 2, 3, \ldots, 1000000}\]
		in od tod razberemo, da je mišljena množica vseh celih števil od ena do milijon. Na ta način lahko podajamo tudi neskončne množice: iz zapisa
		\[\set{1, 2, 3, \ldots}\]
		sklepamo, da gre za množico vseh pozitivnih celih števil.
		
		Težava s tropičjem je seveda, da je ta zapis dvoumen. Množica $\set{1, 2, 3, \ldots}$ bi lahko ravno tako naštevala na primer Fibonaccijeva števila. Nejasnim zapisom se je bolje izogniti, razen če smo popolnoma prepričani, da bo bralec kljub dvoumnosti razumel, kaj točno smo imeli v mislih.
		
		Množice, s katerimi v matematiki delamo, tipično vsebujejo števila, ali pa so vsaj na tak ali drugačen način izpeljane iz številskih množic. Spomnimo se standardnih oznak najpogosteje uporabljanih številskih množic.
		\begin{center}
			\begin{tabular}{|cc|}
				\hline
				\textbf{Množica} & \textbf{Oznaka} \\
				\hline
				množica naravnih števil & $\NN$ \\
				množica celih števil & $\ZZ$ \\
				množica racionalnih števil & $\QQ$ \\
				množica realnih števil & $\RR$ \\
				množica kompleksnih števil & $\CC$ \\
				\hline
			\end{tabular}
		\end{center}
		
		Nekateri $0$ vzamejo za naravno število, nekateri ne. To je v celoti stvar dogovora, kaj pomeni pojem \qt{naravno število}. Za nas bo prišlo bolj prav, če ničlo štejemo kot element množice naravnih števil, torej $\NN = \set{0, 1, 2, 3, \ldots}$.
		
		Interval realnih števil podamo s krajiščema intervala v oklepajih --- okrogli oklepaji ( ) označujejo odprtost intervala (krajišče ni vključeno v interval), oglati oklepaji [ ] pa zaprtost (krajišče je vključeno). Tako se npr.~interval realnih števil od $0$ do $1$, ki ne vsebuje krajišč, označi z $(0, 1)$, če jih vsebuje, pa z $[0, 1]$.
		
		Včasih pridejo prav tudi intervali na drugih množicah kot $\RR$. Zato se dogovorimo, da bomo intervale označevali tako, da podamo množico, ob kateri v indeksu zapišemo krajišči v oklepajih, npr.~$\intco[\NN]{1}{5} = \set{1, 2, 3, 4}$. Realna intervala iz prejšnjega odstavka tako zapišemo kot $\intoo{0}{1}$ in $\intcc{0}{1}$.
		
		Če interval v katero smer gre v nedogled, preprosto zapišemo množico z ustrezno relacijo urejenosti in krajiščem v indeksu. Na primer, $\RR_{> 0}$ označuje množico pozitivnih realnih števil, $\RR_{\geq 0}$ pa množico nenegativnih realnih števil.
		
		\davorin{To bi vsaj bil moj predlog. Na ta način se izognemo dvoumnostim (kar je namen). Na primer, kaj pomeni $\forall\, a > 0$? Če zapišemo $\forall\, a \in \NN_{> 0}$ ali $\forall\, a \in \RR_{> 0}$, je jasno. Razlog, da matematiki \qt{goljufajo} in pridejo skozi brez tega, je (napol dogovorjena in ponotranjena, ampak arbitrarna) izbira črk; vsak izkušen matematik ve, da $\forall\, \epsilon > 0$ pomeni $\forall\, \epsilon \in \RR_{> 0}$. Dodaten problem je, da kasneje uporabljamo urejene pare, ki jih vsi na naši fakulteti pišejo z okroglimi oklepaji. Poskusimo se izogniti zmedi, ali $(a, b)$ pomeni urejeni par ali odprti interval. Če se ne strinjate, popravite in pustite komentar.}
		
		Če imamo dan neki element in neko množico, potem pripadnost tega elementa tej množico izrazimo s simbolom $\in$. Na primer, da je štiri naravno število, zapišemo $4 \in \NN$ (beri: \qt{štiri pripada množici naravnih števil}).
		
		Elementi množic lahko zadoščajo raznim lastnostim. Na primer, recimo, da $\phi$ označuje lastnost \qt{biti manj od pet}; to potem zapišemo
		\[\phi(x) \ = \ \ x < 5.\]
		V tem primeru $x$ imenujemo \df{spremenljivka}, saj ne gre za točno določeno vrednost, pač pa predstavlja splošno število (recimo, da se dogovorimo, da s $\phi$ označujemo lastnost na realnih številih).
		
		Tovrstne lastnosti nam omogočajo, da iz neke množice odberemo elemente z dano lastnostjo in na ta način dobimo novo množico, ki je podmnožica prejšnje. Množico vseh realnih števil, ki so manjša od pet, zapišemo na naslednji način.
		\[\set{x \in \RR}{x < 5}\]
		Seveda, ker je primerjava s števili zelo pogosta lastnost, je uporabno, če uvedemo krajše oznake, ki isto povejo; že prej smo se dogovorili, da tako množico označimo z $\RR_{< 5}$. Za povsem splošne lastnosti pa ne bomo imeli vnaprej dogovorjenih oznak, zato je dobro, da poznamo splošni zapis. Torej, če je $X$ poljubna množica in $\phi$ poljubna lastnost njenih elementov, tedaj podmnožico, ki vsebuje točno tiste elemente množice $X$, ki zadoščajo lastnosti $\phi$, označimo takole.
		\[\set[1]{x \in X}{\phi(x)}\]
		
		Pri tem se zavedajmo: ni pomembno, da spremenljivko označimo ravno z $x$. Zapis
		\[\set[1]{y \in X}{\phi(y)}\]
		še vedno označuje isto množico. V vsakem primeru gre za množico vseh elementov iz $X$ z lastnostjo $\phi$. Pravzaprav sploh ni nujno, da uporabimo črko; poslušimo se lahko kateregakoli simbola. Taisto množico lahko zapišemo tudi $\set{\heartsuit \in X}{\phi(\heartsuit)}$.
		
		Kadar imamo spremenljivko, ki jo lahko preimenujemo, ne da bi spremenili pomen izraza, jo imenujemo \davorin{\df{dummy variable} --- dajmo to posloveniti.}. Takšne primere že dobro poznate; na primer, integral $\int_0^1 x^2 \,dx$ se ne spremeni, če preimenujete spremenljivko in zapišete $\int_0^1 y^2 \,dy$.
		
		\begin{zgled}
			Kako bi zapisali množico vseh sodih naravnih števil? Spomnimo se, da je število sodo, kadar je deljivo z $2$. Za $n \in \NN$ to zapišemo takole: $2 \divides n$ (beri: \qt{dve deli $n$}). Množica sodih naravnih števil se potem zapiše kot
			\[\set{n \in \NN}{2 \divides n}.\]
		\end{zgled}
	
	
	\section{Preslikave}
	
		Množice ne obstajajo povsem ločene ena od druge, pač pa so med sabo povezane s \df{preslikavami} oziroma s tujko \df{funkcijami}\footnote{Nekateri uporabljajo izraz \qt{funkcija} samo za tiste preslikave, ki slikajo v realna ali kompleksna števila, ampak ta uporaba je že nekoliko zastarela. Dandanes večina matematiko besedo \qt{funkcija} obravnava kot sopomenko besede \qt{preslikava}. Tako jo bomo uporabljali tudi v tej knjigi.}. Posamična preslikava slika elemente ene množice po določenem predpisu v elemente druge množice.
		
		Če je $f$ preslikava, ki slika iz množice $X$ v množico $Y$, to zapišemo $f\colon X \to Y$. Rečemo, da je množica $X$ \df{začetna množica} ali \df{domena} preslikave $f$, množica $Y$ pa je \df{ciljna množica} ali \df{kodomena} preslikave $f$. \davorin{Morda dodamo kot možno ime še prevod angleške besede `range', se pravi `razpon'?}
		
		Začetni množici ste v srednji šoli rekli tudi \qt{definicijsko območje}, ampak v tej knjigi bomo morali biti bolj previdni. Predpis za preslikavo po definiciji velja za vse elemente domene, ampak kasneje (v razdelku~\ref{RAZDELEK: Funkcije kot funkcijske relacije}) obravnavamo delne preslikave, ki niso definirane na celi domeni; zanje je torej definicijsko območje manjše kot domena. Držimo se torej raje zgoraj danega poimenovanja.
		
		Običaj je, da predpis preslikave podamo s pomočjo spremenljivke, tipično z oznako $x$. Na primer, če je $f$ preslikava kvadriranja, njen predpis zapišemo kot
		\[f(x) = x^2.\]
		Na tem mestu je potrebno poudariti več reči.
		\begin{itemize}
			\item
				Velikokrat površno rečemo, da zgornji predpis podaja preslikavo. To ni povsem res --- to je zgolj predpis preslikave. Za to, da preslikavo v celoti podamo, je potrebno navesti tri stvari: poleg predpisa še domeno in kodomeno. Vse to je del informacije o preslikavi.
				
				To se jasno pokaže, če začnemo razmišljati o lastnostih preslikav. Se še spomnite iz srednje šole, kaj pomeni, da je preslikava surjektivna? (Bomo ponovili v razdelku~\ref{RAZDELEK: Injektivnost in surjektivnost}.) Če vzamemo, da preslikava $f$ zadošča zgornjemu predpisu in jo obravnavamo kot preslikavo $f\colon \RR \to \RR$, ni surjektivna, če jo obravnavamo kot preslikavo $f\colon \RR_{\geq 0} \to \RR_{\geq 0}$, pa je.
			\item
				Za spremenljivko $x$ velja isto, kot smo razpravljali že v prejšnjem razdelku pri lastnostih elementov množic: spremenljivka $x$ nima vnaprej določene vrednosti, pač pa predstavlja mesto, kamor lahko vstavimo poljubno vrednost. Seveda je potem vseeno, če vzamemo kakšno drugo črko ali čisto drug simbol: $f(y) = y$ določa isti predpis kot $f(x) = x$; prav tako $f(\sun) = \sun^2$. Se pravi, tudi v tem primeru gre za \note{dummy variable}. Če si torej izberemo neko vrednost, jo lahko vstavimo na mesto spremenljivke in poračunamo, npr.~$f(3) = 3^2 = 9$ oziroma $f(2\pi) = (2\pi)^2 = 4\pi^2$. Predstavljajte si, da je spremenljivka pravzaprav škatlica, kamor lahko vstavite vrednost, torej
				\[f(\argbox) = \argbox^2.\]
			\item
				Alternativen način zapisa $f(x) = x^2$ je
				\[f\colon x \mapsto x^2.\]
				Pazimo: navadna puščica $\to$ podaja domeno in kodomeno, kot razloženo zgoraj. Repata puščica $\mapsto$ pa za posamičen element domene pove, v kateri element kodomene se preslika.
				
				Zapis z repato puščico je še posebej uporaben, kadar želimo podati preslikavo, ne da bi nam bilo potrebno izbrati ime zanjo. Na primer, realno funkcijo kvadriranja lahko v celoti podamo takole:
				\begin{align*}
					\RR &\to \RR \\
					x &\mapsto x^2
				\end{align*}
				(prva vrstica pove domeno in kodomeno, druga pa predpis). Tako podanim preslikavam potem rečemo \df{brezimne preslikave} (s tujko \df{anonimne funkcije}). Kasneje (v razdelku~\ref{RAZDELEK: Brezimne preslikave}) bomo spoznali bolj strnjen zapis takih preslikav, ki je še posebej primeren za izvajanje operacij med preslikavami; takrat bomo takšno funkcijo zapisali kot $\xlam{x}[\RR]{x^2}[\RR]$.
		\end{itemize}
	
	
	\section{Logični simboli}\label{RAZDELEK: Logični simboli}
	
		Preproste izjave, kot na primer \nls{$n$ je sodo število.}, že znamo zapisati s simboli: $2 \divides n$. Povečini pa delamo z bolj kompleksnimi, sestavljenimi izjavami. Tudi za te obstaja simbolni zapis; na primer, izjavo \nls{Če je $a$ sodo število, je tudi kvadrat števila $a$ sod.}, zapišemo kot $2 \divides a \implies 2 \divides a^2$. Seveda ta izjava velja za vsa naravna števila (znaš to dokazati?). To zapišemo takole: $\all{a}[\NN]{2 \divides a \implies 2 \divides a^2}$. V tem razdelku si bomo ogledali, kako povezati preprostejše izjave v bolj sestavljene in kako to v splošnem simbolno zapisati.
		
		Kot smo navajeni iz običajnih jezikov, posamične stavke povežemo v sestavljeno poved z \emph{vezniki}. Najpogosteje uporabljeni matematični vezniki so v tabeli~\ref{TABELA: Standardni izjavni vezniki}.
		
		\begin{table}[!ht]
			\centering
			\begin{tabular}{|ccc|}
				\hline
				\textbf{Izjavni veznik} & \textbf{Oznaka} & \textbf{Kako preberemo} \\
				\hline
				negacija & $\lnot{p}$ & ne $p$ \\
				konjunkcija & $p \land q$ & $p$ in $q$ \\
				disjunkcija & $p \lor q$ & $p$ ali $q$ \\
				implikacija & $p \impl q$ & če $p$, potem $q$ \\
				ekvivalenca & $p \lequ q$ & $p$ natanko tedaj, ko $q$ \\
				\hline
			\end{tabular}
			\caption{Standardni izjavni vezniki}\label{TABELA: Standardni izjavni vezniki}
		\end{table}
		
		\begin{opomba}
			V matematiki se za izjavne veznike običajno uporabljajo zgoraj navedene tujke, ampak vsaka od njih seveda ima svoj pomen. Dobesedni prevodi teh tujk so:
			\begin{itemize}
				\item
					negacija $\to$ zanikanje,
				\item
					konjunkcija $\to$ vezava,
				\item
					disjunkcija $\to$ ločitev,
				\item
					implikacija $\to$ vpletenost,
				\item
					ekvivalenca $\to$ enakovrednost.
			\end{itemize}
			Za primerjavo: spomnite se vezalnega in ločnega priredja iz slovenščine!
		\end{opomba}
		
		\begin{zgled}
			Naj $p$ označuje stavek \nls{Zunaj dežuje.} in $q$ stavek \nls{Vzamem dežnik.}. Tedaj $\lnot{p}$ pomeni \nls{Zunaj ne dežuje.} in $p \impl q$ pomeni \nls{Če zunaj dežuje, potem vzamem dežnik.}.
		\end{zgled}
		
		Kose sestavljene izjave lahko veže več kot en veznik. V tem primeru se (tako kot pri računanju s števili) dogovorimo o prednosti veznikov. Po dogovoru je vrstni red veznikov tak, kot v tabeli~\ref{TABELA: Standardni izjavni vezniki}, tj.~najmočneje veže negacija, nato konjunkcija, nato disjunkcija, nato implikacija, nato ekvivalenca. Kadar želimo, da se najprej izvede veznik z nižjo prednostjo, uporabimo oklepaje.
		
		\begin{zgled}
			Označimo sledeče stavke:
			\begin{quote}
				$p$ \ \ldots\ldots\ \nls{Imam čas.} \\
				$q$ \ \ldots\ldots\ \nls{Ostanem doma.}
			\end{quote}
			Tedaj $\lnot{p} \land q$ pomeni isto kot $(\lnot{p}) \land q$, to je \nls{Nimam časa in ostanem doma.}, medtem ko $\lnot(p \land q)$ pomeni \nls{Ni res, da imam čas in ostanem doma.}.
		\end{zgled}
		\davorin{Če komu pade na pamet primer boljših stavkov, je zaželjeno, da popravi\ldots}
		
		Poleg zgoraj navedenih izjavnih veznikov se včasih uporabljajo še sledeči (tabela~\ref{TABELA: Nadaljnji izjavni vezniki}).
		
		\begin{table}[!ht]
			\centering
			\begin{tabular}{|ccc|}
				\hline
				\textbf{Izjavni veznik} & \textbf{Oznaka} & \textbf{Kako preberemo} \\
				\hline
				stroga disjunkcija & $p \xor q$ & bodisi $p$ bodisi $q$ \\
				Shefferjev\tablefootnote{Henry Maurice Sheffer (1882 -- 1964) je bil ameriški logik.} veznik & $p \shf q$ & ne hkrati $p$ in $q$ \\
				Łukasiewiczev\tablefootnote{Jan Łukasiewicz (beri: \hill{u}ukaśj\^{e}vič) (1878 -- 1956) je bil poljski logik in filozof.} veznik & $p \luk q$ & niti $p$ niti $q$ \\
				\hline
			\end{tabular}
			\caption{Nekateri nadaljnji izjavni vezniki}\label{TABELA: Nadaljnji izjavni vezniki}
		\end{table}
		
		Za strogo disjunkcijo (tudi: ekskluzivna disjunkcija, izključitvena disjunkcija) se uporabljajo še druge oznake: $p \oplus q$, $p + q$. Razlika med navadno in strogo disjunkcijo je sledeča: $p \lor q$ pomeni, da \emph{vsaj eden} od $p$ in $q$ velja, medtem ko $p \xor q$ pomeni, da velja \emph{natanko eden}.
		
		\begin{zgled}
			Stavek \nls{Pisni del predmeta je potrebno opraviti s kolokviji ali pisnim izpitom.} je primer navadne disjunkcije (seveda se vam prizna pisni del predmeta tudi, če uspešno odpišete tako kolokvije kot pisni izpit), stavek \nls{Grem bodisi na morje bodisi v hribe.} pa je primer stroge disjunkcije (ne da se biti na dveh mestih hkrati).
		\end{zgled}
		
		Pogosto veznike iz tabele~\ref{TABELA: Nadaljnji izjavni vezniki} (in vse preostale, ki jih nismo navedli) kar izrazimo s standardnimi na sledeči način.
		\begin{center}
			\begin{tabular}{|ccc|}
				\hline
				\textbf{Izjavni veznik} & \multicolumn{2}{c|}{\textbf{Nekatere izražave s standardnimi vezniki}} \\
				\hline
				$p \xor q$ & $(p \lor q) \land \lnot(p \land q)$ & $(p \land \lnot{q}) \lor (\lnot{p} \land q)$ \\
				$p \shf q$ & $\lnot(p \land q)$ & $\lnot{p} \lor \lnot{q}$ \\
				$p \luk q$ & $\lnot(p \lor q)$ & $\lnot{p} \land \lnot{q}$ \\
				\hline
			\end{tabular}
		\end{center}
		
		Včasih pa vendarle raje delamo neposredno z dodatnimi vezniki. Služijo lahko kot koristna okrajšava, so pa še drugi razlogi. Na primer, stroga disjunkcija igra vlogo seštevanja v Boolovem kolobarju (glej~\note{razdelek o Boolovih kolobarjih}), Shefferjev in Łukasiewiczev veznik pa se uporabljata pri preklopnih vezjih, saj je z vsakim od njiju možno izraziti vse izjavne veznike (glej vajo~\ref{VAJA: polni nabori z enim veznikom}). V računalništvu imajo ti trije vezniki standardne oznake XOR, NAND, NOR.
		
		\davorin{Nekje tukaj povejmo, kakšno prednost damo tem trem veznikom v primerjavi s standardnimi. Kateremu dogovoru sledimo?}
		
		Včasih so izjave odvisne od kakšnih parametrov. Na primer, naj $\phi(x)$ pomeni \nls{$x$ je zelen.}; tedaj $\phi(\text{trava})$ pomeni \nls{Trava je zelena.}. Simbol $\phi$ torej predstavlja lastnost določenih objektov. Takšne primere smo imeli že v razdelku~\ref{RAZDELEK: Množice}, kjer smo navedli oznako za podmnožico tistih elementov, ki zadoščajo dani lastnosti.
		
		Lastnosti, odvisne od spremenljivk, lahko \emph{kvantificiramo} po njihovih spremenljivkah, tj.~povemo, \qt{kako pogosto} velja lastnost. Tabela~\ref{TABELA: Kvantifikatorji} podaja najpogosteje uporabljane kvantifikatorje in njihove oznake.
		
		\begin{table}[!ht]
			\centering
			\begin{tabular}{|ccc|}
				\hline
				\textbf{Kvantifikator} & \textbf{Oznaka} & \textbf{Kako preberemo} \\
				\hline
				univerzalni kvantifikator & $\xall{x}[X]{\phi(x)}$ & za vsak $x$ iz $X$ velja lastnost $\phi$ \\
				eksistenčni kvantifikator & $\xsome{x}[X]{\phi(x)}$ & obstaja $x$ iz $X$ z lastnostjo $\phi$ \\
				\note{kako se temu reče?} & $\xexactlyone{x}[X]{\phi(x)}$ & obstaja natanko en $x$ iz $X$ z lastnostjo $\phi$ \\
				\hline
			\end{tabular}
			\caption{Kvantifikatorji}\label{TABELA: Kvantifikatorji}
		\end{table}
		
		Oznaki $\forall$ in $\exists$ sta narobe obrnjena A in E in izhajata iz nemščine (\textbf{a}ll, \textbf{e}xistiert).
		
		Seveda je tudi kvantificirana spremenljivka \note{dummy} in jo lahko poljubno preimenujemo. Izjavi $\xall{x}[X]{\phi(x)}$ in $\xall{y}[X]{\phi(y)}$ povesta natanko isto: vsi elementi množice $X$ imajo lastnost $\phi$.
		
		\begin{zgled}
			Vemo, da za vsako nenegativno realno število obstaja enolično določen nenegativen kvadratni koren; to izjavo lahko zapišemo na sledeči način.
			\[\xall{a}[\RR_{\geq 0}]{\xexactlyone{b}[\RR_{\geq 0}]{b^2 = a}}\]
			Zaradi tega lahko definiramo kvadratni koren kot funkcijo $\sqrt{\phantom{I}}\colon \RR_{\geq 0} \to \RR_{\geq 0}$ \note{več o tem kasneje}.
		\end{zgled}
		
		Po dogovoru kvantifikatorji vežejo šibkeje kot izjavni vezniki. Izjavo, da je vsako celo število bodisi sodo bodisi liho, torej zapišemo takole.
		\[\all[2]{a}[\ZZ]{2 \divides a \xor 2 \divides (a-1)}\]
		
		\begin{zgled}
			Za poljubno naravno število $n \in \NN$ naj $P(n)$ označuje izjavo, da je $n$ praštevilo. Torej, $P$ definiramo takole.
			\[P(n) \dfeq \all[1]{x}[\NN]{x \divides n \implies x = 1 \xor x = n}\]
			(Premisli, kaj bi se zgodilo, če bi namesto stroge disjunkcije vzeli navadno. Bi še vedno dobili pravilni pojem praštevila?)
			
			Naj $S(n)$ označuje, da je $n$ sestavljeno število.
			\[S(n) \dfeq \xsome{x, y}[\intoo[\NN]{1}{n}]{x \cdot y = n}\]
			(Kadar imamo več zaporednih kvantifikatorjev iste vrste, jih po dogovoru lahko strnemo kot zgoraj. Dana formula za $S(n)$ je krajši zapis za $\xsome{x}[\intoo[\NN]{1}{n}]{\xsome{y}[\intoo[\NN]{1}{n}]{x \cdot y = n}}$.)
			
			Zdaj lahko na pregleden način zapišemo, da je vsako naravno število od $2$ naprej bodisi praštevilo bodisi sestavljeno.
			\[\all[1]{n}[\NN_{\geq 2}]{P(n) \xor S(n)}\]
		\end{zgled}
	
	
	\section{Definicije}
	
		\davorin{Predlagam, da v definicijah konsistentno uporabljamo `kadar' namesto `če' (\qt{Funkcija je zvezna, kadar velja to in to.}). V definicijah gre za ekvivalenco, ne implikacijo.}