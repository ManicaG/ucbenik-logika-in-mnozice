\chapter{Preproste množice}
\label{cha:preproste-mnozice}



\section{Kar je že Davorin napisal}

Pojem množice nam omogoča, da več elementov obravnavamo kot eno celoto. To naredi obravnavo bolj obvladljivo in omogoča lažji simbolni zapis.

Množico z majhnim številom elementov lahko zapišemo tako, da naštejemo vse elemente in jih obdamo z zavitimi oklepaji. Torej, če zapišemo
\[A = \set[1]{\text{jabolko}, \text{hruška}, \text{breskev}},\]
s tem trdimo, da je $A$ množica, katere elementi so natanko jabolko, hruška in breskev.

Za vsak element, ki ga zapišemo med zavitimi oklepaji, preprosto trdimo, da pripada množici, in nič več. To pomeni, da vrstni red elementov ni pomemben --- množica
\[\set[1]{\text{breskev}, \text{hruška}, \text{jabolko}}\]
je ista množica kot $A$. V obeh primerih smo preprosto za tri dane elemente rekli, da pripadajo množici. Nadalje, vseeno je, kolikokrat za neki dani element rečemo, da je v množici; po prvi razglasitvi imamo ravno toliko informacije, kot po deseti. Potemtakem so množice $\set{a}$, $\set{a, a}$ in $\set{a, a, a}$ vse medsebojno enake in vsebujejo natanko en element (četudi pri zadnjih dveh morda na prvi pogled izgleda drugače).

Množico, ki ne vsebuje nobenega elementa, imenujemo \df{prazna množica} in jo lahko zapišemo kot $\set{}$. Ker se ta množica tako pogosto pojavlja, ima še dodatno oznako: $\emptyset$.

Če ima množica mnogo elementov, je nepraktično (ali včasih celo nemogoče) vse našteti. Zapišemo lahko na primer
\[\set{1, 2, 3, \ldots, 1000000}\]
in od tod razberemo, da je mišljena množica vseh celih števil od ena do milijon. Na ta način lahko podajamo tudi neskončne množice: iz zapisa
\[\set{1, 2, 3, \ldots}\]
sklepamo, da gre za množico vseh pozitivnih celih števil.

Težava s tropičjem je seveda, da je ta zapis dvoumen. Množica $\set{1, 2, 3, \ldots}$ bi ravno tako lahko naštevala na primer Fibonaccijeva števila. Nejasnim zapisom se je bolje izogniti, razen če smo popolnoma prepričani, da bo bralec kljub dvoumnosti razumel, kaj točno smo imeli v mislih.

Množice, s katerimi v matematiki delamo, tipično vsebujejo števila, ali pa so vsaj na tak ali drugačen način izpeljane iz številskih množic. Spomnimo se standardnih oznak najpogosteje uporabljanih številskih množic.
\begin{center}
\begin{tabular}{|cc|}
\hline
\textbf{Množica} & \textbf{Oznaka} \\
\hline
množica naravnih števil & $\NN$ \\
množica celih števil & $\ZZ$ \\
množica racionalnih števil & $\QQ$ \\
množica realnih števil & $\RR$ \\
množica kompleksnih števil & $\CC$ \\
\hline
\end{tabular}
\end{center}

Nekateri $0$ vzamejo za naravno število, nekateri ne. To je v celoti stvar dogovora, kaj pomeni pojem \qt{naravno število}. Za nas bo prišlo bolj prav, če ničlo štejemo kot element množice naravnih števil, torej $\NN = \set{0, 1, 2, 3, \ldots}$.

Interval realnih števil podamo s krajiščema intervala v oklepajih --- okrogli oklepaji ( ) označujejo odprtost intervala (krajišče ni vključeno v interval), oglati oklepaji [ ] pa zaprtost (krajišče je vključeno). Tako se npr.~interval realnih števil od $0$ do $1$, ki ne vsebuje krajišč, označi z $(0, 1)$, če jih vsebuje, pa z $[0, 1]$.

Včasih pridejo prav tudi intervali na drugih množicah kot $\RR$. Zato se dogovorimo, da bomo intervale označevali tako, da podamo množico, ob kateri v indeksu zapišemo krajišči v oklepajih, npr.~$\intco[\NN]{1}{5} = \set{1, 2, 3, 4}$. Realna intervala iz prejšnjega odstavka tako zapišemo kot $\intoo{0}{1}$ in $\intcc{0}{1}$.

Če interval v katero smer gre v nedogled, preprosto zapišemo množico z ustrezno relacijo urejenosti in krajiščem v indeksu. Na primer, $\RR_{> 0}$ označuje množico pozitivnih realnih števil, $\RR_{\geq 0}$ pa množico nenegativnih realnih števil.

\davorin{To bi vsaj bil moj predlog. Na ta način se izognemo dvoumnostim (kar je namen). Na primer, kaj pomeni $\forall\, a > 0$? Če zapišemo $\forall\, a \in \NN_{> 0}$ ali $\forall\, a \in \RR_{> 0}$, je jasno. Razlog, da matematiki \qt{goljufajo} in pridejo skozi brez tega, je (napol dogovorjena in ponotranjena, ampak arbitrarna) izbira črk; vsak izkušen matematik ve, da $\forall\, \epsilon > 0$ pomeni $\forall\, \epsilon \in \RR_{> 0}$. Dodaten problem je, da kasneje uporabljamo urejene pare, ki jih vsi na naši fakulteti pišejo z okroglimi oklepaji. Poskusimo se izogniti zmedi, ali $(a, b)$ pomeni urejeni par ali odprti interval. Če se ne strinjate, popravite in pustite komentar.}

Če imamo dan neki element in neko množico, potem pripadnost tega elementa tej množico izrazimo s simbolom $\in$. Na primer, da je štiri naravno število, zapišemo $4 \in \NN$ (beri: \qt{štiri pripada množici naravnih števil}).

Elementi množic lahko zadoščajo raznim lastnostim. Na primer, recimo, da $\phi$ označuje lastnost \qt{biti manj od pet}; to potem zapišemo
\[\phi(x) \ = \ \ x < 5.\]
V tem primeru $x$ imenujemo \df{spremenljivka}, saj ne gre za točno določeno vrednost, pač pa predstavlja splošno število (recimo, da se dogovorimo, da s $\phi$ označujemo lastnost na realnih številih).

Tovrstne lastnosti nam omogočajo, da iz neke množice odberemo elemente z dano lastnostjo in na ta način dobimo novo množico, ki je podmnožica prejšnje. Množico vseh realnih števil, ki so manjša od pet, zapišemo na naslednji način.
\[\set{x \in \RR}{x < 5}\]
Seveda, ker je primerjava s števili zelo pogosta lastnost, je uporabno, če uvedemo krajše oznake, ki isto povejo; že prej smo se dogovorili, da tako množico označimo z $\RR_{< 5}$. Za povsem splošne lastnosti pa ne bomo imeli vnaprej dogovorjenih oznak, zato je dobro, da poznamo splošni zapis. Torej, če je $X$ poljubna množica in $\phi$ poljubna lastnost njenih elementov, tedaj podmnožico, ki vsebuje točno tiste elemente množice $X$, ki zadoščajo lastnosti $\phi$, označimo takole.
\[\set[1]{x \in X}{\phi(x)}\]

Pri tem se zavedajmo: ni pomembno, da spremenljivko označimo ravno z $x$. Zapis
\[\set[1]{y \in X}{\phi(y)}\]
še vedno označuje isto množico. V vsakem primeru gre za množico vseh elementov iz $X$ z lastnostjo $\phi$. Pravzaprav sploh ni nujno, da uporabimo črko; poslušimo se lahko kateregakoli simbola. Taisto množico lahko zapišemo tudi $\set{\heartsuit \in X}{\phi(\heartsuit)}$.

Kadar imamo spremenljivko, ki jo lahko preimenujemo, ne da bi spremenili pomen izraza, jo imenujemo \davorin{\df{dummy variable} --- dajmo to posloveniti.}. Takšne primere že dobro poznate; na primer, integral $\int_0^1 x^2 \,dx$ se ne spremeni, če preimenujete spremenljivko in zapišete $\int_0^1 y^2 \,dy$.

\begin{zgled}
Kako bi zapisali množico vseh sodih naravnih števil? Spomnimo se, da je število sodo, kadar je deljivo z $2$. Za $n \in \NN$ to zapišemo takole: $2 \divides n$ (beri: \qt{dve deli $n$}). Množica sodih naravnih števil se potem zapiše kot
\[\set[1]{n \in \NN}{2 \divides n}.\]
\end{zgled}

%%% Local Variables:
%%% mode: latex
%%% TeX-master: "ucbenik-lmn"
%%% End:
