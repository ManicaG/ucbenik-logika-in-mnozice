\chapter{Preproste množice}
\label{cha:preproste-mnozice}


Temeljni gradniki sodobne matematike so \df{množice}, ki so skupki ali zbirke matematičnih
objektov, lahko spet množice. Vsaka množica sestoji iz \df{elementov} in je z njimi
natančno določena. Kadar je $a$ element množice $M$, to zapišemo kot $a \in M$.

Ideja množice kot poljubne zbirke elementov je zavajajoče preprosta, kar so na lastni koži
izkusili matematiki na prelomu iz 19.~v 20.~stoletje. Takrat so že vedeli, da so množice zelo
uporabne in da lahko iz njih tvorimo razne vrste matematičnih objektov. A znameniti
matematik in filozof Bertrand Russell je odkril paradoks, ki se imenuje po njem, in gre
takole. Naj bo~$R$ množica vseh množic, ki niso element same sebe. Ali $R$ je element~$R$?
Če je $R$ element $R$, potem iz definicije $R$ sledi, da $R$ ni element $R$. In če $R$ ni
element $R$, spet iz definicije $R$ sledi, da $R$ je element $R$. Torej $R$ hkrati je in
ni svoj element, kar je protislovje! Russellov paradoks ste morda že spoznali v
priljubljeni različici, ki govori o vaškem brivcu, ki brije vse vaščane, ki ne brijejo
samih sebe.

Russellov paradoks je povzročil pravo krizo v temeljih matematike. Ker so bile množice
nepogrešljivo orodje, jih niso hoteli kar zavreči, po drugi strani pa je bilo treba
preprečiti Russellov in druge paradokse, ki so jih še odkrili. Bertrand Russell je
predlagal rešitev, ki jo je poimenoval \df{teorija tipov}. Russellova teorija tipov je
pomembno vplivala na nadaljni razvoj temeljev matematike, sodobna teorija tipov pa je
pomembno orodje v računalništvu. Tako kot množice so bili tipi skupki elementov, a so
tvorili neskončno hierarhijo, v kateri so bili elementi tipa vedno iz nižjega nivoja
hierarhije kot tip, ki so mu pripadali. Za potrebe večine matematike zadostuje že
preprostejša dvoslojna hierarhija množic in \df{razredov}. Množice smejo biti elementi
množic in razredov, razredi pa ne. Russellov paradoks izgine, ker je $R$ razred vseh
tistih množic, ki niso same svoj element. Vprašanje, ali je $R$ element samega sebe, tako
postane nesmiselno, saj $R$ ni množica. A zaenkrat odložimo podrobnejšo obravnavo razredov
in se raje posvetimo osnovnima pojmoma, množica in preslikava.

V splošni razpravi o množicah, ki bi presegala meje matematične vede, bi se opirali na
zgodovinski in družbeni kontekst, jezikovni izvor in rabo besed `množica', `skupek' in
`zbirka', kognitivno analizo, eksperimente, filozofijo itn. Vsi ti vidiki so za matematike
izjemo koristni, saj iz takih ``pred-matematičnih'' obravnav črpamo sveže zamisli in
matematiko naredimo zares uporabno. Ko pa delujemo znotraj matematike, zunanje vplive
odmislimo in se zanašamo le še na pravila logičnega sklepanja in matematične zakone, da ne
prihaja do nejasnosti in dvomljivih sklepov.

Kot matematiki lahko ustvarimo takšen ali drugačen pojem množice in pri tem imamo popolno
svobodo. Se množica lahko spreminja ali vedno vsebuje iste elemente? Je pomemben vrsti red
elementov v množici? Sme množica biti element same sebe? Ali morajo biti elementi množice
izračunljivi? To so vprašanja, ki nimajo enoznačnega odgovora. In res je znanih več med
seboj nezdružljivih zvrsti teorije množic, ki matematično opredeljujejo različne vidike
običajnega razumevanja besede `množica'. Mi bomo spoznali ``standardno'' teorijo množic,
ki jo uporablja velika večina matematikov.


\section{Princip ekstenzionalnosti}
\label{sec:princip-ekstenzionalnosti}

Zamisel, da je množica natančno določena s svojimi elementi, izrazimo z matematičnim
zakonom, ki mu pravimo \df{princip ekstenzionalnosti}:

\begin{pravilo}[Princip ekstenzionalnosti]
  Množici sta enaki, če vsebujeta iste elemente.
\end{pravilo}

Kaj pravzaprav pomeni, da je to ``pravilo'', ``matematični zakon'' ali ``princip''? So ga
razglasili v parlementu, je to zakon narave, ali morda dogma, ki jo je razglasil profesor
na predavanjih? Bodo tisti, ki principa ekstenzionalnosti ne spoštujejo, deležni Lešnikove
masti? Ne. Matematični zakoni so \emph{dogovori}, nekakšna pravila matematične igre. V
zgodovinskem razvoju matematike so se uveljavili tisti dogovori, ki so bili uporabni v
naravoslovju in tehniki, ali pa so v njih matematiki videli notranjo lepoto in lastno
uporabno vrednost.

Pravkar smo se dogovorili, da bomo obravnavali matematične objekte množice, ki vsebujejo
elemente in da zanje velja princip ekstenzionalnosti. Namesto besed `množica' in `element'
bi lahko izbrali tudi kaki drugi besedi, denimo `zbor' in `član', ali celo `morje' in
`riba', s čimer se matematična vsebina pojmov ne bi čisto nič spremenila, čeprav ne gre
preveč izzivati svojih stanovskih kolegic in kolegov. Strukturo, lastnosti in povezave med
matematičnimi objekti namreč določajo dogovorjeni matematični zakoni in ne besede, s
katerimi jih poimenujemo.

Še enkrat poudarimo, da ima vsakdo, še posebej pa mladi um, popolno svobodo matematičnega
ustvarjanja. Želite razmišljati o drugačnih množicah, ki ne zadoščajo principu
ekstenzionalnsti? Ali pa o številih, ki zadoščajo zakonu $x + x = 0$? O geometriji, v
kateri skozi točko lahko potegnemo dve vzporednici k dani premici? Kar dajte! Pri tem vas
le prosimo, celo zahtevamo, da razmišljate temeljito, vztrajno in globoko, da ste iskreni
do sebe in ostalih ter da svoje zamisli in spoznanja predstavite na matematikom razumljiv
način.

Vrnimo se k našim množicam. Princip ekstenzionalnosti nam pove, da lahko množico podamo
tako, da natančno opredelimo njene elemente. A to ne pomeni, da množica obstaja, brž ko jo
lahko natančno opredelimo! To je pot, ki vodi naravnost do Russelovega paradoksa, saj so
elementi paradoksalne množice~$R$ natančno opredeljeni. Potrebujemo dodatna pravila, ki
določajo dopustne \df{konstrukcije množic}. Izbrati jih moramo previdno, da se izognemo
težavam.

\section{Končne množice}
\label{sec:koncne-mnozice}

Posebej preprosta konstrukcija množic združi končen nabor matematičnih objektov v množico.
Na primer, če so $a$, $b$ in $c$ matematični objekti, potem lahko tvorimo množico
%
\begin{equation*}
  \set{a, b, c}
\end{equation*}
%
katere objekti so natanko $a$, $b$ in $c$. To pomeni, da za vsak matematični objekt~$x$
velja
%
\begin{equation*}
  \text{$x \in \set{a, b, c}$, če in samo če $x = a$ ali $x = b$ ali $x = c$.}
\end{equation*}
%
Fraza ``če in samo če'' tu pomeni, da velja dvoje:
%
\begin{enumerate}
\item Če $x = a$ ali $x = b$ ali $x = c$, potem $x \in \set{a, b, c}$.
\item Če $x \in \set{a, b, c}$, potem $x = a$ ali $x = b$ ali $x = c$.
\end{enumerate}
%
Tako nam na primer prva trditev zagotavlja $1+1 \in \set{1, 2, 4}$, ker velja
vsaj ena od možnosti: $1 + 1 = 1$ ali $1 + 1 = 2$ ali $1 + 1 = 3$. Iz druge trditve sledi, da
$5 \in \set{1, 2, 3}$ ne velja, ker ne velja nobena od možnosti: $5 = 1$ ali $5 = 2$ ali
$5 = 3$.

Splošna konstrukcija končnih množic poteka takole.

\begin{pravilo}
  \label{pravilo:koncna-mnozica}
  Iz poljubnih objektov $a$, $b$, \dots, $z$ lahko tvorimo množico
  $\set{a, b, \ldots, z}$, katere elementi so natanko objekti $a$, $b$, \dots, $z$.
\end{pravilo}

Za trenutek ustavimo tok misli in opozorimo, da zapis s tropičjem `$\ldots$' ni dovolj
natančen, saj dopušča dvoumnosti. Denimo, so elementi množice
%
\begin{equation*}
  \set{3, 5, 7, \ldots, 31},
\end{equation*}
%
liha števila med $3$ in $31$, ali samo praštevila? Zapis res ni dovolj natančen. Kljub
temu tak zapis v praksi uporabljamo, ker v praksi bralec večinoma pravilno ugane, kaj je
bilo mišljeno, saj imamo ljudje zelo podobne sposobnosti prepoznavanja vzorcev. Z
matematičnega vidika pa to ni dopustno, saj lahko tropičje \emph{vedno} razumemo na več
načinov. (Ne verjamete? Naslednji člen v zaporedju $1, 2, 3, \ldots$ je seveda~$5$, ker je
naslednji člen vsota prejšnjih dveh, kot v Fibonaccijevem zaporedju.)

Kot smo že omenili, želimo pojem množice, pri kateri vrstni red elementov ni pomemben.
Torej bi morali biti množici $\set{1, 2}$ in $\set{2, 1}$ enaki. Pa je to res? Velja ena
od treh možnosti:
%
\begin{enumerate}
\item Iz principa ekstenzionalnosti in konstrukcije množic $\set{1, 2}$ in $\set{2, 1}$ sledi, da sta enaki.
\item Iz principa ekstenzionalnosti in konstrukcije množic $\set{1, 2}$ in $\set{2, 1}$ sledi, da nista enaki.
\item Princip ekstenzionalnosti in konstrukcije množic $\set{1, 2}$ in $\set{2, 1}$ ne določajo, ali sta enaki.
\end{enumerate}
%
V prvem primeru bi želeli dokazati enakost. V drugem primeru smo v zagati, saj smo se
dogovorili za matematična pravila, ki imajo neželene posledice. V tretjem primeru moramo
dodati še kakšne nove zakone o množicah. Na srečo obvelja prva možnost.

\begin{trditev}
  Množici $\set{1, 2}$ in $\set{2, 1}$ sta enaki.
\end{trditev}

\begin{dokaz}
  Dokaz, ki ga bomo zapisali je izjemno podroben in ga v praksi matematik ne bi zapisal,
  saj je z njegovim branjem več dela, kot če bi naredili sami. Ker pa želimo pokazati, da
  tudi najbolj trivialna dejstva lahko dokažemo, ga zapišimo.

  Izhajati smemo izključno iz naslednji dejstev:
  %
  \begin{itemize}
  \item princip ekstenzionalnosti,
  \item $x \in \set{1, 2}$, če in samo če $x = 1$ ali $x = 2$,
  \item $x \in \set{2, 1}$, če in samo če $x = 2$ ali $x = 1$.
  \end{itemize}
  %
  Najprej uporabimo princip ekstenzionalnosti, ki zagotavlja, da sta $\set{1, 2}$ in
  $\set{2, 1}$ enaki, če imata iste elemente. Dokažimo torej, da imata iste elemente. To
  naredimo v dveh korakih:
  %
  \begin{enumerate}
  \item Dokažimo, da za vsak element $\set{1, 2}$ dokažemo, da je element $\set{2, 1}$.
    Naj bo $x \in \set{1, 2}$. Iz definicije množice $\set{1, 2}$
    sledi, da je $x = 1$ ali $x = 2$. Obravnavamo dva podprimera:
    %
    \begin{enumerate}
    \item Primer $x = 1$: iz $x = 1$ sledi, da je $x = 2$ ali $x = 1$, zato je $x \in \set{2, 1}$.
    \item Primer $x = 2$: iz $x = 2$ sledi, da je $x = 2$ ali $x = 1$, zato je $x \in \set{2, 1}$.
    \end{enumerate}
    % 
  \item Dokažimo, da za vsak element $\set{2, 1}$ dokažemo, da je element $\set{1, 2}$.

    Ta korak je povsem podoben prvemu, le da je treba povsod zamenjati~$1$ in~$2$.
    Matematik bi zato na tem mestu zapisal, da je drugi korak podoben prevemu in dokaz
    zaključil. A tega tokrat ne bomo storili in bomo zapisali popoln dokaz.

    Naj bo $x \in \set{2, 1}$. Iz definicije množice $\set{2, 1}$ sledi, da je $x = 2$ ali
    $x = 1$. Obravnavamo dva primera:
    %
    \begin{enumerate}
    \item Primer $x = 2$: iz $x = 2$ sledi, da je $x = 1$ ali $x = 2$, zato je $x \in \set{1, 2}$.
    \item Primer $x = 1$: iz $x = 1$ sledi, da je $x = 1$ ali $x = 2$, zato je $x \in \set{1, 2}$. \qedhere
    \end{enumerate}
    %
  \end{enumerate}
\end{dokaz}

Mimogrede, črn kvadratek označuje konec dokaza. Imenuje se tudi ``Halmos'' po matematiku
Paulu Halmosu, ki ga je prvi uporabljal. S podobnim razmislekom, ki ga prepuščamo za vajo,
lahko dokažemo, da ni pomembno, ali se element pojavi enkrat ali večkrat.

\begin{naloga}
  Podrobno dokažite, da sta množici $\set{1, 1, 2}$ in $\set{1, 2}$ enaki.
\end{naloga}

V prejšnji nalogi smo zapisali $\set{1, 1, 2}$. Pa je to sploh dovoljeno?
Pravilo~\ref{pravilo:koncna-mnozica} pravi, da lahko iz objektov $a, b, c, \ldots, z$
tvorimo končno množico $\set{a, b, \ldots, z}$. Nikjer ne piše, da smeta biti $a$ in $b$
enaka, zato je upravičeno vprašanje, ali je dovoljeno za $a$ in $b$ vzeti~$1$. V
matematiki vse razumemo dobesedno. V pravilu~\ref{pravilo:koncna-mnozica} piše ``iz
poljubnih objektov'', torej imamo povsem proste roke. Povedano z drugimi besedami, množico
$\set{1, 1, 2}$ smemo tvoriti, ker nikjer ne piše, da morajo biti elementi različni.

V zvezi s pravilom~\ref{pravilo:koncna-mnozica} se pojavljajo še drugi dvomi. Ali smemo
tvoriti množico, ki ima več elementov, kot je črk abecede? Ali bi bilo pravilo še vedno
isto, če bi namesto ``$a, b, \ldots, z$'' zapisali ``$a, b, \ldots, j$''? Ali smemo
tvoriti množico z nič elementi? Iz nesrečnega tropičja se res ne vidi, kaj je in kaj ni
dovoljeno. Če poškilite v razdelek~\ref{sec:aksiomi-teorije-mnozic}, kjer so našteti
``uradni'' aksiomih teorije množic, tam pravila o končnih množicah ne boste našli, saj
sledi iz treh bolj osnovnih pravil.

\begin{pravilo}
  \label{pravilo:prazna-mnozica}
  Obstaja množica $\emptyset$, ki nima elementov. Imenujemo jo \df{prazna množica}.
\end{pravilo}

\begin{pravilo}
  \label{pravilo:neurejeni-dvojec}
  Za vsak $x$ in $y$ lahko tvorimo množico $\set{x, y}$, imenovano \df{(neurejeni)
    dvojec}. Elementa $\set{x, y}$ sta natanko $x$ in $y$.
\end{pravilo}

\begin{pravilo}
  \label{pravilo:unija}
  Iz poljubnih množic $A$ in $B$, lahko tvorimo množico $A \cup B$, imenovano \df{unija
    $A$ in $B$}. Elementi $A \cup B$ so natanko tisti objekti, ki so elementi $A$ ali
  elementi $B$.
\end{pravilo}

V pravilu~\ref{pravilo:neurejeni-dvojec} smo besedo ``neurejeni'' zapisali v oklepaju, kar
pomeni, da beseda pravzaprav ni pombembna in bi jo lahko tudi izpustili. Se pravi, da
``neurejeni dvojec'' in ``dvojec'' pomenita isto. V primeru nejasnosti raje uporabimo
daljšo obliko.





\subsection{Prazna množica}
\label{sec:prazna-mnozica}

Poseben primer končne množice je $\set{\,}$. To je množica, ki ne vsebuje nobenega
elementa. Imenujemo jo \df{prazna množica} in jo zapišemo tudi kot $\emptyset$.

\andrej{Ali je treba povedati še kaj o prazni množici na tem mestu?}

\section{Enojec}
\label{sec:enojec}

Naslednji primer končne množice je \df{enojec} $\set{a}$, ki vsebuje en sam element $a$.
Pogosto potrebujemo enojec in nas ne zanima, kaj točno je njegov element. V ta namen
uvedemo novo konstrukcijo.

\begin{pravilo}
  \df{Standardni enojec} je množica $\one$, katere edini element je~$\unit$.
\end{pravilo}





Potemtakem so množice $\set{a}$, $\set{a, a}$ in $\set{a, a, a}$ vse medsebojno enake in vsebujejo natanko en element (četudi pri zadnjih dveh morda na prvi pogled izgleda drugače).

\section{Druge osnovne množice}

\andrej{To ne paše sem, ker bi bilo tu dosti bolj naravno nadaljevati s kartezičnim
  produktom. To bomo prestavili na mesto, kjer bo dejansko prišlo prav.}

Množice, s katerimi v matematiki delamo, tipično vsebujejo števila, ali pa so vsaj na tak ali drugačen način izpeljane iz številskih množic. Spomnimo se standardnih oznak najpogosteje uporabljanih številskih množic.
\begin{center}
\begin{tabular}{|cc|}
\hline
\textbf{Množica} & \textbf{Oznaka} \\
\hline
množica naravnih števil & $\NN$ \\
množica celih števil & $\ZZ$ \\
množica racionalnih števil & $\QQ$ \\
množica realnih števil & $\RR$ \\
množica kompleksnih števil & $\CC$ \\
\hline
\end{tabular}
\end{center}

Nekateri $0$ vzamejo za naravno število, nekateri ne. To je v celoti stvar dogovora, kaj pomeni pojem ``naravno število''. Za nas bo prišlo bolj prav, če ničlo štejemo kot element množice naravnih števil, torej $\NN = \set{0, 1, 2, 3, \ldots}$.



\section{Preslikave}

\note{
Preslikava med množicama, podana kot predpis, ki enolično določa vrednost pri vsakem
argumentu. Princip ekstenzionalnosti za preslikave. Kaj bi pomenilo, da princip
ekstenzionalnisti ne velja, na primer, če bi gledali na preslikave kot na programe in bi
bilo pomembno, kako hitro delajo.
}

Množice ne obstajajo povsem ločene ena od druge, pač pa so med sabo povezane s \df{preslikavami} oziroma s tujko \df{funkcijami}\footnote{Nekateri uporabljajo izraz ``funkcija'' samo za tiste preslikave, ki slikajo v realna ali kompleksna števila, ampak ta uporaba je že nekoliko zastarela. Dandanes večina matematiko besedo ``funkcija'' obravnava kot sopomenko besede ``preslikava''. Tako jo bomo uporabljali tudi v tej knjigi.}. Posamična preslikava slika elemente ene množice po določenem predpisu v elemente druge množice.

Če je $f$ preslikava, ki slika iz množice $X$ v množico $Y$, to zapišemo $f\colon X \to Y$. Rečemo, da je množica $X$ \df{začetna množica} ali \df{domena} preslikave $f$, množica $Y$ pa je \df{ciljna množica} ali \df{kodomena} preslikave $f$.

Začetni množici ste v srednji šoli rekli tudi ``definicijsko območje'', ampak v tej knjigi bomo morali biti bolj previdni. Predpis za preslikavo po definiciji velja za vse elemente domene, ampak kasneje (v razdelku~\ref{razdelek:izpeljava-preslikav-iz-relacij}) obravnavamo delne preslikave, ki niso definirane na celi domeni; zanje je torej definicijsko območje manjše kot domena. Držimo se raje zgoraj danega poimenovanja.

Običaj je, da predpis preslikave podamo s pomočjo spremenljivke, tipično z oznako $x$. Na primer, če je $f$ preslikava kvadriranja, njen predpis zapišemo kot
\[f(x) = x^2.\]
Na tem mestu je potrebno poudariti več reči.
\begin{itemize}
\item
Velikokrat površno rečemo, da zgornji predpis podaja preslikavo. To ni povsem res --- to je zgolj predpis preslikave. Za to, da preslikavo v celoti podamo, je potrebno navesti tri stvari: poleg predpisa še domeno in kodomeno. Vse to je del informacije o preslikavi.

\andrej{Tabelarični prikaz funkcije?}

To se jasno pokaže, če začnemo razmišljati o lastnostih preslikav. Se še spomnite iz srednje šole, kaj pomeni, da je preslikava surjektivna? (Bomo ponovili v razdelku~\ref{razdelek:injektivnost-in-surjektivnost}.) Če vzamemo, da preslikava $f$ zadošča zgornjemu predpisu in jo obravnavamo kot preslikavo $f\colon \RR \to \RR$, ni surjektivna, če jo obravnavamo recimo kot preslikavo $f\colon \RR_{\geq 0} \to \RR_{\geq 0}$, pa je.
\item
Za spremenljivko $x$ velja isto, kot smo razpravljali že v prejšnjem razdelku pri lastnostih elementov množic: spremenljivka $x$ nima vnaprej določene vrednosti, pač pa predstavlja mesto, kamor lahko vstavimo poljubno vrednost. Seveda je potem vseeno, če vzamemo kakšno drugo črko ali čisto drug simbol: $f(y) = y^2$ določa isti predpis kot $f(x) = x^2$; prav tako $f(\heartsuit) = \heartsuit^2$. Se pravi, tudi v tem primeru gre za nemo spremenljivko. Če si torej izberemo neko vrednost, jo lahko vstavimo na mesto spremenljivke in izračunamo vrednost dobljenega izraza, npr.~$f(3) = 3^2 = 9$ oziroma $f(2\pi) = (2\pi)^2 = 4\pi^2$. Predstavljajte si, da je spremenljivka pravzaprav škatlica, kamor lahko vstavite vrednost, torej
\[f(\argbox) = \argbox^2.\]
\item
Alternativen način zapisa $f(x) = x^2$ je
\[f\colon x \mapsto x^2.\]
Pazimo: navadna puščica $\to$ podaja domeno in kodomeno, kot razloženo zgoraj. Repata puščica $\mapsto$ pa za posamičen element domene pove, v kateri element kodomene se preslika.

Zapis z repato puščico je še posebej uporaben, kadar želimo podati preslikavo, ne da bi nam bilo potrebno izbrati ime zanjo. Na primer, realno funkcijo kvadriranja lahko v celoti podamo takole:
\begin{align*}
\RR &\to \RR \\
x &\mapsto x^2
\end{align*}
(prva vrstica pove domeno in kodomeno, druga pa predpis). Tako podanim preslikavam potem rečemo \df{brezimne preslikave} (s tujko \df{anonimne funkcije}). Kasneje (v razdelku~\ref{razdelek:brezimne-preslikave}) bomo spoznali bolj strnjen zapis takih preslikav, ki je še posebej primeren za izvajanje operacij med preslikavami; takrat bomo takšno funkcijo zapisali kot $\lam{x \in \RR} x^2$.
\end{itemize}

\note{Sklop (kompozicija, kompozitum) preslikav. Identiteta kot enota za sklapljanje. Razčlenitev (dekompozicija, faktorizacija) preslikav.}

\davorin{Definirati moramo tudi oznako $\set{f(x)}{x \in X}$, kar je druge vrste oznaka kot prej definirana $\set{x \in X}{\phi(x)}$. Se gremo primerjavo s Pythonom (razlika med \texttt{\{f(x) for x in X\}} in \texttt{\{x if phi(x)\}})? Smo matematični hipsterji in uvedemo oznako $\{f(x) \,|\, x \in X \,|\, \phi(x)\}$, ki ustreza \texttt{\{f(x) for x in X if phi(x)\}}, kar bi tudi prišlo prav?}

Zaenkrat smo imeli primere, ko je bil prepis preslikave dan z eno samo spremenljivko, npr.~$f(x) = x^2$. Zelo pogoste so pa tudi \df{preslikave več spremenljivk}, npr.~$f(x, y) = x^2 + y^2$. Že osnovne računske operacije so take --- na primer, pri seštevanju vzamemo \emph{dva} podatka in vrnemo rezultat (vsoto).

V takem primeru je smiselno reči: domena preslikave sestoji iz \df{dvojic} ali \df{parov} števil. Pri seštevanju je to, katero število je prvo, katero pa drugo, sicer nepomembno, pri kakšni drugi operaciji (npr.~že odštevanju), pa je, zato posebej zahtevajmo: gre za \df{urejene dvojice} (\df{pare}). Urejeno dvojico elementov $a$ in $b$ (v tem vrstem redu) po dogovoru zapišemo kot $(a, b)$. Vrednosti $a$ in $b$ imenujemo \df{komponenti} tega para; natančneje, $a$ je \df{prva komponenta}, $b$ pa \df{druga komponenta}.

Če imamo dve množici $A$ in $B$, tedaj množico vseh urejenih dvojic, katerih prva komponenta je element iz $A$, druga komponenta pa element iz $B$, označimo $A \times B$ in imenujemo \df{zmnožek} ali \df{produkt} množic $A$ in $B$. Glede na to, da obstaja mnogo operacij, ki se imenujejo ``produkt'' (poznate že vsaj produkt števil, produkt števila z vektorjem, skalarni produkt vektorjev in vektorski produkt vektorjev, obstaja pa jih še precej več), je koristno produkt množic posebej poimenovati, da ga ločimo od drugih: zanj se je uveljavil izraz \df{kartezični produkt} (izhaja iz imena Cartesius, tj.~latinske različice priimka Renéja Descarta\footnote{René Descartes (1596 -- 1650) je bil francoski filozof, matematik in znanstvenik.}).

Seštevanje potemtakem lahko razumemo kot preslikavo $+\colon \RR \times \RR \to \RR$. V tem smislu še vedno gre za preslikavo, ki dan vhodni podatek preslika v neki rezultat, le da je vhodni podatek dvojica števil, ne pa zgolj eno število. Kadar imamo produkt več enakih faktorjev, ga lahko (kot običajno) zapišemo v obliki potence; pisali bi lahko tudi $+\colon \RR^2 \to \RR$.

Seveda nismo omejeni na preslikave samo ene ali dveh spremenljivk. Nič nam ne preprečuje definirati recimo $f(x, y, z) = 2x + y - 3z$. Smiselna domena te preslikave setoji iz \df{urejenih trojic} števil. V splošnem, če jemljemo elemente iz množic $A$, $B$, $C$, tedaj se množica vseh takih trojic označi z $A \times B \times C$. Prejšnji predpis določa potem preslikavo $f\colon \RR \times \RR \times \RR \to \RR$ (oziroma krajše $f\colon \RR^3 \to \RR$).

Spremenljivk je lahko še več; poleg dvojic in trojic tako dobimo še četverice, peterice, šesterice\ldots V splošnem takšna končna zaporedja elementov imenujemo \df{urejene večterice}. Tudi število spremenljivk je lahko označeno s črko; na primer, preslikava, ki računa povprečje $n$ števil (kjer $n \in \NN_{\geq 1}$), je dana kot
\begin{align*}
\RR^n &\to \RR \\
(x_1, x_2, \ldots, x_n) &\mapsto \frac{x_1 + x_2 + \ldots + x_n}{n}
\end{align*}
(če hočemo poudariti, da imajo naše večterice natanko $n$ komponent, jih imenujemo $n$-terice). Nadlega pri tem je sicer spet dvoumnost tropičja. Deloma jo je možno odpraviti tako, da celotno večterico označimo z eno spremenljivko. Pogosta izbira zapisa je $f(\mathbf{x})$ ali $f(\vec{x})$ (razlog za to je, da lahko večterico vidimo kot vektor).

Marsikdaj želimo delati ne samo z eno preslikavo, pač pa s celo množico preslikav naenkrat. Zato uvedemo: množica vseh preslikav, ki slikajo iz $X$ v $Y$, se označi kot $Y^X$; temu se reče \df{eksponent} množic $X$ in $Y$ (\note{na primernem mestu kasneje} bomo razložili, od kod ta oznaka).

\begin{zgled}
Množico vseh preslikav, ki realna števila slikajo nazaj v realna števila, označimo z $\RR^\RR$. Če nas zanimajo realne preslikave, ki so definirana samo na intervalu $\intoo{-1}{1}$, opazujemo množico $\RR^{\intoo{-1}{1}}$. Definiramo lahko preslikavo
\begin{align*}
\RR^{\intoo{-1}{1}} &\to \RR \\
f &\mapsto f(0),
\end{align*}
ki preslikavam priredi njihovo vrednost v točki $0$. Ta preslikava torej ima za argumente (tj.~vnose) celotne preslikave in ne števila! Sama po sebi je element množice $\RR^{\RR^{\intoo{-1}{1}}}$.
\end{zgled}

\begin{zgled}
Za poljubne množice $A$, $B$, $C$ lahko definiramo sledečo preslikavo, katere argumenti so pari preslikav.
\begin{align*}
B^A \times C^B &\to C^A \\
(f, g) &\mapsto g \circ f
\end{align*}
\end{zgled}


\davorin{Glede na to, da gre za slovenski učbenik, dajem izrazu `preslikava' prednost pred izrazom `funkcija'. Seveda pa sem pojasnil tudi slednji izraz (v prvem poglavju).}

\note{Uvod. Definicijsko območje in zaloga vrednosti \davorin{morda dodamo kot možno ime za zalogo vrednosti še prevod angleške besede `range', se pravi `razpon'?}. Zožitve (tako domene kot kodomene); oznake za to so $\rstr{f}_A$, $\rstr{f}^B$, $\rstr{f}_A^B$. Izvrednotenje (evalvacija) preslikave (če ne bomo tega pojasnili že pri eksponentih množic).}


\section{Kartezični produkt}
\label{sec:kartezicni-produkt}

Pojem splošne konstrukcije množic. Iz množic lahko na različne načine delamo nove množice.
Temu pravimo konstrukcije. Konstrukcijo podamo s pravili.

Urejeni par kot primitivni koncept. Povemo pravila, kako se dela elemente kartezičnega
produkta in kako se jih uporablja (projekciji). Povemo enačbe, ki veljajo (beta in
ekstenzionalnost).

Zapis preslikava, ki slika v produkt.
Zapis preslikava, definirane na produktu.

\section{Vsota}
\label{sec:vsota}

Vsota množic, pri čemer vpeljemo primitivne oznake za injekcije (kar je dosti bolj
smiselno, kot da vedno uporabljamo $\inl$ in $\inr$). Kadar smo leni, uporabljamo $\inl$
in $\inr$.

Elementi $A + A$. Kasneje bomo spoznali tudi operacijo unija, ki jo tudi že poznamo.

Pravila za vsoto množic. Notacija za preslikave, definirane na vsoti, $f : A + B \to C$:
%
\begin{align*}
  f(\inl(x)) &= \cdots x \cdots \\
  f(\inr(y)) &= \cdots y \cdots
\end{align*}
%
in tudi
%
\begin{align*}
  f(z) =
  \begin{cases}
    \cdots x \cdots & \text{če $z = \inl(x)$,} \\
    \cdots y \cdots & \text{če $z = \inl(y)$.}
  \end{cases}
\end{align*}

\section{Eksponent}
\label{sec:eksponent}



Zapis funkcijskega predpisa na razne načine.
\note{Tj.~anonimne oz.~čiste funkcije. Na tem mestu pride tudi $\lambda$-notacija in določena mera $\lambda$-računa.}

Aplikacija kot operacija.

Eksponent kot množica vseh preslikav. Pravila za eksponent: kako naredimo elemente
eksponenta in kako jih uporabimo. Enačbe ($\beta$ in ekstenzionalnost).

Vezane spremenljivke v zapisu funkcijskega predpisa. Substitucija in vezane spremenljivke.

Preslikave iz prazne množice in v prazno množice. Preslikave iz enojca in v enojec.


Transponiranje (currying).


\section{Izomorfizmi med množicami}
\label{sec:izom-med-mnoic}

Definicija inverza in izomorfizma.

Če je preslikava izomorfizem, je tudi njen inverz.

Izomorfnost množic. Je refleksivna, simetrična in tranzitivna (a ne poudarjamo preveč pojma ekvivalence na tem mestu).

Izomorfnost je kongruenca za produkt, vsoto in eksponent.

\section{Aritmetika Tarskega}
\label{sec:aritmetika-tarskega}

Aritmetični zakoni Tarskega za množice. Podobnost z običajno aritmetiko.

Pri asociativnosti produkta obravnavamo $A_1 \times A_2 \times \cdots \times A_n$ in
enojec kot produkt nič množic. Podobno za vsote.


\section{Kar je že Davorin napisal}

Interval realnih števil podamo s krajiščema intervala v oklepajih --- okrogli oklepaji ( ) označujejo odprtost intervala (krajišče ni vključeno v interval), oglati oklepaji [ ] pa zaprtost (krajišče je vključeno). Tako se npr.~interval realnih števil od $0$ do $1$, ki ne vsebuje krajišč, označi z $(0, 1)$, če jih vsebuje, pa z $[0, 1]$.

Včasih pridejo prav tudi intervali na drugih množicah kot $\RR$. Zato se dogovorimo, da bomo intervale označevali tako, da podamo množico, ob kateri v indeksu zapišemo krajišči v oklepajih, npr.~$\intco[\NN]{1}{5} = \set{1, 2, 3, 4}$. Realna intervala iz prejšnjega odstavka tako zapišemo kot $\intoo{0}{1}$ in $\intcc{0}{1}$.

Če interval v katero smer gre v nedogled, preprosto zapišemo množico z ustreznim simbolom za urejenost in krajiščem v indeksu. Na primer, $\RR_{> 0}$ označuje množico pozitivnih realnih števil, $\RR_{\geq 0}$ pa množico nenegativnih realnih števil.

Primerjave med elementi, kot npr.~pravkar podani $>$ in $\geq$, imenujemo \df{relacije} (podrobneje jih bomo spoznali v poglavju~\ref{poglavje:relacije}). Zgornji zapis bomo uporabljali tudi za druge vrste relacij, ne samo za relacije urejenosti. Na primer, množico vseh neničelnih realnih števil zapišemo kot $\RR_{\neq 0}$.

\davorin{To bi vsaj bil moj predlog. Na ta način se izognemo dvoumnostim (kar je namen). Na primer, kaj pomeni $\forall\, a > 0$? Če zapišemo $\forall\, a \in \NN_{> 0}$ ali $\forall\, a \in \RR_{> 0}$, je jasno. Razlog, da matematiki ``goljufajo'' in pridejo skozi brez tega, je (napol dogovorjena in ponotranjena, ampak arbitrarna) izbira črk; vsak izkušen matematik ve, da $\forall\, \epsilon > 0$ pomeni $\forall\, \epsilon \in \RR_{> 0}$. Dodaten problem je, da kasneje uporabljamo urejene pare, ki jih vsi na naši fakulteti pišejo z okroglimi oklepaji. Poskusimo se izogniti zmedi, ali $(a, b)$ pomeni urejeni par ali odprti interval. Če se ne strinjate, popravite in pustite komentar.}

Če imamo dan neki element in neko množico, potem pripadnost tega elementa tej množici izrazimo s simbolom $\in$. Na primer, da je štiri naravno število, zapišemo $4 \in \NN$ (beri: ``štiri pripada množici naravnih števil'').

Elementi množic lahko zadoščajo raznim lastnostim. Na primer, recimo, da $\phi$ označuje lastnost ``biti manj od pet''; to potem zapišemo
\[\phi(x) \ = \ \ x < 5.\]
V tem primeru $x$ imenujemo \df{spremenljivka}, saj ne gre za točno določeno vrednost, pač pa predstavlja splošno število (recimo, da se dogovorimo, da s $\phi$ označujemo lastnost na realnih številih).

Tovrstne lastnosti nam omogočajo, da iz neke množice odberemo elemente z dano lastnostjo in na ta način dobimo novo množico, ki je podmnožica prejšnje. Množico vseh realnih števil, ki so manjša od pet, zapišemo na naslednji način.
\[\set{x \in \RR}{x < 5}\]
Seveda, ker je primerjava s števili zelo pogosta lastnost, je uporabno, če uvedemo krajše oznake, ki povejo isto; že prej smo se dogovorili, da tako množico označimo z $\RR_{< 5}$. Za povsem splošne lastnosti pa ne bomo imeli vnaprej dogovorjenih oznak, zato je dobro, da poznamo splošni zapis. Torej, če je $X$ poljubna množica in $\phi$ poljubna lastnost njenih elementov, tedaj podmnožico, ki vsebuje točno tiste elemente množice $X$, ki zadoščajo lastnosti $\phi$, označimo takole.
\[\set[1]{x \in X}{\phi(x)}\]

Pri tem se zavedajmo: ni pomembno, da spremenljivko označimo ravno z $x$. Zapis
\[\set[1]{y \in X}{\phi(y)}\]
še vedno označuje isto množico. V vsakem primeru gre za množico vseh elementov iz $X$ z lastnostjo $\phi$. Pravzaprav sploh ni nujno, da uporabimo črko; poslužimo se lahko kateregakoli simbola (ki mu nismo predtem že predpisali določenega pomena). Taisto množico lahko zapišemo tudi $\set{\heartsuit \in X}{\phi(\heartsuit)}$.

Kadar imamo spremenljivko, ki jo lahko preimenujemo, ne da bi spremenili pomen izraza, jo imenujemo \df{nema spremenljivka}. Takšne primere že dobro poznate; na primer, integral $\int_0^1 x^2 \,dx$ se ne spremeni, če preimenujete spremenljivko in zapišete $\int_0^1 y^2 \,dy$.

\begin{zgled}
Kako bi zapisali množico vseh sodih naravnih števil? Spomnimo se, da je število sodo, kadar je deljivo z $2$. Za $n \in \NN$ to zapišemo takole: $2 \divides n$ (beri: ``dve deli $n$''). Množica sodih naravnih števil se potem zapiše kot
\[\set[1]{n \in \NN}{2 \divides n}.\]
\end{zgled}


\section{Vaje}


%%% Local Variables:
%%% mode: latex
%%% TeX-master: "ucbenik-lmn"
%%% End:
