\chapter{Preslikave}


\section{Prestavljeno iz matematičnega izražanja sem, začasno}
\label{sec:prest-iz-matem}



\section{Preslikave}

Množice ne obstajajo povsem ločene ena od druge, pač pa so med sabo povezane s \df{preslikavami} oziroma s tujko \df{funkcijami}\footnote{Nekateri uporabljajo izraz \qt{funkcija} samo za tiste preslikave, ki slikajo v realna ali kompleksna števila, ampak ta uporaba je že nekoliko zastarela. Dandanes večina matematiko besedo \qt{funkcija} obravnava kot sopomenko besede \qt{preslikava}. Tako jo bomo uporabljali tudi v tej knjigi.}. Posamična preslikava slika elemente ene množice po določenem predpisu v elemente druge množice.

Če je $f$ preslikava, ki slika iz množice $X$ v množico $Y$, to zapišemo $f\colon X \to Y$. Rečemo, da je množica $X$ \df{začetna množica} ali \df{domena} preslikave $f$, množica $Y$ pa je \df{ciljna množica} ali \df{kodomena} preslikave $f$.

Začetni množici ste v srednji šoli rekli tudi \qt{definicijsko območje}, ampak v tej knjigi bomo morali biti bolj previdni. Predpis za preslikavo po definiciji velja za vse elemente domene, ampak kasneje (v razdelku~\ref{razdelek:izpeljava-preslikav-iz-relacij}) obravnavamo delne preslikave, ki niso definirane na celi domeni; zanje je torej definicijsko območje manjše kot domena. Držimo se raje zgoraj danega poimenovanja.

Običaj je, da predpis preslikave podamo s pomočjo spremenljivke, tipično z oznako $x$. Na primer, če je $f$ preslikava kvadriranja, njen predpis zapišemo kot
\[f(x) = x^2.\]
Na tem mestu je potrebno poudariti več reči.
\begin{itemize}
\item
Velikokrat površno rečemo, da zgornji predpis podaja preslikavo. To ni povsem res --- to je zgolj predpis preslikave. Za to, da preslikavo v celoti podamo, je potrebno navesti tri stvari: poleg predpisa še domeno in kodomeno. Vse to je del informacije o preslikavi.

To se jasno pokaže, če začnemo razmišljati o lastnostih preslikav. Se še spomnite iz srednje šole, kaj pomeni, da je preslikava surjektivna? (Bomo ponovili v razdelku~\ref{razdelek:injektivnost-in-surjektivnost}.) Če vzamemo, da preslikava $f$ zadošča zgornjemu predpisu in jo obravnavamo kot preslikavo $f\colon \RR \to \RR$, ni surjektivna, če jo obravnavamo recimo kot preslikavo $f\colon \RR_{\geq 0} \to \RR_{\geq 0}$, pa je.
\item
Za spremenljivko $x$ velja isto, kot smo razpravljali že v prejšnjem razdelku pri lastnostih elementov množic: spremenljivka $x$ nima vnaprej določene vrednosti, pač pa predstavlja mesto, kamor lahko vstavimo poljubno vrednost. Seveda je potem vseeno, če vzamemo kakšno drugo črko ali čisto drug simbol: $f(y) = y^2$ določa isti predpis kot $f(x) = x^2$; prav tako $f(\heartsuit) = \heartsuit^2$. Se pravi, tudi v tem primeru gre za \note{dummy variable}. Če si torej izberemo neko vrednost, jo lahko vstavimo na mesto spremenljivke in izračunamo vrednost dobljenega izraza, npr.~$f(3) = 3^2 = 9$ oziroma $f(2\pi) = (2\pi)^2 = 4\pi^2$. Predstavljajte si, da je spremenljivka pravzaprav škatlica, kamor lahko vstavite vrednost, torej
\[f(\argbox) = \argbox^2.\]
\item
Alternativen način zapisa $f(x) = x^2$ je
\[f\colon x \mapsto x^2.\]
Pazimo: navadna puščica $\to$ podaja domeno in kodomeno, kot razloženo zgoraj. Repata puščica $\mapsto$ pa za posamičen element domene pove, v kateri element kodomene se preslika.

Zapis z repato puščico je še posebej uporaben, kadar želimo podati preslikavo, ne da bi nam bilo potrebno izbrati ime zanjo. Na primer, realno funkcijo kvadriranja lahko v celoti podamo takole:
\begin{align*}
\RR &\to \RR \\
x &\mapsto x^2
\end{align*}
(prva vrstica pove domeno in kodomeno, druga pa predpis). Tako podanim preslikavam potem rečemo \df{brezimne preslikave} (s tujko \df{anonimne funkcije}). Kasneje (v razdelku~\ref{razdelek:brezimne-preslikave}) bomo spoznali bolj strnjen zapis takih preslikav, ki je še posebej primeren za izvajanje operacij med preslikavami; takrat bomo takšno funkcijo zapisali kot $\xlam{x}[\RR]{x^2}[\RR]$.
\end{itemize}

\note{Sklop (kompozicija, kompozitum) preslikav. Identiteta kot enota za sklapljanje. Razčlenitev (dekompozicija, faktorizacija) preslikav.}

\davorin{Definirati moramo tudi oznako $\set{f(x)}{x \in X}$, kar je druge vrste oznaka kot prej definirana $\set{x \in X}{\phi(x)}$. Se gremo primerjavo s Pythonom (razlika med \texttt{\{f(x) for x in X\}} in \texttt{\{x if phi(x)\}})? Smo matematični hipsterji in uvedemo oznako $\{f(x) \,|\, x \in X \,|\, \phi(x)\}$, ki ustreza \texttt{\{f(x) for x in X if phi(x)\}}, kar bi tudi prišlo prav?}

Zaenkrat smo imeli primere, ko je bil prepis preslikave dan z eno samo spremenljivko, npr.~$f(x) = x^2$. Zelo pogoste so pa tudi \df{preslikave več spremenljivk}, npr.~$f(x, y) = x^2 + y^2$. Že osnovne računske operacije so take --- na primer, pri seštevanju vzamemo \emph{dva} podatka in vrnemo rezultat (vsoto).

V takem primeru je smiselno reči: domena preslikave sestoji iz \df{dvojic} ali \df{parov} števil. Pri seštevanju je to, katero število je prvo, katero pa drugo, sicer nepomembno, pri kakšni drugi operaciji (npr.~že odštevanju), pa je, zato posebej zahtevajmo: gre za \df{urejene dvojice} (\df{pare}). Urejeno dvojico elementov $a$ in $b$ (v tem vrstem redu) po dogovoru zapišemo kot $(a, b)$. Vrednosti $a$ in $b$ imenujemo \df{komponenti} tega para; natančneje, $a$ je \df{prva komponenta}, $b$ pa \df{druga komponenta}.

Če imamo dve množici $A$ in $B$, tedaj množico vseh urejenih dvojic, katerih prva komponenta je element iz $A$, druga komponenta pa element iz $B$, označimo $A \times B$ in imenujemo \df{zmnožek} ali \df{produkt} množic $A$ in $B$. Glede na to, da obstaja mnogo operacij, ki se imenujejo \qt{produkt} (poznate že vsaj produkt števil, produkt števila z vektorjem, skalarni produkt vektorjev in vektorski produkt vektorjev, obstaja pa jih še precej več), je koristno produkt množic posebej poimenovati, da ga ločimo od drugih: zanj se je uveljavil izraz \df{kartezični produkt} (izhaja iz imena Cartesius, tj.~latinske različice priimka Renéja Descarta\footnote{René Descartes (1596 -- 1650) je bil francoski filozof, matematik in znanstvenik.}).

Seštevanje potemtakem lahko razumemo kot preslikavo $+\colon \RR \times \RR \to \RR$. V tem smislu še vedno gre za preslikavo, ki dan vhodni podatek preslika v neki rezultat, le da je vhodni podatek dvojica števil, ne pa zgolj eno število. Kadar imamo produkt več enakih faktorjev, ga lahko (kot običajno) zapišemo v obliki potence; pisali bi lahko tudi $+\colon \RR^2 \to \RR$.

Seveda nismo omejeni na preslikave samo ene ali dveh spremenljivk. Nič nam ne preprečuje definirati recimo $f(x, y, z) = 2x + y - 3z$. Smiselna domena te preslikave setoji iz \df{urejenih trojic} števil. V splošnem, če jemljemo elemente iz množic $A$, $B$, $C$, tedaj se množica vseh takih trojic označi z $A \times B \times C$. Prejšnji predpis določa potem preslikavo $f\colon \RR \times \RR \times \RR \to \RR$ (oziroma krajše $f\colon \RR^3 \to \RR$).

Spremenljivk je lahko še več; poleg dvojic in trojic tako dobimo še četverice, peterice, šesterice\ldots V splošnem takšna končna zaporedja elementov imenujemo \df{urejene večterice}. Tudi število spremenljivk je lahko označeno s črko; na primer, preslikava, ki računa povprečje $n$ števil (kjer $n \in \NN_{\geq 1}$), je dana kot
\begin{align*}
\RR^n &\to \RR \\
(x_1, x_2, \ldots, x_n) &\mapsto \frac{x_1 + x_2 + \ldots + x_n}{n}
\end{align*}
(če hočemo poudariti, da imajo naše večterice natanko $n$ komponent, jih imenujemo $n$-terice). Nadlega pri tem je sicer spet dvoumnost tropičja. Deloma jo je možno odpraviti tako, da celotno večterico označimo z eno spremenljivko. Pogosta izbira zapisa je $f(\mathbf{x})$ ali $f(\vec{x})$ (razlog za to je, da lahko večterico vidimo kot vektor).

Marsikdaj želimo delati ne samo z eno preslikavo, pač pa s celo množico preslikav naenkrat. Zato uvedemo: množica vseh preslikav, ki slikajo iz $X$ v $Y$, se označi kot $Y^X$; temu se reče \df{eksponent} množic $X$ in $Y$ (\note{na primernem mestu kasneje} bomo razložili, od kod ta oznaka).

\begin{zgled}
Množico vseh preslikav, ki realna števila slikajo nazaj v realna števila, označimo z $\RR^\RR$. Če nas zanimajo realne preslikave, ki so definirana samo na intervalu $\intoo{-1}{1}$, opazujemo množico $\RR^{\intoo{-1}{1}}$. Definiramo lahko preslikavo
\begin{align*}
\RR^{\intoo{-1}{1}} &\to \RR \\
f &\mapsto f(0),
\end{align*}
ki preslikavam priredi njihovo vrednost v točki $0$. Ta preslikava torej ima za argumente (tj.~vnose) celotne preslikave in ne števila! Sama po sebi je element množice $\RR^{\RR^{\intoo{-1}{1}}}$.
\end{zgled}

\begin{zgled}
Za poljubne množice $A$, $B$, $C$ lahko definiramo sledečo preslikavo, katere argumenti so pari preslikav.
\begin{align*}
B^A \times C^B &\to C^A \\
(f, g) &\mapsto g \circ f
\end{align*}
\end{zgled}


\davorin{Glede na to, da gre za slovenski učbenik, dajem izrazu `preslikava' prednost pred izrazom `funkcija'. Seveda pa sem pojasnil tudi slednji izraz (v prvem poglavju).}

\note{Uvod. Definicijsko območje in zaloga vrednosti \davorin{morda dodamo kot možno ime za zalogo vrednosti še prevod angleške besede `range', se pravi `razpon'?}. Zožitve (tako domene kot kodomene); oznake za to so $\rstr{f}_A$, $\rstr{f}^B$, $\rstr{f}_A^B$. Izvrednotenje (evalvacija) preslikave (če ne bomo tega pojasnili že pri eksponentih množic).}


\section{Brezimne preslikave}\label{razdelek:brezimne-preslikave}

\note{Tj.~anonimne oz.~čiste funkcije. Na tem mestu pride tudi $\lambda$-notacija in določena mera $\lambda$-računa.}


\section{Slike in praslike}

Preslikava kot taka nam pove za posamične elemente, kam se slikajo. Marsikdaj pa nas zanima več: kam se slikajo celotne množice elementov. Na primer, zanima nas lahko, v kaj se projicira neko prostorsko telo na ravnino.

\note{luštna slika projekcije nekega prostorskega objekta na neko ravnino}

Da dobimo sliko celotne množice, moramo zbrati skupaj slike vseh posamičnih elementov množice. Smiselna je torej naslednja definicija.

\begin{definicija}\label{definicija:slika}
Naj bo $f\colon X \to Y$ preslikava. \df{Slika} množice $A \subseteq X$ je označena in definirana kot
\[\img{f}{A} \dfeq \set[1]{f(x)}{x \in A} = \set[1]{y \in Y}{\xsome{x}[A]{y = f(x)}}.\]
Ta predpis definira preslikavo $\img{f}\colon \pst(X) \to \pst(Y)$.
\end{definicija}

\begin{opomba}
Kot običajno, obstajajo različne oznake v uporabi. Sliko $\img{f}{A}$ se označuje tudi kot $f[A]$ ali celo kar kot $f(A)$. V slednjem primeru se predpostavlja zadostna matematična zrelost bralca, da zna razbrati, kdaj $f$ označuje preslikavo $f\colon X \to Y$, kdaj pa preslikavo $f\colon \pst(X) \to \pst(Y)$.

V tej knjigi se bomo načrtno izogibali takšnim dvoumnostim in za sliko dosledno uporabljali oznako iz definicije~\ref{definicija:slika}.
\end{opomba}

\begin{vaja}
Prepričaj se, da za poljubno preslikavo $f\colon X \to Y$ velja sledeče:
\begin{itemize}
\item
$\img{f}{X} = \rn{f}$,
\item
$\img{f}{\emptyset} = \emptyset$,
\item
$\img[1]{f}{\set{x}} = \set[1]{f(x)}$ za vsak $x \in X$.
\end{itemize}
\end{vaja}

\note{primeri in lastnosti slik že tu ali kasneje skupaj s primeri/lastnostmi praslik?}

Včasih pa imamo obratno nalogo: iz dane slike ugotoviti, kaj vse se je z neko preslikavo vanjo preslikalo. Zato vpeljemo še sledečo definicijo.

\begin{definicija}\label{definicija:praslika}
Naj bo $f\colon X \to Y$ preslikava. \df{Praslika} množice $B \subseteq Y$ je označena in definirana kot
\[\pim{f}{B} \dfeq \set[1]{x \in X}{f(x) \in B}.\]
Ta predpis definira preslikavo $\pim{f}\colon \pst(Y) \to \pst(X)$.
\end{definicija}

\begin{opomba}
Tudi za prasliko obstajajo različne oznake. Praslika $\pim{f}{B}$ se označi tudi kot $f^{-1}[B]$ ali kar kot $f^{-1}(B)$. V slednjem primeru se spet zanašamo na izkušenost bralca, da praslike $f^{-1}\colon \pst(Y) \to \pst(X)$ ne zamenja z obratom $f^{-1}\colon Y \to X$. Slednji morda sploh ne obstaja! (Praslika seveda obstaja za vse funkcije.)

Če obrat funkcije obstaja, tedaj velja $\pim[1]{f}{\set{y}} = \set[1]{f^{-1}(y)}$ za vsak $y \in Y$ (premisli!), kar nekoliko pojasni oznako $f^{-1}$ tudi za prasliko. Kljub vsemu, z namenom izogibanja dvoumnostim se bomo v tej knjigi skrbno držali oznake iz definicije~\ref{definicija:praslika} za prasliko.

Ko smo že pri alternativnih, potencialno zavajajočih oznakah: pri prasliki enojca se tipično izpuščajo zaviti oklepaji, torej se namesto $\pim[1]{f}{\set{y}}$ piše $\pim{f}{y}$ (ali celo $f^{-1}(y)$).
\end{opomba}

\note{primeri, vaje}

\note{lastnosti: ohranjanje unij, presekov, komplementov}


\section{Injektivnost in surjektivnost}\label{razdelek:injektivnost-in-surjektivnost}

\note{Vključno z ekvivalenco z mono- in epimorfizmi.}


\section{Bijektivnost in obratne preslikave}\label{razdelek:bijektivnost-in-obratne-preslikave}

Kot dobro veste že iz srednje šole, nam injektivnost in surjektivnost omogočata definicijo bijektivnosti.

\begin{definicija}
Preslikava je \df{bijektivna}, kadar je injektivna in surjektivna.
\end{definicija}

To pomeni: če imamo bijektivno preslikavo (na kratko kar: \df{bijekcijo}) $f\colon X \to Y$, smo povezali elemente množice $X$ z elementi množice $Y$, in sicer tako, da vsakemu elementu v katerikoli od množic $X$ oz.~$Y$ pripišemo natanko en element druge množice.

\note{slika dveh množic s poparjenimi pikami}

Rečemo, da so elementi množice $X$ v \df{bijektivni korespondenci} (ali po slovensko \df{povratno enolični zvezi}) z elementi množice $Y$. Bijektivnost se na grafih kaže takole: preslikava je bijektivna, kadar vsaka vodoravnica seka njen graf natanko enkrat.

Bijektivne preslikave igrajo pomembno vlogo v matematiki. Oglejmo si tri primere.
\begin{itemize}
\item
Če imamo povratno enolično zvezo med elementi dveh množic, je jasno, da imata isto število elementov. To nam omogoča definicijo \df{kardinalnosti} množic --- glej poglavje~\note{o kardinalnosti}.
\item
Predstavljajmo si, da so elementi neke množice $X$ imena za določene objekte. Na bijektivno preslikavo $f\colon X \to Y$ lahko potem gledamo kot na preimenovanje teh objektov. Seveda preimenovanje ne spremeni narave (ali če hočete natančnejši izraz, matematične strukture) objektov --- z drugimi besedami, $X$ in $Y$ se razlikujeta zgolj po imenih svojih elementov. To nas privede do pojma \df{izomorfizma}. Za več podrobnosti glej poglavje~\note{o strukturiranih množicah}.
\item
Če imamo povratno enolično zvezo med elementi množic $X$ in $Y$, potem ta zveza ne podaja zgolj preslikave v smeri $X \to Y$, pač pa tudi v smeri $Y \to X$, ker za vsak element iz $Y$ obstaja enolično določen element iz $X$, ki se vanj preslika. Z drugimi besedami, bijektivne preslikave imajo \df{obrate}.
\end{itemize}

Povejmo več o obratih preslikav. Začnimo s formalno definicijo.

\begin{definicija}
Naj bo $f\colon X \to Y$ poljubna preslikava. Za preslikavo $g\colon Y \to X$ rečemo, da je \df{obrat} ali \df{inverz} preslikave $f$, kadar velja
\[g \circ f = \id[X] \qquad\qquad \text{in} \qquad\qquad f \circ g = \id[Y].\]
Z drugimi besedami, $g$ je obrat $f$, kadar slika v nasprotni smeri in za vsak $x \in X$ velja $g\big(f(x)\big) = x$ ter za vsak $y \in Y$ velja $f\big(g(y)\big) = y$. Kadar obrat preslikave $f$ obstaja, rečemo, da je $f$ \df{obrnljiva} (ali \df{invertibilna}) preslikava.
\end{definicija}

\begin{zgled}\label{zgled:logaritmiranje-je-obratno-od-eksponenciranja}
Kot veš že iz srednje šole, logaritmiranje je obratno od eksponenciranja. Če smo natančnejši: preslikavi $\xlam{x}[\RR]{b^x}[\RR_{> 0}]$ in $\xlam{x}[\RR_{>0}]{\log_b x}[\RR]$ sta si obratni pri vsaki osnovi $b \in \RR_{> 0} \setminus \set{1}$.
\end{zgled}

\begin{vaja}\label{vaja:enolicnost-obrata-preslikave}
Dokaži: če sta $g$ in $h$ obrata iste preslikave $f$, tedaj $g = h$.
\end{vaja}

Vaja~\ref{vaja:enolicnost-obrata-preslikave} pove, da je obrat funkcije enolično določen, tj.~vsaka funkcija ima kvečjemu en obrat. Zato lahko uvedemo izrecno oznako: obrat preslikave $f$ (kadar obstaja) označimo z $f^{-1}$. Velja torej: kadar je preslikava $f\colon X \to Y$ obrnljiva, določa preslikavo $f^{-1}\colon Y \to X$.

Ta oznaka je nekoliko nerodna --- pomembno se je zavedati, da $f^{-1}(x)$ pomeni obrat preslikave $f$, izvrednoten na $x$, medtem kot $\big(f(x)\big)^{-1}$ pomeni obratna vrednost (v smislu deljenja) izvrednotenja preslikave $f$ na $x$. Za primerjavo, kot omenjeno v zgledu~\ref{zgled:logaritmiranje-je-obratno-od-eksponenciranja}, je obrat eksponenciranja logaritmiranje, medtem ko je obratna vrednost od $b^x$ enaka $(b^x)^{-1} = \frac{1}{b^x} = b^{-x}$.

\begin{vaja}
Premisli: če ima preslikava $f$ obrat $f^{-1}$, tedaj je tudi $f^{-1}$ obrnljiva preslikava in velja $(f^{-1})^{-1} = f$ (torej, obrat obrata je izvorna preslikava).
\end{vaja}

\begin{vaja}
Pogosto rečemo, da sta seštevanje in odštevanje obratni operaciji. Strogo vzeto, ti dve operaciji nista obratni kot preslikavi, saj obe slikata (recimo, da ju gledamo na realnih številih) $\RR \times \RR \to \RR$, tj.~ne slikata v nasprotnih smereh. Ugotovi, v kakšnem smislu točno sta seštevanje in odštevanje obratni, tj.~kateri dve preslikavi sta pravzaprav druga drugi obratni.
\end{vaja}

Zakaj se sploh ukvarjamo z obrati? Pogosto obravnavamo preslikavo, ki izhaja iz nekega konkretnega (na primer fizikalnega) problema, v smislu, da preslikava vzame začetne podatke in nam vrne, kaj se bo na koncu zgodilo. Marsikdaj pa hočemo rešiti obraten problem: želimo določene končne rezultate in se sprašujemo, kakšni morajo biti začetni pogoji, da jih bomo dosegli. V takem primeru pride prav obratna preslikava.

Kot omenjeno, je obrat preslikave enoličen. Ne velja pa, da za poljubne preslikave sploh obstaja. Na primer, naj bo $f$ edina možna preslikava $\set{0, 1} \to \set{\unit}$, torej tista, ki tako $0$ kot $1$ preslika v $\unit$. Nobena preslikava $g\colon \set{\unit} \to \set{0, 1}$ ne more biti obrat preslikave $f$, saj je $g \circ f$ gotovo konstantna in potemtakem ne more biti identiteta na $\set{0, 1}$.

Kdaj torej obstaja obrat preslikave?

\begin{trditev}
Za poljubno preslikavo $f\colon X \to Y$ sta ekvivalentni sledeči trditvi.
\begin{enumerate}
\item
Preslikava $f$ je obrnljiva.
\item
Preslikava $f$ je bijektivna.
\end{enumerate}
\end{trditev}

\begin{dokaz}
\begin{implproof}{1}{2}
Predpostavljamo, da obstaja obrat $f^{-1}$.

Dokažimo, da je $f$ injektivna. Vzemimo poljubna $x, y \in X$, za katera velja $f(x) = f(y)$. Tedaj $x = f^{-1}\big(f(x)\big) = f^{-1}\big(f(y)\big) = y$.

Dokažimo, da je $f$ surjektivna. Vzemimo poljuben $y \in Y$. Tedaj $y = f\big(f^{-1}(y)\big)$.
\end{implproof}
\begin{implproof}{2}{1}
Če je $f$ bijekcija, za vsak $y \in Y$ velja, da je $\pim[1]{f}{\set{y}}$ enojec (glej \note{ustrezne predhodne trditve v razdelku o injektivnosti in surjektivnosti}). Definirajmo $g\colon Y \to X$ na naslednji način: za vsak $y \in Y$ naj bo $g(y)$ tisti element $x \in X$, za katerega velja $\pim[1]{f}{\set{y}} = \set{x}$. \note{Iz lastnosti praslike sledi, da je $g$ obrat $f$.}
\end{implproof}
\end{dokaz}

Iz dokaza te trditve vidimo, da bi bilo koristno imeti oznako za \qt{tisti element}, če želimo podajati tovrstne preslikave s simboli. Naj bo $\phi$ lastnost elementov množice $X$ (torej predikat $\phi\colon X \to \tvs$), ki je resnična za natanko en element. Dogovorimo se, da
\[\xthat{x}[X]{\phi(x)}\]
pomeni \qt{tisti (edini) element množice $X$, ki ima lastnost $\phi$} (simbolček na začetku je mala grška črka jota). Zdaj lahko izrecno zapišemo: če je $f\colon X \to Y$ bijekcija, tedaj je njen obrat $f^{-1}\colon Y \to X$ dan s predpisom
\[f^{-1}(y) = \that[1]{x}[X]{f(x) = y}.\]

\davorin{Andrej, omenjal si, da želiš imeti to oznako. Če sem kaj zgrešil, prosim popravi.}

Zaenkrat smo to joto uporabljali zgolj kot okrajšavo za stavek v običajnem jeziku, ampak če želimo $\iota$-izraze uporabljati v matematičnih dokazih, jim moramo dati natančen matematični pomen. Definirajmo torej joto formalno matematično.

Naj bo $X$ poljubna množica. Na njej imamo enakost; obravnavajmo jo na tem mestu kot lastnost dvojic elementov iz $X$, torej kot predikat $=_X\colon X \times X \to \tvs$ (za vsak par elementov vrnemo resničnostno vrednost izjave, da sta komponenti para enaki). Transponirajmo to preslikavo; dobimo $\transposed{=_X}\colon X \to \tvs^X$. Ta transponiranka je injektivna: če se za $a, b \in X$ preslikavi $\xlam{x}[X]{a = x}$ in $\xlam{x}[X]{b = x}$ ujemata, se ujemata tudi njuni vrednosti pri $b$. Ker drži $b = b$, potem drži tudi $a = b$.

Če zožimo kodomeno preslikave $\transposed{=_X}$ na njeno sliko, potemtakem dobimo bijekcijo. Naj bo jota njen obrat, torej $\iota \dfeq \big(\rstr{\transposed{=_X}}^{\rn{\transposed{=_X}}}\big)^{-1}$. V tem smislu je zgornja oznaka $\xthat{x}[X]{\phi(x)}$ okrajšava za $\iota\big(\xlam{x}[X]{\phi(x)}\big)$ (kar bi seveda lahko še skrajšali do $\iota(\phi)$, ampak v praksi je to običajno manj zgovorno).


%%% Local Variables:
%%% mode: latex
%%% TeX-master: "ucbenik-lmn"
%%% End:
