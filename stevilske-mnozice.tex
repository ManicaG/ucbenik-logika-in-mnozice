\chapter{Številske množice}

Številske množice (naravna števila, cela števila, \ldots) poznate že od nekdaj. O njih imate zadosti občutka oz.~intuitivne predstave, da jih lahko uporabljate in pridete do pravilnih rezultatov. Tudi v tej knjigi smo jih že kar naprej izkoriščali za razne primere.

Ampak intuitivna predstava je tudi vse, kar zaenkrat imamo o številskih množicah. Nismo še podali natančne matematične definicije zanje, na osnovi katere bi lahko neizpodbitno dokazovali izreke o njih.

Za vajo lahko sami premislite, ali bi znali na tem mestu podati natančno definicijo, kaj pomeni biti naravno, celo, racionalno oz.~realno število. Definicija seveda mora biti natančna --- npr.~reči, da so realna števila tista, ki ležijo na številski premici, ni zadovoljiva definicija (vsaj ne, če ne pojasnite nedvoumno, kaj pomeni \qt{številska premica} in kaj pomeni \qt{ležati} na njej).

V tem poglavju se bomo sistematično lotili obravnave najpogosteje uporabljanih številskih množic. Podali bomo njihove konstrukcije, karakterizacije in temeljne lastnosti.


\section{Naravna števila}

Če vas kdo vpraša, kako dobiti vsa naravna števila, verjetno odgovorite nekaj v naslednjem smislu: naravna števila so $0$ in vsa tista števila, ki jih dobite s prištevanjem enice, tj.~jemanjem naslednika. Torej, začnemo z $0$, vzamemo naslednika in dobimo $1$, nato še enkrat vzamemo naslednika in dobimo $2$ itd.

Prvi, ki je znal to intuitivno predstavo preliti v natančno matematično definicijo, je bil Peano\footnote{Giuseppe Peano (1858 -- 1932) je bil italijanski matematik.} komaj dobro stoletje nazaj. Pogoje, ki jih zahtevamo za neko množico, da jo lahko imenujemo \qt{množica naravnih števil}, po njem imenujemo \df{Peanovi aksiomi}. \davorin{Nekje bomo predebatirali, kaj je aksiom in zakaj jih uporabljamo. Peanove aksiome povežimo s tem.}

Če boste brskali po literaturi, boste naleteli na mnogo različnih inačic Peanovih aksiomov. Mi bomo izbrali sledečo jedrnato različico.

\begin{definicija}[Peano]\label{definicija:naravna-stevila}
\df{Množica naravnih števil} je množica (običajno označena z $\NN$), skupaj z izbranim njenim elementom (običajno označenim z $0$, kar beremo \qt{ničla} ali \qt{nič}) in preslikavo na tej množici (običajno označeno z $s\colon \NN \to \NN$, ki jo imenujemo \qt{naslednik}), kadar veljajo naslednje lastnosti:
\begin{itemize}
\item
$s$ je injektivna preslikava,
\item
$0 \notin \rn{s}$,
\item
velja načelo \df{matematične indukcije}: če je $\phi$ predikat na $\NN$, za katerega velja
\[\phi(0) \qquad\qquad \text{in} \qquad\qquad \all[2]{n}[\NN]{\phi(n) \implies \phi\big(s(n)\big)},\]
tedaj $\phi$ velja za vse elemente $\NN$.
\end{itemize}
\end{definicija}

Poskusimo si zdaj natančno pojasniti pomen teh pogojev.

S pomočjo elementa $0$ in preslikave $s$ lahko v nedogled generiramo elemente množice $\NN$. Začnemo z $0$, nato vzamemo naslednika in dobimo $s(0)$, nato vzamemo naslednika tega elementa in dobimo $s(s(0))$, nato naslednika $s(s(s(0)))$ itd. Takšen zapis je sicer precej nepraktičen --- si predstavljate, da rečete \qt{dobimo se čez naslednika od naslednika od naslednika od naslednika od naslednika ničle ur} (namesto \qt{dobimo se čez pet ur})? Zato sprejmemo dogovor: $s(0)$ označimo krajše z $1$ in preberemo \qt{ena}, $s(s(0))$ označimo z $2$ in preberemo \qt{dve} in tako naprej.\footnote{Trenutno dogovorjena sistematična imena za števila gredo do \df{centiljona}, ki ga zapišemo z enico, ki ji sledi 600 ničel (vsaj pri nas; ponekod po svetu centiljon pomeni enica s 303 ničlami). To pomeni, da lahko sistematično izrazimo števila do $10^{606}-1$ (= devetsto devetindevetdeset centiljard devetsto devetindevetdeset centiljonov devetsto devetindevetdeset novemnonagintiljard\ldots). Nekateri razširijo to lestvico še z nadaljnjimi latinskimi izpeljankami, obstajajo pa tudi posebna imena za nekatera posamična velika števila, na primer \df{gugol} za $10^{100}$ (od tod izhaja ime spletnega brskalnika Google).}

Smo na ta način dobili neskončno različnih elementov $\NN$? Če ne bi zahtevali zgornjih pogojev, to ne bi bilo nujno. Lahko bi se namreč zaciklali (v smislu, da je naslednik nekega elementa element, ki smo ga že prej navedli).

Včasih je takšno zaciklanje nekaj, kar dejansko hočemo. Na primer, pri algebri boste spoznali tako imenovane \df{ciklične grupe}. Ciklično grupo z $n$ elementi označimo $\ZZ_n$, njene elemente pa kar z $0, 1, \ldots, n-1$. Spodaj je slika ciklične grupe $\ZZ_5$.

\note{slika usmerjenega grafa, ki predstavlja $\ZZ_5$}

Puščice označujejo, kako slika naslednik v tej grupi: naslednik $0$ je $1$, naslednik $1$ je $2$, naslednik $2$ je $3$, naslednik $3$ je $4$, nato pa se zacikla in naslednik $4$ je $0$.

Pogoj $0 \notin \rn{s}$ reče, da nič ni naslednik nobenega naravnega števila. Na ta način se izognemo, da bi naravna števila tvorila ciklično grupo.

Obstaja pa še en način, kako se lahko jemanje naslednika zacikla. Vzemimo spodnji primer.

\note{slike polgrupe $\set{0, \ldots, 4}$, ki se zacikla $4 \to 2$}

Nasledniki se lahko zaciklajo tudi pri elementu, ki ni $0$. V danem primeru je naslednik $0$ element $1$, naslednik $1$ je $2$, naslednik $2$ je $3$, naslednik $3$ je $4$, naslednik $4$ pa je $2$.

Zakaj naravna števila niso taka? Ker v danem primeru $s$ ni injektivna preslikava. Pogoj o injektivnosti nam v bistvu pove sledeče: naravna števila se ne morejo zaciklati pri nobenem nasledniku.

Vidimo, da se naravna števila ne morejo zaciklati niti na začetku (pri $0$) niti nekje vmes v verigi naslednikov --- torej gredo v nedogled, kot želimo. Z drugimi besedami, $0$, $s(0)$, $s(s(0))$, $s(s(s(0)))$,\ldots so medsebojno različni elementi množice $\NN$ in naravnih števil je posledično neskončno.

Čemu pa služi zadnji pogoj iz definicije~\ref{definicija:naravna-stevila}, tj.~načelo o indukciji? Že brez tega pogoja vemo, da so $0$, $s(0)$, $s(s(0))$, $s(s(s(0)))$,\ldots naravna števila, česar pa ne vemo, je, da so to \emph{vsa} naravna števila --- da torej ni nobenih drugih.

\begin{vaja}
Premisli, da množica $\RR_{> -1}$ z naslednikom $s(x) \dfeq x+1$ zadošča vsem pogojem iz definicije~\ref{definicija:naravna-stevila}, razen načelu indukcije.
\end{vaja}

Vidimo, da bi brez načela indukcije lahko imeli v množici $\NN$ odvečna števila (takšna, ki jih ne štejemo kot naravna). S predpostavko o indukciji se to ne more zgoditi. Ta namreč pravi: če neka lastnost velja za začetni element verige $0$, $s(0)$, $s(s(0))$, $s(s(s(0)))$,\ldots in če lahko sklepamo, da kakor hitro ta lastnost velja za določen element verige, velja tudi za naslednjega, potem ta lastnost velja za vsa naravna števila. Če za lastnost vzamemo \qt{biti element te verige}, iz načela o indukciji sklenemo, da se vsako naravno število nahaja nekje v tej verigi. Peanovi aksiomi torej podajajo strukturo, ki ustreza naši intuitivni predstavi množice naravnih števil.

\subsection{Indukcija}

Glede na to, da je načelo o matematični indukciji eden od osnovnih aksiomov, s katerimi so naravna števila podana, ne preseneča, da je indukcija eden najpogostejših načinov, kako dokazujemo izjave na naravnih številih. V tem podrazdelku si bomo podrobneje ogledali, kako deluje indukcija v praksi.

Z matematično indukcijo dokazujemo univerzalno kvantificirane izjave na naravnih številih, torej izjave oblike
\[\xall{n}[\NN]{\phi(n)}.\]
Po načelu indukcije za dokaz take izjave zadostuje narediti naslednje. Najprej dokažemo
\[\phi(0)\]
(da torej lastnost $\phi$ velja za začetno naravno število). To imenujemo \df{temelj} ali \df{osnova} ali \df{baza} indukcije. Nato dokažemo izjavo
\[\all[2]{n}[\NN]{\phi(n) \implies \phi\big(s(n)\big)};\]
to imenujemo \df{indukcijski korak}. Z besedami, dokažemo, da kakor hitro velja lastnost $\phi$ za neko naravno število, mora veljati ta lastnost tudi za naslednje.

Intuitivno je jasno, da to mora delovati. Temelj indukcije nam pove, da dana lastno velja za $0$. Ker zdaj vemo, da velja za $0$, mora po indukcijskem koraku veljati za naslednika ničle, torej za $1$. Zdaj vemo, da velja za $1$, torej mora po indukcijskem koraku veljati tudi za $2$. Tako nadaljujemo: sklepamo, da lastnost velja za $3$, nato za $4$ in tako naprej. Ker se vsa naravna števila pojavijo v verigi naslednikov ničle, mora z indukcijo dokazana lastnost dejansko veljati za vsa naravna števila.

Oglejmo si nekaj dokazov konkretnih izjav z indukcijo.

\note{preprosti primeri dokazov z indukcijo}

\note{Od tod naprej obravnavano variante indukcije.}

\note{indukcija na $\NN_{\geq n}$ ($n \in \NN$) in na $\ZZ_{\geq n}$ ($n \in \ZZ$)}

\note{indukcija v dve smeri na $\ZZ$}

\note{dvojna indukcija na $\NN \times \NN$ in splošneje večkratna gnezdena indukcija na $\NN^k$}

\note{indukcija s parametrom na $\NN \times X$}

\note{krepka indukcija (indukcijski korak se skliče na vsa manjša števila namesto na predhodnika; baza krepke indukcije ni potrebna)}

\subsection{Rekurzija}
\subsection{Računske operacije}
\subsection{Urejenost}
\subsection{Nadaljnje karakterizacije}


\section{Cela števila}
\section{Racionalna števila}
\section{Realna števila}
\section{Kompleksna števila}

\davorin{Se ustavimo že pri realnih številih? Gremo še dlje do kvaternionov?}


%%% Local Variables:
%%% mode: latex
%%% TeX-master: "ucbenik-lmn"
%%% End:
