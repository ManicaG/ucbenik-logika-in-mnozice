\documentclass[11pt,a4paper,twoside]{book}


%%%%%%%%%%%%%%%%%%%%%%%%%%%%%%%%%%%%%%%%%%%%%%%%%%%%%%%%%
%%%  Imported Packages
%%%%%%%%%%%%%%%%%%%%%%%%%%%%%%%%%%%%%%%%%%%%%%%%%%%%%%%%%
\usepackage[slovene]{babel}
\usepackage[utf8]{inputenc}
\usepackage[T1]{fontenc}

\usepackage{url}
\usepackage{ifthen}
\usepackage{amssymb}
\usepackage{amsmath}
\usepackage{mathabx}
\usepackage{theorem}
\usepackage{textgreek}
\usepackage{wasysym}
\usepackage{phonetic}
\usepackage{bm}
\usepackage{tablefootnote}
\usepackage{color}
%\usepackage[most]{tcolorbox}
\usepackage{tikz}
%\usepackage{tikzsymbols}
\usepackage{tkz-graph}
\usepackage{xparse}
\usepackage{mathrsfs}
\usepackage{ulem}
\usepackage{charter}


%%%%%%%%%%%%%%%%%%%%%%%%%%%%%%%%%%%%%%%%%%%%%%%%%%%%%%%%%%%%%
%%%  Theorems etc.
%%%%%%%%%%%%%%%%%%%%%%%%%%%%%%%%%%%%%%%%%%%%%%%%%%%%%%%%%%%%%
        {
                \theorembodyfont{\itshape}

                \newtheorem{izrek}{Izrek}[section]
                \newtheorem{lema}[izrek]{Lema}
                \newtheorem{trditev}[izrek]{Trditev}
                \newtheorem{posledica}[izrek]{Posledica}
        }

        {
                \theorembodyfont{\rmfamily}
                \newtheorem{definicija}[izrek]{Definicija}
                \newtheorem{opomba}[izrek]{Opomba}
                \newtheorem{primer}[izrek]{Primer}
                \newtheorem{zgled}[izrek]{Zgled}
                \newtheorem{vaja}[izrek]{Vaja}
        }

%%%%%%  Proofs
%%%%%%%%%%%%%%%%%%%%%%%%%%%%%%%%%%%%%%%%%%%%%%%%%%%%%%%%%%%%%
        \newenvironment{dokaz}{
                \goodbreak\par
                \textit{Dokaz.}%
        }{%
                \nopagebreak
                \hfill{\vrule width 1ex height 1ex depth 0ex}
                \medskip
                \goodbreak
        }
%%%%%%%%%%%%%%%%%%%%%%%%%%%%%%%%%%%%%%%%%%%%%%%%%%%%%%%%%%%%%%%%%%%%%%%%%%%%%%%%%%%%%%%%%%%%%%%%%%%%%%%%%%%%%%%%%%%%%%


%%%%%%%%%%%%%%%%%%%%%%%%%%%%%%%%%%%%%%%%%%%%%%%%%%%%%%%%%%%%%%%%%%%%%%%%%%%%%%%%%%%%%%%%%%%%%%%%%%%%%%%%%%%%%%%%%%%%%%
%%%  Commands
%%%%%%%%%%%%%%%%%%%%%%%%%%%%%%%%%%%%%%%%%%%%%%%%%%%%%%%%%%%%%%%%%%%%%%%%%%%%%%%%%%%%%%%%%%%%%%%%%%%%%%%%%%%%%%%%%%%%%%


%%%%%%  Auxiliary
%%%%%%%%%%%%%%%%%%%%%%%%%%%%%%%%%%%%%%%%%%%%%%%%%%%%%%%%%%%%%
\newcommand{\sizedescriptor}[2]
{
\ifthenelse{\equal{#1}{0}}{}{
\ifthenelse{\equal{#1}{1}}{\big}{
\ifthenelse{\equal{#1}{2}}{\Big}{
\ifthenelse{\equal{#1}{3}}{\bigg}{
\ifthenelse{\equal{#1}{4}}{\Bigg}{
#2}}}}}
}

\newcommand{\someref}{{\small\textcolor{blue}{[\textbf{ref.}]}}}
\newcommand{\intermission}{\bigskip\medskip}
\newcommand{\qt}[1]{{\quotedblbase}{#1}{‘‘}}  % text in quotation marks
\newcommand{\nls}[1]{\qt{\textit{#1}}}  % sentence in a natural language

\definecolor{andrejcolor}{rgb}{0.7,0,0.7}
\definecolor{davorincolor}{rgb}{0,0.45,1}
\definecolor{markocolor}{rgb}{0.7,0.4,0}
\definecolor{matijacolor}{rgb}{0,0.6,0.4}

\newcommand{\andrej}[1]{{\small\textcolor{andrejcolor}{(#1 \ \mbox{--Andrej})}}}
\newcommand{\davorin}[1]{{\small\textcolor{davorincolor}{(#1 \ \mbox{--Davorin})}}}
\newcommand{\marko}[1]{{\small\textcolor{markocolor}{(#1 \ \mbox{--Marko})}}}
\newcommand{\matija}[1]{{\small\textcolor{matijacolor}{(#1 \ \mbox{--Matija})}}}

\definecolor{notecolor}{rgb}{0.6,0.5,0.7}
\newcommand{\note}[1]{{\small\textcolor{notecolor}{(#1)}}}
\newcommand{\alert}[1]{{\small\textcolor{red}{\textbf{#1}}}}


%%%%%%  Logical Quantifiers, λ- and ι-Expressions
%%%%%%%%%%%%%%%%%%%%%%%%%%%%%%%%%%%%%%%%%%%%%%%%%%%%%%%%%%%%%

%  no parenthesis (add x in front of the name of the command)
\NewDocumentCommand{\xall}
{m O{\empty} m}
{\forall\, {#1} \ifthenelse{\equal{#2}{}}{}{\in{#2}} \,.\, {#3}}
\NewDocumentCommand{\xsome}
{m O{\empty} m}
{\exists\, {#1} \ifthenelse{\equal{#2}{}}{}{\in{#2}} \,.\, {#3}}
\NewDocumentCommand{\xexactlyone}
{m O{\empty} m}
{\exists\;\!!\, {#1} \ifthenelse{\equal{#2}{}}{}{\in{#2}} \,.\, {#3}}
\NewDocumentCommand{\xlam}
{m O{\empty} m O{\empty}}
{\lambda{#1} \ifthenelse{\equal{#2}{}}{}{\in{#2}} \,.\, {#3} \ifthenelse{\equal{#4}{}}{}{\in{#4}}}
\NewDocumentCommand{\xthat}
{m O{\empty} m}
{\iota{#1} \ifthenelse{\equal{#2}{}}{}{\in{#2}} \,.\, {#3}}

%  with parenthesis -- the first optional argument is the size (values 0-4)
\NewDocumentCommand{\all}
{O{auto} m O{\empty} m}
{\xall{#2}[#3]{\sizedescriptor{#1}{\left}( {#4} \sizedescriptor{#1}{\right})}}
\NewDocumentCommand{\some}
{O{auto} m O{\empty} m}
{\xsome{#2}[#3]{\sizedescriptor{#1}{\left}( {#4} \sizedescriptor{#1}{\right})}}
\NewDocumentCommand{\exactlyone}
{O{auto} m O{\empty} m}
{\xexactlyone{#2}[#3]{\sizedescriptor{#1}{\left}( {#4} \sizedescriptor{#1}{\right})}}
\NewDocumentCommand{\lam}
{O{auto} m O{\empty} m O{\empty}}
{\xlam{#2}[#3]{\sizedescriptor{#1}{\left}( {#4} \sizedescriptor{#1}{\right})}[#5]}
\NewDocumentCommand{\that}
{O{auto} m O{\empty} m}
{\xthat{#2}[#3]{\sizedescriptor{#1}{\left}( {#4} \sizedescriptor{#1}{\right})}}


%%%%%%  Logic
%%%%%%%%%%%%%%%%%%%%%%%%%%%%%%%%%%%%%%%%%%%%%%%%%%%%%%%%%%%%%
\newcommand{\tvs}{\Omega}  % set of truth values
\newcommand{\true}{\top}  % truth
\newcommand{\false}{\bot}  % falsehood
\newcommand{\etrue}{\boldsymbol{\top}}  % emphasized truth
\newcommand{\efalse}{\boldsymbol{\bot}}  % emphasized falsehood
\newcommand{\impl}{\Rightarrow}  % implication sign
\newcommand{\revimpl}{\Leftarrow}  % reverse implication sign
\newcommand{\lequ}{\Leftrightarrow}  % equivalence sign
\newcommand{\xor}{\mathbin{\veebar}}  % exclusive disjunction sign
\newcommand{\shf}{\mathbin{\uparrow}}  % Sheffer connective
\newcommand{\luk}{\mathbin{\downarrow}}  % Łukasiewicz connective


%%%%%%  Sets
%%%%%%%%%%%%%%%%%%%%%%%%%%%%%%%%%%%%%%%%%%%%%%%%%%%%%%%%%%%%%
%  \set{1, 2, 3}         ->  {1, 2, 3}
%  \set{a \in X}{a < 1}  ->  {a ∈ X | a < 1}
\NewDocumentCommand{\set}
{O{auto} m G{\empty}}
{\sizedescriptor{#1}{\left}\{ {#2} \ifthenelse{\equal{#3}{}}{}{ \; \sizedescriptor{#1}{\middle}| \; {#3}} \sizedescriptor{#1}{\right}\}}
%\newcommand{\vsubset}{\Mapstochar\cap}
%\newcommand{\finseq}[1]{{#1}^*}
\newcommand{\pst}{\mathcal{P}}
\renewcommand{\complement}[1]{{#1}^C}


%%%%%%  Number Sets, Intervals
%%%%%%%%%%%%%%%%%%%%%%%%%%%%%%%%%%%%%%%%%%%%%%%%%%%%%%%%%%%%%
\newcommand{\NN}{\mathbb{N}}
\newcommand{\ZZ}{\mathbb{Z}}
\newcommand{\QQ}{\mathbb{Q}}
\newcommand{\RR}{\mathbb{R}}
\newcommand{\CC}{\mathbb{C}}
\newcommand{\intoo}[3][\RR]{{#1}_{(#2, #3)}}
\newcommand{\intcc}[3][\RR]{{#1}_{[#2, #3]}}
\newcommand{\intoc}[3][\RR]{{#1}_{(#2, #3]}}
\newcommand{\intco}[3][\RR]{{#1}_{[#2, #3)}}


%%%%%%  Maps and Relations
%%%%%%%%%%%%%%%%%%%%%%%%%%%%%%%%%%%%%%%%%%%%%%%%%%%%%%%%%%%%%
\newcommand{\id}[1][]{\textrm{Id}_{#1}}  % identity map
\newcommand{\argbox}{{\;\!\fbox{\phantom{M}}\;\!}}  % box for a function argument
\newcommand{\rstr}[1]{\left.{#1}\right|}  % map restriction
\newcommand{\im}{\mathrm{im}}  % map image
\newcommand{\parto}{\mathrel{\rightharpoonup}}  % partial mapping sign
\NewDocumentCommand{\rel}
{O{\empty} O{\empty}}
{\ifthenelse{\equal{#1}{}}{\mathscr{R}}{{#1} \mathrel{\mathscr{R}} {#2}}}  % a relation
\NewDocumentCommand{\srel}
{O{\empty} O{\empty}}
{\ifthenelse{\equal{#1}{}}{\mathscr{S}}{{#1} \mathrel{\mathscr{S}} {#2}}}  % a second relation
\newcommand{\dom}{\mathrm{dom}}  % domain
\newcommand{\cod}{\mathrm{cod}}  % codomain
\newcommand{\dd}[1]{D_{#1}}  % domain of definition
\newcommand{\rn}[1]{Z_{#1}}  % range
\newcommand{\graph}[1]{\Gamma_{#1}}  % graph of a (partial) function
\NewDocumentCommand{\img}  % image
{O{\empty} m G{\empty}}
{{#2}_*\ifthenelse{\equal{#3}{}}{}{\!\sizedescriptor{#1}{\left}( {#3} \sizedescriptor{#1}{\right})}}
\NewDocumentCommand{\pim}  % preimage
{O{\empty} m G{\empty}}
{{#2}^*\ifthenelse{\equal{#3}{}}{}{\!\sizedescriptor{#1}{\left}( {#3} \sizedescriptor{#1}{\right})}}
\newcommand{\ec}[2][]{[\:\!{#2}\:\!]_{#1}}  % equivalence class
\newcommand{\transposed}[1]{\widehat{#1}}


%%%%%%  Misc.
%%%%%%%%%%%%%%%%%%%%%%%%%%%%%%%%%%%%%%%%%%%%%%%%%%%%%%%%%%%%%
\newcommand{\divides}[2]{#1 \mid #2} % deljivost

\newcommand{\df}[1]{\emph{\textbf{#1}}}  % defined notion
\newcommand{\oper}{\mathop{\circledast}}  % symbol for a general operation
\newcommand{\ism}{\cong}  % isomorphic
\newcommand{\equ}{\sim}  % equivalent
\newcommand{\dfeq}{\mathrel{\mathop:}=}  % definitional equality
\newcommand{\revdfeq}{=\mathrel{\mathop:}}  % reverse definitional equality
\newcommand{\isdefined}[1]{{#1}\!\downarrow}  % given value is defined
\newcommand{\kleq}{\simeq}  % Kleene equality
\newcommand{\claim}[3]{{#1} \;\colon\; \frac{#2}{#3}}  % claim, divided on context, assumptions, conclusions
\newcommand{\unit}{\mathord{\boldsymbol{*}}}  % element in a generic singleton
\NewDocumentEnvironment{implproof}  % proof of an implication
{O{\empty} G{\empty} O{=>} G{\empty}}
{
\begin{description}
\item[\quad$\sizedescriptor{#1}{\left}({#2}
\ifthenelse{\equal{#3}{=>}}{\impl}{
\ifthenelse{\equal{#3}{<=}}{\revimpl}{
\ifthenelse{\equal{#3}{->}}{\rightarrow}{
\ifthenelse{\equal{#3}{<-}}{\leftarrow}{
#3
}}}} {#4}\sizedescriptor{#1}{\right})$]\ \vspace{0.3em}\\
}
{
\end{description}
}


%%%%%%%%%%%%%%%%%%%%%%%%%%%%%%%%%%%%%%%%%%%%%%%%%%%%%%%%%%%%%%%%%%%%%%%%%%%%%%%%%%%%%%%%%%%%%%%%%%%%%%%%%%%%%%%%%%%%%%

%%% Local Variables:
%%% mode: latex
%%% TeX-master: "ucbenik-lmn"
%%% End:




%%%%%%%%%%%%%%%%%%%%%%%%%%%%%%%%%%%%%%%%%%%%%%%%%%%%%%%%%%%%%%%%%%%%%%%%%%%%%%%%%%%%%%%%%%%%%%%%%%%%%%%%%%%%%%%%%%%%%%
%%  Page Style & Margins (A4 page = 210mm x 297mm)

\setlength{\textwidth}{15cm}
\setlength{\textheight}{224mm}

\setlength{\topmargin}{0cm}
\setlength{\evensidemargin}{0cm}
\setlength{\oddsidemargin}{\paperwidth}
\addtolength{\oddsidemargin}{-\textwidth}
\addtolength{\oddsidemargin}{-2in}

\renewcommand{\baselinestretch}{1.25}
\setlength{\parskip}{1.5ex}


%%%%%%%%%%%%%%%%%%%%%%%%%%%%%%%%%%%%%%%%%%%%%%%%%%%%%%%%%%%%%%%%%%%%%%%%%%%%%%%%%%%%%%%%%%%%%%%%%%%%%%%%%%%%%%%%%%%%%%






\begin{document}

   %--------------------------------------------------------------------
   %--------------------------------------------------------------------
   % TITLE PAGE


        \title{\Huge \textbf{\textsc{Logika in množice}}}
        \author{A.~Bauer, D.~Lešnik, M.~Petkovšek, M.~Pretnar}

        \maketitle


   %--------------------------------------------------------------------
   %--------------------------------------------------------------------
   % Foreword

   \chapter*{Predgovor}%\addcontentsline{toc}{chapter}{\numberline{}Predgovor}


        %--------------------------------------------------------------------
        %--------------------------------------------------------------------
        % TOC


        \tableofcontents
        \listoftables


        %--------------------------------------------------------------------
        %--------------------------------------------------------------------
        % BODY


        \chapter{Matematično izražanje}\label{POGLAVJE: Matematično izražanje}

	\note{Za začetek bom vnašal oporne točke besedila. Slog bo verjetno treba še popraviti in besedilo dopolniti. --Davorin}
	
	\textcolor{red}{\small \textbf{Če je možno, prosim uporabljajte tabulatorje namesto presledkov za zamike v latex kodi in koda naj nima izrecno vnešenih prelomov vrstic, pač pa se v urejevalniku besedila uporablja avtomatski word wrap, ki se prilagaja širini okna. --Davorin}}
	
	Za matematično delo je bistveno, da se lahko zanašamo na pravilnost naših trditev. To pomeni:
	\begin{itemize}
		\item
			matematične izjave morajo imeti \emph{nedvoumen pomen},
		\item
			matematične izjave lahko \emph{dokažemo}.
	\end{itemize}
	
	Stavki v običajnih jezikih nimajo nedvoumnega pomena, zato matematične izjave raje podamo v \emph{matematičnem jeziku}. Za to potrebujemo \qt{matematično abecedo}, tj.~simbolni zapis, v katerem podamo izjave. Tega obravnavamo v naslednjem razdelku, dokazovanje matematičnih izjav pa v razdelku za tem.
	
	\section{Simbolni zapis}\label{RAZDELEK: Simbolni zapis}
	
		Za množice, s katerimi najpogosteje delamo, obstajajo standardne oznake (tabela~\ref{TABELA: standardne številske množice}).
		
		\begin{table}[!ht]
			\centering
			\begin{tabular}{|cc|}
				\hline
				\textbf{Množica} & \textbf{Oznaka} \\
				\hline
				množica naravnih števil & $\NN$ \\
				množica celih števil & $\ZZ$ \\
				množica racionalnih števil & $\QQ$ \\
				množica realnih števil & $\RR$ \\
				množica kompleksnih števil & $\CC$ \\
				\hline
			\end{tabular}
			\caption{standardne številske množice}\label{TABELA: standardne številske množice}
		\end{table}
		
		Nekateri $0$ vzamejo za naravno število, nekateri ne. To je v celoti stvar dogovora, kaj pomeni pojem \qt{naravno število}. Za nas bo prišlo bolj prav, če ničlo štejemo kot element množice naravnih števil, torej $\NN = \set{0, 1, 2, 3, \ldots}$.
		
		Interval realnih števil podamo s krajiščema intervala v oklepajih --- okrogli oklepaji ( ) označujejo odprtost intervala (krajišče ni vključeno v interval), oglati oklepaji [ ] pa zaprtost (krajišče je vključeno). Tako se npr.~interval realnih števil od $0$ do $1$, ki ne vsebuje krajišč, označi z $(0, 1)$, če jih vsebuje, pa z $[0, 1]$.
		
		Včasih pridejo prav tudi intervali na drugih množicah kot $\RR$. Zato se dogovorimo, da bomo intervale označevali tako, da podamo množico, ob kateri v indeksu zapišemo krajišči v oklepajih, npr.~$\intco[\NN]{1}{5} = \set{1, 2, 3, 4}$. Realna intervala iz prejšnjega odstavka tako zapišemo kot $\intoo{0}{1}$ in $\intcc{0}{1}$.
		
		Če interval v katero smer gre v nedogled, preprosto zapišemo množico z ustrezno relacijo urejenosti in krajiščem v indeksu. Na primer, $\RR_{> 0}$ označuje množico pozitivnih realnih števil, $\RR_{\geq 0}$ pa množico nenegativnih realnih števil.
		
		\note{To bi vsaj bil moj predlog. Na ta način se izognemo dvoumnostim (kar je namen). Na primer, kaj pomeni $\forall\, a > 0$? Če zapišemo $\forall\, a \in \NN_{> 0}$ ali $\forall\, a \in \RR_{> 0}$, je jasno. Razlog, da matematiki \qt{goljufajo} in pridejo skozi brez tega, je (napol dogovorjena in ponotranjena, ampak arbitrarna) izbira črk; vsak izkušen matematik ve, da $\forall\, \epsilon > 0$ pomeni $\forall\, \epsilon \in \RR_{> 0}$. Če se ne strinjate, popravite in pustite komentar. --Davorin}
		
		Izjavo, da je $2$ naravno število, zapišemo takole: $2 \in \NN$ (beri: $2$ pripada množici naravnih števil). Kako zapišemo, da je $a$ sodo število? Število je sodo, kadar je deljivo z $2$, torej pišemo $2 | a$ (beri: $2$ deli $a$).
		
		Če imamo več izjav, jih lahko strnemo v sestavljeno izjavo. Na primer, izjavo \nls{Če je $a$ sodo število, je tudi kvadrat števila $a$ sod.}, zapišemo kot $2 | a \implies 2 | a^2$.
		
		Seveda ta izjava velja za vsa naravna števila (znaš to dokazati?). To zapišemo takole: $\all{a}{\NN}{2 | a \implies 2 | a^2}$.
		
		Kot smo navajeni iz običajnih jezikov, posamične stavke povežemo v sestavljeno poved z \emph{vezniki}. Najpogosteje uporabljeni matematični vezniki so v tabeli~\ref{TABELA: standardni izjavni vezniki}.
		
		\begin{table}[!ht]
			\centering
			\begin{tabular}{|ccc|}
				\hline
				\textbf{Izjavni veznik} & \textbf{Oznaka} & \textbf{Kako preberemo} \\
				\hline
				negacija & $\lnot{p}$ & ne $p$ \\
				konjunkcija & $p \land q$ & $p$ in $q$ \\
				disjunkcija & $p \lor q$ & $p$ ali $q$ \\
				implikacija & $p \impl q$ & če $p$, potem $q$ \\
				ekvivalenca & $p \lequ q$ & $p$ natanko tedaj, ko $q$ \\
				\hline
			\end{tabular}
			\caption{standardni izjavni vezniki}\label{TABELA: standardni izjavni vezniki}
		\end{table}
		
		\begin{opomba}
			V matematiki se za izjavne veznike običajno uporabljajo zgoraj navedene tujke, ampak vsaka od njih seveda ima svoj pomen. Dobesedni prevodi teh tujk so:
			\begin{itemize}
				\item
					negacija $\to$ zanikanje,
				\item
					konjunkcija $\to$ vezava,
				\item
					disjunkcija $\to$ ločitev,
				\item
					implikacija $\to$ vpletenost,
				\item
					ekvivalenca $\to$ enakovrednost.
			\end{itemize}
			Za primerjavo: spomnite se vezalnega in ločnega priredja iz slovenščine!
		\end{opomba}
		
		\begin{zgled}
			Naj $p$ označuje stavek \nls{Zunaj dežuje.} in $q$ stavek \nls{Vzamem dežnik.}. Tedaj $\lnot{p}$ pomeni \nls{Zunaj ne dežuje.} in $p \impl q$ pomeni \nls{Če zunaj dežuje, potem vzamem dežnik.}.
		\end{zgled}
		
		Kose sestavljene izjave lahko veže več kot en veznik. V tem primeru se (tako kot pri računanju s števili) dogovorimo o prednosti veznikov. Po dogovoru je vrstni red veznikov tak, kot v tabeli~\ref{TABELA: standardni izjavni vezniki}, tj.~najmočneje veže negacija, nato konjunkcija, nato disjunkcija, nato implikacija, nato ekvivalenca. Kadar želimo, da se najprej izvede veznik z nižjo prednostjo, uporabimo oklepaje.
		
		\begin{zgled}
			Označimo sledeče stavke:
			\begin{quote}
				$p$ \ \ldots\ldots\ \nls{Imam čas.} \\
				$q$ \ \ldots\ldots\ \nls{Ostanem doma.}
			\end{quote}
			Tedaj $\lnot{p} \land q$ pomeni isto kot $(\lnot{p}) \land q$, to je \nls{Nimam časa in ostanem doma.}, medtem ko $\lnot(p \land q)$ pomeni \nls{Ni res, da imam čas in ostanem doma.}.
		\end{zgled}
		\note{Če komu pade na pamet primer boljših stavkov, je zaželjeno, da popravi\ldots --Davorin}
		
		Poleg zgoraj navedenih izjavnih veznikov se včasih uporabljajo še sledeči (tabela~\ref{TABELA: nadaljnji izjavni vezniki}).
		
		\begin{table}[!ht]
			\centering
			\begin{tabular}{|ccc|}
				\hline
				\textbf{Izjavni veznik} & \textbf{Oznaka} & \textbf{Kako preberemo} \\
				\hline
				stroga disjunkcija & $p \xor q$ & bodisi $p$ bodisi $q$ \\
				Shefferjev\tablefootnote{Henry Maurice Sheffer (1882 -- 1964) je bil ameriški logik.} veznik & $p \shf q$ & ne hkrati $p$ in $q$ \\
				Łukasiewiczev\tablefootnote{Jan Łukasiewicz (beri: \hill{u}ukaśj\^{e}vič) (1878 -- 1956) je bil poljski logik in filozof.} veznik & $p \luk q$ & niti $p$ niti $q$ \\
				\hline
			\end{tabular}
			\caption{nekateri nadaljnji izjavni vezniki}\label{TABELA: nadaljnji izjavni vezniki}
		\end{table}
		
		Za strogo disjunkcijo (tudi: ekskluzivna disjunkcija, izključitvena disjunkcija) se uporabljajo še druge oznake: $p \oplus q$, $p + q$. Razlika med navadno in strogo disjunkcijo je sledeča: $p \lor q$ pomeni, da \emph{vsaj eden} od $p$ in $q$ velja, medtem ko $p \xor q$ pomeni, da velja \emph{natanko eden}.
		
		\begin{zgled}
			Stavek \nls{Pisni del predmeta je potrebno opraviti s kolokviji ali pisnim izpitom.} je primer navadne disjunkcije (seveda se vam prizna pisni del predmeta tudi, če uspešno odpišete tako kolokvije kot pisni izpit), stavek \nls{Grem bodisi na morje bodisi v hribe.} pa je primer stroge disjunkcije (ne da se biti na dveh mestih hkrati).
		\end{zgled}
		
		Običajno veznike iz tabele~\ref{TABELA: nadaljnji izjavni vezniki} (in vse preostale, ki jih nismo navedli) izrazimo s standardnimi (glej tabelo~\ref{TABELA: izražava nadaljnjih izjavnih veznikov s standardnimi}), včasih pa je uporabno delati neposredno z njimi. Na primer, stroga disjunkcija služi kot seštevanje v Boolovem kolobarju (glej~\note{razdelek o Boolovih kolobarjih}), Shefferjev in Łukasiewiczev veznik pa se uporabljata pri preklopnih vezjih, saj je z vsakim od njiju možno izraziti vse izjavne veznike (glej~\note{razdelek o polnih naborih}). V računalništvu imajo ti trije vezniki standardne oznake XOR, NAND, NOR.
		
		\begin{table}[!ht]
			\centering
			\begin{tabular}{|ccc|}
				\hline
				\textbf{Izjavni veznik} & \multicolumn{2}{c|}{\textbf{Nekatere izražave s standardnimi vezniki}} \\
				\hline
				$p \xor q$ & $(p \lor q) \land \lnot(p \land q)$ & $(p \land \lnot{q}) \lor (\lnot{p} \land q)$ \\
				$p \shf q$ & $\lnot(p \land q)$ & $\lnot{p} \lor \lnot{q}$ \\
				$p \luk q$ & $\lnot(p \lor q)$ & $\lnot{p} \land \lnot{q}$ \\
				\hline
			\end{tabular}
			\caption{izražava nadaljnjih izjavnih veznikov s standardnimi}\label{TABELA: izražava nadaljnjih izjavnih veznikov s standardnimi}
		\end{table}
		
		\note{Na tem mestu povejmo, kakšno prednost damo tem trem veznikom v primerjavi s standardnimi. Kateremu dogovoru sledimo?}
		
		Včasih so izjave odvisne od kakšnih parametrov. Na primer, naj $\phi(x)$ pomeni \nls{$x$ je zelen.}; tedaj $\phi(\text{trava})$ pomeni \nls{Trava je zelena.}. Takim odvisnim izjavam rečemo \df{predikati} in izražajo lastnosti, ki jim parametri (\qt{spremenljivke}) lahko zadoščajo.
		
		Predikate lahko \emph{kvantificiramo} po njihovih spremenljivkah, tj.~povemo, \qt{kako pogosto} velja lastnost, dana s predikatom. Tabela~\ref{TABELA: kvantifikatorji} podaja najpogosteje uporabljane kvantifikatorje in njihove oznake.
		
		\begin{table}[!ht]
			\centering
			\begin{tabular}{|ccc|}
				\hline
				\textbf{Kvantifikator} & \textbf{Oznaka} & \textbf{Kako preberemo} \\
				\hline
				univerzalni kvantifikator & $\xall{x}{X}{\phi(x)}$ & za vsak $x$ iz $X$ velja lastnost $\phi$ \\
				eksistenčni kvantifikator & $\xsome{x}{X}{\phi(x)}$ & obstaja $x$ iz $X$ z lastnostjo $\phi$ \\
				\note{kako se temu reče?} & $\xexactlyone{x}{X}{\phi(x)}$ & obstaja natanko en $x$ iz $X$ z lastnostjo $\phi$ \\
				\hline
			\end{tabular}
			\caption{kvantifikatorji}\label{TABELA: kvantifikatorji}
		\end{table}
		
		Oznaki $\forall$ in $\exists$ sta narobe obrnjena A in E in izhajata iz nemščine (\textbf{a}ll, \textbf{e}xistiert).
		
		\begin{zgled}
			Vemo, da za vsako nenegativno realno število obstaja enolično določen nenegativen kvadratni koren; to izjavo lahko zapišemo na sledeči način.
			\[\xall{a}{\RR_{\geq 0}}\xexactlyone{b}{\RR_{\geq 0}}{b^2 = a}\]
			Zaradi tega lahko definiramo kvadratni koren kot funkcijo $\sqrt{\phantom{I}}\colon \RR_{\geq 0} \to \RR_{\geq 0}$.
		\end{zgled}
		
		Po dogovoru kvantifikatorji vežejo šibkeje kot izjavni vezniki. Izjavo, da je vsako celo število bodisi liho bodisi sodo, torej zapišemo takole.
		\[\all[2]{a}{\ZZ}{2 | a \xor 2 | (a-1)}\]
		
		\note{Se že na tem mestu predebatirajo vezane oz.~nevezane spremenljivke ter preimenovanje spremenljivk? Kaj je slovenski prevod za `dummy variable'?}
		
		\begin{zgled}
			Za poljubno naravno število $n \in \NN$ naj $P(n)$ označuje izjavo, da je $n$ praštevilo. Torej, $P$ definiramo takole.
			\[P(n) \dfeq \all[1]{x}{\NN}{x | n \implies x = 1 \xor x = n}\]
			(Premisli, kaj bi se zgodilo, če bi namesto stroge disjunkcije vzeli navadno. Bi še vedno dobili pravilni pojem praštevila?)
			
			Naj $S(n)$ označuje, da je $n$ sestavljeno število.
			\[S(n) \dfeq \xsome{x, y}{\intoo[\NN]{1}{n}}{x \cdot y = n}\]
			(Kadar imamo več zaporednih kvantifikatorjev iste vrste, jih po dogovoru lahko strnemo kot zgoraj. Dana formula za $S(n)$ je krajši zapis za $\xsome{x}{\intoo[\NN]{1}{n}}\xsome{y}{\intoo[\NN]{1}{n}}{x \cdot y = n}$.)
			
			Zdaj lahko na pregleden način zapišemo, da je vsako naravno število od $2$ naprej bodisi praštevilo bodisi sestavljeno.
			\[\all[1]{n}{\NN_{\geq 2}}{P(n) \xor S(n)}\]
		\end{zgled}
		
		\note{
			Ideje za VAJE:\\
				\hbox{}\qquad\qquad * Napiši te in te z besedami podane matematične izjave simbolno.\\
				\hbox{}\qquad\qquad * Za te in te \qt{življenjske} izjave vsak osnoven sestavni kos označi s črko in zapiši sestavljeno izjavo z mešanico veznikov in kvantifikatorjev.\\
				\hbox{}\qquad\qquad * ...
		}
	
	
	\section{Pravila dokazovanja}\label{RAZDELEK: Pravila dokazovanja}
	
		Matematične izsledke običajno podajamo preko jasno izraženih izjav. Med študijem matematike hitro opazite, da se takšne izjave podajajo pod imeni \quotesinglbase{izrek}', \quotesinglbase{trditev}', \quotesinglbase{lema}', \quotesinglbase{posledica}' in podobno. Kdaj uporabiti katerega teh imen ni natanko določeno, pač pa je prepuščeno presoji matematika. Približno vodilo je naslednje:
		\begin{itemize}
			\item
				\df{izrek}: osrednji, bistven rezultat,
			\item
				\df{trditev}: stranski rezultat,
			\item
				\df{lema}: rezultat, ki sam po sebi nima toliko vsebine, se pa uporabi pri dokazovanju pomembnejšega rezultata,\footnote{Sicer ni nujno, da se resnična pomembnost izjav takoj pokaže. Mnogo je primerov, ko se kak matematični članek po določenem času začne ceniti ne toliko zaradi glavnega izreka, pač pa zaradi neke leme, ki se je za dokaz glavnega izreka uporabila.}
			\item
				\df{posledica}: rezultat, ki je zanimiv sam po sebi, ki pa hitro sledi iz predhodne izjave.
		\end{itemize}
		
		Če skrbno analizirate izreke, trditve itd.~s predavanj (ali matematičnih člankov), opazite, da sestojijo iz treh delov: \note{kontekst, predpostavke, sklepi}
	
	
	\section{Definicije}
        \chapter{Logika}\label{POGLAVJE: Logika}

	\note{uvod}
	
	
	\section{Izjavni vezniki}
	
		V razdelku~\ref{RAZDELEK: Logični simboli} smo omenili nekaj izjavnih veznikov, podali oznake zanje in opisali njihov intuitivni pomen. Ampak če se hočemo zanašati na pravilnost naših sklepov, moramo tem oznakam dati \emph{formalni matematični pomen}.
		
		Če imamo neko izjavo, lahko določimo njeno resničnost, tj.~povemo, do kolikšne mere je resnična. Temu rečemo \df{resničnostna vrednost} izjave. Množico vseh možnih resničnostnih vrednosti označimo z $\tvs$. Seveda ni kaj dosti možnih resničnostnih vrednosti: to sta \df{resnica} (dogovorimo se, da bomo zanjo uporabljali oznako $\true$) in \df{neresnica} (oznaka $\false$). Se pravi, $\tvs = \set{\true, \false}$.
		
		\begin{opomba}
			Logiki, kjer sta edini resničnostni vrednosti resnica in neresnica, rečemo \df{dvovrednostna} oziroma \df{klasična logika}. Obstajajo splošnejše vrste logike, kjer je $\set{\true, \false}$ prava podmnožica $\tvs$, ampak v tej knjigi se bomo omejili na klasično logiko, na katero ste navajeni in ki se uporablja v večjem delu matematike.
		\end{opomba}
		
		\davorin{Kako izrecno bomo ločevali med izjavami in njihovimi logičnimi vrednostmi?}
		
		Izjavne veznike lahko potem formalno podamo kot preslikave. Na primer, negacija je preslikava $\lnot\colon \tvs \to \tvs$ (vsaki resničnostni vrednosti pripišemo njeno nasprotno vrednost). Preslikavo, definirano na majhni končni množici, lahko preprosto podamo s tabelo vseh njenih vrednosti. V primeru izjavnih veznikov takim tabelam rečemo \df{resničnostne tabele}. Resničnostna tabela za negacijo izgleda takole.
		\begin{center}
			\begin{tabular}{c|c}
				$p$ & $\lnot{p}$ \\
				\hline
				$\true$ & $\false$ \\
				$\false$ & $\true$
			\end{tabular}
		\end{center}
		Ta tabela povsem natančno definira negacijo kot preslikavo $\lnot\colon \tvs \to \tvs$. Seveda smo negacijo definirali tako, kot bi pričakovali: negacija resnice je neresnica, negacija neresnice je resnica.
		
		Podobno lahko naredimo z ostalimi izjavnimi vezniki, le da preostali vežejo dve izjavi. Se pravi, npr.~konjunkcija vzame dve resničnostni vrednosti in vrne resničnostno vrednost, ki pove, ali sta obe dani vrednosti resnični. Konjunkcijo lahko torej interpretiramo kot preslikavo $\land\colon \tvs \times \tvs \to \tvs$ (ali na kratko $\land\colon \tvs^2 \to \tvs$).
		
		V splošnem definiramo, da je \df{$n$-mestni izjavni veznik} preslikava oblike $\tvs^n \to \tvs$. Negacija je torej enomestni izjavni veznik, ostali vezniki, ki smo jih do zdaj omenili, pa so dvomestni.
		
		Definirajmo zdaj konjunkcijo natančno preko resničnostne tabele. Množica $\tvs \times \tvs$ ima štiri elemente --- vse možne pare, sestavljene iz $\true$ oz.~$\false$. Intuitivni pomen konjunkcije razumemo: konjunkcija dveh izjav je resnična natanko tedaj, ko sta obe izjavi resnični. To nas vodi do naslednje tabele.
		\begin{center}
			\begin{tabular}{cc|c}
				$p$ & $q$ & $p \land q$ \\
				\hline
				$\true$ & $\true$ & $\true$ \\
				$\true$ & $\false$ & $\false$ \\
				$\false$ & $\true$ & $\false$ \\
				$\false$ & $\false$ & $\false$
			\end{tabular}
		\end{center}
		
		Za disjunkcijo smo že rekli, da pride v dveh različicah: navadna pomeni, da vsaj ena od izjav velja, izključitvena pa pomeni, da velja natanko ena od izjav. Posledično je torej smiselno definirati funkciji $\lor, \xor\colon \tvs \times \tvs \to \tvs$ na sledeči način.
		\begin{center}
			\begin{tabular}{cc|cc}
				$p$ & $q$ & $p \lor q$ & $p \xor q$ \\
				\hline
				$\true$ & $\true$ & $\true$ & $\false$ \\
				$\true$ & $\false$ & $\true$ & $\true$ \\
				$\false$ & $\true$ & $\true$ & $\true$ \\
				$\false$ & $\false$ & $\false$ & $\false$
			\end{tabular}
		\end{center}
		Bodi pozoren na razliko med zadnjima dvema stolpcema!
		
		Obenem lahko še na hitro opravimo z veznikoma $\shf$ in $\luk$. Spomnimo se, da $p \shf q$ pomeni \qt{ne hkrati $p$ in $q$}, medtem ko $p \luk q$ pomeni \qt{niti $p$ niti $q$}.
		\begin{center}
			\begin{tabular}{cc|cc}
				$p$ & $q$ & $p \shf q$ & $p \luk q$ \\
				\hline
				$\true$ & $\true$ & $\false$ & $\false$ \\
				$\true$ & $\false$ & $\true$ & $\false$ \\
				$\false$ & $\true$ & $\true$ & $\false$ \\
				$\false$ & $\false$ & $\true$ & $\true$
			\end{tabular}
		\end{center}
		
		Implikacija je nekoliko bolj subtilna. Kaj točno trdimo z izjavo $p \impl q$, se pravi, kakor hitro velja $p$, mora veljati tudi $q$? No, če $p$ ne velja, potem sploh nismo postavili nobenega pogoja --- izjava je avtomatično izpolnjena. Če $p$ velja, pa zraven zahtevamo še $q$. Resničnostna tabela za implikacijo je potemtakem sledeča.
		\begin{center}
			\begin{tabular}{cc|c}
				$p$ & $q$ & $p \impl q$ \\
				\hline
				$\true$ & $\true$ & $\true$ \\
				$\true$ & $\false$ & $\false$ \\
				$\false$ & $\true$ & $\true$ \\
				$\false$ & $\false$ & $\true$
			\end{tabular}
		\end{center}
		
		Ekvivalenca je spet preprosta --- izjavi sta ekvivalentni, kadar imata isto resničnostno vrednost. Od tod dobimo sledečo resničnostno tabelo.
		\begin{center}
			\begin{tabular}{cc|c}
				$p$ & $q$ & $p \lequ q$ \\
				\hline
				$\true$ & $\true$ & $\true$ \\
				$\true$ & $\false$ & $\false$ \\
				$\false$ & $\true$ & $\false$ \\
				$\false$ & $\false$ & $\true$
			\end{tabular}
		\end{center}
		
		Za lažjo referenco zberimo resničnostne tabele vseh do zdaj omenjenih veznikov na eno mesto (tabela~\ref{TABELA: Resničnostna tabela osnovnih izjavnih veznikov}).
		
		\begin{table}[!ht]
			\centering
			\begin{tabular}{c|c}
				$p$ & $\lnot{p}$ \\
				\hline
				$\true$ & $\false$ \\
				$\false$ & $\true$
			\end{tabular}
			\qquad\quad
			\begin{tabular}{cc|ccccccc}
				$p$ & $q$ & $p \land q$ & $p \lor q$ & $p \xor q$ & $p \shf q$ & $p \luk q$ & $p \impl q$ & $p \lequ q$ \\
				\hline
				$\true$ & $\true$ & $\true$ & $\true$ & $\false$ & $\false$ & $\false$ & $\true$ & $\true$ \\
				$\true$ & $\false$ & $\false$ & $\true$ & $\true$ & $\true$ & $\false$ & $\false$ & $\false$ \\
				$\false$ & $\true$ & $\false$ & $\true$ & $\true$ & $\true$ & $\false$ & $\true$ & $\false$ \\
				$\false$ & $\false$ & $\false$ & $\false$ & $\false$ & $\true$ & $\true$ & $\true$ & $\true$
			\end{tabular}
			\caption{Resničnostna tabela osnovnih izjavnih veznikov}\label{TABELA: Resničnostna tabela osnovnih izjavnih veznikov}
		\end{table}
		
		Zdaj ko imamo natančno definicijo izjavnih veznikov, lahko trditve v zvezi z njimi tudi formalno utemeljimo. Na primer, spomnimo se, da smo že malo po omembi veznikov $\xor$, $\shf$, $\luk$ podali njihovo izražavo z vezniki $\lnot$, $\land$, $\lor$. Če na glas preberemo vse izjave, nam je intuitivno jasno, katere se ujemajo in zakaj, ampak zdaj lahko dejansko preverimo, da te izražave veljajo.
		
		Na primer, kaj pomeni, da se $p \luk q$ lahko izrazi kot $\lnot(p \lor q)$? To pomeni, da sta funkciji $\tvs \times \tvs \to \tvs$, dani s predpisoma $(p, q) \mapsto p \luk q$ in $(p, q) \mapsto \lnot(p \lor q)$, enaki. (Slednja funkcija je sestavljena, tj.~sklop dveh funkcij. Lahko bi tudi zapisali, da velja $\luk = \lnot \circ \lor$.) Funkciji z isto domeno in kodomeno sta enaki, kadar pri vsakem argumentu vrneta isti vrednosti, kar v našem primeru pomeni, da imata enaka stolpca v resničnostni tabeli. Poračunajmo torej vse izraze v danih izražavah. Ko dobimo enake rezultate, bomo vedeli, da izražave dejansko veljajo.
		
		\begin{center}
			\begin{tabular}{cc|cccccc}
				$p$ & $q$ & $p \shf q$ & $p \land q$ & $\lnot(p \land q)$ & $\lnot{p}$ & $\lnot{q}$ & $\lnot{p} \lor \lnot{q}$ \\
				\hline
				$\true$ & $\true$ & $\efalse$ & $\true$ & $\efalse$ & $\false$ & $\false$ & $\efalse$ \\
				$\true$ & $\false$ & $\etrue$ & $\false$ & $\etrue$ & $\false$ & $\true$ & $\etrue$ \\
				$\false$ & $\true$ & $\etrue$ & $\false$ & $\etrue$ & $\true$ & $\false$ & $\etrue$ \\
				$\false$ & $\false$ & $\etrue$ & $\false$ & $\etrue$ & $\true$ & $\true$ & $\etrue$
			\end{tabular}
		\end{center}
		
		\begin{center}
			\begin{tabular}{cc|cccccc}
				$p$ & $q$ & $p \luk q$ & $p \lor q$ & $\lnot(p \lor q)$ & $\lnot{p}$ & $\lnot{q}$ & $\lnot{p} \land \lnot{q}$ \\
				\hline
				$\true$ & $\true$ & $\efalse$ & $\true$ & $\efalse$ & $\false$ & $\false$ & $\efalse$ \\
				$\true$ & $\false$ & $\efalse$ & $\true$ & $\efalse$ & $\false$ & $\true$ & $\efalse$ \\
				$\false$ & $\true$ & $\efalse$ & $\true$ & $\efalse$ & $\true$ & $\false$ & $\efalse$ \\
				$\false$ & $\false$ & $\etrue$ & $\false$ & $\etrue$ & $\true$ & $\true$ & $\etrue$
			\end{tabular}
		\end{center}
		
		\begin{center}
			\begin{tabular}{cc|ccccc}
				$p$ & $q$ & $p \xor q$ & $p \lor q$ & $p \land q$ & $\lnot(p \land q)$ & $(p \lor q) \land \lnot(p \land q)$  \\
				\hline
				$\true$ & $\true$ & $\efalse$ & $\true$ & $\true$ & $\false$ & $\efalse$ \\
				$\true$ & $\false$ & $\etrue$ & $\true$ & $\false$ & $\true$ & $\etrue$ \\
				$\false$ & $\true$ & $\etrue$ & $\true$ & $\false$ & $\true$ & $\etrue$ \\
				$\false$ & $\false$ & $\efalse$ & $\false$ & $\false$ & $\true$ & $\efalse$
			\end{tabular}
		\end{center}
		
		\begin{center}
			\begin{tabular}{cc|cccccc}
				$p$ & $q$ & $p \xor q$ & $\lnot{q}$ & $p \land \lnot{q}$ & $\lnot{p}$ & $\lnot{p} \land q$ & $(p \land \lnot{q}) \lor (\lnot{p} \land q)$  \\
				\hline
				$\true$ & $\true$ & $\efalse$ & $\false$ & $\false$ & $\false$ & $\false$ & $\efalse$ \\
				$\true$ & $\false$ & $\etrue$ & $\true$ & $\true$ & $\false$ & $\false$ & $\etrue$ \\
				$\false$ & $\true$ & $\etrue$ & $\false$ & $\false$ & $\true$ & $\true$ & $\etrue$ \\
				$\false$ & $\false$ & $\efalse$ & $\true$ & $\false$ & $\true$ & $\false$ & $\efalse$
			\end{tabular}
		\end{center}
		
		Kako simbolno zapisati, da sta dve izražavi enaki? Lahko bi pisali
		\[\big((p, q) \mapsto p \shf q\big) = \big((p, q) \mapsto \lnot(p \land q)\big),\]
		ampak to je nekoliko nerodno in nepregledno. Kasneje (v razdelku~\note{o anonimnih funkcijah}) se bomo naučili $\lambda$-notacijo, s katero dobimo
		\[\xlam{(p, q)}[\tvs^2]{p \shf q} = \xlam{(p, q)}[\tvs^2]{\lnot(p \land q)},\]
		ampak to je še vedno nepregledno. Uveljavil se je običaj, da se izraze, ki so enakovredni v smislu, da dajo isti rezultat pri vsaki izbiri argumentov, poveže s simbolom $\equiv$, torej zapišemo
		\[p \shf q \equiv \lnot(p \land q).\]
		Konkretno za izraze v logiki se uporablja tudi $\sim$, se pravi, zapišemo lahko tudi
		\[p \shf q \sim \lnot(p \land q).\]
		V tej knjigi se bomo držali uporabe simbola $\equiv$. \davorin{Recimo. Po mojem je to boljše, ker lahko $\equiv$ uporabljamo še za druge funkcije (npr.~$f(x) \equiv 0$ pomeni, da je $f$ konstantno enaka $0$, medtem ko $f(x) = 0$ predstavlja enačbo, s katero iščemo ničle funkcije) in ker bomo kasneje $\sim$ uporabljali za ekvivalenčne relacije.}
		
		Med drugim smo s temi tabelami izpeljali tako imenovana \df{de Morganova zakona} za izjavno logiko \davorin{Verjetno je smiselno specificirati \qt{za izjavno logiko}. Imeli bomo namreč še zakona za predikatno logiko (za $\forall$ in $\exists$) ter za množice (za preseke in unije).}, ki povesta, kako negacija vpliva na konjunkcijo in disjunkcijo:
		\[\lnot(p \land q) \equiv \lnot{p} \lor \lnot{q},\]
		\[\lnot(p \lor q) \equiv \lnot{p} \land \lnot{q}.\]
		To je smiselno: kadar ni res, da veljata oba $p$ in $q$, vsaj eden od njiju ne velja. Kadar ni res, da velja vsaj eden od njiju, nobeden od njiju ne velja.
		
		Z resničnostnimi tabelami lahko preverimo še mnoge druge formule. \df{Zakon dvojne negacije} pravi $\lnot\lnot{p} \equiv p$, tj.~če dvakrat zanikamo izjavo, dobimo izjavo, enakovredno začetni. Poračunajmo tabelo.
		
		\begin{center}
			\begin{tabular}{c|ccc}
				$p$ & $\lnot{p}$ & $\lnot\lnot{p}$ & $p$ \\
				\hline
				$\true$ & $\false$ & $\etrue$ & $\etrue$ \\
				$\false$ & $\true$ & $\efalse$ & $\efalse$
			\end{tabular}
		\end{center}
		
		Spomnimo se: za poljubno dvomestno operacijo $\oper$ na neki množici $X$ rečemo, da je
		\begin{itemize}
			\item
				\df{izmenljiva} ali \df{komutativna}, kadar velja $a \oper b = b \oper a$ za vse $a, b \in X$ (na kratko: $a \oper b \equiv b \oper a$),
			\item
				\df{družitvena} \davorin{ne spomnim se --- kako se že temu reče po slovensko?} ali \df{asociativna}, kadar velja $(a \oper b) \oper c = a \oper (b \oper c)$ za vse $a, b, c \in X$ (na kratko: $(a \oper b) \oper c \equiv a \oper (b \oper c)$),
			\item
				\df{idempotentna} \davorin{a imamo slovenski izraz za to?}, kadar velja $a \oper a = a$ za vse $a \in X$ (torej $a \oper a \equiv a$).
		\end{itemize}
		
		Preverimo z resničnostno tabelo, da je konjunkcija komutativna, torej $p \land q \equiv q \land p$.
		
		\begin{center}
			\begin{tabular}{cc|ccccc}
				$p$ & $q$ & $p \land q$ & $q \land p$ \\
				\hline
				$\true$ & $\true$ & $\etrue$ & $\etrue$ \\
				$\true$ & $\false$ & $\efalse$ & $\efalse$ \\
				$\false$ & $\true$ & $\efalse$ & $\efalse$ \\
				$\false$ & $\false$ & $\efalse$ & $\efalse$
			\end{tabular}
		\end{center}
		
		Še hitreje lahko preverimo, da je konjunkcija idempotentna.
		
		\begin{center}
			\begin{tabular}{c|cc}
				$p$ & $p \land p$ & $p$ \\
				\hline
				$\true$ & $\etrue$ & $\etrue$ \\
				$\false$ & $\efalse$ & $\efalse$
			\end{tabular}
		\end{center}
		
		Kako pa preveriti, da je konjunkcija asociativna, torej $(p \land q) \land r \equiv p \land (q \land r)$? Vidimo, da v teh izrazih nastopajo tri spremenljivke in torej potrebujemo resničnostno tabelo, kjer upoštevamo vseh osem možnosti za izbiro $p$, $q$, $r$.
		
		\begin{center}
			\begin{tabular}{ccc|cccc}
				$p$ & $q$ & $r$ & $p \land q$ & $(p \land q) \land r$ & $q \land r$ & $p \land (q \land r)$ \\
				\hline
				$\true$ & $\true$ & $\true$ & $\true$ & $\etrue$ & $\true$ & $\etrue$ \\
				$\true$ & $\true$ & $\false$ & $\true$ & $\efalse$ & $\false$ & $\efalse$ \\
				$\true$ & $\false$ & $\true$ & $\false$ & $\efalse$ & $\false$ & $\efalse$ \\
				$\true$ & $\false$ & $\false$ & $\false$ & $\efalse$ & $\false$ & $\efalse$ \\
				$\false$ & $\true$ & $\true$ & $\false$ & $\efalse$ & $\true$ & $\efalse$ \\
				$\false$ & $\true$ & $\false$ & $\false$ & $\efalse$ & $\false$ & $\efalse$ \\
				$\false$ & $\false$ & $\true$ & $\false$ & $\efalse$ & $\false$ & $\efalse$ \\
				$\false$ & $\false$ & $\false$ & $\false$ & $\efalse$ & $\false$ & $\efalse$
			\end{tabular}
		\end{center}
		
		To pomeni, da lahko v izrazih, kjer nastopa več zaporednih konjunkcij, spuščamo oklepaje: namesto $p \land (\lnot{q} \land r)$ pišemo kar $p \land \lnot{q} \land r$.
		
		Enako velja tudi za disjunkcijo.
		
		\begin{vaja}
			Dokaži, da je disjunkcija komutativna, asociativna in idempotentna!
		\end{vaja}
		
		Preostali dvomestni vezniki, ki smo jih omenili, ne zadoščajo vsem trem lastnostim naenkrat.
		
		\begin{vaja}
			Preveri, kateri znani dvomestni izjavni vezniki so komutativni, asociativni oziroma idempotentni!
		\end{vaja}
		
		Ko rešite zgornjo vajo, boste med drugim opazili: implikacija ni komutativna. To pomeni, da lahko definiramo nov izjavni veznik $\revimpl$ na naslednji način: $p \revimpl q \dfeq q \impl p$ za vse $p, q \in \tvs$. Z drugimi besedami, $\revimpl$ je dan s sledečo resničnostno tabelo.
		\begin{center}
			\begin{tabular}{cc|c}
				$p$ & $q$ & $p \revimpl q$ \\
				\hline
				$\true$ & $\true$ & $\true$ \\
				$\true$ & $\false$ & $\true$ \\
				$\false$ & $\true$ & $\false$ \\
				$\false$ & $\false$ & $\true$
			\end{tabular}
		\end{center}
		
		\note{dokazi s pomočjo resničnostnih tabel še vseh ostalih formul, ki jih hočemo imeti, med drugim distributivnosti}
		
		Do zdaj smo omenili zgolj nekaj posamičnih izjavnih veznikov. Koliko pa je vseh skupaj? Spomnimo se, da je $n$-mestni izjavni veznik definiran kot preslikava $\tvs^n \to \tvs$. Množica $\tvs^n$ vsebuje vse urejene $n$-terice elementov $\true$ in $\false$; teh je $2^n$ (za vsako od $n$ mest v $n$-terici imamo dve možnosti in vse te izbire so neodvisne med sabo). Za vsako od teh $2^n$ večteric imamo dve možnosti, kam jo preslikamo: v $\true$ ali v $\false$. Vseh možnosti --- torej vseh $n$-mestnih veznikov --- je potemtakem $2^{2^n}$. (Vseh izjavnih veznikov, ko dopuščamo vse možne $n$, je seveda neskončno.)
		
		Za boljšo predstavo si oglejmo vse $n$-mestne veznike za majhne $n \in \NN$. Prva možnost je $n = 0$. Formula nam pravi, da je število ničmestnih izjavnih veznikov enako $2^{2^0} = 2^1 = 2$. Kaj pomeni, da pri nič vhodnih podatkih vrnemo $\true$ ali $\false$? To pomeni, da preprosto izberemo resničnostno vrednost --- z drugimi besedami, ničmestni izjavni vezniki so isto kot resničnostne vrednosti.
		
		Koliko je vseh enomestnih izjavnih veznikov? Formula pravi $2^{2^1} = 2^2 = 4$. Zapišimo vse možnosti.
		
		\begin{center}
			\begin{tabular}{c|cccc}
				$p$ &&&& \\
				\hline
				$\true$ & $\true$ & $\false$ & $\true$ & $\false$ \\
				$\false$ & $\true$ & $\false$ & $\false$ & $\true$
			\end{tabular}
		\end{center}
		
		Vidimo: vsi enomestni izjavni vezniki so obe konstantni funkciji v $\tvs$, identiteta na $\tvs$ in negacija.
		
		Kar se dvomestnih veznikov tiče, vidimo, da jih je $2^{2^2} = 2^4 = 16$.
		
		\begin{vaja}
			Preveri, da so vsi dvomestni vezniki natanko: konstanta z vrednostjo $\top$, projekcija na prvo komponento (tj.~$(p, q) \mapsto p$), projekcija na drugo komponento (tj.~$(p, q) \mapsto q$), konjunkcija $\land$, disjunkcija $\lor$, implikacija $\impl$, povratna implikacija $\revimpl$, ekvivalenca $\lequ$ in negacije vseh teh.
		\end{vaja}
		
		Tromestnih veznikov je že $2^{2^3} = 2^8 = 256$ in ne bomo vseh naštevali. Kako pa bi kakega dobili? Preprost način je, da vzamemo tri spremenljivke in jih združimo z večimi znanimi vezniki, na primer $(p, q, r) \mapsto p \land \lnot{q} \impl r$.\footnote{Načeloma sploh ni nujno, da vse tri spremenljivke dejansko uporabimo. Na primer, $(p, q, r) \mapsto p \land q$ še vedno podaja tromestni veznik, saj gre za preslikavo $\tvs^3 \to \tvs$.}
		
		Seveda se pojavi vprašanje, kako podati izjavne veznike, ki jih ne bi mogli sestaviti iz osnovnih. Izkaže se, da to ni problem: \emph{vsak veznik (ne glede na mestnost) je možno izraziti z osnovnimi}; pravzaprav zadostujejo že $\lnot$, $\land$ in $\lor$.
		
		Ideja je sledeča. Katerikoli izjavni veznik je oblike $V\colon \tvs^n \to \tvs$ in v celoti podan z resničnostno tabelo. Vzemimo konkreten primer; naj bo $V$ tromestni veznik, podan z naslednjo tabelo.
		
		\begin{center}
			\begin{tabular}{ccc|c}
				$p$ & $q$ & $r$ & $V(p, q, r)$ \\
				\hline
				$\true$ & $\true$ & $\true$ & $\false$ \\
				$\true$ & $\true$ & $\false$ & $\true$ \\
				$\true$ & $\false$ & $\true$ & $\true$ \\
				$\true$ & $\false$ & $\false$ & $\false$ \\
				$\false$ & $\true$ & $\true$ & $\true$ \\
				$\false$ & $\true$ & $\false$ & $\true$ \\
				$\false$ & $\false$ & $\true$ & $\false$ \\
				$\false$ & $\false$ & $\false$ & $\false$
			\end{tabular}
		\end{center}
		
		Tedaj lahko rečemo: $V$ je resničen tedaj, ko smo v 2., 3., 5.~ali 6.~vrstici. Kdaj smo v drugi vrstici? Točno tedaj, ko $p$ in $q$ veljata, $r$ pa ne, se pravi, ko velja $p \land q \land \lnot{r}$. Podobno naredimo še za preostale vrstice: tretja je določena s $p \land \lnot{q} \land r$, peta z $\lnot{p} \land q \land r$ in šesta z $\lnot{p} \land q \land \lnot{r}$. Potemtakem lahko zapišemo:
		\[V(p, q, r) \equiv (p \land q \land \lnot{r}) \lor (p \land \lnot{q} \land r) \lor (\lnot{p} \land q \land r) \lor (\lnot{p} \land q \land \lnot{r}).\]
		Temu rečemo \df{disjunktivna normalna oblika} (s kratico DNO) veznika $V$.
		
		Obstaja še dualna oblika take izražave. Lahko si rečemo tudi, da je $V$ resničen, kadar nismo v 1., 4., 7.~oz.~8.~vrstici. Kdaj nismo v prvi vrstici? Kadar niso vsi $p$, $q$, $r$ resnični, torej ko je vsaj eden od njih neresničen --- s formulo $\lnot{p} \lor \lnot{q} \lor \lnot{r}$. Kdaj nismo v četrti vrstici? Ko ni res, da je $p$ resničen, $q$ in $r$ pa ne, torej ko prekršimo vsaj enega teh pogojev, kar nam da formulo $\lnot{p} \lor q \lor r$. Podobno sklepamo, da nismo v sedmi vrstici, kadar velja $p \lor q \lor \lnot{r}$, in da nismo v osmi vrstici, kadar velja $p \lor q \lor r$. To nam da sledečo izražavo za $V$:
		\[V(p, q, r) \equiv (\lnot{p} \lor \lnot{q} \lor \lnot{r}) \land (\lnot{p} \lor q \lor r) \land (p \lor q \lor \lnot{r}) \land (p \lor q \lor r).\]
		Temu rečemo \df{konjunktivna normalna oblika} (s kratico KNO) veznika $V$.
		
		Spremenljivkam in njihovim negacijam z eno besedo rečemo \df{literali}. Disjunktivna normalna oblika je torej disjunkcija konjunkcij literalov, konjunktivna normalna oblika pa konjunkcija disjunkcij literalov.
		
		Iz tega primera je jasno, kako postopamo za poljuben izjavni veznik in zanj zapišemo DNO ali KNO. Opazimo: dolžina posamičnega člena, ki ga omejujejo oklepaji, je vedno enaka (vsebuje toliko literalov, kolikor je mestnost veznika), število teh členov pa razberemo iz stolpca, ki podaja vrednosti veznika v resničnostni tabeli. V primeru DNO je to število enako številu resnic $\true$, v primeru KNO pa številu neresnic $\false$. V zgornjem primeru sta bili DNO in KNO enako dolgi, ker smo imeli štiri $\true$ in $\false$, v splošnem pa se nam morda bolj splača uporabiti eno obliko kot drugo. Na primer, DNO implikacije se glasi $p \impl q \equiv (p \land q) \lor (\lnot{p} \land q) \lor (\lnot{p} \land \lnot{q})$, KNO pa je precej krajša: $p \impl q \equiv \lnot{p} \lor q$.
		
		Vidimo pa, da tu naletimo na problem: kaj se zgodi, če se katera resničnostna vrednost v stolpcu veznika sploh ne pojavi --- z drugimi besedami, kaj če je funkcija, ki podaja veznik, konstantna? Najprej dajmo takim ime: izjavni veznik, ki je pri vseh argumentih resničen, se imenuje \df{istorečje} ali \df{tavtologija}, izjavni veznik, ki je vedno neresničen, pa se imenuje \df{protislovje} ali \df{kontradikcija}.
		
		Za istorečje lahko vedno (ne glede na mestnost) zapišemo DNO (ki je sicer najdaljša možna), medtem ko bi KNO načeloma bila konjunkcija nič členov. Je to smiselno? V bistvu ja: če zahtevamo, da hkrati velja nič pogojev, je naša zahteva vedno izpolnjena. V tem smislu je konjunkcija nič členov enaka $\true$.
		
		Poglejmo podobne primere iz računstva. Kaj je vsota nič členov? Odgovor je seveda $0$. To je enota za seštevanje, kar je smiselno: če nič členom prištejemo en člen, moramo imeti zgolj ta člen. Podobno sklepamo: zmnožek nič členov je enota za množenje $1$ --- če nič faktorjem dodamo še en faktor, imamo skupaj zgolj ta faktor. Spomni se tudi: $a^0 = 1$ in $0! = 1$. To, da je ničkratna uporabe neke operacije enaka enoti za to operacijo, se izide tudi za konjunkcijo: dejansko velja $p \land \true \equiv p \equiv \true \land p$ (preveri z resničnostno tabelo!).
		
		Enak razmislek velja za protislovje. Zanj lahko zapišemo KNO na običajen način, medtem ko bi DNO bila disjunkcija nič členov. Smiselno je, da je disjunkcija nič členov enaka $\false$, tako zaradi tega, ker je $\false$ enota za disjunkcijo (preveri!), kot zaradi čisto intuitivnega razmisleka: kdaj je vsaj eden člen od nič členov resničen? Nikoli.
		
		Vseeno je nekoliko nerodno delati s konjunkcijo ali disjunkcijo nič členov --- kako točno bi to zapisali? Da velja $V(p_1, p_2, \ldots, p_n) \equiv $? Če nič ne zapišemo, kako sploh vemo, ali smo mislili na ničkratno konjunkcijo, disjunkcijo ali katerokoli drugo operacijo? Nekateri se zato preprosto dogovorijo, da ne dopuščajo ničkratnih operacij v DNO oz.~KNO in potem štejejo, da istorečja nimajo KNO, protislovja pa ne DNO.
		
		Tudi če ne dopuščamo ničkratnih operacij, pa še vedno velja: vsak izjavni veznik z mestnostjo vsaj $1$ ima vsaj eno od DNO oz.~KNO in ga torej lahko izrazimo samo z negacijo, konjunkcijo in disjunkcijo. Družini izjavnih veznikov, s katerimi lahko izrazimo vse veznike z mestnostjo vsaj $1$, rečemo \df{poln nabor}. Na kratko lahko torej rečemo, da je $\set{\lnot, \land, \lor}$ poln nabor.
		
		Jasno, če je neka množica veznikov poln nabor, je tudi vsaka njena nadmnožica poln nabor. Sledi, da je tudi na primer $\set{\lnot, \land, \lor, \impl}$ poln nabor.
		
		Spomnimo se zdaj de Morganovih zakonov in zakona o dvojni negaciji --- iz njih lahko izpeljemo $p \land q \equiv \lnot(\lnot{p} \lor \lnot{q})$ in $p \lor q \equiv \lnot(\lnot{p} \land \lnot{q})$. Se pravi, konjunkcijo lahko izrazimo z disjunkcijo in negacijo in prav tako lahko disjunkcijo izrazimo s konjunkcijo in negacijo. To pomeni, da sta že $\set{\lnot, \lor}$ in $\set{\lnot, \land}$ polna nabora! Se pravi, vse veznike s pozitivno mestnostjo je možno izraziti že samo z dvema.
		
		Je možno iti še dlje in najti en sam veznik, s katerim lahko izrazimo ostale? Odgovor je da: $\set{\shf}$ in $\set{\luk}$ sta polna nabora. (Izkaže se, da sta to edina taka veznika med dvomestniki vezniki.)
		
		\begin{vaja}\label{VAJA: polni nabori z enim veznikom}
			\
			\begin{enumerate}
				\item
					Izrazi negacijo samo z veznikom $\shf$. Izrazi še konjunkcijo ali disjunkcijo samo z veznikom $\shf$. Sklepaj, da je $\set{\shf}$ poln nabor.
				\item
					Izrazi negacijo samo z veznikom $\luk$. Izrazi še konjunkcijo ali disjunkcijo samo z veznikom $\luk$. Sklepaj, da je $\set{\luk}$ poln nabor.
			\end{enumerate}
		\end{vaja}
		
		\davorin{Bi na tem mestu predebatirali preklopna vezja?}
		
		\davorin{Mogoče lahko zavoljo celovitosti podamo karakterizacijo polnih naborov kot izrek (in se za dokaz skličemo na literaturo). Nabor je poln, če za vsako sledečih lastnosti obstaja veznik v njem, ki jo prekrši: ohranjanje resnice, ohranjanje neresnice, monotonost, sebi-dualnost, afinost (kot polinom Žegalkina).}
	
	
	\section{Predikati in kvantifikatorji}
	
		\note{\qt{Lastnostim} elementov množic, ki smo jih prej uporabljali za podajanje podmnožic in pri kvantifikatorjih, zdaj \qt{uradno} rečemo \df{predikati} in jih formalno definiramo: predikat na množici $X$ je preslikava $X \to \tvs$. Karakteristične preslikave podmnožic. Spomnimo se kvantifikatorjev. \davorin{Si jih drznemo definirati kot preslikave na eksponentih $\tvs^X$?} Povemo, da lahko imajo predikati več spremenljivk (lahko so definirani na produktu) in da lahko kvantificiramo po samo nekaterih. Vezane, nevezane spremenljivke. Pravila, ki veljajo za kvantifikatorje (de Morgan itd.).}
        \chapter{Dokazovanje}\label{poglavje:dokazovanje}

        Matematične izsledke običajno podajamo preko jasno izraženih izjav. Med študijem matematike hitro opazite, da se takšne izjave podajajo pod imeni `izrek', `trditev', `lema', {posledica} in podobno. Kdaj uporabiti katerega teh imen ni natanko določeno, pač pa je prepuščeno presoji matematika. Približno vodilo je naslednje:
        \begin{itemize}
                \item
                        \df{izrek}: osrednji, bistven rezultat,
                \item
                        \df{trditev}: stranski rezultat,
                \item
                        \df{lema}: rezultat, ki sam po sebi nima toliko vsebine, se pa uporabi pri dokazovanju pomembnejšega rezultata,\footnote{Sicer ni nujno, da se resnična pomembnost izjav takoj pokaže. Mnogo je primerov, ko se kak matematični članek po določenem času začne ceniti ne toliko zaradi glavnega izreka, pač pa zaradi neke leme, ki se je za dokaz glavnega izreka uporabila.}
                \item
                        \df{posledica}: rezultat, ki je zanimiv sam po sebi, ki pa hitro sledi iz predhodne izjave.
        \end{itemize}

        Če skrbno analizirate izreke, trditve itd.~s predavanj (ali iz matematičnih člankov), opazite, da sestojijo iz treh delov: kontekst, predpostavke, sklepi.
        \begin{itemize}
                \item
                        \df{Kontekst} pove, katere objekte obravnavamo in kakšne vrste so.
                \item
                        \df{Predpostavke} so izjave, ki jih privzamemo.
                \item
                        \df{Sklepi} so izjave, ki jih (pri danih predpostavkah) dokazujemo.
        \end{itemize}

        Oglejmo si konkreten primer. Rolleov izrek je znan in uporaben izrek v analizi (če ga še niste spoznali, ga boste v kratkem).

        \begin{izrek}[Rolle]
                Naj bo $f$ realna funkcija, definirana na intervalu $\intcc{a}{b}$, kjer sta $a$ in $b$ realni števili in $a < b$. Če je $f$ zvezna na celem $\intcc{a}{b}$ in odvedljiva na odprtem intervalu $\intoo{a}{b}$ ter zavzame enaki vrednosti v krajiščih, tj.~$f(a) = f(b)$, tedaj ima $f$ stacionarno točko na $\intoo{a}{b}$.
        \end{izrek}

        Analizirajmo, kaj so kontekst, predpostavke in sklepi pri tem izreku.

        \begin{itemize}
                \item
                        Kontekst je sledeč:
                        \[a \in \RR, \qquad b \in \RR_{> a}, \qquad f \in \RR^{\intcc{a}{b}}.\]
                        To so objekti (in njihove vrste), o katerih govori izrek. Smiselno je, da jih zapišemo v tem vrstnem redu; na primer, $f$ zapišemo nazadnje, saj je njena domena odvisna od $a$ in $b$. Kadar imamo objekte, ki so neodvisni med sabo, jih lahko zapišemo v poljubnem vrstnem redu.
                \item
                        Predpostavke so tri. Vsako navedimo v običajnem jeziku in nato še s simbolnim matematičnim zapisom.
                        \begin{itemize}
                                \item
                                        $f$ je zvezna na $\intcc{a}{b}$.
                                        \[
                                                \hspace{-2em}
                                                \all{x \in \intcc{a}{b}}
                                                        \all{\epsilon \in \RR_{> 0}}
                                                                \some{\delta \in \RR_{> 0}}
                                                                        \all{y \in \intcc{a}{b}}
                                                                                (|x - y| < \delta \impl \big|f(x) - f(y)\big| < \epsilon
                                                                        )
                                        \]
                                \item
                                        $f$ je odvedljiva na $\intoo{a}{b}$.
                                        \begin{multline*}
                                                \all{x \in \intoo{a}{b}}
                                                        \some{v \in \RR}
                                                                \all{\epsilon \in \RR_{> 0}}
                                                                        \some{\delta \in \RR_{> 0}}
                                                \all{h \in \RR_{\neq 0}} \\
                                                        (|h| < \delta \implies \Big|\frac{f(x + h) - f(x)}{h} - v\Big| < \epsilon)
                                        \end{multline*}
                                \item
                                        $f$ na krajiščih intervala zavzame enaki vrednosti.
                                        \[f(a) = f(b)\]
                        \end{itemize}
                        Če se vam morda zdita formuli za zveznost in odvedljivost begajoči, imate dve tolažbi. Prva je ta, da se boste čez čas takšnih formul navadili. ;) Druga je, da so tudi drugi matematiki leni po naravi in zato uvedejo oznake za daljše izraze, ki se pogosto uporabljajo. Zgornja zveznost se na krajše zapiše $f \in \mathcal{C}(\intcc{a}{b})$ ($\mathcal{C}$ kot ``continuous'', tj.~zvezen), odvedljivost pa $f \in \mathcal{D}^1(\intoo{a}{b})$ ($\mathcal{D}$ kot ``differentiable'', tj.~odvedljiv, enka pa pomeni ``(vsaj) enkrat odvedljiv'').
                \item
                        Sklep je eden: $f$ ima stacionarno točko na $\intoo{a}{b}$, kar simbolno zapišemo takole.
                        \[\some{x \in \intoo{a}{b}} f'(x) = 0\]
        \end{itemize}

        V splošnem imamo določeno mero svobode, kako natančno razčleniti izrek. Na primer, za Rolleov izrek bi lahko kontekst zapisali tudi kot $a \in \RR, b \in \RR, f \in \RR^{\intcc{a}{b}}$ in pogoj $a < b$ dodali med predpostavke.

        Da ne bomo pisali dolgih seznamov, se dogovorimo za sledeče oznake. Izrek podamo tako, da najprej zapišemo kontekst, nato dvopičje, nato narišemo vodoravno črto, nad črto zapišemo predpostavke (ločene z vejicami), pod črto pa sklepe (ločene z vejicami). Rolleov izrek bi potemtakem povzeli takole.
        \[\claim{a \in \RR, b \in \RR_{> a}, f \in \RR^{\intcc{a}{b}}}{f \in \mathcal{C}(\intcc{a}{b}), f \in \mathcal{D}^1(\intoo{a}{b}), f(a) = f(b)}{\some{x \in \intoo{a}{b}} f'(x) = 0}\]

        V splošnem velja: vse proste spremenljivke, ki se pojavijo v predpostavkah ali sklepih, morajo biti navedene v kontekstu. Po domače povedano: če trdite, da za neko stvar nekaj velja, morate najprej povedati, o kateri stvari sploh govorite.

        Medtem ko je za težje matematične izreke potrebno obilo ustvarjalnosti, da se jih dokaže, pa lažje trditve pogosto lahko avtomatično dokažemo (dobesedno --- obstajajo avtomatični dokazovalniki \davorin{koliko povemo na to temo?}), pa tudi za težje je pomembno, da vemo, kako pristopiti k dokazu. Gre za to, da za vse logične veznike in kvantifikatorje obstajajo splošna pravila, kako ravnamo, če nastopajo kot predpostavke oziroma kot sklepi. To si bomo zdaj ogledali.

        \begin{itemize}
                \item\textbf{Konjunkcija}
                        \begin{itemize}
                                \item
                                        Če $p \land q$ nastopa kot \emph{predpostavka}:
                                        \begin{quote}
                                                predpostavko $p \land q$ nadomestimo s predpostavkama $p$, $q$ (to se pravi, pri dokazovanju lahko uporabimo tako predpostavko $p$ kot predpostavko $q$). S simboli, od trditve
                                                \[\claim{\Gamma}{\Pi', p \land q, \Pi''}{\Sigma}\]
                                                preidemo do trditve
                                                \[\claim{\Gamma}{\Pi', p, q, \Pi''}{\Sigma}\]
                                                (pri zapisih splošnih izrekov bomo kontekst označevali z $\Gamma$, predpostavke s $\Pi$ in sklepe s $\Sigma$).
                                        \end{quote}
                                \item
                                        Če $p \land q$ nastopa kot \emph{sklep}:
                                        \begin{quote}
                                                sklep $p \land q$ dokažemo tako, da dokažemo posebej $p$ in posebej $q$. S simboli:
                                                \[\claim{\Gamma}{\Pi}{\Sigma', p \land q, \Sigma''}\]
                                                preoblikujemo v
                                                \[\claim{\Gamma}{\Pi}{\Sigma', p, q, \Sigma''}\]
                                                (in se zavedamo, da je za dokaz izreka potrebno dokazati \emph{vse} sklepe).
                                        \end{quote}
                        \end{itemize}
                \item\textbf{Disjunkcija}
                        \begin{itemize}
                                \item
                                        Če $p \lor q$ nastopa kot \emph{predpostavka}:
                                        \begin{quote}
                                                ločimo primere: sklepe dokažemo posebej pri predpostavki $p$ (skupaj z ostalimi predpostavkami) in posebej pri predpostavki $q$ (skupaj z ostalimi). Torej, dokazati
                                                \[\claim{\Gamma}{\Pi', p \lor q, \Pi''}{\Sigma}\]
                                                pomeni isto, kot dokazati tako
                                                \[\claim{\Gamma}{\Pi', p, \Pi''}{\Sigma} \qquad \text{kot} \qquad \claim{\Gamma}{\Pi', q, \Pi''}{\Sigma}.\]
                                        \end{quote}
                                \item
                                        Če $p \lor q$ nastopa kot \emph{sklep}:
                                        \begin{quote}
                                                izberemo si enega od $p$, $q$ in ga dokažemo. Se pravi, če imamo
                                                \[\claim{\Gamma}{\Pi}{\Sigma', p \lor q, \Sigma''},\]
                                                si izberemo eno od trditev
                                                \[\claim{\Gamma}{\Pi}{\Sigma', p, \Sigma''} \qquad \text{oziroma} \qquad \claim{\Gamma}{\Pi}{\Sigma', q, \Sigma''}\]
                                                in jo izpeljemo.
                                        \end{quote}
                        \end{itemize}
                \item\textbf{Implikacija}
                        \begin{itemize}
                                \item
                                        Če $p \impl q$ nastopa kot \emph{predpostavka}:
                                        \begin{quote}
                                                če nam kadarkoli uspe izpeljati $p$, lahko dodamo $q$ med predpostavke. Torej, če znamo dokazati
                                                \[\claim{\Gamma}{\Pi', p \impl q, \Pi''}{q},\]
                                                potem za dokaz
                                                \[\claim{\Gamma}{\Pi', p \impl q, \Pi''}{\Sigma}\]
                                                zadostuje dokazati
                                                \[\claim{\Gamma}{\Pi', p \impl q, q, \Pi''}{\Sigma}\]
                                                (kar je lažje, ker imamo eno predpostavko več). To je smiselno: če vemo, da velja $p \impl q$ in dodatno ugotovimo, da velja $p$, potem vemo, da velja tudi $q$.
                                        \end{quote}
                                \item
                                        Če $p \impl q$ nastopa kot \emph{sklep}:
                                        \begin{quote}
                                                sklep $p \impl q$ nadomestimo s $q$, medtem ko $p$ dodamo med predpostavke. Pojasnimo. Trditev $p \impl q$ trdi nekaj samo v primeru, kadar $p$ velja --- v nasprotnem primeru je avtomatično resnična in ni ničesar za dokazati. Torej se lahko omejimo na primer, ko $p$ velja, se pravi, lahko predpostavimo $p$. Kadar $p$ velja, pa trditev $p \impl q$ pravi, da mora veljati tudi $q$. To pomeni, da pri predpostavki $p$ dokazujemo $q$. Simbolno, da dokažemo
                                                \[\claim{\Gamma}{\Pi}{\Sigma', p \impl q, \Sigma''},\]
                                                zadostuje dokazati
                                                \[\claim{\Gamma}{\Pi}{\Sigma', \Sigma''} \qquad \text{in} \qquad \claim{\Gamma}{\Pi, p}{q}.\]
                                        \end{quote}
                        \end{itemize}
                \item\textbf{Univerzalni kvantifikator}
                        \begin{itemize}
                                \item
                                        Če $\all{x \in X} \phi(x, y)$ nastopa kot \emph{predpostavka}:
                                        \begin{quote}
                                                če vemo za (ali med dokazom najdemo) katerikoli konkreten element $a \in X$, tedaj lahko med predpostavke dodamo $\phi(a, y)$. Namreč, če vemo, da lastnost $\phi$ (z morebitnimi nadaljnjimi parametri) velja za vse elemente množice $X$, potem ta lastnost velja za poljuben konkreten element. Simbolno, od
                                                \[\claim{\Gamma', a \in X, \Gamma''}{\Pi', \all{x \in X} \phi(x, y), \Pi''}{\Sigma}\]
                                                preidemo do
                                                \[\claim{\Gamma', a \in X, \Gamma''}{\Pi', \all{x \in X} \phi(x, y), \phi(a, y), \Pi''}{\Sigma}.\]
                                        \end{quote}
                                \item
                                        Če $\all{x \in X} \phi(x, y)$ nastopa kot \emph{sklep}:
                                        \begin{quote}
                                                v kontekst dodamo $x \in X$, sklep $\all{x \in X} \phi(x, y)$ pa nadomestimo s sklepom $\phi(x, y)$. S simboli, od
                                                \[\claim{\Gamma}{\Pi}{\Sigma', \all{x \in X} \phi(x, y), \Sigma''}\]
                                                preidemo do
                                                \[\claim{\Gamma, x \in X}{\Pi}{\Sigma', \phi(x, y), \Sigma''}\]
                                                Zakaj tako postopamo in kaj smo s tem pravzaprav naredili? Premislimo: želimo dokazati, da neka lastnost velja za vse elemente dane množice $X$. Če ima $X$ slučajno samo končno mnogo elementov, bi lahko lastnost preverili za vsakega posebej, ampak povečini delamo z neskončnimi množicami, kjer to ne deluje. Morda ima množica $X$ kakšno posebno lastnost, zaradi katere lahko univerzalni kvantifikator dokažemo na svojevrsten način (na primer, univerzalno kvantificirane izjave nad $\NN$ lahko dokazujemo z matematično indukcijo --- glej \note{razdelek o naravnih številih}), ampak to se zgodi v izjemnih primerih.

                                                V splošnem nimamo druge možnosti, kot da si izberemo simbol (tipično kar spremenljivko v kvantifikatorju), ki nam predstavlja poljuben, katerikoli element množice in zanj dokažemo želeno lastnost. Ideja je, da spremenljivka spet nastopa v vlogi ``škatlice'', kamor lahko vstavimo poljuben element množice $X$. Če nam je dokaz lastnosti uspel, ne da bi za spremenljivko predpostavili karkoli več, kot da predstavlja element množice $X$, tedaj dobimo dokaz lastnosti za katerikoli dejanski element množice $X$ tako, da v dobljeni dokaz namesto spremenljivke vstavimo ta element. Na ta način smo potem dejansko dobili dokaz lastnosti za vse elemente množice $X$.

                                                Besedni dokazi univerzalno kvantificirane izjave se zato tipično začnejo takole: ``Vzemimo poljuben $x \in X$. Dokažimo, da zanj velja dana lastnost.''
                                        \end{quote}
                        \end{itemize}
                \item\textbf{Eksistenčni kvantifikator}
                        \begin{itemize}
                                \item
                                        Če $\some{x \in X} \phi(x, y)$ nastopa kot \emph{predpostavka}:
                                        \begin{quote}
                                                v kontekst dodamo $x \in X$, eksistenčno predpostavko pa nadomestimo s $\phi(x, y)$. S simboli,
                                                \[\claim{\Gamma}{\Pi', \some{x \in X} \phi(x, y), \Pi''}{\Sigma}\]
                                                popravimo v
                                                \[\claim{\Gamma, x \in X}{\Pi', \phi(x, y), \Pi''}{\Sigma}.\]
                                                Zakaj to deluje? Naša predpostavka je, da v množici $X$ obstaja element z lastnostjo $\phi$ (z morebitnimi nadaljnjimi parametri). Torej si lahko vzamemo neki konkreten element množice $X$ s to lastnostjo, ki ga lahko uporabljamo kasneje v dokazu (za to ga moramo nekako označiti; v praksi ga tipično označimo kar z isto spremenljivko, kot v kvantifikatorju).
                                        \end{quote}
                                \item
                                        Če $\some{x \in X} \phi(x, y)$ nastopa kot \emph{sklep}:
                                        \begin{quote}
                                                da dokažemo eksistenčno izjavo, moramo podati neki konkreten element $x \in X$ in zanj dokazati dano lastnost $\phi(x, y)$. \davorin{Hm, kako točno to zapišemo simbolno v zgornji obliki?}
                                        \end{quote}
                        \end{itemize}
        \end{itemize}

        V zgornjem seznamu nismo omenili vseh veznikov in kvantifikatorjev. To je zato, ker jih pri dokazovanju nadomestimo z zgornjimi. Konkretno:
        \begin{itemize}
                \item
                        Za negacijo velja $\lnot{p} \equiv p \impl \false$. Med drugim to pomeni, da $\lnot{p}$ dokažemo na sledeči način: predpostavimo $p$ in iz tega izpeljemo neresnico.
                \item
                        Za ekvivalenco velja $p \lequ q \equiv (p \impl q) \land (p \revimpl q)$. To pomeni, da ekvivalenco dokažemo tako, da dokažemo implikacijo med $p$ in $q$ v obe smeri --- se pravi, enkrat predpostavimo $p$ in izpeljemo $q$, drugič pa predpostavimo $q$ in izpeljemo $p$.
                \item
                        Za veznike $\xor$, $\shf$, $\luk$ si preprosto izberemo eno od izražav z negacijo, konjunkcijo in disjunkcijo in nato delamo z njo.
                \item
                        Kvantifikator $\exactlyone{x \in X} \phi(x, y)$ ločimo na dva dela: na obstoj in enoličnost, in vsakega posebej dokažemo. Se pravi, skličemo se na izražavo
                        \[\exactlyone{x \in X} \phi(x, y) \equiv \some{x \in X} \phi(x, y) \land \all{a, b \in X} (\phi(a, y) \land \phi(b, y) \implies a = b).\]
                        Včasih je lažje, če najprej dokažemo obstoj elementa in ta element pri dokazu enoličnosti že uporabimo, torej dokazujemo izražavo
                        \[\exactlyone{x \in X} \phi(x, y) \equiv \some{x \in X} (\phi(x, y) \land \all{a \in X} (\phi(a, y) \implies a = x)).\]
        \end{itemize}

        Seveda ne bo možno dokazati vsakega izreka s slepim sledenjem zgornjim pravilom; včasih moramo uporabiti še kakšno dodatno strategijo. Spodnji dve sta zelo pogosti.
        \begin{itemize}
                \item
                        Med predpostavke dodamo trditev, za katero že vemo, da je resnična. Morda gre za trditev, ki smo jo že dokazali, morda pa gre kar za istorečje. Pogost primer tega je, da uporabimo zakon o izključenem tretjem in za dodatno predpostavko vzamemo $p \lor \lnot{p}$ (kjer je $p$ katerakoli konkretna izjava). Po zgornjih pravilih to potem pomeni, da ločimo primere in trditev dokažemo posebej pri predpostavki $p$ ter posebej pri predpostavki $\lnot{p}$.
                \item
                        Nekatere predpostavke ali sklepe nadomestimo z enakovrednimi izjavami. Na primer, velja
                        \[p \lor q \equiv \lnot(\lnot{p} \land \lnot{q}) \equiv \lnot{p} \impl q \equiv \lnot{q} \impl p.\]
                        To pomeni, da lahko disjunkcijo (poleg zgoraj omenjenega načina) dokažemo tudi tako, da predpostavimo, da nobena od možnosti ne velja, in od tod izpeljemo neresnico, ali pa predpostavimo, da ena od možnosti ne velja, in od tod izpeljemo drugo.

                        Zelo pogosta uporaba te ideje je \df{dokaz s protislovjem}, ki temelji na zakonu o dvojni negaciji $p \equiv \lnot\lnot{p}$. Izjavo torej lahko dokažemo tako, da predpostavimo njeno negacijo, in od tod izpeljemo neresnico. Tipičen besedni dokaz s protislovjem izgleda takole: ``Dokazujemo $p$. Pa recimo, da $p$ ne velja. Potem /neki sklepi/. To je v nasprotju s tem, kar smo dokazali prej, torej smo izpeljali protislovje. Se pravi, ni možno, da $p$ ne bi veljal, torej mora veljati.''
        \end{itemize}

        \note{mnogo zgovornih primerov dokazov, ki ponazorijo zgornje postopke}


\section{Vaje}


%%% Local Variables:
%%% mode: latex
%%% TeX-master: "ucbenik-lmn"
%%% End:


        \chapter{Konstrukcije množic}
                \section{Preprosti primeri}
                        \note{prazna množica, enojci}
                \section{Podmnožice}
                        \davorin{Če \qt{embedding} prevajamo kot \qt{vložitev}, kako potem prevedemo \qt{inclusion}? Imamo sicer tujko \qt{inkluzija}, ampak fino bi bilo imeti še slovenski izraz. Vključitev?}
                \section{Potenčna množica}
                        \davorin{Verjetno je smiselno, da ta razdelek sledi razdelku o podmnožicah. Morda kar združimo ta dva razdelka?}
                \section{Družine množic}
                \section{Produkt množic}
                \section{Vsota množic}
                \section{Unija in presek}
                \section{Eksponentna množica}
                        \davorin{Vrstni red teh razdelkov bomo najbrž še premešali.}

        \chapter{Preslikave}



\section{Slike in praslike}

Preslikava kot taka nam pove za posamične elemente, kam se slikajo. Marsikdaj pa nas zanima več: kam se slikajo celotne množice elementov. Na primer, zanima nas lahko, v kaj se projicira neko prostorsko telo na ravnino.

\note{luštna slika projekcije nekega prostorskega objekta na neko ravnino}

Da dobimo sliko celotne množice, moramo zbrati skupaj slike vseh posamičnih elementov množice. Smiselna je torej naslednja definicija.

\begin{definicija}\label{definicija:slika}
Naj bo $f\colon X \to Y$ preslikava. \df{Slika} množice $A \subseteq X$ je označena in definirana kot
\[\img{f}{A} \dfeq \set[1]{f(x)}{x \in A} = \set[1]{y \in Y}{\some{x \in A} y = f(x)}.\]
Ta predpis definira preslikavo $\img{f}\colon \pst(X) \to \pst(Y)$.
\end{definicija}

\begin{opomba}
Kot običajno, obstajajo različne oznake v uporabi. Sliko $\img{f}{A}$ se označuje tudi kot $f[A]$ ali celo kar kot $f(A)$. V slednjem primeru se predpostavlja zadostna matematična zrelost bralca, da zna razbrati, kdaj $f$ označuje preslikavo $f\colon X \to Y$, kdaj pa preslikavo $f\colon \pst(X) \to \pst(Y)$.

V tej knjigi se bomo načrtno izogibali takšnim dvoumnostim in za sliko dosledno uporabljali oznako iz definicije~\ref{definicija:slika}.
\end{opomba}

\begin{naloga}
Prepričaj se, da za poljubno preslikavo $f\colon X \to Y$ velja sledeče:
\begin{itemize}
\item
$\img{f}{X} = \rn{f}$,
\item
$\img{f}{\emptyset} = \emptyset$,
\item
$\img[1]{f}{\set{x}} = \set[1]{f(x)}$ za vsak $x \in X$.
\end{itemize}
\end{naloga}

\note{primeri in lastnosti slik že tu ali kasneje skupaj s primeri/lastnostmi praslik?}

Včasih pa imamo obratno nalogo: iz dane slike ugotoviti, kaj vse se je z neko preslikavo vanjo preslikalo. Zato vpeljemo še sledečo definicijo.

\begin{definicija}\label{definicija:praslika}
Naj bo $f\colon X \to Y$ preslikava. \df{Praslika} množice $B \subseteq Y$ je označena in definirana kot
\[\pim{f}{B} \dfeq \set[1]{x \in X}{f(x) \in B}.\]
Ta predpis definira preslikavo $\pim{f}\colon \pst(Y) \to \pst(X)$.
\end{definicija}

\begin{opomba}
Tudi za prasliko obstajajo različne oznake. Praslika $\pim{f}{B}$ se označi tudi kot $f^{-1}[B]$ ali kar kot $f^{-1}(B)$. V slednjem primeru se spet zanašamo na izkušenost bralca, da praslike $f^{-1}\colon \pst(Y) \to \pst(X)$ ne zamenja z obratom $f^{-1}\colon Y \to X$. Slednji morda sploh ne obstaja! (Praslika seveda obstaja za vse funkcije.)

Če obrat funkcije obstaja, tedaj velja $\pim[1]{f}{\set{y}} = \set[1]{f^{-1}(y)}$ za vsak $y \in Y$ (premisli!), kar nekoliko pojasni oznako $f^{-1}$ tudi za prasliko. Kljub vsemu, z namenom izogibanja dvoumnostim se bomo v tej knjigi skrbno držali oznake iz definicije~\ref{definicija:praslika} za prasliko.

Ko smo že pri alternativnih, potencialno zavajajočih oznakah: pri prasliki enojca se tipično izpuščajo zaviti oklepaji, torej se namesto $\pim[1]{f}{\set{y}}$ piše $\pim{f}{y}$ (ali celo $f^{-1}(y)$).
\end{opomba}

\note{primeri, vaje}

\note{lastnosti: ohranjanje unij, presekov, komplementov}


\section{Injektivnost in surjektivnost}\label{razdelek:injektivnost-in-surjektivnost}

\note{Vključno z ekvivalenco z mono- in epimorfizmi.}


\section{Bijektivnost in obratne preslikave}\label{razdelek:bijektivnost-in-obratne-preslikave}

Kot dobro veste že iz srednje šole, nam injektivnost in surjektivnost omogočata definicijo bijektivnosti.

\begin{definicija}
Preslikava je \df{bijektivna}, kadar je injektivna in surjektivna.
\end{definicija}

To pomeni: če imamo bijektivno preslikavo (na kratko kar: \df{bijekcijo}) $f\colon X \to Y$, smo povezali elemente množice $X$ z elementi množice $Y$, in sicer tako, da vsakemu elementu v katerikoli od množic $X$ oz.~$Y$ pripišemo natanko en element druge množice.

\note{slika dveh množic s poparjenimi pikami}

Rečemo, da so elementi množice $X$ v \df{bijektivni korespondenci} (ali po slovensko \df{povratno enolični zvezi}) z elementi množice $Y$. Bijektivnost se na grafih kaže takole: preslikava je bijektivna, kadar vsaka vodoravnica seka njen graf natanko enkrat.

Bijektivne preslikave igrajo pomembno vlogo v matematiki. Oglejmo si tri primere.
\begin{itemize}
\item
Če imamo povratno enolično zvezo med elementi dveh množic, je jasno, da imata isto število elementov. To nam omogoča definicijo \df{kardinalnosti} množic --- glej poglavje~\note{o kardinalnosti}.
\item
Predstavljajmo si, da so elementi neke množice $X$ imena za določene objekte. Na bijektivno preslikavo $f\colon X \to Y$ lahko potem gledamo kot na preimenovanje teh objektov. Seveda preimenovanje ne spremeni narave (ali če hočete natančnejši izraz, matematične strukture) objektov --- z drugimi besedami, $X$ in $Y$ se razlikujeta zgolj po imenih svojih elementov. To nas privede do pojma \df{izomorfizma}. Za več podrobnosti glej poglavje~\note{o strukturiranih množicah}.
\item
Če imamo povratno enolično zvezo med elementi množic $X$ in $Y$, potem ta zveza ne podaja zgolj preslikave v smeri $X \to Y$, pač pa tudi v smeri $Y \to X$, ker za vsak element iz $Y$ obstaja enolično določen element iz $X$, ki se vanj preslika. Z drugimi besedami, bijektivne preslikave imajo \df{obrate}.
\end{itemize}

Povejmo več o obratih preslikav. Začnimo s formalno definicijo.

\begin{definicija}
Naj bo $f\colon X \to Y$ poljubna preslikava. Za preslikavo $g\colon Y \to X$ rečemo, da je \df{obrat} ali \df{inverz} preslikave $f$, kadar velja
\[g \circ f = \id[X] \qquad\qquad \text{in} \qquad\qquad f \circ g = \id[Y].\]
Z drugimi besedami, $g$ je obrat $f$, kadar slika v nasprotni smeri in za vsak $x \in X$ velja $g\big(f(x)\big) = x$ ter za vsak $y \in Y$ velja $f\big(g(y)\big) = y$. Kadar obrat preslikave $f$ obstaja, rečemo, da je $f$ \df{obrnljiva} (ali \df{invertibilna}) preslikava.
\end{definicija}

\begin{zgled}\label{zgled:logaritmiranje-je-obratno-od-eksponenciranja}
Kot veš že iz srednje šole, logaritmiranje je obratno od eksponenciranja. Če smo natančnejši: preslikavi $\lam{x \in \RR}  b^x$ in $\lam{x \in \RR_{>0}} \log_b x$ sta si obratni pri vsaki osnovi $b \in \RR_{> 0} \setminus \set{1}$.
\end{zgled}

\begin{naloga}\label{naloga:enolicnost-obrata-preslikave}
Dokaži: če sta $g$ in $h$ obrata iste preslikave $f$, tedaj $g = h$.
\end{naloga}

Vaja~\ref{naloga:enolicnost-obrata-preslikave} pove, da je obrat funkcije enolično določen, tj.~vsaka funkcija ima kvečjemu en obrat. Zato lahko uvedemo izrecno oznako: obrat preslikave $f$ (kadar obstaja) označimo z $f^{-1}$. Velja torej: kadar je preslikava $f\colon X \to Y$ obrnljiva, določa preslikavo $f^{-1}\colon Y \to X$.

Ta oznaka je nekoliko nerodna --- pomembno se je zavedati, da $f^{-1}(x)$ pomeni obrat preslikave $f$, izvrednoten na $x$, medtem kot $\big(f(x)\big)^{-1}$ pomeni obratna vrednost (v smislu deljenja) izvrednotenja preslikave $f$ na $x$. Za primerjavo, kot omenjeno v zgledu~\ref{zgled:logaritmiranje-je-obratno-od-eksponenciranja}, je obrat eksponenciranja logaritmiranje, medtem ko je obratna vrednost od $b^x$ enaka $(b^x)^{-1} = \frac{1}{b^x} = b^{-x}$.

\begin{naloga}
Premisli: če ima preslikava $f$ obrat $f^{-1}$, tedaj je tudi $f^{-1}$ obrnljiva preslikava in velja $(f^{-1})^{-1} = f$ (torej, obrat obrata je izvorna preslikava).
\end{naloga}

\begin{naloga}
Pogosto rečemo, da sta seštevanje in odštevanje obratni operaciji. Strogo vzeto, ti dve operaciji nista obratni kot preslikavi, saj obe slikata (recimo, da ju gledamo na realnih številih) $\RR \times \RR \to \RR$, tj.~ne slikata v nasprotnih smereh. Ugotovi, v kakšnem smislu točno sta seštevanje in odštevanje obratni, tj.~kateri dve preslikavi sta pravzaprav druga drugi obratni.
\end{naloga}

Zakaj se sploh ukvarjamo z obrati? Pogosto obravnavamo preslikavo, ki izhaja iz nekega konkretnega (na primer fizikalnega) problema, v smislu, da preslikava vzame začetne podatke in nam vrne, kaj se bo na koncu zgodilo. Marsikdaj pa hočemo rešiti obraten problem: želimo določene končne rezultate in se sprašujemo, kakšni morajo biti začetni pogoji, da jih bomo dosegli. V takem primeru pride prav obratna preslikava.

Kot omenjeno, je obrat preslikave enoličen. Ne velja pa, da za poljubne preslikave sploh obstaja. Na primer, naj bo $f$ edina možna preslikava $\set{0, 1} \to \set{\unit}$, torej tista, ki tako $0$ kot $1$ preslika v $\unit$. Nobena preslikava $g\colon \set{\unit} \to \set{0, 1}$ ne more biti obrat preslikave $f$, saj je $g \circ f$ gotovo konstantna in potemtakem ne more biti identiteta na $\set{0, 1}$.

Kdaj torej obstaja obrat preslikave?

\begin{trditev}
Za poljubno preslikavo $f\colon X \to Y$ sta ekvivalentni sledeči trditvi.
\begin{enumerate}
\item
Preslikava $f$ je obrnljiva.
\item
Preslikava $f$ je bijektivna.
\end{enumerate}
\end{trditev}

\begin{proof}
\begin{implproof}{1}{2}
Predpostavljamo, da obstaja obrat $f^{-1}$.

Dokažimo, da je $f$ injektivna. Vzemimo poljubna $x, y \in X$, za katera velja $f(x) = f(y)$. Tedaj $x = f^{-1}\big(f(x)\big) = f^{-1}\big(f(y)\big) = y$.

Dokažimo, da je $f$ surjektivna. Vzemimo poljuben $y \in Y$. Tedaj $y = f\big(f^{-1}(y)\big)$.
\end{implproof}
\begin{implproof}{2}{1}
Če je $f$ bijekcija, za vsak $y \in Y$ velja, da je $\pim[1]{f}{\set{y}}$ enojec (glej \note{ustrezne predhodne trditve v razdelku o injektivnosti in surjektivnosti}). Definirajmo $g\colon Y \to X$ na naslednji način: za vsak $y \in Y$ naj bo $g(y)$ tisti element $x \in X$, za katerega velja $\pim[1]{f}{\set{y}} = \set{x}$. \note{Iz lastnosti praslike sledi, da je $g$ obrat $f$.}
\end{implproof}
\end{proof}

Iz dokaza te trditve vidimo, da bi bilo koristno imeti oznako za ``tisti element'', če želimo podajati tovrstne preslikave s simboli. Naj bo $\phi$ lastnost elementov množice $X$ (torej predikat $\phi\colon X \to \tvs$), ki je resnična za natanko en element. Dogovorimo se, da
\[\that{x \in X} \phi(x)\]
pomeni ``tisti (edini) element množice $X$, ki ima lastnost $\phi$'' (simbolček na začetku je mala grška črka jota). Zdaj lahko izrecno zapišemo: če je $f\colon X \to Y$ bijekcija, tedaj je njen obrat $f^{-1}\colon Y \to X$ dan s predpisom
\[f^{-1}(y) = \that{x \in X} (f(x) = y).\]

\davorin{Andrej, omenjal si, da želiš imeti to oznako. Če sem kaj zgrešil, prosim popravi.}

Zaenkrat smo to joto uporabljali zgolj kot okrajšavo za stavek v običajnem jeziku, ampak če želimo $\iota$-izraze uporabljati v matematičnih dokazih, jim moramo dati natančen matematični pomen. Definirajmo torej joto formalno matematično.

Naj bo $X$ poljubna množica. Na njej imamo enakost; obravnavajmo jo na tem mestu kot lastnost dvojic elementov iz $X$, torej kot predikat $=_X\colon X \times X \to \tvs$ (za vsak par elementov vrnemo resničnostno vrednost izjave, da sta komponenti para enaki). Transponirajmo to preslikavo; dobimo $\transposed{=_X}\colon X \to \tvs^X$. Ta transponiranka je injektivna: če se za $a, b \in X$ preslikavi $\lam{x \in X} (a = x)$ in $\lam{x \in X} (b = x)$ ujemata, se ujemata tudi njuni vrednosti pri $b$. Ker drži $b = b$, potem drži tudi $a = b$.

Če zožimo kodomeno preslikave $\transposed{=_X}$ na njeno sliko, potemtakem dobimo bijekcijo. Naj bo jota njen obrat, torej $\iota \dfeq \big(\rstr{\transposed{=_X}}^{\rn{\transposed{=_X}}}\big)^{-1}$. V tem smislu je zgornja oznaka $\that{x \in X} \phi(x)$ okrajšava za $\iota (\lam{x \in X} \phi(x))$ (kar bi seveda lahko še skrajšali do $\iota(\phi)$, ampak v praksi je to običajno manj zgovorno).


\section{Vaje}


%%% Local Variables:
%%% mode: latex
%%% TeX-master: "ucbenik-lmn"
%%% End:

        \chapter{Relacije}\label{POGLAVJE: Relacije}

        \section{Splošno o relacijah}

                V matematiki pogosto želimo izraziti, da so določeni objekti v nekem odnosu, npr.~eno število je večje od drugega; temu s tujko rečemo \df{relacija}. Kako to formalno izraziti? Ideja je, da relacijo podamo z množico vseh skupin elementov, ki so v relaciji. Na primer, relacijo $\leq$ na naravnih številih podamo kot podmnožico
                \[\set[1]{(a, b) \in \NN \times \NN}{\xsome{n}[\NN]{a + n = b}}.\]
                Torej, število $a$ je v relaciji $\leq$ s številom $b$ takrat, ko par $(a, b)$ pripada tej množici.

                Splošne relacije so lahko med poljubno mnogo elementi iz poljubnih (ne nujno istih) množic. Na primer, relacija komplanarnosti štirih točk v prostoru je podmnožica produkta $\RR^3 \times \RR^3 \times \RR^3 \times \RR^3$, relacija pripadnosti $\in$ med elementi neke množice $X$ in podmnožicami množice $X$ pa je podmnožica produkta $X \times \pst(X)$.

                Splošna definicija relacije je potemtakem naslednja.
                \begin{definicija}
                        \df{Relacija} na družini množic $\mathscr{D}$ je podmnožica produkta $\prod_{X \in \mathscr{D}} X$, skupaj s podatkom, za katero družino $\mathscr{D}$ gre.
                \end{definicija}

                \begin{opomba}\label{OPOMBA: definicija relacij}
                        Kaj mislimo tu z izrazom \qt{skupaj s podatkom}? Določena podmnožica ima mnogo nadmnožic in podatek, med elementi katerih množic opazujemo odnos, je za relacijo prav tako pomemben, saj so od tega odvisne lastnosti relacije. Lastnosti relacij obravnavamo kasneje v razdelku~\ref{RAZDELEK: Lastnosti relacij}, ampak če že zdaj damo primer: $\set{(a, a)}{a \in \NN}$ je refleksivna kot relacija na naravnih številih (tj.~kot podmnožica $\NN \times \NN$), ne pa tudi kot relacija na celih številih (tj.~kot podmnožica $\ZZ \times \ZZ$).

                        Kako \qt{priložiti} podatek o družini? Ena možnost je, da relacijo podamo kot urejeni par $\rel = (R, \mathscr{D})$, kjer $R \subseteq \prod_{X \in \mathscr{D}} X$. Še ena možnost je, da relacijo podamo kot družino preslikav $(\rel \to X)_{X \in \mathscr{D}}$, ki skupaj porodijo inkluzijo $\rel \hookrightarrow \prod_{X \in \mathscr{D}} X$. Ampak načeloma je povsem vseeno, ali vzamemo katero od teh dveh možnosti ali še kaj tretjega. V tej knjigi se ne bomo omejevali na posamičen formalen zapis za relacijo, bo pa seveda v vseh primerih jasno, za katero družino gre.
                \end{opomba}

                \davorin{Verjetno bi bilo smiselno omeniti še možnost podajanja relacije kot predikat $\prod_{X \in \mathscr{D}} X \to \tvs$.}

                V praksi se povečini uporabljajo relacije med dvema elementoma.
                \begin{definicija}
                        \df{Dvomestna} (ali \df{dvojiška} ali \df{binarna}) \df{relacija} $\rel$ med elementi množic $X$ in $Y$ je podmnožica produkta $X \times Y$, skupaj s podatkom o $X$ in $Y$. Za takšno relacijo definiramo:
                        \begin{itemize}
                                \item
                                        množica $X$ je \df{začetna množica} ali \df{domena} relacije $\rel$, kar označimo $\dom(\rel)$,
                                \item
                                        množica $Y$ je \df{ciljna množica} ali \df{kodomena} relacije $\rel$, kar označimo $\cod(\rel)$,
                                \item
                                        \df{definicijsko območje} ali \df{nosilec} relacije $\rel$ je množica $\dd{\rel} \dfeq \set{x \in X}{\xsome{y}[Y]{\rel[x][y]}}$ (torej $\dd{\rel} \subseteq \dom(\rel)$),
                                \item
                                        \df{zaloga vrednosti} ali \df{slika} \note{razpon?} relacije $\rel$ je množica $\rn{\rel} \dfeq \set{y \in Y}{\xsome{x}[X]{\rel[x][y]}}$ (torej $\dd{\rel} \subseteq \cod(\rel)$).
                        \end{itemize}
                \end{definicija}

                Skoraj vse relacije, ki nas zanimajo v tej knjigi, so dvomestne. Zato se dogovorimo, da z izrazom \qt{relacija} vselej mislimo dvomestno relacijo, razen če je izrecno rečeno drugače.

                Če je $\rel \subseteq X \times Y$ relacija, potemtakem lahko zapišemo, da sta $x \in X$ in $y \in Y$ v relaciji $\rel$ takole: $(x, y) \in \rel$. Ampak to vodi do čudnih zapisov v primeru običajnih relacij, npr.~$(2, 3) \in \mathnormal{<}$. To je bolje zapisati $2 < 3$ in posledično se dogovorimo, da v primeru dvojiške relacije raje uporabljamo zapis $\rel[x][y]$.

                Povečini se še dodatno omejimo na relacije z isto domeno in kodomeno.
                \begin{definicija}
                        \df{Dvomestna} (\df{dvojiška}, \df{binarna}) \df{relacija} na množici $X$ je podmnožica produkta $X \times X$, skupaj s podatkom o $X$.
                \end{definicija}

                Takšne relacije lahko lepo ponazorimo z usmerjenimi grafi. Graf relacije $\rel \subseteq X \times X$ je definiran takole: vozlišča grafa so elementi množice $X$ in za vsaka dva elementa $a, b \in X$, za katera velja $\rel[a][b]$, narišemo puščico od $a$ do $b$.

                \GraphInit[vstyle = Normal]
                \tikzset
                {
                        EdgeStyle/.append style = {->, bend left}
                }

                \begin{zgled}\label{ZGLED: graf relacije}
                        Naj bo $X = \set{A, B, C, D, E, F}$ in naj bo
                        \[\rel \dfeq \set{...}\]
                        relacija na $X$. Njen graf izgleda takole.

                        \note{graf relacije $\rel$}
                        %\begin{center}
                                %\begin{tikzpicture}
                                        %\SetGraphUnit{3}
                                        %\Vertex[Math=true, x=0, y=0]{A}
                                        %\Vertex[Math=true, x=3, y=2]{B}
                                        %\Vertex[Math=true, x=2, y=-3]{C}
                                        %\Vertex[Math=true, x=6, y=1]{D}
                                        %\Vertex[Math=true, x=8, y=-1]{E}
                                        %\Vertex[Math=true, x=10, y=2]{F}
                                        %
                                        %\Edge(A)(B)
                                        %\Loop[dist = 5em, dir = EA](B)
                                %\end{tikzpicture}
                        %\end{center}
                        %\davorin{izgled grafa je še treba popraviti}
                \end{zgled}


        \section{Operacije z relacijami}\label{RAZDELEK: Operacije z relacijami}

                Običajno je, da iz že danih matematičnih objektov lahko skonstruiramo nove preko določenih operacij. Z relacijami ni nič drugače; v tem razdelku si bomo ogledali običajne operacije na relacijah.

                Ker so relacije podmnožice, imamo za začetek vse operacije na podmnožicah. Torej, za poljubno družino $(\rel_i)_{i \in I}$ podmnožic produkta $X \times Y$ sta tudi unija $\bigcup_{i \in I} \rel_i$ in presek $\bigcap_{i \in I} \rel_i$ relaciji. Če je $\rel \subseteq X \times Y$ relacija, je njena komplementarna relacija $\complement{\rel} = X \times Y \setminus \rel \ \subseteq \ X \times Y$.

                Posebej imamo \df{prazno relacijo} $\emptyset \subseteq X \times Y$ (nobena dva elementa nista v relaciji) in \df{polno relacijo} $X \times Y \subseteq X \times Y$ (vsaka dva elementa sta v relaciji), ki sta si medsebojno komplementarni.

                Poleg operacij, ki jih relacije podedujejo od podmnožic, imamo še operacije, ki upoštevajo produktno strukturo.

                Če so $X$, $Y$, $Z$ množice in $\rel \subseteq X \times Y$, $\srel \subseteq Y \times Z$ relaciji, tedaj je \df{sklop} (\df{kompozicija}, \df{kompozitum}) \df{relacij} definiran kot
                \[\srel \circ \rel \dfeq \set[1]{(x, z) \in X \times Z}{\some{y}[Y]{\rel[x][y] \land \srel[y][z]}}\]
                (po vzoru preslikav tudi sklop relacij pišemo v obratnem vrstnem redu; glej razdelek~\ref{RAZDELEK: Izpeljava preslikav iz relacij}). Opazimo: domena $\srel \circ \rel$ je domena $\rel$, kodomena $\srel \circ \rel$ je kodomena $\srel$. Sklapljanje je asociativna operacija, torej pri sklopu večih relacij oklepaji niso pomembni.

                \begin{vaja}
                        Dokaži, da je sklapljanje relacij asociativno!
                \end{vaja}

                Večkraten sklop relacije $\rel \subseteq X \times X$ same s sabo označimo
                \[\rel^n \dfeq \underbrace{\rel \circ \rel \circ \ldots \circ \rel}_{\text{$n$ $\rel$-jev}}\]
                za $n \in \NN_{\geq 2}$. Seveda je smiselno definirati, da je $\rel^1$ enak $\rel$ in da je $\rel^0$ relacija enakosti na množici $X$, saj je to enota za sklapljanje relacij na $X$, tj.~$=_X \circ \rel = \rel = \rel \circ =_X$ (premisli, da je to res!).

                \begin{zgled}
                        Naj bo $\rel \subseteq X \times X$ relacija. Tedaj iz grafa relacije zlahka razberemo, kaj je $\rel^n$: elementa $a, b \in X$ sta v relaciji $\rel^n$ natanko tedaj, ko imamo pot dolžine $n$ od $a$ do $b$ (to deluje tudi za $n = 1$ in $n = 0$). Naj primer, če je $\rel$ relacija iz zgleda~\ref{ZGLED: graf relacije}, tedaj graf relacije $\rel^3$ izgleda takole.

                        \note{graf $\rel^3$}
                \end{zgled}

                Za poljubno relacijo $\rel \subseteq X \times Y$ definiramo \df{obratno} (\df{inverzno}) \df{relacijo} kot
                \[\rel^{-1} \dfeq \set{(y, x) \in Y \times X}{\rel[x][y]}\]
                (torej ima obratna relacija glede na izvorno zamenjano domeno in kodomeno). Posledično lahko za poljubno relacijo $\rel \subseteq X \times X$ definiramo njeno potenco s poljubno celo stopnjo: $\rel^{-n} \dfeq (\rel^{-1})^n = (\rel^n)^{-1}$.

                \begin{vaja}
                        Preveri, da velja $(\srel \circ \rel)^{-1} = \rel^{-1} \circ \srel^{-1}$!
                \end{vaja}

                \begin{zgled}
                        Graf relacije, ki je obratna relaciji $\rel \subseteq X \times X$, dobimo tako, da v grafu relacije $\rel$ obrnemo puščice. Na primer, če je $\rel$ relacija iz zgleda~\ref{ZGLED: graf relacije}, tedaj graf relacije $\rel^{-1}$ izgleda takole.

                        \note{graf $\rel^{-1}$}
                \end{zgled}

                \begin{zgled}
                        Naj bo $L$ množica ljudi. Vpeljimo oznake za naslednje relacije na $L$:
                        \begin{itemize}
                                \item
                                        $\texttt{St}$ je relacija \qt{je starš od},
                                \item
                                        $\texttt{Oč}$ je relacija \qt{je oče od},
                                \item
                                        $\texttt{Ma}$ je relacija \qt{je mati od},
                                \item
                                        $\texttt{Si}$ je relacija \qt{je sin od},
                                \item
                                        $\texttt{Hč}$ je relacija \qt{je hči od},
                                \item
                                        $\texttt{Br}$ je relacija \qt{je brat od},
                                \item
                                        $\texttt{Se}$ je relacija \qt{je sestra od}
                        \end{itemize}

                        Na primer: Marko $\texttt{Br}$ Metka pomeni \qt{Marko je brat od Metke.} (oz.~v lepši slovenščini \qt{Marko je Metkin brat.}).

                        Velja med drugim:

                        \begin{tabular}{l}
                                $\texttt{Oč} \cup \texttt{Ma} = \texttt{St}$, \\
                                $\texttt{St} \circ \texttt{St} = \texttt{St}^2 = \text{\qt{je stari starš od}}$, \\
                                $\texttt{St} \circ \texttt{Br} = \text{\qt{je stric od}}$, \\
                                $\texttt{Br} \cup \texttt{Se} = \text{\qt{je sorojenec od}}$, \\
                                $\texttt{St}^{-1} = \text{\qt{je otrok od}}$, \\
                                $\bigcup_{n \in \NN_{\geq 1}} \texttt{St}^n = \text{\qt{je prednik od}}$, \\
                                $\bigcup_{n \in \NN_{\geq 1}} \texttt{St}^{-n} = \text{\qt{je potomec od}}$, \\
                                $\texttt{St} \circ (\texttt{Br} \cup \texttt{Se}) \circ \texttt{Hč} = \text{\qt{je sestrična od}}$.
                        \end{tabular}

                        Sklapljanje relacij ni komutativno; na primer $\texttt{Ma} \circ \texttt{Oč}$ je stari oče po materini strani, $\texttt{Oč} \circ \texttt{Ma}$ pa stara mama po očetovi strani.

                        \davorin{V tem zgledu sicer predpostavljamo, da je vsaka oseba bodisi moškega bodisi ženskega spola, kar ni čisto res. Ima kdo kakšno idejo, kako se temu izogniti (in še vedno imeti lahko razumljiv zgled)?}
                \end{zgled}

                \note{Na smiselnem mestu omenimo še zožitve relacij (tako domene kot kodomene).}


        \section{Lastnosti relacij}\label{RAZDELEK: Lastnosti relacij}

                Vemo, da so na primer racionalna števila uporabnejša od celih, saj lahko v okviru njih dodatno delimo --- z drugimi besedami, racionalna števila imajo več uporabne \emph{strukture} oz.~več uporabnih \emph{lastnosti}. Podobno za relacije obstajajo lastnosti, ki so se skozi prakso izkazale za zelo uporabne. Nekatere izmed njih si bomo ogledali v tem razdelku.

                Vse sledeče lastnosti se nanašajo na dvomestno relacijo z isto domeno in kodomeno.

                \begin{definicija}
                        Naj bo $\rel \subseteq X \times X$ relacija.
                        \begin{itemize}
                                \item
                                        Relacija $\rel$ je \df{povratna} (ali \df{refleksivna}), kadar velja
                                        \[\xall{x}[X]{\rel[x][x]},\]
                                        tj.~vsak element je v relaciji s samim sabo.
                                \item
                                        Relacija $\rel$ je \df{nepovratna} (ali \df{irefleksivna}), kadar velja
                                        \[\xall{x}[X]{\lnot(\rel[x][x])},\]
                                        tj.~noben element ni v relaciji s samim sabo.
                                \item
                                        Relacija $\rel$ je \df{somerna} (ali \df{simetrična}), kadar velja
                                        \[\all{x, y}[X]{\rel[x][y] \implies \rel[y][x]},\]
                                        tj.~če je en element v relaciji z drugim, je tudi drugi s prvim.
                                \item
                                        Relacija $\rel$ je \df{protisomerna} (ali \df{antisimetrična}), kadar velja
                                        \[\all{x, y}[X]{\rel[x][y] \land \rel[y][x] \implies x = y},\]
                                        tj.~dva elementa sta obojestransko v relaciji samo v primeru, če gre za en in isti element.

                                        \davorin{Mogoče pretiravam s slovenskimi imeni\ldots}
                                \item
                                        Relacija $\rel$ je \df{nesomerna} (ali \df{asimetrična}), kadar velja
                                        \[\xall{x, y}[X]{\lnot(\rel[x][y] \land \rel[y][x])},\]
                                        tj.~nobena dva elementa nista obojestransko v relaciji.
                                \item
                                        Relacija $\rel$ je \df{prehodna} (ali \df{tranzitivna}), kadar velja
                                        \[\all{x, y, z}[X]{\rel[x][y] \land \rel[y][z] \implies \rel[x][z]},\]
                                        tj.~če je en element v relaciji z drugim in drugi s tretjim, je tudi prvi v relaciji s tretjim.
                                \item
                                        Relacija $\rel$ je \df{neprehodna} (ali \df{intranzitivna}), kadar velja
                                        \[\xall{x, y, z}[X]{\lnot(\rel[x][y] \land \rel[y][z] \land \rel[x][z])},\]
                                        tj.~če je en element v relaciji z drugim in drugi s tretjim, prvi ne more tudi biti v relaciji s tretjim.
                                \item
                                        Relacija $\rel$ je \df{enolična}, kadar velja
                                        \[\all{x, y, z}[X]{\rel[x][y] \land \rel[x][z] \implies y = z},\]
                                        tj.~vsak element je v relaciji s kvečjemu enim elementom.
                                \item
                                        Relacija $\rel$ je \df{celovita}, kadar velja
                                        \[\xall{x}[X]{\xsome{y}[Y]{\rel[x][y]}},\]
                                        tj.~vsak element je v relaciji z vsaj enim elementom, se pravi $\dd{f} = \dom(f)$.
                                \item
                                        Relacija $\rel$ je \df{sovisna}, kadar velja
                                        \[\all{x, y}[X]{x \neq y \implies \rel[x][y] \lor \rel[y][x]},\]
                                        tj.~za vsaka dva različna elementa velja, da je eden od njiju v relaciji z drugim.
                                \item
                                        Relacija $\rel$ je \df{strogo sovisna}, kadar velja
                                        \[\all{x, y}[X]{\rel[x][y] \lor \rel[y][x]},\]
                                        tj.~za vsaka dva elementa velja, da je eden od njiju v relaciji z drugim.
                        \end{itemize}
                \end{definicija}

                \begin{zgled}
                        Za nekaj običajnih relacij si oglejmo njihove lastnosti.
                        \begin{itemize}
                                \item
                                        Relacija $\leq$ na $\NN$, $\ZZ$, $\QQ$, $\RR$ je refleksivna, antisimetrična, tranzitivna in strogo sovisna.
                                \item
                                        Relacija $<$ na $\NN$, $\ZZ$, $\QQ$, $\RR$ je irefleksivna, asimetrična, tranzitivna in sovisna.
                                \item
                                        Relacija deljivosti $|$ na $\NN_{\geq 1}$ je refleksivna, antisimetrična in tranzitivna.
                                \item
                                        Relacija $\subseteq$ na $\pst(X)$ je refleksivna, antisimetrična in tranzitivna.
                                \item
                                        Relacija enakosti $=_X$ na katerikoli množici $X$ je refleksivna, simetrična, antisimetrična, tranzitivna in enolična.
                        \end{itemize}
                \end{zgled}

                Lastnosti operacij smo podali z izjavami, ampak lahko jih na ekvivalenten način podamo z operacijami ali lastnostmi grafa --- glej tabelo~\ref{TABELA: Lastnosti relacije}.

                \davorin{Ko \LaTeX\ hoče biti neumen, zna biti precej neumen. Tabelo~\ref{TABELA: Lastnosti relacije} vrže na konec celotnega poglavja, čeprav mu je zapovedano, da jo naj da \emph{prav sem}.}

                \begin{table}[!ht]
                        \centering
                        \newcommand{\opis}[1]{\begin{minipage}{0.45\textwidth}\begin{center}{#1}\end{center}\end{minipage}}
                        \def\arraystretch{3}
                        \begin{tabular}{|ccc|}
                                \hline
                                \textbf{Lastnost relacije} & \textbf{Izražava z operacijami} & \textbf{Lastnost grafa} \\
                                \hline
                                refleksivnost & $=_X \subseteq \rel$ & \opis{Vsako vozlišče ima zanko.} \\
                                irefleksivnost & $=_X \cap \rel = \emptyset$ & \opis{Nobeno vozlišče nima zanke.} \\
                                simetričnost & $\rel = \rel^{-1}$ & \opis{Vsaka puščica ima nasprotno puščico.} \\
                                antisimetričnost & $\rel \cap \rel^{-1} \subseteq =_X$ & \opis{Edine puščice z nasprotnimi puščicami so zanke.} \\
                                asimetričnost & $\rel \cap \rel^{-1} = \emptyset$ & \opis{Nobena puščica nima nasprotne puščice.} \\
                                tranzitivnost & $\rel^2 \subseteq \rel$ & \opis{Za vsako pot pozitivne dolžine obstaja puščica, ki gre od začetka do konca poti.} \\
                                intranzitivnost & $\rel^2 \cap \rel = \emptyset$ & \opis{Za nobeno pot pozitivne dolžine ne obstaja puščica, ki gre od začetka do konca poti.} \\
                                enoličnost & $\rel \circ \rel^{-1} \subseteq =_X$ & \opis{Iz vsakega vozlišča gre kvečjemu ena puščica.} \\
                                celovitost & $=_X \subseteq \rel^{-1} \circ \rel$ & \opis{Iz vsakega vozlišča gre vsaj ena puščica.} \\
                                sovisnost & $=_X \cup \rel \cup \rel^{-1} = X$ & \opis{Vsaki dve različni vozlišči sta povezani s puščico.} \\
                                stroga sovisnost & $\rel \cup \rel^{-1} = X$ & \opis{Vsaki dve vozlišči sta povezani s puščico.} \\
                                \hline
                        \end{tabular}
                        \caption{Lastnosti relacije $\rel \subseteq X \times X$ in njihove karakterizacije}\label{TABELA: Lastnosti relacije}
                \end{table}

                \begin{vaja}
                        Dokaži, da so vse karakterizacije v vsaki vrstici tabele~\ref{TABELA: Lastnosti relacije} res ekvivalentne!
                \end{vaja}

                Marsikdaj imamo sledeči problem: za določene pare elementov $(x_i, y_i)_{i \in I}$ hočemo, da so v neki relaciji in relacija mora zadoščati predpisani lastnosti. Kako definirati takšno relacijo? Smiselna izbira je vzeti najmanjšo relacijo s predpisano lastnostjo, ki vsebuje vse $(x_i, y_i)$. V ta namen definiramo pojem ogrinjače relacij.

                \begin{definicija}
                        Naj bo $\rel \subseteq X \times X$ relacija in $\mathscr{L}$ lastnost relacij na $X$. Najmanjša relacija na $X$, ki vsebuje $\rel$ in ima lastnost $\mathscr{L}$, se imenuje \df{$\mathscr{L}$-ogrinjača} ali \df{$\mathscr{L}$-ovojnica} relacije $\rel$.
                \end{definicija}

                Ogrinjača relacije je dobro definirana (v smislu, da je enolično določena): če imamo dve relaciji $\rel$ in $\srel$, ki obe vsebujeta dano relacijo in imata lastnost $\mathscr{L}$ ter sta najmanjši taki, mora potem veljati, da sta vsebovani ena v drugi, tj.~$\rel \subseteq \srel$ in $\srel \subseteq \rel$, kar pomeni, da sta enaki.

                Ni pa nujno, da ogrinjača dane relacije za dano lastnost sploh obstaja --- na primer, irefleksivna ogrinjača ne obstaja za nobeno relacijo, ki ni že sama po sebi irefleksivna (premisli, zakaj). Seveda, če relacija je irefleksivna, tedaj je svoja lastna irefleksivna ogrinjača. To očitno velja v splošnem: če ima relacija lastnost $\mathscr{L}$, je enaka svoji $\mathscr{L}$-ogrinjači.

                Premislimo, kdaj smo lahko gotovi, da ogrinjača obstaja.

                \begin{definicija}
                        Naj bo $X$ množica in $\mathscr{L}$ lastnost relacij na $X$. Rečemo, da je $\mathscr{L}$ \df{presečno dedna}, kadar velja: poljuben presek relacij na $X$ z lastnostjo $\mathscr{L}$ prav tako ima lastnost $\mathscr{L}$.
                \end{definicija}

                \begin{vaja}\label{VAJA: presečna dednost zaprta za konjunkcije}
                        Dokaži: konjunkcija končno mnogo presečno dednih lastnosti relacij na dani množici je presečno dedna.
                \end{vaja}

                \begin{trditev}\label{TRDITEV: obstoj ogrinjače iz presečne dednosti}
                        Če je $\mathscr{L}$ presečno dedna lastnost relacij na $X$, tedaj za vsako relacijo $\rel$ na $X$ obstaja njena $\mathscr{L}$-ogrinjača, in sicer je enaka preseku vseh relacij na $X$, ki vsebujejo $\rel$ in imajo lastnost $\mathscr{L}$.
                \end{trditev}

                \begin{dokaz}
                        Naj bo $\srel$ presek vseh relacij na $X$, ki vsebujejo $\rel$ in imajo lastnost $\mathscr{L}$. Posledično je $\srel$ vsebovana v vseh relacijah na $X$ z lastnostjo $\mathscr{L}$, ki vsebujejo $\rel$. Ker je $\mathscr{L}$ presečno dedna lastnost, jo ima tudi $\srel$.
                \end{dokaz}

                Kako pa vemo, kdaj je lastnost presečno dedna? Včasih lahko to razberemo kar iz oblike logične formule, s katero je lastnost podana.

                \begin{izrek}\label{IZREK: presečna dednost iz logične oblike}
                        Naj bo $\mathscr{L}$ lastnost relacij na množici $X$, ki jo lahko za poljubno relacijo $\rel$ podamo z zapisom oblike
                        \[\all[1]{x_1, x_2, \ldots, x_n}[X]{\phi(\rel, x_1, x_2, \ldots, x_n) \implies \psi(\rel, x_1, x_2, \ldots, x_n)},\]
                        kjer sta $\phi(\rel, x_1, x_2, \ldots, x_n)$ in $\psi(\rel, x_1, x_2, \ldots, x_n)$ konjunkciji končno mnogo členov oblike $\rel[x_i][x_j]$ --- v posebnem primeru je lahko $\phi(\rel, x_1, x_2, \ldots, x_n)$ konjunkcija nič členov in potem je $\mathscr{L}$ podana z zapisom oblike
                        \[\xall{x_1, x_2, \ldots, x_n}[X]{\psi(\rel, x_1, x_2, \ldots, x_n)}.\]
                        Tedaj je $\mathscr{L}$ presečno dedna lastnost in torej ima vsaka relacija na $X$ $\mathscr{L}$-ogrinjačo.
                \end{izrek}

                \begin{dokaz}
                        Naj bo $(\rel_i)_{i \in I}$ poljubna družina relacij na $X$ z lastnostjo $\mathscr{L}$ in naj bo $\rel \dfeq \bigcap_{i \in I} \rel_I$ njen presek. Dokazujemo, da $\mathscr{L}$ velja za $\rel$.

                        Vzemimo poljubne $x_1, x_2, \ldots, x_n \in X$, za katere velja $\phi(\rel, x_1, x_2, \ldots, x_n)$. Ker je $\phi(\rel, x_1, x_2, \ldots, x_n)$ konjunkcija členov oblike $\rel[x_i][x_j]$, velja tudi $\phi(\rel_i, x_1, x_2, \ldots, x_n)$ za vsak $i \in I$. Po predpostavki torej velja $\psi(\rel_i, x_1, x_2, \ldots, x_n)$ za vsak $i \in I$.

                        Vzemimo poljuben člen $\rel[x_a][x_b]$ iz $\psi(\rel, x_1, x_2, \ldots, x_n)$. Videli smo, da velja $x_a \mathrel{\rel_i} x_b$ za vsak $i \in I$, torej velja $\rel[x_a][x_b]$.

                        Vidimo, da pod našimi predpostavkami velja $\psi(\rel, x_1, x_2, \ldots, x_n)$. Sklenemo, da velja lastnost $\mathscr{L}$ za relacijo $\rel$.
                \end{dokaz}

                \begin{posledica}\label{POSLEDICA: obstoj ogrinjač}
                        Za naslednje lastnosti relacij (in njihovo poljubno konjunkcijo) vselej obstaja ogrinjača: refleksivnost, simetričnost, tranzitivnost.
                \end{posledica}

                \begin{dokaz}
                        Vse izmed naštetih lastnosti se po definiciji dajo zapisati v obliki iz izreka~\ref{IZREK: presečna dednost iz logične oblike}. Za njihovo konjunkcijo glej še vajo~\ref{VAJA: presečna dednost zaprta za konjunkcije} in trditev~\ref{TRDITEV: obstoj ogrinjače iz presečne dednosti}.
                \end{dokaz}

                \begin{vaja}
                        Dokaži, da za poljubno relacijo $\rel$ na množici $X$ velja spodnja tabela!
                        \begin{center}
                                \begin{tabular}{|c|c|}
                                        \hline
                                        \textbf{Lastnost} & \textbf{Ogrinjača relacije $\rel$} \\
                                        \hline
                                        refleksivnost & $\rel \cup =_X$ \\
                                        simetričnost & $\rel \cup \rel^{-1}$ \\
                                        tranzitivnost & $\bigcup_{n \in \NN_{\geq 1}} \rel^n$ \\
                                        \hline
                                \end{tabular}
                        \end{center}
                \end{vaja}

                \note{ena izmed nalog: Za relacijo $\rel[n][(n+1)]$ na $\NN$ (ali $\ZZ$) preveri, da je njena tranzitivna ogrinjača $<$.}


        \section{Izpeljava preslikav iz relacij}\label{RAZDELEK: Izpeljava preslikav iz relacij}

                Ko definiramo temeljne matematične pojme, imamo določeno mero izbire, kaj vzamemo za izvoren pojem, kaj pa definiramo preko drugih pojmov. V tej knjigi smo od začetka vzeli preslikave za bolj osnoven pojem in relacije lahko definiramo s pomočjo preslikav (kot omenjeno v opombi~\ref{OPOMBA: definicija relacij}, relacijo lahko definiramo kot družino preslikav), lahko pa postopamo tudi obratno --- pojem preslikave izpeljemo iz pojma relacije. Kako to gre, si bomo pogledali v tem razdelku.

                \begin{definicija}
                        \df{Delna preslikava} (ali \df{delna funkcija} ali \df{parcialna funkcija}) je enolična dvomestna relacija.
                \end{definicija}

                Kot dvomestna relacija ima vsaka delna preslikava določeno domeno, kodomeno, definicijsko območje in zalogo vrednosti. Če je $f$ delna preslikava z domeno $X$ in kodomeno $Y$, to zapišemo kot $f\colon X \parto Y$.

                V primeru delne preslikave podmnožico produkta domene in kodomene, ki določa relacijo, označimo z $\graph{f}$ in imenujemo \df{graf} delne preslikave $f$ (ne zamešaj tega s prej definiranim pojmom grafa relacije --- prejšnji pojem je pomenil graf v smislu teorije grafov, sedanji pojem pa graf v smislu preslikav). Delna preslikava je torej v celoti podana z informacijo o domeni, kodomeni in grafu.

                Ideja je, da za delno preslikavo $f\colon X \parto Y$ za vsak $x \in \dd{f}$ obstaja natanko en $y \in Y$, s katerim je $x$ v relaciji. To potem zapišemo $f(x) = y$. Torej, če je $x$ v definicijskem območju, rečemo, da je $f(x)$ definiran, kar zapišemo $\isdefined{f(x)}$, in v tem primeru je $f(x)$ enak vrednosti, s katero je $x$ v relaciji. V nasprotnem primeru rečemo, da $f(x)$ ni definiran.

                Če imamo dve vrednosti, ki morda nista definirani, ni posebej smiselno pisati enakosti med njima. Smiselna relacija med njima je \df{Kleenejeva enakost}, kar pišemo $f(x) \kleq g(y)$, kar pomeni naslednje: leva stran $f(x)$ je definirana natanko tedaj, ko je definirana desna stran $g(y)$, in če sta obe definirani, sta enaki.

                \begin{zgled}
                        Deljenje na realnih številih lahko obravnavamo kot delno preslikavo $/\colon \RR \times \RR \parto \RR$; njeno definicijsko območje je $\dd{/} = \RR \times \RR_{\neq 0}$. Za vsak $x \in \RR$ velja $\frac{x}{x^2} \kleq \frac{1}{x}$, ne pa tudi $\frac{x^2}{x} \kleq x$ (premisli, zakaj).
                \end{zgled}

                \begin{zgled}
                        Delne preslikave so zelo uporabne v računalništvu. Za algoritme pričakujemo, da jim podamo vhodne podatke in bodo potem vrnili željene izhodne podatke. Zgodi se pa lahko, da se algoritem pri nekaterih vhodnih podatkih nikoli ne ustavi (ali javi napako), se pravi, ne dobimo rezultata. Če je $P$ množica možnih podatkov, lahko poljuben algoritem obravnavamo kot delno preslikavo $P \parto P$.\footnote{Natančneje, to velja za deterministične algoritme (takšne, ki se pri enakih vhodnih podatkih vedno enako obnašajo). V primeru nedeterminističnih algoritmov dobimo splošno relacijo na $P$.}

                        Izkaže se, da za nekatere probleme ne obstaja računski postopek, ki bi pri vseh možnih vnosih vrnil pravilen odgovor. Primer tega je \df{problem zaustavitve}: želimo algoritem, ki kot vhodna podatka sprejme poljuben algoritem in poljuben vnos ter se odloči, ali se dani algoritem pri danem vnosu ustavi. Kakršenkoli program, ki sprejme takšna podatka in nikoli ne vrne napačnega rezultata, nujno določa delno preslikavo, ki ni povsod definirana. \davorin{Verjetno bomo nekje hoteli imeti razdelek o diagonalizaciji; morda lahko tja dodamo dokaz te trditve.}
                \end{zgled}

                \begin{definicija}
                        \df{Preslikava} (ali \df{funkcija}) je celovita (z drugimi besedami, povsod definirana) delna preslikava. Če je domena preslikave $f$ množica $X$ in kodomena množica $Y$, to zapišemo kot $f\colon X \to Y$.
                \end{definicija}

                Seveda lahko vsako delno preslikavo zožimo do preslikave: delna preslikava $f\colon X \parto Y$ porodi preslikavo $\rstr{f}_{\dd{f}}\colon \dd{f} \to Y$.

                \begin{vaja}
                        Operacijo sklapljanja $\circ$ smo definirali za splošne relacije (razdelek~\ref{RAZDELEK: Operacije z relacijami}). Preveri, da se ta definicija ujema z običajno definicijo sklapljanja preslikav. Premisli še, kaj je sklop delnih preslikav.
                \end{vaja}


        \section{Relacije urejenosti}\label{RAZDELEK: Relacije urejenosti}

                Že od začetka tega poglavja kot klasične primere relacij podajamo razne urejenosti, kot $\leq$ in $<$. V tem razdelku si bomo ogledali, kakšne lastnosti morajo imeti relacije, da na določen način \qt{urejajo} množico.

                Sledeča definicija povzame štiri tipične primere relacij urejenosti.

                \begin{definicija}
                        Naj bo $X$ množica in $\preceq$ relacija na $X$. Tedaj:
                        \begin{itemize}
                                \item
                                        relacija $\preceq$ je \df{šibka urejenost}, kadar je refleksivna in tranzitivna,
                                \item
                                        relacija $\preceq$ je \df{delna urejenost}, kadar je antisimetrična šibka urejenost (tj.~refleksivna, tranzitivna, antisimetrična),
                                \item
                                        relacija $\preceq$ je \df{linearna urejenost}, kadar je strogo sovisna delna urejenost (tj.~refleksivna, tranzitivna, antisimetrična, strogo sovisna),
                                \item
                                        relacija $\preceq$ je \df{stroga linearna urejenost}, kadar je irefleksivna, tranzitivna in sovisna.
                        \end{itemize}
                        \davorin{Poimenovanja v zvezi s sovisnostjo in strogostjo sem povzel po trenutnih predavanjih iz Logike in množic, ampak mislim, da bi se strogost lahko naredila bolj konsistentna.}
                \end{definicija}

                V tej definiciji smo uporabili znak $\preceq$ za relacijo. Pogosto uporabimo kakšen takšen znak, če hočemo sugerirati, da gre za relacijo urejenosti.

                Tipična primera linearne oz.~stroge linearne urejenosti sta relaciji $\leq$ in $<$ na številskih množicah $\NN$, $\ZZ$, $\QQ$, $\RR$. Tipičen primer delne urejenosti je relacija inkluzije $\subseteq$ na katerikoli potenčni množici $\pst(X)$ (če ima $X$ vsaj dva elementa, ta relacija ne bo linearna).

                Primere šibkih urejenosti pogosto dobimo na sledeči način. Naj bo $f\colon X \to Y$ preslikava in $\preceq_Y$ neka relacija urejenosti na $Y$. Za poljubna $a, b \in X$ definirajmo
                \[a \preceq_X b \dfeq f(a) \preceq_Y f(b).\]
                Tudi če je $\preceq_Y$ močnejše vrste relacija --- delna ali linearna urejenost --- je relacija $\preceq_X$ v splošnem zgolj šibka urejenost na $X$.

                \note{še več primerov}

                \note{razlaga imen relacij}

                \note{najmanjši/največji, minimalni/maksimalni elementi, natančne meje}


        \section{Ekvivalenčne relacije in kvocientne množice}

                Ena temeljnih matematičnih dejavnosti je \df{abstrakcija} \davorin{pojmovanje?}, tj.~iz posamičnih primerov izluščimo njihovo temeljno, bistveno lastnost in potem delamo s to lastnostjo. \davorin{To je pomembna stvar. Dajmo to razlago čimbolj izboljšati.} Na primer, vemo, kaj pomeni \qt{pet rac}, \qt{pet avtov}, \qt{pet sekund}, ampak kaj pomeni \qt{pet}?

                V tem razdelku si bomo ogledali, kako lahko formalno abstrahiramo pojme s posamičnih primerov s pomočjo ekvivalenčnih relacij.

                \begin{definicija}
                        \df{Ekvivalenčna relacija} je relacija, ki je refleksivna, simetrična in tranzitivna.
                \end{definicija}

                Ekvivalenčne relacije tipično označimo z $\equ$ (obstaja več načinov, kako to preberemo: vijuga, tilda, kača\ldots) ali čim podobnih.

                \begin{zgled}
                        Vsaka množica $X$ ima najmanjšo ekvivalenčno relacijo --- enakost $=_X$ --- in največjo --- polno relacijo $X \times X$.
                \end{zgled}

                \begin{zgled}
                        Za poljubni celi števili $a, b \in \ZZ$ definiramo: $a$ je v relaciji z $b$, kadar sta $a$ in $b$ iste parnosti. To določa ekvivalenčno relacijo na $\ZZ$.
                \end{zgled}

                Za poljubno relacijo $\rel \subseteq X \times X$ in poljuben $a \in X$ lahko definiramo
                \[\ec[\rel]{a} \dfeq \set{x \in X}{\rel[a][x]}.\]
                Torej, $\ec[\rel]{a}$ sestoji iz vseh elementov, s katerimi je $a$ v relaciji. V primeru, da imamo ekvivalenčno relacijo $\equ$, imenujemo množico $\ec[\equ]{a}$ \df{ekvivalenčni razred} elementa $a$. Kadar je jasno, za katero ekvivalenčno relacijo gre, pogosto ekvivalenčne razrede krajše označujemo kar z $\ec{a}$.

                Bistvo ekvivalenčne relacije je, da ekvivalenčni razredi tvorijo razbitje množice.

                \davorin{Razbitje množice bomo verjetno definirali že prej, najbrž pri vsotah množic. Če ne, potem na tem mestu pride še definicija razbitja.}

                \davorin{Marko raje uporablja izraz `razdelitev množice', ker se mu `razbitje množice' zdi preveč \qt{nasilno}. ;) Kakšna so mnenja drugih? Kateri izraz bi uporabljali?}

                \begin{izrek}[ekvivalenčne relacije natanko ustrezajo razbitjem]
                        Naj bo $X$ poljubna množica.
                        \begin{enumerate}
                                \item
                                        Naslednji trditvi sta ekvivalentni za vsako relacijo $\rel$ na $X$.
                                        \begin{enumerate}
                                                \item
                                                        $\rel$ je ekvivalenčna relacija.
                                                \item
                                                        $\set[1]{\ec[\rel]{a}}{a \in X}$ je razbitje množice $X$.
                                        \end{enumerate}
                                \item
                                        Za vsako razbitje množice $X$ obstaja enolično določena ekvivalenčna relacija $\equ$ na $X$, tako da je razbitje enako $\set[1]{\ec[\equ]{a}}{a \in X}$.
                        \end{enumerate}
                \end{izrek}

                \begin{dokaz}
                        \begin{enumerate}
                                \item
                                        \begin{itemize}
                                                \item\proven{$(\text{\textit{a}} \impl \text{\textit{b}})$}
                                                \item\proven{$(\text{\textit{b}} \impl \text{\textit{a}})$}
                                        \end{itemize}
                                \item
                        \end{enumerate}
                        \note{dokončaj dokaz}
                \end{dokaz}

                Če je $\equ$ ekvivalenčna relacija na množici $X$, tedaj množico vseh njenih ekvivalenčnih razredov označimo z
                \[X/_\equ \dfeq \set[1]{\ec{a}}{a \in X}\]
                in imenujemo \df{kvocientna množica} množice $X$ po relaciji $\equ$.

                \note{kvocientna množica kot množica abstrahiranih pojmov}

                \begin{vaja}
                        Iz posledice~\ref{POSLEDICA: obstoj ogrinjač} sklepamo, da za vsako relacijo na katerikoli množici obstaja njena ekvivalenčna ogrinjača. Dokaži: če je $\rel$ relacija na množici $X$, tedaj je njena ekvivalenčna ogrinjača enaka
                        \[\bigcup_{n \in \NN} (\rel \cup \rel^{-1})^n.\]
                \end{vaja}

                \begin{vaja}
                        Naj bo $(X, \preceq)$ šibka urejenost. Za poljubna $a, b \in X$ definiramo
                        \[a \approx b \dfeq a \preceq b \land b \preceq a.\]
                        \begin{enumerate}
                                \item
                                        Preveri, da je $\approx$ ekvivalenčna relacija na množici $X$.
                                \item
                                        Na kvocientni množici $X/_\approx$ definiramo relacijo $\leq$ na sledeči način: za poljubna $a, b \in X$ naj velja
                                        \[\ec{a} \leq \ec{b} \dfeq a \preceq b.\]
                                        Dokaži, da ta predpis podaja dobro definirano relacijo na $X/_\approx$.
                                \item
                                        Dokaži: $(X/_\approx, \leq)$ je delna urejenost.
                        \end{enumerate}
                        To je kanoničen način, kako šibko urejenost okrepimo do delne urejenosti.
                \end{vaja}

                \note{Kakšen zanimiv zgled uporabe te vaje?}

                Ko smo obravnavali bijekcije v razdelku~\ref{RAZDELEK: Bijektivnost in obratne preslikave}, smo omenili, zakaj je uporabno imeti obrate preslikav. Težava je seveda, da imajo samo bijekcije obrate (v smislu, da so tudi obrati preslikave --- kot relacije seveda imajo obrate), medtem ko včasih želimo obrniti tudi druge preslikave.

                Vzemimo na primer eksponentno funkcijo $\xlam{x}{e^x}$. Če jo obravnavamo kot preslikavo $\RR \to \RR$, seveda nima obrata, saj ni surjektivna. Ideja je, da zožimo kodomeno do zaloge vrednosti --- preslikava $\xlam{x}{e^x}\colon \RR \to \RR_{> 0}$ je bijektivna in posledično lahko definiramo njen obrat (naravni logaritem) $\ln\colon \RR_{> 0} \to \RR$.

                To je standarden trik, če preslikava ni surjektivna. Kaj pa, če ni injektivna? Pogosto v tem primeru zožimo še domeno na območje, na katerem je preslikava injektivna. Na primer, preslikavo $\xlam{x}{x^2}\colon \RR \to \RR$ zožimo do bijekcije $\xlam{x}{x^2}\colon \RR_{\geq 0} \to \RR_{\geq 0}$, kjer imamo obrat $\xlam{x}{\sqrt{x}}\colon \RR_{\geq 0} \to \RR_{\geq 0}$.

                Ima pa ta prostop težave. Prvič, v nasprotju z zožanje kodomene pri zožitvi domene izgubimo določeno količino informacije o preslikavi (kam so se preslikale vrednosti, ki so prej bile v domeni, zdaj pa niso več?). Drugič, izbira zožene domene ni kanonična. Preslikavo $\xlam{x}{x^2}$ bi ravno tako lahko zožili na $\RR_{\leq 0} \to \RR_{\geq 0}$ ali na $\QQ_{\geq 0} \cup (\RR \setminus \QQ)_{\leq 0} \to \RR_{\geq 0}$ ali celo do $\emptyset \to \emptyset$ ali še neskončno drugih možnosti, pri katerih dobimo bijekcijo.

                S pomočjo kvocientov lahko rešimo te probleme in najdemo kanoničen način, kako preslikavo popraviti do injektivne (in če dodamo še zožitev kodomene, do surjektivne in torej v celoti do bijektivne). Vemo že, da sta injektivnost in surjektivnost dualni (razdelek~\ref{RAZDELEK: Injektivnost in surjektivnost}). Kaj je dualno zožitvi kodomene? Odgovor: kvocient domene. Namreč, če zožimo množico, je tako, kot da jo zdaj gledamo od precej bliže --- vidimo samo manjše območje okoli sebe. Kvocienti počnejo obratno --- tako je, kot če bi množico pogledali od precej daleč. Ne vidimo več posamičnih potez, pač pa se te združijo v bolj splošne oblike. (Seveda se ta dualnost, tako kot pri injektivnosti in surjektivnosti, da utemeljiti tudi formalno matematično. \davorin{Bomo govorili o zožkih in kozožkih?})

                \begin{izrek}[naravna razčlenitev preslikave]
                        Za vsako preslikavo $f\colon X \to Y$ obstaja (kanonična) razčlenitev
                        \[f = i \circ \tilde{f} \circ q,\]
                        kjer je $q$ surjekcija, $\tilde{f}$ bijekcija in $i$ injekcija. Konkretno, $q\colon X \to X/_\equ$ je naravna kvocientna preslikava $q(x) = \ec{x}$, pri čemer je ekvivalenčna relacija $\equ$ na $X$ definirana kot
                        \[a \equ b \dfeq f(a) = f(b),\]
                        preslikava $i\colon \rn{f} \hookrightarrow Y$ je vključitev zaloge vrednosti v kodomeno, preslikava $\tilde{f}\colon X/_\equ \to \rn{f}$ pa je v celoti določena s pogojem
                        \[\tilde{f}([x]) = f(x)\]
                        (med drugim to pomeni, da sta množici $X/_\equ$ in $\im(f)$ v bijektivni korespondenci). \davorin{To je vir raznih izrekov o izomorfizmih v algebri. A povemo kaj na to temo?}

                        Za ponazoritev, imamo spodnji diagram.

                        \note{diagram s tikz}
                \end{izrek}

                \begin{dokaz}
                        \note{napiši dokaz}
                \end{dokaz}

        \chapter{Strukturirane množice}
                \note{Struktura na množici. Morfizmi, ki to strukturo ohranjajo. Izomorfnost. Definicija strukturirane množice preko njene karakterizacije --- potrebna obstoj in enoličnost (do izomorfizma). Primeri. Posebej primeri struktur urejenosti (izhaja iz razdelka o strukturah urejenosti v poglavju o relacijah) in osnovnih algebrskih struktur (pride prav kasneje pri konstrukciji številskih množic). Urejenostna in algebrska struktura se združita v pojmu mreže. Definicija (polnih) Boolovih mrež/kolobarjev in povezava z logiko. Širša slika strukturiranih množic --- kategorije.}

        \chapter{Številske množice}
                \note{Karakterizacije številskih množic. Dokaz obstoja in enoličnosti (do izomorfizma).}
                \section{Naravna števila}
                \section{Cela števila}
                \section{Racionalna števila}
                \section{Realna števila}
                \section{Kompleksna števila}
                        \davorin{Se ustavimo že pri realnih številih? Gremo še dlje do kvaternionov?}

        \chapter{Indukcija}
                \note{Dobro osnovano urejene in dobro urejene množice. Indukcija na dobro osnovano urejenih množicah. Strukturna indukcija.}

        \chapter{Aksiomatska teorija množic}
                \section{Zermelo-Fraenklovi aksiomi}
                \section{Aksiom izbire}
                \section{Kumulativna hierarhija}

        \chapter{Kardinalna števila}
                \section{Končnost in neskončnost}
                \section{Števnost}
                \section{Kardinalnost množice}

        \chapter{Ordinalna števila}
                \davorin{Mogoče združimo kardinalna in ordinalna števila v eno poglavje?}


\end{document}