\documentclass[11pt,a4paper,twoside]{book}

%%%%%%%%%%%%%%%%%%%%%%%%%%%%%%%%%%%%%%%%%%%%%%%%%%%%%%%%%
%%%  Imported Packages
%%%%%%%%%%%%%%%%%%%%%%%%%%%%%%%%%%%%%%%%%%%%%%%%%%%%%%%%%
\usepackage[slovene]{babel}
\usepackage[utf8]{inputenc}
\usepackage[T1]{fontenc}

\usepackage{ifthen}
\usepackage{amssymb}
\usepackage{amsmath}
\usepackage{amsthm} % Must come after amsmath!
\usepackage{textgreek}
\usepackage{phonetic}
\usepackage{tablefootnote}
\usepackage{xcolor}
\usepackage{tikz}
\usepackage{tkz-graph}
\usepackage{xparse}
\usepackage{mathrsfs}

\usepackage{charter} %% charter font za glavno besedilo

%%%%%%%%%%%%%%%%%%%%%%%%%%%%%%%%%%%%%%%%%%%%%%%%%%%%%%%%%%%%%
%%  Page Style & Margins (A4 page = 210mm x 297mm)

% PAGE GEOMETRY
\usepackage[papersize={210mm,297mm}, % A4
            twoside,
            includehead,
            top=1in, % margina na vrhu strani
            bottom=1in, % margina na dnu strani
            inner=1in, % margina na notranji strani strani
            outer=1in, % margina na zunanji strani strani
            bindingoffset=0pt % dodatna margina na notranji strani
           ]{geometry}

%%%%%%%%%%%%%%%%%%%%%%%%%%%%%%%%%%%%%%%%%%%%%%%%%%%%%%%%%%%%%
% HYPERLINKING AND PDF METADATA
\usepackage[backref=page,
            colorlinks,
            citecolor=linkcolor,
            linkcolor=linkcolor,
            urlcolor=linkcolor,
            unicode,
            pdfauthor={Andrej Bauer, Davorin Lešnik},
            pdftitle={Logika in množice},
            pdfsubject={matematika},
            pdfkeywords={logika,množice,osnove matematike}]{hyperref}
\renewcommand{\backref}[1]{}
\renewcommand{\backrefalt}[4]{%
   \ifcase #1 %
   (Ni citirano.)
   \or
   (Citirano na strani\ #2.)
   \else
   (Citirano na straneh\ #2.)
   \fi}

\definecolor{linkcolor}{rgb}{0,0,0} % Barva hiperpovezav

%%%%%%%%%%%%%%%%%%%%%%%%%%%%%%%%%%%%%%%%%%%%%%%%%%%%%%%%%%%%%
%%%% Header and footers
%%%%%%%%%%%%%%%%%%%%%%%%%%%%%%%%%%%%%%%%%%%%%%%%%%%%%%%%%%%%%
\usepackage{fancyhdr} % To set headers and footers
\pagestyle{fancyplain}
\setlength{\headheight}{15pt}
\renewcommand{\chaptermark}[1]{\markboth{\textsc{Poglavje \thechapter. #1}}{}}
\renewcommand{\sectionmark}[1]{\markright{\textsc{\thesection\ #1}}}

\lhead[\fancyplain{}{{\thepage}}]%
      {\fancyplain{}{\nouppercase{\rightmark}}}
\rhead[\fancyplain{}{\nouppercase{\leftmark}}]%
      {\fancyplain{}{\thepage}}
\cfoot[]{}
\lfoot[]{}
\rfoot[]{}

%%%%%%%%%%%%%%%%%%%%%%%%%%%%%%%%%%%%%%%%%%%%%%%%%%%%%%%%%%%%%
%%%  Theorems etc.
%%%%%%%%%%%%%%%%%%%%%%%%%%%%%%%%%%%%%%%%%%%%%%%%%%%%%%%%%%%%%
{
\theoremstyle{theorem}
\newtheorem{izrek}{Izrek}[section]
\newtheorem{lema}[izrek]{Lema}
\newtheorem{trditev}[izrek]{Trditev}
\newtheorem{posledica}[izrek]{Posledica}
}

{
\theoremstyle{definition}
\newtheorem{definicija}[izrek]{Definicija}
\newtheorem{opomba}[izrek]{Opomba}
\newtheorem{primer}[izrek]{Primer}
\newtheorem{zgled}[izrek]{Zgled}
\newtheorem{naloga}[izrek]{Naloga}
\newtheorem{vaja}{Vaja}[chapter]
}

%%%%%%  Proofs
%%%%%%%%%%%%%%%%%%%%%%%%%%%%%%%%%%%%%%%%%%%%%%%%%%%%%%%%%%%%%
\newenvironment{dokaz}{%
\goodbreak\par\textit{Dokaz.}%
}{%
\nopagebreak%
\hfill{\vrule width 1ex height 1ex depth 0ex}%
\medskip%
\goodbreak%
}

%%%%%%%%%%%%%%%%%%%%%%%%%%%%%%%%%%%%%%%%%%%%%%%%%%%%%%%%%%%%%
%%%%%% Macros
%%%%%%%%%%%%%%%%%%%%%%%%%%%%%%%%%%%%%%%%%%%%%%%%%%%%%%%%%%%%%%%%%%%%%%%%%%%%%%%%%%%%%%%%%%%%%%%%%%%%%%%%%%%%%%%%%%%%%%
%%%  Commands
%%%%%%%%%%%%%%%%%%%%%%%%%%%%%%%%%%%%%%%%%%%%%%%%%%%%%%%%%%%%%%%%%%%%%%%%%%%%%%%%%%%%%%%%%%%%%%%%%%%%%%%%%%%%%%%%%%%%%%


%%%%%%  Auxiliary
%%%%%%%%%%%%%%%%%%%%%%%%%%%%%%%%%%%%%%%%%%%%%%%%%%%%%%%%%%%%%
\newcommand{\sizedescriptor}[2]
{
\ifthenelse{\equal{#1}{0}}{}{
\ifthenelse{\equal{#1}{1}}{\big}{
\ifthenelse{\equal{#1}{2}}{\Big}{
\ifthenelse{\equal{#1}{3}}{\bigg}{
\ifthenelse{\equal{#1}{4}}{\Bigg}{
#2}}}}}
}

\newcommand{\someref}{{\small\textcolor{blue}{[\textbf{ref.}]}}}
\newcommand{\intermission}{\bigskip\medskip}
\newcommand{\qt}[1]{{\quotedblbase}{#1}{‘‘}}  % text in quotation marks
\newcommand{\nls}[1]{\qt{\textit{#1}}}  % sentence in a natural language

\definecolor{andrejcolor}{rgb}{0.7,0,0.7}
\definecolor{davorincolor}{rgb}{0,0.45,1}
\definecolor{markocolor}{rgb}{0.7,0.4,0}
\definecolor{matijacolor}{rgb}{0,0.6,0.4}

\newcommand{\andrej}[1]{{\small\textcolor{andrejcolor}{(#1 \ \mbox{--Andrej})}}}
\newcommand{\davorin}[1]{{\small\textcolor{davorincolor}{(#1 \ \mbox{--Davorin})}}}
\newcommand{\marko}[1]{{\small\textcolor{markocolor}{(#1 \ \mbox{--Marko})}}}
\newcommand{\matija}[1]{{\small\textcolor{matijacolor}{(#1 \ \mbox{--Matija})}}}

\definecolor{notecolor}{rgb}{0.6,0.5,0.7}
\newcommand{\note}[1]{{\small\textcolor{notecolor}{(#1)}}}
\newcommand{\alert}[1]{{\small\textcolor{red}{\textbf{#1}}}}


%%%%%%  Logical Quantifiers, λ- and ι-Expressions
%%%%%%%%%%%%%%%%%%%%%%%%%%%%%%%%%%%%%%%%%%%%%%%%%%%%%%%%%%%%%

%  no parenthesis (add x in front of the name of the command)
\NewDocumentCommand{\xall}
{m O{\empty} m}
{\forall\, {#1} \ifthenelse{\equal{#2}{}}{}{\in{#2}} \,.\, {#3}}
\NewDocumentCommand{\xsome}
{m O{\empty} m}
{\exists\, {#1} \ifthenelse{\equal{#2}{}}{}{\in{#2}} \,.\, {#3}}
\NewDocumentCommand{\xexactlyone}
{m O{\empty} m}
{\exists\;\!!\, {#1} \ifthenelse{\equal{#2}{}}{}{\in{#2}} \,.\, {#3}}
\NewDocumentCommand{\xlam}
{m O{\empty} m O{\empty}}
{\lambda{#1} \ifthenelse{\equal{#2}{}}{}{\in{#2}} \,.\, {#3} \ifthenelse{\equal{#4}{}}{}{\in{#4}}}
\NewDocumentCommand{\xthat}
{m O{\empty} m}
{\iota{#1} \ifthenelse{\equal{#2}{}}{}{\in{#2}} \,.\, {#3}}

%  with parenthesis -- the first optional argument is the size (values 0-4)
\NewDocumentCommand{\all}
{O{auto} m O{\empty} m}
{\xall{#2}[#3]{\sizedescriptor{#1}{\left}( {#4} \sizedescriptor{#1}{\right})}}
\NewDocumentCommand{\some}
{O{auto} m O{\empty} m}
{\xsome{#2}[#3]{\sizedescriptor{#1}{\left}( {#4} \sizedescriptor{#1}{\right})}}
\NewDocumentCommand{\exactlyone}
{O{auto} m O{\empty} m}
{\xexactlyone{#2}[#3]{\sizedescriptor{#1}{\left}( {#4} \sizedescriptor{#1}{\right})}}
\NewDocumentCommand{\lam}
{O{auto} m O{\empty} m O{\empty}}
{\xlam{#2}[#3]{\sizedescriptor{#1}{\left}( {#4} \sizedescriptor{#1}{\right})}[#5]}
\NewDocumentCommand{\that}
{O{auto} m O{\empty} m}
{\xthat{#2}[#3]{\sizedescriptor{#1}{\left}( {#4} \sizedescriptor{#1}{\right})}}


%%%%%%  Logic
%%%%%%%%%%%%%%%%%%%%%%%%%%%%%%%%%%%%%%%%%%%%%%%%%%%%%%%%%%%%%
\newcommand{\tvs}{\Omega}  % set of truth values
\newcommand{\true}{\top}  % truth
\newcommand{\false}{\bot}  % falsehood
\newcommand{\etrue}{\boldsymbol{\top}}  % emphasized truth
\newcommand{\efalse}{\boldsymbol{\bot}}  % emphasized falsehood
\newcommand{\impl}{\Rightarrow}  % implication sign
\newcommand{\revimpl}{\Leftarrow}  % reverse implication sign
\newcommand{\lequ}{\Leftrightarrow}  % equivalence sign
\newcommand{\xor}{\mathbin{\veebar}}  % exclusive disjunction sign
\newcommand{\shf}{\mathbin{\uparrow}}  % Sheffer connective
\newcommand{\luk}{\mathbin{\downarrow}}  % Łukasiewicz connective


%%%%%%  Sets
%%%%%%%%%%%%%%%%%%%%%%%%%%%%%%%%%%%%%%%%%%%%%%%%%%%%%%%%%%%%%
%  \set{1, 2, 3}         ->  {1, 2, 3}
%  \set{a \in X}{a < 1}  ->  {a ∈ X | a < 1}
\NewDocumentCommand{\set}
{O{auto} m G{\empty}}
{\sizedescriptor{#1}{\left}\{ {#2} \ifthenelse{\equal{#3}{}}{}{ \; \sizedescriptor{#1}{\middle}| \; {#3}} \sizedescriptor{#1}{\right}\}}
%\newcommand{\vsubset}{\Mapstochar\cap}
%\newcommand{\finseq}[1]{{#1}^*}
\newcommand{\pst}{\mathcal{P}}
\renewcommand{\complement}[1]{{#1}^C}


%%%%%%  Number Sets, Intervals
%%%%%%%%%%%%%%%%%%%%%%%%%%%%%%%%%%%%%%%%%%%%%%%%%%%%%%%%%%%%%
\newcommand{\NN}{\mathbb{N}}
\newcommand{\ZZ}{\mathbb{Z}}
\newcommand{\QQ}{\mathbb{Q}}
\newcommand{\RR}{\mathbb{R}}
\newcommand{\CC}{\mathbb{C}}
\newcommand{\intoo}[3][\RR]{{#1}_{(#2, #3)}}
\newcommand{\intcc}[3][\RR]{{#1}_{[#2, #3]}}
\newcommand{\intoc}[3][\RR]{{#1}_{(#2, #3]}}
\newcommand{\intco}[3][\RR]{{#1}_{[#2, #3)}}


%%%%%%  Maps and Relations
%%%%%%%%%%%%%%%%%%%%%%%%%%%%%%%%%%%%%%%%%%%%%%%%%%%%%%%%%%%%%
\newcommand{\id}[1][]{\textrm{Id}_{#1}}  % identity map
\newcommand{\argbox}{{\;\!\fbox{\phantom{M}}\;\!}}  % box for a function argument
\newcommand{\rstr}[1]{\left.{#1}\right|}  % map restriction
\newcommand{\im}{\mathrm{im}}  % map image
\newcommand{\parto}{\mathrel{\rightharpoonup}}  % partial mapping sign
\NewDocumentCommand{\rel}
{O{\empty} O{\empty}}
{\ifthenelse{\equal{#1}{}}{\mathscr{R}}{{#1} \mathrel{\mathscr{R}} {#2}}}  % a relation
\NewDocumentCommand{\srel}
{O{\empty} O{\empty}}
{\ifthenelse{\equal{#1}{}}{\mathscr{S}}{{#1} \mathrel{\mathscr{S}} {#2}}}  % a second relation
\newcommand{\dom}{\mathrm{dom}}  % domain
\newcommand{\cod}{\mathrm{cod}}  % codomain
\newcommand{\dd}[1]{D_{#1}}  % domain of definition
\newcommand{\rn}[1]{Z_{#1}}  % range
\newcommand{\graph}[1]{\Gamma_{#1}}  % graph of a (partial) function
\NewDocumentCommand{\img}  % image
{O{\empty} m G{\empty}}
{{#2}_*\ifthenelse{\equal{#3}{}}{}{\!\sizedescriptor{#1}{\left}( {#3} \sizedescriptor{#1}{\right})}}
\NewDocumentCommand{\pim}  % preimage
{O{\empty} m G{\empty}}
{{#2}^*\ifthenelse{\equal{#3}{}}{}{\!\sizedescriptor{#1}{\left}( {#3} \sizedescriptor{#1}{\right})}}
\newcommand{\ec}[2][]{[\:\!{#2}\:\!]_{#1}}  % equivalence class
\newcommand{\transposed}[1]{\widehat{#1}}


%%%%%%  Misc.
%%%%%%%%%%%%%%%%%%%%%%%%%%%%%%%%%%%%%%%%%%%%%%%%%%%%%%%%%%%%%
\newcommand{\divides}[2]{#1 \mid #2} % deljivost

\newcommand{\df}[1]{\emph{\textbf{#1}}}  % defined notion
\newcommand{\oper}{\mathop{\circledast}}  % symbol for a general operation
\newcommand{\ism}{\cong}  % isomorphic
\newcommand{\equ}{\sim}  % equivalent
\newcommand{\dfeq}{\mathrel{\mathop:}=}  % definitional equality
\newcommand{\revdfeq}{=\mathrel{\mathop:}}  % reverse definitional equality
\newcommand{\isdefined}[1]{{#1}\!\downarrow}  % given value is defined
\newcommand{\kleq}{\simeq}  % Kleene equality
\newcommand{\claim}[3]{{#1} \;\colon\; \frac{#2}{#3}}  % claim, divided on context, assumptions, conclusions
\newcommand{\unit}{\mathord{\boldsymbol{*}}}  % element in a generic singleton
\NewDocumentEnvironment{implproof}  % proof of an implication
{O{\empty} G{\empty} O{=>} G{\empty}}
{
\begin{description}
\item[\quad$\sizedescriptor{#1}{\left}({#2}
\ifthenelse{\equal{#3}{=>}}{\impl}{
\ifthenelse{\equal{#3}{<=}}{\revimpl}{
\ifthenelse{\equal{#3}{->}}{\rightarrow}{
\ifthenelse{\equal{#3}{<-}}{\leftarrow}{
#3
}}}} {#4}\sizedescriptor{#1}{\right})$]\ \vspace{0.3em}\\
}
{
\end{description}
}


%%%%%%%%%%%%%%%%%%%%%%%%%%%%%%%%%%%%%%%%%%%%%%%%%%%%%%%%%%%%%%%%%%%%%%%%%%%%%%%%%%%%%%%%%%%%%%%%%%%%%%%%%%%%%%%%%%%%%%

%%% Local Variables:
%%% mode: latex
%%% TeX-master: "ucbenik-lmn"
%%% End:


\begin{document}

%--------------------------------------------------------------------
%--------------------------------------------------------------------
% TITLE PAGE

\title{Logika in množice} % če se spremeni naslov, je treba spremeniti tudi zgoraj v paketu hyperref
\author{Andrej Bauer \and Davorin Lešnik}
\maketitle

%--------------------------------------------------------------------
%--------------------------------------------------------------------
% Foreword

\chapter*{Predgovor}%\addcontentsline{toc}{chapter}{\numberline{}Predgovor}

%--------------------------------------------------------------------
%--------------------------------------------------------------------
% TOC

\tableofcontents
% \listoftables % To se meni zdi nepotrebno, zakaj se to daje v knjige? (Andrej)

%--------------------------------------------------------------------
%--------------------------------------------------------------------
% BODY

\chapter*{Pomembnejši makroji (razlaga uporabe)}

\davorin{To poglavje je namenjeno zgolj za nas pisce, ne pa za bralce. Tu razložim uporabo nekaterih latexovskih makrojev, ki sem jih definiral. Če dodate svoje, katerih uporaba ni očitna, njihovo razlago prosim dodajte sem.}

\newcommand{\ponazoritev}[2]{

\medskip
\begin{tabular}{lll}
\textbf{Koda:} && \texttt{#1} \\[1ex]
\textbf{Prikaz:} && {#2}
\end{tabular}
\bigskip

}

\section*{Nekateri zapisi}

Za podajanje latexovskih ukazov uporabimo \ltc{ltc}. \davorin{Okolje \texttt{verbatim} ne deluje dobro znotraj makrojev, ampak če kdo ve, kako to razrešiti, naj popravi.}
\ponazoritev{\ltc{ltc}\{sqrt$\backslash$\{2$\backslash$\}\}}{\ltc{sqrt\{2\}}}

Narekovaje pišemo tako, kot je to običajno v {\LaTeX}u, saj lahko kasneje določimo, kako
se jih dejansko prikazuje.

Včasih bomo želeli podati stavek v naravnem jeziku (namesto v simbolnem matematičnem).
\ponazoritev{\ltc{nls}\{Stavek v naravnem jeziku.\}}{\nls{Stavek v naravnem jeziku.}}

Za definirani izraz uporabimo \ltc{df}.
\ponazoritev{Funkcija je \ltc{df}\{zvezna\}, kadar\ltc{ldots}}{Funkcija je \df{zvezna}, kadar\ldots}

Za definicijsko enakost uporabimo \ltc{dfeq} oz.~za enakost v nasprotni smeri \ltc{revdfeq}.
\ponazoritev{\textdollar{f(x,y) \ltc{dfeq} x + y}\textdollar}{$f(x,y) \dfeq x + y$}
\ponazoritev{\textdollar{e\^{}2 + \ltc{pi} \ltc{revdfeq} a}\textdollar}{$e^2 + \pi \revdfeq a$}

\section*{Množice}

Za množice uporabimo ukaz \ltc{set}. Podamo lahko enega ali dva argumenta.
\ponazoritev{\textdollar{\ltc{set}\{1,2,3\}}\textdollar}{$\set{1,2,3}$}
\ponazoritev{\textdollar{\ltc{set}\{x \ltc{in} \ltc{RR}\}\{x > 1\}}\textdollar}{$\set{x \in \RR}{x > 1}$}

Zaviti oklepaji se samodejno prilagajajo velikosti besedila.
\ponazoritev{\textdollar{\ltc{set}\{1, \ltc{displaystyle}\{\ltc{frac}\{3\}\{4\}\}\} \ltc{cup} \ltc{set}\{x \ltc{in} \ltc{NN}\}\{x > 2\^{}\{2\^{}\{100\}\}\}}\textdollar}{$\set{1, \displaystyle{\frac{3}{4}}} \cup \set{x \in \NN}{x > 2^{2^{100}}}$}

Če nam privzeta velikost oklepajev ni všeč, jo lahko spremenimo z izbirnim parametrom, ki je število od 0 do 4.
\ponazoritev{\textdollar{\ltc{set}[0]\{0\}, \ltc{set}[1]\{1\}, \ltc{set}[2]\{2\}, \ltc{set}[3]\{3\}, \ltc{set}[4]\{4\}}\textdollar}{$\set[0]{0}, \set[1]{1}, \set[2]{2}, \set[3]{3}, \set[4]{4}$}

Za generični enojec uporabimo ukaz \ltc{one}, za njegov element pa \ltc{unit}.
\ponazoritev{\textdollar{\ltc{one} = \ltc{set}\{\ltc{unit}\}}\textdollar}{$\one = \set{\unit}$}

\section*{Intervali}

\davorin{Glej diskusijo, ki se trenutno nahaja v razdelku~2.1 (ampak to se bo spremenilo).}

Za intervale uporabljamo ukaze \ltc{intoo}, \ltc{intoc}, \ltc{intco}, \ltc{intcc}, kjer \texttt{o} označuje odprtost, \texttt{c} pa zaprtost intervala. Krajišči intervala podamo kot argumenta.
\ponazoritev{\textdollar{\ltc{intoo}\{0\}\{1\}, \ltc{intoc}\{2\}\{3\}, \ltc{intco}\{4\}\{5\}, \ltc{intcc}\{6\}\{7\}}\textdollar}{$\intoo{0}{1}, \intoc{2}{3}, \intco{4}{5}, \intcc{6}{7}$}

Če želimo interval na neki drugi množici kot $\RR$, podamo to množico kot izbirni argument.
\ponazoritev{\textdollar{\ltc{intco}[\ltc{NN}]\{1\}\{5\} = \ltc{set}\{1,2,3,4\}}\textdollar}{$\intco[\NN]{1}{5} = \set{1,2,3,4}$}

\section*{Kvantifikatorji, $\lambda$- in $\iota$-izrazi}

Vsi kvantifikatorji imajo enako obliko, ponazorimo jo z univerzalnim kvantifikatorjem:
%
% Meni se ful ful ne da uporabljati teh makrojev, da bo 2% lepše, to bomo itak izbrisali.
% (Andrej)
%
\begin{itemize}
\item Koda: \verb|\all{x \in A} \Phi|
\item Prikaz: $\all{x \in A} \Phi$
\end{itemize}
%
Če želimo oklepaje okoli $\Phi$, jih enostavno napišemo. Če želimo imeti neomejen kvantifikator, lahko napišemo \verb|\all{x} \Phi| itd.

Ostali kvantifikatorji si:
%
\begin{itemize}
\item eksistenčni: \verb|\some{x \in A} \Phi|, dobimo $\some{x \in A} \Phi$
\item enolični obtoj: \verb|\exactlyone{x \in A} \Phi|, dobimo $\exactlyone{x \in A} \Phi$
\item funkcija: \verb|\lam{x \in A} e|, dobimo $\lam{x \in A} e$
\item opis: \verb|\that{x \in A} \Phi|, dobimo $\that{x \in A} \Phi$
\end{itemize}


\section*{Kanonične projekcije in injekcije}

Nismo še sprejeli odločitve, kako bomo označevali projekcije oz.~injekcije pri dvojiških produktih oz.~vsotah. Tudi ko jo bomo, bomo verjetno šli skozi več iteracij. Imejmo torej makroje zanje, ki jih bomo lahko na koncu poljubno spreminjali.

\davorin{Projekcije in injekcije so naravne preslikave. Tega verjetno ne bomo omenjali študentom, dobro pa bi bilo, da se sami tega zavedamo in izrecno pišemo indekse komponent. Na ta način se izognemo zmedi v situacijah, kjer obravnavamo več kot en (ko)produkt.}

Ukazi za leve oz.~desne projekcije oz.~injekcije so sledeči.
\begin{center}
\begin{tabular}{c|cc}
& leva & desna \\
\hline
projekcija & \ltc{lpr} & \ltc{rpr} \\
injekcija & \ltc{lin} & \ltc{rin}
\end{tabular}
\end{center}

Tem ukazom kot izbirna parametra podamo faktorja oz.~sumanda.
\ponazoritev{\textdollar{X \ltc{stackrel}\{\ltc{lpr}[X][Y]\}\{\ltc{longleftarrow}\} X \ltc{times} Y \ltc{stackrel}\{\ltc{rpr}[X][Y]\}\{\ltc{longrightarrow}\} Y}\textdollar}{$X \stackrel{\lpr[X][Y]}{\longleftarrow} X \times Y \stackrel{\rpr[X][Y]}{\longrightarrow} Y$}
\ponazoritev{\textdollar{X \ltc{stackrel}\{\ltc{lin}[X][Y]\}\{\ltc{longrightarrow}\} X + Y \ltc{stackrel}\{\ltc{rin}[X][Y]\}\{\ltc{longleftarrow}\} Y}\textdollar}{$X \stackrel{\lin[X][Y]}{\longrightarrow} X + Y \stackrel{\rin[X][Y]}{\longleftarrow} Y$}

%%% Local Variables:
%%% mode: latex
%%% TeX-master: "ucbenik-lmn"
%%% End:

\chapter{Matematično izražanje}\label{POGLAVJE: Matematično izražanje}

	\note{Za začetek bom vnašal oporne točke besedila. Slog bo verjetno treba še popraviti in besedilo dopolniti. --Davorin}
	
	\textcolor{red}{\small \textbf{Če je možno, prosim uporabljajte tabulatorje namesto presledkov za zamike v latex kodi in koda naj nima izrecno vnešenih prelomov vrstic, pač pa se v urejevalniku besedila uporablja avtomatski word wrap, ki se prilagaja širini okna. --Davorin}}
	
	Za matematično delo je bistveno, da se lahko zanašamo na pravilnost naših trditev. To pomeni:
	\begin{itemize}
		\item
			matematične izjave morajo imeti \emph{nedvoumen pomen},
		\item
			matematične izjave lahko \emph{dokažemo}.
	\end{itemize}
	
	Stavki v običajnih jezikih nimajo nedvoumnega pomena, zato matematične izjave raje podamo v \emph{matematičnem jeziku}. Za to potrebujemo \qt{matematično abecedo}, tj.~simbolni zapis, v katerem podamo izjave. Tega obravnavamo v naslednjem razdelku, dokazovanje matematičnih izjav pa v razdelku za tem.
	
	\section{Simbolni zapis}\label{RAZDELEK: Simbolni zapis}
	
		Za množice, s katerimi najpogosteje delamo, obstajajo standardne oznake (tabela~\ref{TABELA: standardne številske množice}).
		
		\begin{table}[!ht]
			\centering
			\begin{tabular}{|cc|}
				\hline
				\textbf{Množica} & \textbf{Oznaka} \\
				\hline
				množica naravnih števil & $\NN$ \\
				množica celih števil & $\ZZ$ \\
				množica racionalnih števil & $\QQ$ \\
				množica realnih števil & $\RR$ \\
				množica kompleksnih števil & $\CC$ \\
				\hline
			\end{tabular}
			\caption{standardne številske množice}\label{TABELA: standardne številske množice}
		\end{table}
		
		Nekateri $0$ vzamejo za naravno število, nekateri ne. To je v celoti stvar dogovora, kaj pomeni pojem \qt{naravno število}. Za nas bo prišlo bolj prav, če ničlo štejemo kot element množice naravnih števil, torej $\NN = \set{0, 1, 2, 3, \ldots}$.
		
		Interval realnih števil podamo s krajiščema intervala v oklepajih --- okrogli oklepaji ( ) označujejo odprtost intervala (krajišče ni vključeno v interval), oglati oklepaji [ ] pa zaprtost (krajišče je vključeno). Tako se npr.~interval realnih števil od $0$ do $1$, ki ne vsebuje krajišč, označi z $(0, 1)$, če jih vsebuje, pa z $[0, 1]$.
		
		Včasih pridejo prav tudi intervali na drugih množicah kot $\RR$. Zato se dogovorimo, da bomo intervale označevali tako, da podamo množico, ob kateri v indeksu zapišemo krajišči v oklepajih, npr.~$\intco[\NN]{1}{5} = \set{1, 2, 3, 4}$. Realna intervala iz prejšnjega odstavka tako zapišemo kot $\intoo{0}{1}$ in $\intcc{0}{1}$.
		
		Če interval v katero smer gre v nedogled, preprosto zapišemo množico z ustrezno relacijo urejenosti in krajiščem v indeksu. Na primer, $\RR_{> 0}$ označuje množico pozitivnih realnih števil, $\RR_{\geq 0}$ pa množico nenegativnih realnih števil.
		
		\note{To bi vsaj bil moj predlog. Na ta način se izognemo dvoumnostim (kar je namen). Na primer, kaj pomeni $\forall\, a > 0$? Če zapišemo $\forall\, a \in \NN_{> 0}$ ali $\forall\, a \in \RR_{> 0}$, je jasno. Razlog, da matematiki \qt{goljufajo} in pridejo skozi brez tega, je (napol dogovorjena in ponotranjena, ampak arbitrarna) izbira črk; vsak izkušen matematik ve, da $\forall\, \epsilon > 0$ pomeni $\forall\, \epsilon \in \RR_{> 0}$. Če se ne strinjate, popravite in pustite komentar. --Davorin}
		
		Izjavo, da je $2$ naravno število, zapišemo takole: $2 \in \NN$ (beri: $2$ pripada množici naravnih števil). Kako zapišemo, da je $a$ sodo število? Število je sodo, kadar je deljivo z $2$, torej pišemo $2 | a$ (beri: $2$ deli $a$).
		
		Če imamo več izjav, jih lahko strnemo v sestavljeno izjavo. Na primer, izjavo \nls{Če je $a$ sodo število, je tudi kvadrat števila $a$ sod.}, zapišemo kot $2 | a \implies 2 | a^2$.
		
		Seveda ta izjava velja za vsa naravna števila (znaš to dokazati?). To zapišemo takole: $\all{a}{\NN}{2 | a \implies 2 | a^2}$.
		
		Kot smo navajeni iz običajnih jezikov, posamične stavke povežemo v sestavljeno poved z \emph{vezniki}. Najpogosteje uporabljeni matematični vezniki so v tabeli~\ref{TABELA: standardni izjavni vezniki}.
		
		\begin{table}[!ht]
			\centering
			\begin{tabular}{|ccc|}
				\hline
				\textbf{Izjavni veznik} & \textbf{Oznaka} & \textbf{Kako preberemo} \\
				\hline
				negacija & $\lnot{p}$ & ne $p$ \\
				konjunkcija & $p \land q$ & $p$ in $q$ \\
				disjunkcija & $p \lor q$ & $p$ ali $q$ \\
				implikacija & $p \impl q$ & če $p$, potem $q$ \\
				ekvivalenca & $p \lequ q$ & $p$ natanko tedaj, ko $q$ \\
				\hline
			\end{tabular}
			\caption{standardni izjavni vezniki}\label{TABELA: standardni izjavni vezniki}
		\end{table}
		
		\begin{opomba}
			V matematiki se za izjavne veznike običajno uporabljajo zgoraj navedene tujke, ampak vsaka od njih seveda ima svoj pomen. Dobesedni prevodi teh tujk so:
			\begin{itemize}
				\item
					negacija $\to$ zanikanje,
				\item
					konjunkcija $\to$ vezava,
				\item
					disjunkcija $\to$ ločitev,
				\item
					implikacija $\to$ vpletenost,
				\item
					ekvivalenca $\to$ enakovrednost.
			\end{itemize}
			Za primerjavo: spomnite se vezalnega in ločnega priredja iz slovenščine!
		\end{opomba}
		
		\begin{zgled}
			Naj $p$ označuje stavek \nls{Zunaj dežuje.} in $q$ stavek \nls{Vzamem dežnik.}. Tedaj $\lnot{p}$ pomeni \nls{Zunaj ne dežuje.} in $p \impl q$ pomeni \nls{Če zunaj dežuje, potem vzamem dežnik.}.
		\end{zgled}
		
		Kose sestavljene izjave lahko veže več kot en veznik. V tem primeru se (tako kot pri računanju s števili) dogovorimo o prednosti veznikov. Po dogovoru je vrstni red veznikov tak, kot v tabeli~\ref{TABELA: standardni izjavni vezniki}, tj.~najmočneje veže negacija, nato konjunkcija, nato disjunkcija, nato implikacija, nato ekvivalenca. Kadar želimo, da se najprej izvede veznik z nižjo prednostjo, uporabimo oklepaje.
		
		\begin{zgled}
			Označimo sledeče stavke:
			\begin{quote}
				$p$ \ \ldots\ldots\ \nls{Imam čas.} \\
				$q$ \ \ldots\ldots\ \nls{Ostanem doma.}
			\end{quote}
			Tedaj $\lnot{p} \land q$ pomeni isto kot $(\lnot{p}) \land q$, to je \nls{Nimam časa in ostanem doma.}, medtem ko $\lnot(p \land q)$ pomeni \nls{Ni res, da imam čas in ostanem doma.}.
		\end{zgled}
		\note{Če komu pade na pamet primer boljših stavkov, je zaželjeno, da popravi\ldots --Davorin}
		
		Poleg zgoraj navedenih izjavnih veznikov se včasih uporabljajo še sledeči (tabela~\ref{TABELA: nadaljnji izjavni vezniki}).
		
		\begin{table}[!ht]
			\centering
			\begin{tabular}{|ccc|}
				\hline
				\textbf{Izjavni veznik} & \textbf{Oznaka} & \textbf{Kako preberemo} \\
				\hline
				stroga disjunkcija & $p \xor q$ & bodisi $p$ bodisi $q$ \\
				Shefferjev\tablefootnote{Henry Maurice Sheffer (1882 -- 1964) je bil ameriški logik.} veznik & $p \shf q$ & ne hkrati $p$ in $q$ \\
				Łukasiewiczev\tablefootnote{Jan Łukasiewicz (beri: \hill{u}ukaśj\^{e}vič) (1878 -- 1956) je bil poljski logik in filozof.} veznik & $p \luk q$ & niti $p$ niti $q$ \\
				\hline
			\end{tabular}
			\caption{nekateri nadaljnji izjavni vezniki}\label{TABELA: nadaljnji izjavni vezniki}
		\end{table}
		
		Za strogo disjunkcijo (tudi: ekskluzivna disjunkcija, izključitvena disjunkcija) se uporabljajo še druge oznake: $p \oplus q$, $p + q$. Razlika med navadno in strogo disjunkcijo je sledeča: $p \lor q$ pomeni, da \emph{vsaj eden} od $p$ in $q$ velja, medtem ko $p \xor q$ pomeni, da velja \emph{natanko eden}.
		
		\begin{zgled}
			Stavek \nls{Pisni del predmeta je potrebno opraviti s kolokviji ali pisnim izpitom.} je primer navadne disjunkcije (seveda se vam prizna pisni del predmeta tudi, če uspešno odpišete tako kolokvije kot pisni izpit), stavek \nls{Grem bodisi na morje bodisi v hribe.} pa je primer stroge disjunkcije (ne da se biti na dveh mestih hkrati).
		\end{zgled}
		
		Običajno veznike iz tabele~\ref{TABELA: nadaljnji izjavni vezniki} (in vse preostale, ki jih nismo navedli) izrazimo s standardnimi (glej tabelo~\ref{TABELA: izražava nadaljnjih izjavnih veznikov s standardnimi}), včasih pa je uporabno delati neposredno z njimi. Na primer, stroga disjunkcija služi kot seštevanje v Boolovem kolobarju (glej~\note{razdelek o Boolovih kolobarjih}), Shefferjev in Łukasiewiczev veznik pa se uporabljata pri preklopnih vezjih, saj je z vsakim od njiju možno izraziti vse izjavne veznike (glej~\note{razdelek o polnih naborih}). V računalništvu imajo ti trije vezniki standardne oznake XOR, NAND, NOR.
		
		\begin{table}[!ht]
			\centering
			\begin{tabular}{|ccc|}
				\hline
				\textbf{Izjavni veznik} & \multicolumn{2}{c|}{\textbf{Nekatere izražave s standardnimi vezniki}} \\
				\hline
				$p \xor q$ & $(p \lor q) \land \lnot(p \land q)$ & $(p \land \lnot{q}) \lor (\lnot{p} \land q)$ \\
				$p \shf q$ & $\lnot(p \land q)$ & $\lnot{p} \lor \lnot{q}$ \\
				$p \luk q$ & $\lnot(p \lor q)$ & $\lnot{p} \land \lnot{q}$ \\
				\hline
			\end{tabular}
			\caption{izražava nadaljnjih izjavnih veznikov s standardnimi}\label{TABELA: izražava nadaljnjih izjavnih veznikov s standardnimi}
		\end{table}
		
		\note{Na tem mestu povejmo, kakšno prednost damo tem trem veznikom v primerjavi s standardnimi. Kateremu dogovoru sledimo?}
		
		Včasih so izjave odvisne od kakšnih parametrov. Na primer, naj $\phi(x)$ pomeni \nls{$x$ je zelen.}; tedaj $\phi(\text{trava})$ pomeni \nls{Trava je zelena.}. Takim odvisnim izjavam rečemo \df{predikati} in izražajo lastnosti, ki jim parametri (\qt{spremenljivke}) lahko zadoščajo.
		
		Predikate lahko \emph{kvantificiramo} po njihovih spremenljivkah, tj.~povemo, \qt{kako pogosto} velja lastnost, dana s predikatom. Tabela~\ref{TABELA: kvantifikatorji} podaja najpogosteje uporabljane kvantifikatorje in njihove oznake.
		
		\begin{table}[!ht]
			\centering
			\begin{tabular}{|ccc|}
				\hline
				\textbf{Kvantifikator} & \textbf{Oznaka} & \textbf{Kako preberemo} \\
				\hline
				univerzalni kvantifikator & $\xall{x}{X}{\phi(x)}$ & za vsak $x$ iz $X$ velja lastnost $\phi$ \\
				eksistenčni kvantifikator & $\xsome{x}{X}{\phi(x)}$ & obstaja $x$ iz $X$ z lastnostjo $\phi$ \\
				\note{kako se temu reče?} & $\xexactlyone{x}{X}{\phi(x)}$ & obstaja natanko en $x$ iz $X$ z lastnostjo $\phi$ \\
				\hline
			\end{tabular}
			\caption{kvantifikatorji}\label{TABELA: kvantifikatorji}
		\end{table}
		
		Oznaki $\forall$ in $\exists$ sta narobe obrnjena A in E in izhajata iz nemščine (\textbf{a}ll, \textbf{e}xistiert).
		
		\begin{zgled}
			Vemo, da za vsako nenegativno realno število obstaja enolično določen nenegativen kvadratni koren; to izjavo lahko zapišemo na sledeči način.
			\[\xall{a}{\RR_{\geq 0}}\xexactlyone{b}{\RR_{\geq 0}}{b^2 = a}\]
			Zaradi tega lahko definiramo kvadratni koren kot funkcijo $\sqrt{\phantom{I}}\colon \RR_{\geq 0} \to \RR_{\geq 0}$.
		\end{zgled}
		
		Po dogovoru kvantifikatorji vežejo šibkeje kot izjavni vezniki. Izjavo, da je vsako celo število bodisi liho bodisi sodo, torej zapišemo takole.
		\[\all[2]{a}{\ZZ}{2 | a \xor 2 | (a-1)}\]
		
		\note{Se že na tem mestu predebatirajo vezane oz.~nevezane spremenljivke ter preimenovanje spremenljivk? Kaj je slovenski prevod za `dummy variable'?}
		
		\begin{zgled}
			Za poljubno naravno število $n \in \NN$ naj $P(n)$ označuje izjavo, da je $n$ praštevilo. Torej, $P$ definiramo takole.
			\[P(n) \dfeq \all[1]{x}{\NN}{x | n \implies x = 1 \xor x = n}\]
			(Premisli, kaj bi se zgodilo, če bi namesto stroge disjunkcije vzeli navadno. Bi še vedno dobili pravilni pojem praštevila?)
			
			Naj $S(n)$ označuje, da je $n$ sestavljeno število.
			\[S(n) \dfeq \xsome{x, y}{\intoo[\NN]{1}{n}}{x \cdot y = n}\]
			(Kadar imamo več zaporednih kvantifikatorjev iste vrste, jih po dogovoru lahko strnemo kot zgoraj. Dana formula za $S(n)$ je krajši zapis za $\xsome{x}{\intoo[\NN]{1}{n}}\xsome{y}{\intoo[\NN]{1}{n}}{x \cdot y = n}$.)
			
			Zdaj lahko na pregleden način zapišemo, da je vsako naravno število od $2$ naprej bodisi praštevilo bodisi sestavljeno.
			\[\all[1]{n}{\NN_{\geq 2}}{P(n) \xor S(n)}\]
		\end{zgled}
		
		\note{
			Ideje za VAJE:\\
				\hbox{}\qquad\qquad * Napiši te in te z besedami podane matematične izjave simbolno.\\
				\hbox{}\qquad\qquad * Za te in te \qt{življenjske} izjave vsak osnoven sestavni kos označi s črko in zapiši sestavljeno izjavo z mešanico veznikov in kvantifikatorjev.\\
				\hbox{}\qquad\qquad * ...
		}
	
	
	\section{Pravila dokazovanja}\label{RAZDELEK: Pravila dokazovanja}
	
		Matematične izsledke običajno podajamo preko jasno izraženih izjav. Med študijem matematike hitro opazite, da se takšne izjave podajajo pod imeni \quotesinglbase{izrek}', \quotesinglbase{trditev}', \quotesinglbase{lema}', \quotesinglbase{posledica}' in podobno. Kdaj uporabiti katerega teh imen ni natanko določeno, pač pa je prepuščeno presoji matematika. Približno vodilo je naslednje:
		\begin{itemize}
			\item
				\df{izrek}: osrednji, bistven rezultat,
			\item
				\df{trditev}: stranski rezultat,
			\item
				\df{lema}: rezultat, ki sam po sebi nima toliko vsebine, se pa uporabi pri dokazovanju pomembnejšega rezultata,\footnote{Sicer ni nujno, da se resnična pomembnost izjav takoj pokaže. Mnogo je primerov, ko se kak matematični članek po določenem času začne ceniti ne toliko zaradi glavnega izreka, pač pa zaradi neke leme, ki se je za dokaz glavnega izreka uporabila.}
			\item
				\df{posledica}: rezultat, ki je zanimiv sam po sebi, ki pa hitro sledi iz predhodne izjave.
		\end{itemize}
		
		Če skrbno analizirate izreke, trditve itd.~s predavanj (ali matematičnih člankov), opazite, da sestojijo iz treh delov: \note{kontekst, predpostavke, sklepi}
	
	
	\section{Definicije}
\chapter{Preproste množice}
\label{cha:preproste-mnozice}


Temeljni gradniki sodobne matematike so \df{množice}, ki so skupki ali zbirke matematičnih
objektov, lahko spet množice. Vsaka množica sestoji iz \df{elementov} in je z njimi
natančno določena. Kadar je $a$ element množice $M$, to zapišemo kot $a \in M$.

Ideja množice kot poljubne zbirke elementov je zavajajoče preprosta, kar so na lastni koži
izkusili matematiki na prelomu iz 19.~v 20.~stoletje. Takrat so že vedeli, da so množice zelo
uporabne in da lahko iz njih tvorimo razne vrste matematičnih objektov. A znameniti
matematik in filozof Bertrand Russell je odkril paradoks, ki se imenuje po njem, in gre
takole. Naj bo~$R$ množica vseh množic, ki niso element same sebe. Ali $R$ je element~$R$?
Če je $R$ element $R$, potem iz definicije $R$ sledi, da $R$ ni element $R$. In če $R$ ni
element $R$, spet iz definicije $R$ sledi, da $R$ je element $R$. Torej $R$ hkrati je in
ni svoj element, kar je protislovje! Russellov paradoks ste morda že spoznali v
priljubljeni različici, ki govori o vaškem brivcu, ki brije vse vaščane, ki ne brijejo
samih sebe.

Russellov paradoks je povzročil pravo krizo v temeljih matematike. Ker so bile množice
nepogrešljivo orodje, jih niso hoteli kar zavreči, po drugi strani pa je bilo treba
preprečiti Russellov in druge paradokse, ki so jih še odkrili. Bertrand Russell je
predlagal rešitev, ki jo je poimenoval \df{teorija tipov}. Russellova teorija tipov je
pomembno vplivala na nadaljni razvoj temeljev matematike, sodobna teorija tipov pa je
pomembno orodje v računalništvu. Tako kot množice so bili tipi skupki elementov, a so
tvorili neskončno hierarhijo, v kateri so bili elementi tipa vedno iz nižjega nivoja
hierarhije kot tip, ki so mu pripadali. Za potrebe večine matematike zadostuje že
preprostejša dvoslojna hierarhija množic in \df{razredov}. Množice smejo biti elementi
množic in razredov, razredi pa ne. Russellov paradoks izgine, ker je $R$ razred vseh
tistih množic, ki niso same svoj element. Vprašanje, ali je $R$ element samega sebe, tako
postane nesmiselno, saj $R$ ni množica. A zaenkrat odložimo podrobnejšo obravnavo razredov
in se raje posvetimo osnovnima pojmoma, množica in preslikava.

V splošni razpravi o množicah, ki bi presegala meje matematične vede, bi se opirali na
zgodovinski in družbeni kontekst, jezikovni izvor in rabo besed `množica', `skupek' in
`zbirka', kognitivno analizo, eksperimente, filozofijo itn. Vsi ti vidiki so za matematike
izjemo koristni, saj iz takih ``pred-matematičnih'' obravnav črpamo sveže zamisli in
matematiko naredimo zares uporabno. Ko pa delujemo znotraj matematike, zunanje vplive
odmislimo in se zanašamo le še na pravila logičnega sklepanja in matematične zakone, da ne
prihaja do nejasnosti in dvomljivih sklepov.

Kot matematiki lahko ustvarimo takšen ali drugačen pojem množice in pri tem imamo popolno
svobodo. Se množica lahko spreminja ali vedno vsebuje iste elemente? Je pomemben vrsti red
elementov v množici? Sme množica biti element same sebe? Ali morajo biti elementi množice
izračunljivi? To so vprašanja, ki nimajo enoznačnega odgovora. In res je znanih več med
seboj nezdružljivih zvrsti teorije množic, ki matematično opredeljujejo različne vidike
običajnega razumevanja besede `množica'. Mi bomo spoznali ``standardno'' teorijo množic,
ki jo uporablja velika večina matematikov.


\section{Načelo ekstenzionalnosti}
\label{sec:nacelo-ekstenzionalnosti}

Zamisel, da je množica natančno določena s svojimi elementi, izrazimo z matematičnim
zakonom, ki mu pravimo \df{načelo ekstenzionalnosti}:

\begin{pravilo}[Ekstenzionalnost množic]
  Množici sta enaki, če vsebujeta iste elemente.
\end{pravilo}

Kaj pravzaprav pomeni, da je to ``pravilo'', ``matematični zakon'' ali ``načelo''? So ga
razglasili v parlementu, je to zakon narave, ali morda dogma, ki jo je razglasil profesor
na predavanjih? Bodo tisti, ki načela ekstenzionalnosti ne spoštujejo, deležni Lešnikove
masti? Ne. Matematični zakoni so \emph{dogovori}, nekakšna pravila matematične igre. V
zgodovinskem razvoju matematike so se uveljavili tisti dogovori, ki so bili uporabni v
naravoslovju in tehniki, ali pa so v njih matematiki videli notranjo lepoto in lastno
uporabno vrednost.

Pravkar smo se dogovorili, da bomo obravnavali matematične objekte množice, ki vsebujejo
elemente in da zanje velja načelo ekstenzionalnosti. Namesto besed `množica' in `element'
bi lahko izbrali tudi kaki drugi besedi, denimo `zbor' in `član', ali celo `morje' in
`riba', s čimer se matematična vsebina pojmov ne bi čisto nič spremenila, čeprav ne gre
preveč izzivati svojih stanovskih kolegic in kolegov. Strukturo, lastnosti in povezave med
matematičnimi objekti namreč določajo dogovorjeni matematični zakoni in ne besede, s
katerimi jih poimenujemo.

Še enkrat poudarimo, da ima vsakdo, še posebej pa mladi um, popolno svobodo matematičnega
ustvarjanja. Želite razmišljati o drugačnih množicah, ki ne zadoščajo načelou
ekstenzionalnsti? Ali pa o številih, ki zadoščajo zakonu $x + x = 0$? O geometriji, v
kateri skozi točko lahko potegnemo dve vzporednici k dani premici? Kar dajte! Pri tem vas
le prosimo, celo zahtevamo, da razmišljate temeljito, vztrajno in globoko, da ste iskreni
do sebe in ostalih ter da svoje zamisli in spoznanja predstavite na matematikom razumljiv
način.

Vrnimo se k našim množicam. Načelo ekstenzionalnosti nam pove, da lahko množico podamo
tako, da natančno opredelimo njene elemente. A to ne pomeni, da množica obstaja, brž ko jo
lahko natančno opredelimo! To je pot, ki vodi naravnost do Russelovega paradoksa, saj so
elementi paradoksalne množice~$R$ natančno opredeljeni. Potrebujemo dodatna pravila, ki
določajo dopustne \df{konstrukcije množic}. Izbrati jih moramo previdno, da se izognemo
težavam.

\section{Končne množice}
\label{sec:koncne-mnozice}

Posebej preprosta konstrukcija množic združi končen nabor matematičnih objektov v množico.
Na primer, če so $a$, $b$ in $c$ matematični objekti, potem lahko tvorimo množico
%
\begin{equation*}
  \set{a, b, c}
\end{equation*}
%
katere objekti so natanko $a$, $b$ in $c$. To pomeni, da za vsak matematični objekt~$x$
velja
%
\begin{equation*}
  \text{$x \in \set{a, b, c}$, če in samo če $x = a$ ali $x = b$ ali $x = c$.}
\end{equation*}
%
Fraza ``če in samo če'' tu pomeni, da velja dvoje:
%
\begin{enumerate}
\item Če $x = a$ ali $x = b$ ali $x = c$, potem $x \in \set{a, b, c}$.
\item Če $x \in \set{a, b, c}$, potem $x = a$ ali $x = b$ ali $x = c$.
\end{enumerate}
%
Tako nam na primer prva trditev zagotavlja $1+1 \in \set{1, 2, 4}$, ker velja
vsaj ena od možnosti: $1 + 1 = 1$ ali $1 + 1 = 2$ ali $1 + 1 = 3$. Iz druge trditve sledi, da
$5 \in \set{1, 2, 3}$ ne velja, ker ne velja nobena od možnosti: $5 = 1$ ali $5 = 2$ ali
$5 = 3$.

Splošna konstrukcija končnih množic poteka takole.

\begin{pravilo}
  \label{pravilo:koncna-mnozica}
  Za vse objekte $a$, $b$, \dots, $z$ je $\set{a, b, \ldots, z}$ množica, katere elementi
  so natanko objekti $a$, $b$, \dots, $z$.
\end{pravilo}

Za trenutek ustavimo tok misli in opozorimo, da zapis s tropičjem `$\ldots$' ni dovolj
natančen, saj dopušča dvoumnosti. Denimo, so elementi množice
%
\begin{equation*}
  \set{3, 5, 7, \ldots, 31},
\end{equation*}
%
liha števila med $3$ in $31$, ali samo praštevila? Zapis res ni dovolj natančen. Kljub
temu tak zapis v praksi uporabljamo, ker v praksi bralec večinoma pravilno ugane, kaj je
bilo mišljeno, saj imamo ljudje zelo podobne sposobnosti prepoznavanja vzorcev. Z
matematičnega vidika pa to ni dopustno, saj lahko tropičje \emph{vedno} razumemo na več
načinov. (Ne verjamete? Naslednji člen v zaporedju $1, 2, 3, \ldots$ je seveda~$5$, ker je
naslednji člen vsota prejšnjih dveh, kot v Fibonaccijevem zaporedju.)

Kot smo že omenili, želimo pojem množice, pri kateri vrstni red elementov ni pomemben.
Torej bi morali biti množici $\set{1, 2}$ in $\set{2, 1}$ enaki. Pa je to res? Velja ena
od treh možnosti:
%
\begin{enumerate}
\item Iz načela ekstenzionalnosti in konstrukcije množic $\set{1, 2}$ in $\set{2, 1}$ sledi, da sta enaki.
\item Iz načela ekstenzionalnosti in konstrukcije množic $\set{1, 2}$ in $\set{2, 1}$ sledi, da nista enaki.
\item Načelo ekstenzionalnosti in konstrukcije množic $\set{1, 2}$ in $\set{2, 1}$ ne določajo, ali sta enaki.
\end{enumerate}
%
V prvem primeru bi želeli dokazati enakost. V drugem primeru smo v zagati, saj smo se
dogovorili za matematična pravila, ki imajo neželene posledice. V tretjem primeru moramo
dodati še kakšne nove zakone o množicah. Na srečo obvelja prva možnost.

\begin{trditev}
  Množici $\set{1, 2}$ in $\set{2, 1}$ sta enaki.
\end{trditev}

\begin{proof}
  Dokaz, ki ga bomo zapisali je izjemno podroben in ga v praksi matematik ne bi zapisal,
  saj je z njegovim branjem več dela, kot če bi naredili sami. Ker pa želimo pokazati, da
  tudi najbolj trivialna dejstva lahko dokažemo, ga zapišimo.

  Izhajati smemo izključno iz naslednji dejstev:
  %
  \begin{itemize}
  \item načelo ekstenzionalnosti,
  \item $x \in \set{1, 2}$, če in samo če $x = 1$ ali $x = 2$,
  \item $x \in \set{2, 1}$, če in samo če $x = 2$ ali $x = 1$.
  \end{itemize}
  %
  Najprej uporabimo načelo ekstenzionalnosti, ki zagotavlja, da sta $\set{1, 2}$ in
  $\set{2, 1}$ enaki, če imata iste elemente. Dokažimo torej, da imata iste elemente. To
  naredimo v dveh korakih:
  %
  \begin{enumerate}
  \item Dokažimo, da za vsak element $\set{1, 2}$ dokažemo, da je element $\set{2, 1}$.
    Naj bo $x \in \set{1, 2}$. Iz definicije množice $\set{1, 2}$
    sledi, da je $x = 1$ ali $x = 2$. Obravnavamo dva podprimera:
    %
    \begin{enumerate}
    \item Primer $x = 1$: iz $x = 1$ sledi, da je $x = 2$ ali $x = 1$, zato je $x \in \set{2, 1}$.
    \item Primer $x = 2$: iz $x = 2$ sledi, da je $x = 2$ ali $x = 1$, zato je $x \in \set{2, 1}$.
    \end{enumerate}
    %
  \item Dokažimo, da za vsak element $\set{2, 1}$ dokažemo, da je element $\set{1, 2}$.

    Ta korak je povsem podoben prvemu, le da je treba povsod zamenjati~$1$ in~$2$.
    Matematik bi zato na tem mestu zapisal, da je drugi korak podoben prevemu in dokaz
    zaključil. A tega tokrat ne bomo storili in bomo zapisali popoln dokaz.

    Naj bo $x \in \set{2, 1}$. Iz definicije množice $\set{2, 1}$ sledi, da je $x = 2$ ali
    $x = 1$. Obravnavamo dva primera:
    %
    \begin{enumerate}
    \item Primer $x = 2$: iz $x = 2$ sledi, da je $x = 1$ ali $x = 2$, zato je $x \in \set{1, 2}$.
    \item Primer $x = 1$: iz $x = 1$ sledi, da je $x = 1$ ali $x = 2$, zato je $x \in \set{1, 2}$. \qedhere
    \end{enumerate}
    %
  \end{enumerate}
\end{proof}

Mimogrede, črn kvadratek označuje konec dokaza. Imenuje se tudi ``Halmos'' po matematiku
Paulu Halmosu, ki ga je prvi uporabljal. S podobnim razmislekom, ki ga prepuščamo za vajo,
lahko dokažemo, da ni pomembno, ali se element pojavi enkrat ali večkrat.

\begin{naloga}
  Podrobno dokažite, da sta množici $\set{1, 1, 2}$ in $\set{1, 2}$ enaki.
\end{naloga}

V prejšnji nalogi smo zapisali $\set{1, 1, 2}$. Pa je to sploh dovoljeno?
Pravilo~\ref{pravilo:koncna-mnozica} pravi, da lahko iz objektov $a, b, c, \ldots, z$
tvorimo končno množico $\set{a, b, \ldots, z}$. Nikjer ne piše, da smeta biti $a$ in $b$
enaka, zato je upravičeno vprašanje, ali je dovoljeno za $a$ in $b$ vzeti~$1$. V
matematiki vse razumemo dobesedno. V pravilu~\ref{pravilo:koncna-mnozica} piše ``Za vse
objekte'', torej imamo povsem proste roke. Povedano z drugimi besedami, množico
$\set{1, 1, 2}$ smemo tvoriti, ker nikjer ne piše, da morajo biti elementi različni.

V zvezi s pravilom~\ref{pravilo:koncna-mnozica} se pojavljajo še drugi dvomi. Ali smemo
tvoriti množico, ki ima več elementov, kot je črk abecede? Ali bi bilo pravilo še vedno
isto, če bi namesto ``$a, b, \ldots, z$'' zapisali ``$a, b, \ldots, j$''? Ali smemo
tvoriti množico z nič elementi? Če namreč vstavimo nič elementov, se pravilo glasi ``Za
vse objekte je $\set{\,}$ množica, katere elementi so natanko objekti,'' kar je vsaj
nenavadno. Iz nesrečnega tropičja se res ne vidi, kaj je in kaj ni dovoljeno. Če poškilite
v razdelek~\ref{sec:aksiomi-teorije-mnozic}, kjer so našteti ``uradni'' aksiomih teorije
množic, tam pravila o končnih množicah ne boste našli, saj sledi iz treh bolj osnovnih
pravil.

\begin{pravilo}
  \label{pravilo:prazna-mnozica}
  \df{Prazna množica} $\emptyset$ je množica, ki nima elementov.
\end{pravilo}

\begin{pravilo}
  \label{pravilo:neurejeni-dvojec}
  Za vsak $x$ in $y$ je \df{(neurejeni) par} ali \df{dvojec} $\set{x, y}$ množica, katere
  elementa sta natanko $x$ in $y$.
\end{pravilo}

\begin{pravilo}
  \label{pravilo:unija}
  Za vsaki množici $A$ in $B$ je \df{unija $A \cup B$} množica, ki ima za elemente
  natanko vse objekte, ki so element $A$ ali element $B$.
\end{pravilo}

V pravilu~\ref{pravilo:neurejeni-dvojec} smo besedo ``neurejeni'' zapisali v oklepaju, kar
pomeni, da beseda pravzaprav ni pombembna in bi jo lahko tudi izpustili. Se pravi, da
``neurejeni dvojec'' in ``dvojec'' pomenita isto. V primeru nejasnosti raje uporabimo
daljšo obliko.

Tri nova pravila skupaj nadomestijo pravilo~\ref{pravilo:koncna-mnozica} in odstranijo
marsikateri dvom o uporabi. Prvo pravilo pojasni, da lahko tvorimo množico brez elementov.
Poleg oznake $\emptyset$ je za prazno množico smiselno uporabiti tudi zapis $\set{\,}$.

Drugo pravilo pove, kako lahko tvorimo množico z dvema elementoma, pa tudi z enim.
Spomnimo se, pravila je treba brati dobesedno: za $x$ in $y$ bi lahko vzeli dvakrat isti
objekt~$z$ in tvorili množico $\set{z, z}$, ki ima natanko elementa $z$ in $z$. To je
pravzaprav množica z enim samim elementom $z$, zato ji pravimo tudi \df{enojec} in jo
zapišemo~$\set{z}$.

Tretje pravilo nam omogoča, da tvorimo večje množice. Denimo, množico z elementi $a$, $b$,
$c$ lahko tvorimo kot unijo
%
\begin{equation*}
  \set{a, b} \cup \set{c}.
\end{equation*}
%
To ni edini način, enako množico lahko dobimo na več načinov:
%
\begin{equation*}
  (\set{a} \cup \set{b}) \cup \set{c}
  \quad\text{ali}\quad
  \set{b} \cup \set{c, a}
  \quad\text{ali}\quad
  \set{a,c,a} \cup \set{b,c}
  \quad\text{itn.}
\end{equation*}
%
Seveda bi morali dokazati, da so vse te množice enake, a tega ne bomo storili.

Pogosto nam bo prišlo prav, da bomo imeli pri roki množico z enim elementom, pri čemer nam
bo vseeno, kaj ta element je. V ta namen postavimo pravilo, ki zagotavlja obstoj množice z
enim elementom.

\begin{pravilo}
  \label{pravilo:enojec}
  \df{Standardni enojec} je množica~$one$, katere edini element je~$\unit$.
\end{pravilo}

Morda se zdi nenavadno, da množico označimo s številom, a ta občutek bo hitro izginil, ko
bomo računali z množicami. Pravaprav bi lahko prazno množico označili z nič $\mathbf{0}$,
in nekateri matematiki to dejansko počnejo.

Edini element množice $\one$ smo označili z nenavadnim zapisom $\unit$. Na tem mestu ne
bomo pojasnili, zakaj pišemo tako, radovedneži pa lahko pogledajo v
razdelek~\ref{sec:aritmetika-tarskega}. Mimogrede, seveda velja $\one = \set{\unit}$.

Pravilo~\ref{pravilo:enojec} ni nujno potrebno, saj lahko tvorimo veliko različnih enojcev
kar sami $\set{\emptyset}$, $\set{42}$, $\set{\set{\emptyset}}$ itn. Ali je kateri od njih
``prvi med enakimi'' in bi ga lahko uporabljali kot ``standardni'' enojec? Ker je odgovor
v veliki meri stvar osebnega mnenja, je bolje, da razglasimo pravilo, ki ustoliči
standardni enojec. S prazno množico nimamo podobnih težav, saj je ena sama.

% \subsection{Druge množice}

% \andrej{To ne paše sem, ker bi bilo tu dosti bolj naravno nadaljevati s preslikavami.
%  To bomo prestavili na mesto, kjer bo dejansko prišlo prav.}

% Množice, s katerimi v matematiki delamo, tipično vsebujejo števila, ali pa so vsaj na tak ali drugačen način izpeljane iz številskih množic. Spomnimo se standardnih oznak najpogosteje uporabljanih številskih množic.
% \begin{center}
% \begin{tabular}{|cc|}
% \hline
% \textbf{Množica} & \textbf{Oznaka} \\
% \hline
% množica naravnih števil & $\NN$ \\
% množica celih števil & $\ZZ$ \\
% množica racionalnih števil & $\QQ$ \\
% množica realnih števil & $\RR$ \\
% množica kompleksnih števil & $\CC$ \\
% \hline
% \end{tabular}
% \end{center}

% Nekateri $0$ vzamejo za naravno število, nekateri ne. To je v celoti stvar dogovora, kaj pomeni pojem ``naravno število''. Za nas bo prišlo bolj prav, če ničlo štejemo kot element množice naravnih števil, torej $\NN = \set{0, 1, 2, 3, \ldots}$.

\section{Preslikave}

Temelj matematike ne tvorijo le množice, ampak tudi drugi matematični pojmi. Prvi izmed
njih je \df{preslikava}, oziroma s tujko \df{funkcija}.\footnote{Nekateri uporabljajo
  izraz ``funkcija'' samo za tiste preslikave, ki slikajo v realna ali kompleksna števila,
  vendar to navado izpodriva računalništvo, saj funkcije v programskih jezikih nimajo
  omejitev. Dandanes večina matematikov besedo ``funkcija'' obravnava kot sopomenko besede
  ``preslikava'' in tako jo bomo uporabljali tudi mi.} V srednji šoli ste že spoznali
nekatere preslikave, kot so na primer linearne preslikave, trigonometrijske funkcije,
logaritem itd. Nas pa ne bodo zanimale posamezne preslikave, ali posebne lastnosti
preslikav, ampak preslikave na splošno.

Vsaka preslikava ima tri sestavne dele: \df{domeno} ali \df{začetno množico},
\df{kodomeno} ali \df{ciljno množico} in \df{predpis}. Domeni se pogosto reče tudi
\df{definicijsko območje}. Če govorimo o preslikavi, ki ima domeno~$X$ in kodomeno~$Y$, to
ponazorimo s puščico med $X$ in $Y$, takole
%
\begin{equation*}
  \xymatrix{
    {X} \ar[r] &
    {Y}
  }
\end{equation*}
%
Če želimo preslikavo poimenovati, na primer $f$, zapišemo
%
\begin{equation*}
  \xymatrix{
   {f : X} \ar[r] &
    {Y}
  }
  \qquad\text{ali}\qquad
  \xymatrix{
   {X} \ar[r]^{f} &
   {Y}
  }
\end{equation*}
%
Pravimo, da je \df{$f$ preslikava iz $X$ v $Y$}. Zapis nad puščico je prikladen, kadar
imamo opravka z večimi preslikavami, ki jih predstavimo z diagramom. Na primer,
%
\begin{equation*}
  \xymatrix{
    {X} \ar[r] &
    {Y} \ar[r]^{f} &
    {Z}  &
    {W} \ar[l]_{g}
  }
\end{equation*}
%
nam pove, da imamo opravka z (neimenovano) preslikavo iz $X$ v $Y$, s preslikavo $f$ iz
$Y$ v $Z$ in s preslikavo $g$ is $W$ v $Z$. Diagrami so lahko še precej bolj zapleteni.

Tretji del preslikave je predpis, ki določa, kako elemente domene preslikamo v elemente
kodomene. Kaj pravzaprav to pomeni? Možnih je več odgovorov. V srednji šoli predpis
enačimo z matematično formulo, ki spremenljivko preslika v vrednost, na primer $x$ slika v
$2 \sin(x + \pi/4)$. S simboli to zapišemo
%
\begin{equation*}
  x \mapsto 2 \sin(x + \pi/4).
\end{equation*}
%
in preberemo ``$x$ se slika v dvakrat sinus od $x$ plus pi četrtin.''
%
Matematiki smo natančni, zato ne mešamo uporabe puščic $\to$ in $\mapsto$. Navadna puščica
se uporablja pri oznaki domene in kodomene, repata pa v predpisu. V računalništvu besedo
`predpis' razumemo kot `programska koda' in o preslikavah razmišljajo kar kot o
algoritmih --- tudi to je eden od možnih pogledov na preslikave.

V teoriji množic razumemo besedo `predpis' kot kakršnokoli prirejanje med elementi množic
domene~$X$ in kodomene~$Y$, mora pa veljati:
%
\begin{itemize}
\item \df{celovitost}: vsakemu elementu iz $X$ je prirejen vsaj en element iz $Y$,
\item \df{enoličnost}: če sta elementu $x$ prirejena $y \in Y$ in $z \in Y$, potem $y = z$.
\end{itemize}


\subsection{Funkcijski predpisi}
\label{sec:funkcijski-predpisi}

Predpise lahko podamo na različne načine, najbolj pogost pa je \df{funkcijski predpis}, ki
se mu še posebej posvetimo in se ob njem naučimo nekaj natančnosti. Funkcijski predpis ima
obliko
%
\begin{equation*}
  x \mapsto \cdots,
\end{equation*}
%
ki smo jo že videli maloprej. Na desni, lahko namesto $\cdots$ zapišemo izraz, v katerem
se sme pojaviti simbol~$x$, denimo
%
\begin{equation*}
  x \mapsto 1 + x^2.
\end{equation*}
%
Ni nujno, da se~$x$ pojavi, denimo $x \mapsto 42$ vsakemu elementu iz domene priredi
število $42$. V funkcijskem predpisu se smejo pojaviti tudi drugi simboli, ki jim
pravimo \df{parametri}. Tako je
%
\begin{equation*}
  x \mapsto a \cdot x + b
\end{equation*}
%
funkcijski predpis s parametroma $a$ in $b$, ki elementu $x$ priredi element $a \cdot x + b$.

Spremenljivka $x$ nima v naprej določene vrednosti, pač pa kaže, kam lahko vstavimo
elemente domene. Pravimo, da je $x$ \df{vezana spremenljivka}, kar pomeni, da je veljavna
le v funkcijskem predpisu, nanj je vezana, in da ni pomembno, s katerim simbolom jo
označimo. Tako sta funkcijska predpisa
%
\begin{equation*}
  x \mapsto 1 + x^2
  \qquad\text{in}\qquad
  a \mapsto 1 + a^2
\end{equation*}
%
enaka in lahko bi celo pisali $\Box \mapsto 1 + \Box^2$ ali
$\heartsuit \mapsto 1 + \heartsuit^2$.

V funkcijskem predpisu mora na levi stati en sam simbol, ki na desni kaže, kam je treba
vstaviti element iz domene. Tako
%
\begin{equation*}
  \sin(x) \mapsto \cos(2 x),
  \qquad
  3 + 2 \mapsto 5
  \qquad\text{in}\qquad
  \sin(x) \mapsto 2 \cdot \sin(x)
\end{equation*}
%
\emph{niso} veljavni funkcijski predpisi.

Seveda dopuščamo možnost, da se vezana spremenljivka pojavi enkrat, večkrat ali sploh ne.
Funkcijska predpisa
%
%
\begin{equation*}
  x \mapsto 42
  \qquad\text{in}\qquad
  x \mapsto x \cdot \sin(x)
\end{equation*}
%
sta torej veljavna.

Če želimo preslikavo z danim funkcijskim predpisom poimenovati, na primer $f$, zapišemo
%
\begin{equation*}
  f : x \mapsto 1 + x^2.
\end{equation*}
%
To preberemo ``$f$ slika $x$ v ena plus $x$ na kvadrat.'' Običajna sta tudi zapisa
%
\begin{equation*}
  f(x) = 1 + x^2
  \qquad\text{in}\qquad
  f(x) \dfeq 1 + x^2.
\end{equation*}
%
Funkcijske predpise je podrobno prvi preučeval Alonzo Church,\footnote{Alonzo Church
  (1903--1995) je bil ameriški matematik in logik, ki je pomembno prispeval k razvoju
  logike in teoretičnega računalništva. Njegov študent, Dana Stewarta Scott, je imel
  študenta Marka Petkovška in Andreja Bauerja, slednji pa je imel študenta Davorina
  Lešnika.} ki je uporabljal zapis
%
\begin{equation*}
  \lambda x \,.\, 1 + x^2
\end{equation*}
%
in teorijo funkcijskih predpisov poimenoval \df{$\lambda$-račun}. V logiki se je njegov
zapis obdržal in se uveljavil tudi v programski jezikih:
%
\begin{itemize}
\item v Pythonu pišemo \verb|lambda x : 1+x**2|,
\item v Haskellu pišemo \verb|\x -> 1+x**2| in
\item v OCamlu pišemo \verb|fun x => 1+x*x|.
\end{itemize}
%
Predvsem v programiranju funkcijskim predpisom pravijo tudi \df{anonimne} ali \df{brezimne
  preslikave}.

Nekateri starejši zapisi funkcijskih predpisov so slabi, a jih ljudje vztrajno
uporabljajo. Opozorimo le na en slab zapis, ki povzroča precej preglavic, ne da bi se
matematiki tega zares zavedali. Funkcijski predpis mora določati vezano spremenljivko,
sicer ne vemo, kako vstaviti vrednosti, a na žalost jo matematiki pogosto izpustijo skupaj
z $\mapsto$, da ostane samo izraz na desni.
%
Težava je v tem, da se lahko v funkcijskem predpisu pojavi več kot en simbol. Če vam na primer povem, da imam v mislih funkcijski predpis
%
\begin{equation*}
  a \cdot x + b
\end{equation*}
%
boste vsi mislili, da je mišljeno $x \mapsto a \cdot x + b$. A pravzprav bi lahko bilo
tudi $a \mapsto a \cdot x + b$ ali $b \mapsto a \cdot x + b$ ali celo
$t \mapsto a \cdot x + b$! Namreč, nič ni narobe s funkcijskim predpisom, v katerem se
pojavijo dodatni simboli.

Morda pa lahko vezano spremenljivko in $\mapsto$ brez škode izpustimo, če v izrazu nastopa
samo en simbol, denimo $1 + x^2?$
%
A spet bi zabredli v težave. Je $42$ število ali funkcijski predpis $x \mapsto 42$? Je
$1 + x^2$ funkcijski predpis $x \mapsto 1 + x^2$ ali $a \mapsto 1 + x^2$?

Velikokrat površno rečemo, da funkcijski predpis podaja preslikavo. To ni res, saj smo že
prej povedali, da ima vsaka preslikava tri sestavne dele: domeno, kodomeno in prirejanje.
Res, če ne poznamo domene, ne moremo preveriti, ali je funkcijski predpis celovit. Denimo,
funkcijski predpis
%
\begin{equation*}
  x \mapsto \frac{x}{x^2 - 2}
\end{equation*}
%
ni celovit, če je domena množica realnih števil, in je celovit, če je domena množica
racionalnih števil. Tudi kodomeno moramo poznati, sicer ne moremo določiti nekaterih
lastnosti preslikave, kot je na primer surjektivnost, glej
razdelek~\ref{razdelek:injektivnost-in-surjektivnost}.



\subsection{Ostali načini podajanja preslikav}
\label{sec:ostali-predpisi}

Funkcijski predpisi niso edini način za podajanje prirejanja, zato omenimo še nekatere
druge.

Preslikavo s končno domeno lahko podamo s tabelo, na primer:
%
\begin{center}
  $f : \set{1, 2, 3, 5} \to \set{10, 20, 30}$

  \medskip

  \begin{tabular}{|c|c|} \hline
    1 & 10 \\ \hline
    2 & 10 \\ \hline
    3 & 20 \\ \hline
    5 & 10 \\ \hline
  \end{tabular}
\end{center}
%
To seveda pomeni, da $f$ elementu $1$ priredi $10$, $2$ priredi $10$, $3$ priredi $20$ in $5$
priredi $10$. Tabelo lahko predstavimo na različne načine, lahko kar naštejemo vsa prirejanja:
%
\begin{align*}
  f(1) &= 10 \\
  f(2) &= 10 \\
  f(3) &= 20 \\
  f(5) &= 10.
\end{align*}
%
Tudi
%
\begin{align*}
  1 &\mapsto 10 \\
  2 &\mapsto 10 \\
  3 &\mapsto 20 \\
  5 &\mapsto 10.
\end{align*}
%
je še vedno le tabela, ki prikazuje prirejanje. Ne sme nas motiti dejstvo, da smo
$\mapsto$ uporabili za naštevanje prirejanj, namesto za funkcijski prdpis.

Preslikava je lahko določena tudi z opisom računskega postopka, pravimo mu \df{algoritem},
s pomočjo katerega izračunamo vrednost preslikave pri danem argumentu. Paziti moramo, da je
opis postopka res natančen in nedvoumen, lahko ga kar zapišemo kot program. Teoretični
računalničar bi pripomnil, da je treba pri tem izbrati programski jezik, ki ima ustrezno
matematično definicijo.

Preslikave lahko podamo tudi tako, da opišemo pogoje, pri katerih je element kodomene
prirejen elementu domene. Na primer, preslikavo $f : \NN \to \ZZ$ bi lahko definirali z
zahtevo, da naravnemu številu $n \in \NN$ priredimo celo število $k \in \ZZ$, kadar velja
%
\begin{equation*}
  k^2 \leq n < (k+1)^2.
\end{equation*}
%
To prirejanje je veljavno, če je celovito in enolično, česar ne bomo preverjali, lahko pa
poskusite sami. Nekaj prirejanj $f$ prikazuje naslednja razpredelnica:
%
\begin{align*}
0 &\mapsto 0   &   4 &\mapsto 2   &    8  &\mapsto 2   &   12 &\mapsto 3 \\
1 &\mapsto 1   &   5 &\mapsto 2   &    9  &\mapsto 3   &   13 &\mapsto 3 \\
2 &\mapsto 1   &   6 &\mapsto 2   &    10 &\mapsto 3   &   14 &\mapsto 3 \\
3 &\mapsto 1   &   7 &\mapsto 2   &    11 &\mapsto 3   &   15 &\mapsto 3
\end{align*}
%
Ali znate z besedami opisati preslikavo~$f$?

V splošnem je lahko preslikava podana s precej zapleteno konstrukcijo, ki zahteva veliko
preverjanja in dokazovanja. Osnovne načine podajanja preslikav bomo spoznali skupaj s
konstrukcijami množic.


\subsection{Aplikacija in substitucija}
\label{sec:aplikacija-in-subsitucija}

Do sedaj smo se ukvarjali s tem, kako preslikavo podamo, zdaj pa se vprašajmo, kako lahko
preslikavo uporabimo. Če je $f : X \to Y$ preslikava iz $X$ v $Y$ in je $x \in X$, potem
lahko \df{$f$ uporabimo na $x$} in dobimo \df{vrednost} preslikave~$f$ pri
\df{argumentu}~$x$, to je tisti edini element $Y$, ki ga~$f$ priredi~$x$. Vrednost $f$
pri~$x$ zapišemo
%
\begin{equation*}
  f(x)
  \qquad\text{ali}\qquad
  f\,x
\end{equation*}
%
in preberemo ``$f$ od $x$'' ali ``$f$ pri $x$''. Izraz $f(x)$, oziroma $f\,x$, se imenuje
\df{aplikacija}. Večinoma se uporablja zapis z oklepaji, a ne vedno: navajeni smo pisati
$\ln 2$ in $\sin \alpha$ namesto $\ln(2)$ in $\sin(\alpha)$. Oklepaje izpuščamo tudi v
nekaterih programskih jezikih in občasno v algebri.

V analizi je uveljavljen še en zapis za aplikacijo, ki se uporablja za zaporedja. Namreč,
zaporedje ni nič drugega kot preslikava $a : \NN \to \RR$ iz naravnih v realna števila.
Aplikacijo $a(n)$, ki označuje $n$-ti člen zaporedja, ponavadi pišemo~$a_n$, torej
argument podpišemo.

Preslikavo lahko uporabimo na argumentu tudi, če je nismo poimenovali. Na primer,
preslikavo $\RR \to \RR$, podano s funkcijskim predpisom
%
\begin{equation*}
  x \mapsto 1 + x^2
\end{equation*}
%
uporabimo na argumentu~$3$:
%
\begin{equation*}
  (x \mapsto 1 + x^2)(3).
\end{equation*}
%
Se vam zdi tak zapis nenavaden? Verjetno, a pomislite, zakaj je tako: ker običajno
preslikave poimenujemo in se nanje vedno sklicujemo z njihovim imenom. Prav nobenega
razloga ni, da ne bi s funkcijskimi predpisi delali tako, kot s števili, vektorji in
ostalimi matematičnimi objekti, na katere smo že navajeni. Računalničarji radi rečejo, da
je treba tudi preslikave obravnavati kot ``enakopravne državljane''. Pravi imajo, zato
bomo vadili uporabo funkcijskih predpisov ter z njimi delali, kot da niso nič posebnega,
saj niso!

Kako pravzaprav določimo vrednost funkcije pri danem argumentu? To je odvisno od tega,
kako je podano prirejanje. Če imamo tabelarični prikaz, poiščemo argument v levem stolpcu
in pogledamo v desni stolpec. Če je preslikava podana s funkcijskim predpisom, argument
vstavimo v predpis. Na primer, če je $f : \RR \to \RR$ podana s funkcijskim predpisom
%
\begin{equation*}
  f(x) = 1 + x^2,
\end{equation*}
%
potem je vrednost $f(3)$ enaka $1 + 3^2$, kar je seveda enako~$10$, a to zahteva dodaten
račun, ki nas v tem trenutku ne zanima. Pravimo, da smo simbol~$x$ \df{zamenjali} ali
\df{substituirali} s~$3$, oziroma da smo~$3$ \df{vstavili} v~$f$ namesto~$x$. Seveda lahko
vstavimo argument neposredno v funkcijski predpis, zato je aplikacija
%
\begin{equation*}
  (x \mapsto 1 + x^2)(3)
\end{equation*}
%
seveda spet enaka $1 + 3^2$.

Preslikavo smemo uporabiti na poljubnem elementu domene, ki je lahko zapisan na bolj ali
manj zapleten način, pri čemer gre še vedno samo za zamenjavo. Na primer, v zgornjo
preslikavo~$f$ lahko vstavimo $3 + 4$ in dobimo $1 + (3 + 4)^2$ ali pa za neki $u \in \RR$
vstavimo $u + 2$ in dobimo $1 + (u + 2)^2$. V razdelku~\ref{sec:eksponent} bomo spoznali
še dodatna pravila za vstavljanje izrazov, ki se vrtijo okoli vezanih spremenljivk.


\subsection{Načelo ekstenzionalnosti preslikav}

Kot smo že omenili, je možih več pogledov na preslikave. Ali je pomembno, kako učinkovito
računamo vrednosti preslikave? Vsekakor, ampak ali naj to pomeni, da sta preslikavi
različni, če imata enake vrednosti, a je ena podana z učinkovitim pravilom in druga z
neučinkovitim? V matematiki je odgovor nikalen.

\begin{pravilo}[Ekstenzionalnost preslikav]
  Preslikavi sta enaki, če imata enaki domeni in kodomeni ter imata za vse argumente
  enaki vrednosti.
\end{pravilo}

Natančneje, če sta $f : A \to B$ in $g : C \to D$ preslikavi in velja $A = C$, $B = D$ ter
za vsak $x \in A$ velja $f(x) = g(x)$, tedaj velja $f = g$.

Takoj opozorimo na razliko med
%
\begin{equation*}
  f(x) = g(x)
  \qquad\text{in}\qquad
  f = g
\end{equation*}
%
saj bi marsikdo trdil, da med njima ni razlike. Levi izraz pravi, da sta $f(x)$ in $g(x)$
enaka elementa množice $C$, desni pa da sta $f$ in~$g$ enaki preslikavi iz $A$ v $B$. Na
sploh je treba razlikovati med $f$ in $f(x)$, saj to nikakor nista enaka objekta: prvi je
preslikava, drugi pa vrednost te preslikave pri~$x$. Verjetno nihče ne bi trdil, da je
preslikava $\cos$ isto kot $\cos \frac{\pi}{4}$, ali ne? Isti razmislek veleva, da
$\cos x$ ni isto kot $\cos$, če tudi si mislimo, da je $x$ poljuben. Zmeda izhaja iz
neprimernega zapisa preslikav. Če bi že od malih nog pravilno uporabljali funkcijske
predpise, bi seveda vedeli, da načelo ekstenzionalnosti za preslikave zagotavlja enakost
~$\cos$ in $x \mapsto \cos x$, oba pa sta različna od $\cos x$, ki sploh ni preslikava,
ampak neko realno število. Čeprav je število $\cos x$ odvisno od parametra~$x$, je še
vedno le število.

V bran tradicionalnemu zapisu pa moramo vseeno povedati, da se lahko \emph{dogovorimo} za
nekoliko napačen zapis, če to ne povzroča zmede. S tem se izognemu preveč birokratskemu
pisanju nebistvenih podrobnosti in lahko bistveno izboljšamo komunikacijo in razumevanje
med izkušenimi matematiki. A začetnikom priporočamo, da v dobrobit boljšega razumevanja
snovi vsaj na začetku študija raje vztrajajo pri doslednem zapisu.

Vrnimo se še k načelu ekstenzionalnosti preslikav. Ali ne pravzaprav očitno, da sta
preslikavi enaki, če imata enaki domeni, kodomeni in vrednosti? Morda res, a to ni razlog,
da tega ne bi eksplicitno zapisali. Vsak matematik vam ve povedati kako zgodbo o tem,
kako se je v dokazu skrivala napako ravno tam, kjer je bilo nekaj ``očitno''. Poleg tega
pa si lahko predstavljamo razmere, v katerih je smiselno razlikovati med dvema
preslikavama, ki imata vedno enake vrednosti, denimo v programiranju, kjer je učinkovitost
zelo pomembna.

%% STAR MATERIAL OD DAVORINA. Preveriti, kaj od tega je treba dati v besedilo, in kam.

% Množice ne obstajajo ločene ena od druge pač pa so med sabo povezane s
% \df{preslikavami} oziroma s tujko \df{funkcijami}.  Posamična preslikava slika elemente ene
% množice po določenem predpisu v elemente druge množice.

% Če je $f$ preslikava, ki slika iz množice $X$ v množico $Y$, to zapišemo
% %
% \begin{equation*}
%   f : X \to Y.
% \end{equation*}
% %
% Rečemo, da je množica~$X$ \df{začetna množica} ali \df{domena} preslikave~$f$, množica~$Y$
% pa je \df{ciljna množica} ali \df{kodomena} preslikave $f$.


% Običaj je, da predpis preslikave podamo s pomočjo spremenljivke, tipično z oznako $x$. Na primer, če je $f$ preslikava kvadriranja, njen predpis zapišemo kot
% \[f(x) = x^2.\]
% Na tem mestu je potrebno poudariti več reči.
% \begin{itemize}
% \item
% Velikokrat površno rečemo, da zgornji predpis podaja preslikavo. To ni povsem res --- to je zgolj predpis preslikave. Za to, da preslikavo v celoti podamo, je potrebno navesti tri stvari: poleg predpisa še domeno in kodomeno. Vse to je del informacije o preslikavi.

% To se jasno pokaže, če začnemo razmišljati o lastnostih preslikav. Se še spomnite iz srednje šole, kaj pomeni, da je preslikava surjektivna? (Bomo ponovili v razdelku~\ref{razdelek:injektivnost-in-surjektivnost}.) Če vzamemo, da preslikava $f$ zadošča zgornjemu predpisu in jo obravnavamo kot preslikavo $f\colon \RR \to \RR$, ni surjektivna, če jo obravnavamo recimo kot preslikavo $f\colon \RR_{\geq 0} \to \RR_{\geq 0}$, pa je.
% \item
% Za spremenljivko $x$ velja isto, kot smo razpravljali že v prejšnjem razdelku pri lastnostih elementov množic: spremenljivka $x$ nima vnaprej določene vrednosti, pač pa predstavlja mesto, kamor lahko vstavimo poljubno vrednost. Seveda je potem vseeno, če vzamemo kakšno drugo črko ali čisto drug simbol: $f(y) = y^2$ določa isti predpis kot $f(x) = x^2$; prav tako $f(\heartsuit) = \heartsuit^2$. Se pravi, tudi v tem primeru gre za nemo spremenljivko. Če si torej izberemo neko vrednost, jo lahko vstavimo na mesto spremenljivke in izračunamo vrednost dobljenega izraza, npr.~$f(3) = 3^2 = 9$ oziroma $f(2\pi) = (2\pi)^2 = 4\pi^2$. Predstavljajte si, da je spremenljivka pravzaprav škatlica, kamor lahko vstavite vrednost, torej
% \[f(\argbox) = \argbox^2.\]
% \item
% Alternativen način zapisa $f(x) = x^2$ je
% \[f\colon x \mapsto x^2.\]
% Pazimo: navadna puščica $\to$ podaja domeno in kodomeno, kot razloženo zgoraj. Repata puščica $\mapsto$ pa za posamičen element domene pove, v kateri element kodomene se preslika.

% Zapis z repato puščico je še posebej uporaben, kadar želimo podati preslikavo, ne da bi nam bilo potrebno izbrati ime zanjo. Na primer, realno funkcijo kvadriranja lahko v celoti podamo takole:
% \begin{align*}
% \RR &\to \RR \\
% x &\mapsto x^2
% \end{align*}
% (prva vrstica pove domeno in kodomeno, druga pa predpis). Tako podanim preslikavam potem rečemo \df{brezimne preslikave} (s tujko \df{anonimne funkcije}). Kasneje (v razdelku~\ref{razdelek:brezimne-preslikave}) bomo spoznali bolj strnjen zapis takih preslikav, ki je še posebej primeren za izvajanje operacij med preslikavami; takrat bomo takšno funkcijo zapisali kot $\lam{x \in \RR} x^2$.
% \end{itemize}

% \note{Sklop (kompozicija, kompozitum) preslikav. Identiteta kot enota za sklapljanje. Razčlenitev (dekompozicija, faktorizacija) preslikav.}

% \davorin{Definirati moramo tudi oznako $\set{f(x)}{x \in X}$, kar je druge vrste oznaka kot prej definirana $\set{x \in X}{\phi(x)}$. Se gremo primerjavo s Pythonom (razlika med \texttt{\{f(x) for x in X\}} in \texttt{\{x if phi(x)\}})? Smo matematični hipsterji in uvedemo oznako $\{f(x) \,|\, x \in X \,|\, \phi(x)\}$, ki ustreza \texttt{\{f(x) for x in X if phi(x)\}}, kar bi tudi prišlo prav?}

% Zaenkrat smo imeli primere, ko je bil prepis preslikave dan z eno samo spremenljivko, npr.~$f(x) = x^2$. Zelo pogoste so pa tudi \df{preslikave več spremenljivk}, npr.~$f(x, y) = x^2 + y^2$. Že osnovne računske operacije so take --- na primer, pri seštevanju vzamemo \emph{dva} podatka in vrnemo rezultat (vsoto).

% V takem primeru je smiselno reči: domena preslikave sestoji iz \df{dvojic} ali \df{parov} števil. Pri seštevanju je to, katero število je prvo, katero pa drugo, sicer nepomembno, pri kakšni drugi operaciji (npr.~že odštevanju), pa je, zato posebej zahtevajmo: gre za \df{urejene dvojice} (\df{pare}). Urejeno dvojico elementov $a$ in $b$ (v tem vrstem redu) po dogovoru zapišemo kot $(a, b)$. Vrednosti $a$ in $b$ imenujemo \df{komponenti} tega para; natančneje, $a$ je \df{prva komponenta}, $b$ pa \df{druga komponenta}.

% Če imamo dve množici $A$ in $B$, tedaj množico vseh urejenih dvojic, katerih prva komponenta je element iz $A$, druga komponenta pa element iz $B$, označimo $A \times B$ in imenujemo \df{zmnožek} ali \df{produkt} množic $A$ in $B$. Glede na to, da obstaja mnogo operacij, ki se imenujejo ``produkt'' (poznate že vsaj produkt števil, produkt števila z vektorjem, skalarni produkt vektorjev in vektorski produkt vektorjev, obstaja pa jih še precej več), je koristno produkt množic posebej poimenovati, da ga ločimo od drugih: zanj se je uveljavil izraz \df{kartezični produkt} (izhaja iz imena Cartesius, tj.~latinske različice priimka Renéja Descarta\footnote{René Descartes (1596 -- 1650) je bil francoski filozof, matematik in znanstvenik.}).

% Seštevanje potemtakem lahko razumemo kot preslikavo $+\colon \RR \times \RR \to \RR$. V tem smislu še vedno gre za preslikavo, ki dan vhodni podatek preslika v neki rezultat, le da je vhodni podatek dvojica števil, ne pa zgolj eno število. Kadar imamo produkt več enakih faktorjev, ga lahko (kot običajno) zapišemo v obliki potence; pisali bi lahko tudi $+\colon \RR^2 \to \RR$.

% Seveda nismo omejeni na preslikave samo ene ali dveh spremenljivk. Nič nam ne preprečuje definirati recimo $f(x, y, z) = 2x + y - 3z$. Smiselna domena te preslikave setoji iz \df{urejenih trojic} števil. V splošnem, če jemljemo elemente iz množic $A$, $B$, $C$, tedaj se množica vseh takih trojic označi z $A \times B \times C$. Prejšnji predpis določa potem preslikavo $f\colon \RR \times \RR \times \RR \to \RR$ (oziroma krajše $f\colon \RR^3 \to \RR$).

% Spremenljivk je lahko še več; poleg dvojic in trojic tako dobimo še četverice, peterice, šesterice\ldots V splošnem takšna končna zaporedja elementov imenujemo \df{urejene večterice}. Tudi število spremenljivk je lahko označeno s črko; na primer, preslikava, ki računa povprečje $n$ števil (kjer $n \in \NN_{\geq 1}$), je dana kot
% \begin{align*}
% \RR^n &\to \RR \\
% (x_1, x_2, \ldots, x_n) &\mapsto \frac{x_1 + x_2 + \ldots + x_n}{n}
% \end{align*}
% (če hočemo poudariti, da imajo naše večterice natanko $n$ komponent, jih imenujemo $n$-terice). Nadlega pri tem je sicer spet dvoumnost tropičja. Deloma jo je možno odpraviti tako, da celotno večterico označimo z eno spremenljivko. Pogosta izbira zapisa je $f(\mathbf{x})$ ali $f(\vec{x})$ (razlog za to je, da lahko večterico vidimo kot vektor).

% Marsikdaj želimo delati ne samo z eno preslikavo, pač pa s celo množico preslikav naenkrat. Zato uvedemo: množica vseh preslikav, ki slikajo iz $X$ v $Y$, se označi kot $Y^X$; temu se reče \df{eksponent} množic $X$ in $Y$ (\note{na primernem mestu kasneje} bomo razložili, od kod ta oznaka).

% \begin{zgled}
% Množico vseh preslikav, ki realna števila slikajo nazaj v realna števila, označimo z $\RR^\RR$. Če nas zanimajo realne preslikave, ki so definirana samo na intervalu $\intoo{-1}{1}$, opazujemo množico $\RR^{\intoo{-1}{1}}$. Definiramo lahko preslikavo
% \begin{align*}
% \RR^{\intoo{-1}{1}} &\to \RR \\
% f &\mapsto f(0),
% \end{align*}
% ki preslikavam priredi njihovo vrednost v točki $0$. Ta preslikava torej ima za argumente (tj.~vnose) celotne preslikave in ne števila! Sama po sebi je element množice $\RR^{\RR^{\intoo{-1}{1}}}$.
% \end{zgled}

% \begin{zgled}
% Za poljubne množice $A$, $B$, $C$ lahko definiramo sledečo preslikavo, katere argumenti so pari preslikav.
% \begin{align*}
% B^A \times C^B &\to C^A \\
% (f, g) &\mapsto g \circ f
% \end{align*}
% \end{zgled}


% \davorin{Glede na to, da gre za slovenski učbenik, dajem izrazu `preslikava' prednost pred izrazom `funkcija'. Seveda pa sem pojasnil tudi slednji izraz (v prvem poglavju).}

% \note{Uvod. Definicijsko območje in zaloga vrednosti \davorin{morda dodamo kot možno ime za zalogo vrednosti še prevod angleške besede `range', se pravi `razpon'?}. Zožitve (tako domene kot kodomene); oznake za to so $\rstr{f}_A$, $\rstr{f}^B$, $\rstr{f}_A^B$. Izvrednotenje (evalvacija) preslikave (če ne bomo tega pojasnili že pri eksponentih množic).}


\section{Zmnožek}
\label{sec:kartezicni-produkt}

Množice lahko \df{tvorimo} ali \df{konstruiramo} iz drugih množic na različne načine. V
tem poglavju bomo spoznali tri osnovne konstrukcije, ostale pa kasneje, ko bomo že nekaj
vedeli o logiki. Najprej obravnavajmo kartezični produkt ali zmnožek množic.

Takoj se zastavi vprašanje, kako sploh opisati novo konstrukcijo množic. Načelo
ekstenzionalnosti pove, da je množica opredeljena s svojimi elementi. Torej moramo
pojasniti, kaj so elementi nove množice, se pravi, kako jih vpeljemo, kaj lahko z njimi
počnemo in kakšne so njihove zakonitosti. Natančneje, novo konstrukcijo množic
določajo naslednja pravila:
%
\begin{enumerate}
\item pravilo \df{tvorbe}, ki vpelje novo množico,
\item pravila \df{vpeljave} elementov, ki podajo operacije, s katerimi gradimo elemente,
\item pravila \df{uporabe}, ki podajo opreacije, s katerimi razgradimo elemente,
\item \df{enačbe}, ki veljajo med temi operacijami.
\end{enumerate}
%
Najbolje je, da si postopek ogledamo na primeru.

\begin{pravilo}[Tvorba zmnožka]
  \label{pravilo:zmnozek-tvorba}
  Za vsaki množici $A$ in $B$ je $A \times B$ množica, ki se imenuje \df{zmnožek} ali
  \df{kartezični produkt} $A$ in $B$.
\end{pravilo}

\noindent
%
Pravilo tvorbe pove, da lahko tvorimo novo množico $A \times B$, ne pove pa, kakšne
elemente ima, kar je vsebina naslednjih dveh pravil, ki povesta, kako sestavimo in
razstavimo elemente zmnožka.

\begin{pravilo}[Vpeljava urejenih parov]
  \label{pravilo:zmnozek-vpeljava}
  %
  Za vse $a \in A$ in $b \in B$ je $(a, b) \in A \times B$. Element $(a, b)$ imenujemo
  \df{urejeni par}.
\end{pravilo}

\begin{pravilo}[Uporaba urejenih parov]
  \label{pravilo:zmnozek-uporaba}
    %
  Za vsak $p \in A \times B$ je $\fst(p) \in A$ \df{prva projekcija} in $\snd(p) \in B$
  \df{druga projekcija} elementa~$p$.
\end{pravilo}

Nazadnje podamo še enačbe.

\begin{pravilo}[Računsko pravilo za urejene pare]
  \label{pravilo:zmnozek-racunanje}
  Za vse $a \in A$, $b \in B$ velja $\fst(a, b) = a$ in $\snd(a, b) = b$.
\end{pravilo}

\begin{pravilo}[Ekstenzionalnost urejenih parov]
  \label{pravilo:zmnozek-ekstenzionalnost}
  Za vse $p, q \in A \times B$ velja: če $\fst(p) = \fst(q)$ in $\snd(p) = \snd(q)$,
  potem $p = q$.
\end{pravilo}

\noindent
%
Računsko pravilo se tako imenuje, ker lahko z njim poenostavljamo izraze, drugo pa je
načelo ekstenzionalnosti, ker pravi, da je urejeni par določen s prvo in drugo projekcijo.

Malo bolj naivna konstrukcija zmnožka bi se glasila takole: kartezični produkt
$A \times B$ je množica vseh urejenih parov $(a, b)$, kjer je $a \in A$ in $b \in B$. A
taka konstrukcija ni popolna, saj ne pove, kaj lahko z urejenim parom počnemo. Kako naj
vemo, da iz $(a, b)$ lahko izluščimo $a$ in $b$, in kako preverimo, ali sta dva urejena
para enaka? Če takih zadev ne določimo, bi lahko kdo mislil, da je urejeni par kaka druga
neželena operacija, seštevanje, unija, ali kdovekaj se mota po človeških glavah.

\begin{trditev}
  Vsi elementi kartezičnega produkta so urejeni pari.
\end{trditev}

\begin{proof}
  Ali je trditev sploh treba dokazovati? Vsekakor, saj pravila govorijo le o tem, da
  urejeni pari so elementi kartezičnega produkta, ne pa tudi, da so to edini elementi.
  Spet bomo napisali zelo podroben dokaz.

  Naj bosta $A$ in $B$ množici in $p \in A \times B$. Dokazati moramo, da je $p$ enak
  urejenemu paru, se pravi, da iščemo taka $a \in A$ in $b \in B$, da velja $p = (a, b)$.
  Vzemimo $a = \fst(p)$ in $b = \snd(p)$ term preverimo, da velja $p = (a, b)$. V ta namen
  uporabimo načelo ekstenzionalnosti za pare, ki nam da želeni zaključek, če dokažemo
  %
  \begin{equation*}
    \fst(p) = \fst(a, b)
    \qquad\text{in}\qquad
    \snd(p) = \snd(a, b).
  \end{equation*}
  %
  Prva enačba velja, ker uporabimo definicijo $a$ in $b$ ter računska pravila:
  %
  \begin{equation*}
    \fst(a,b) = \fst(\fst(p), \snd(p)) = \fst(p) = a
  \end{equation*}
  %
  Prav tako velja druga enačba:
  %
  \begin{equation*}
    \snd(a,b) = \snd(\fst(p), \snd(p)) = \snd(p) = b. \qedhere
  \end{equation*}
  %
\end{proof}

Tako podrobne dokaze je težko brati, bolje bi bilo, če bi jih preverili kar z
računalnikom. Bistvo dokaza bi lahko izrazili z eno samo opazko: vsak element
$p \in A \times B$ je urejeni par, namreč $p = (\fst(p), \snd(p))$.






\section{Vsota}
\label{sec:vsota}

Vsota množic, pri čemer vpeljemo primitivne oznake za injekcije (kar je dosti bolj
smiselno, kot da vedno uporabljamo $\inl$ in $\inr$). Kadar smo leni, uporabljamo $\inl$
in $\inr$.

Elementi $A + A$. Kasneje bomo spoznali tudi operacijo unija, ki jo tudi že poznamo.

Pravila za vsoto množic. Notacija za preslikave, definirane na vsoti, $f : A + B \to C$:
%
\begin{align*}
  f(\inl(x)) &= \cdots x \cdots \\
  f(\inr(y)) &= \cdots y \cdots
\end{align*}
%
in tudi
%
\begin{align*}
  f(z) =
  \begin{cases}
    \cdots x \cdots & \text{če $z = \inl(x)$,} \\
    \cdots y \cdots & \text{če $z = \inl(y)$.}
  \end{cases}
\end{align*}

\section{Eksponent}
\label{sec:eksponent}



Zapis funkcijskega predpisa na razne načine.
\note{Tj.~anonimne oz.~čiste funkcije. Na tem mestu pride tudi $\lambda$-notacija in določena mera $\lambda$-računa.}

Aplikacija kot operacija.

Eksponent kot množica vseh preslikav. Pravila za eksponent: kako naredimo elemente
eksponenta in kako jih uporabimo. Enačbe ($\beta$ in ekstenzionalnost).

Vezane spremenljivke v zapisu funkcijskega predpisa. Substitucija in vezane spremenljivke.

Preslikave iz prazne množice in v prazno množice. Preslikave iz enojca in v enojec.


Transponiranje (currying).


\section{Izomorfizmi med množicami}
\label{sec:izom-med-mnoic}

Definicija inverza in izomorfizma. \davorin{inverz = obrat}

\davorin{Hm\ldots Pa poskusimo\ldots\\ izomorfizem = istoličje,\\ morfizem = ličje.\\ Dobesedno prevedeno je homomorfizem = enakoličje, ampak to ni najboljši prevod v tem kontekstu. Predlagam\\ homomorfizem = naličje,\\ ker `naličen' (med drugim) pomeni `podoben', `soroden'.}

Če je preslikava izomorfizem, je tudi njen inverz.

Izomorfnost množic. Je refleksivna, simetrična in tranzitivna (a ne poudarjamo preveč pojma ekvivalence na tem mestu).

Izomorfnost je kongruenca za produkt, vsoto in eksponent.

\section{Aritmetika Tarskega}
\label{sec:aritmetika-tarskega}

Aritmetični zakoni Tarskega za množice. Podobnost z običajno aritmetiko.

Pri asociativnosti produkta obravnavamo $A_1 \times A_2 \times \cdots \times A_n$ in
enojec kot produkt nič množic. Podobno za vsote.

Tu je treba pojasniti, zakaj pišemo $\unit$ za element $\one$.


\section{Kar je že Davorin napisal}

Interval realnih števil podamo s krajiščema intervala v oklepajih --- okrogli oklepaji ( ) označujejo odprtost intervala (krajišče ni vključeno v interval), oglati oklepaji [ ] pa zaprtost (krajišče je vključeno). Tako se npr.~interval realnih števil od $0$ do $1$, ki ne vsebuje krajišč, označi z $(0, 1)$, če jih vsebuje, pa z $[0, 1]$.

Včasih pridejo prav tudi intervali na drugih množicah kot $\RR$. Zato se dogovorimo, da bomo intervale označevali tako, da podamo množico, ob kateri v indeksu zapišemo krajišči v oklepajih, npr.~$\intco[\NN]{1}{5} = \set{1, 2, 3, 4}$. Realna intervala iz prejšnjega odstavka tako zapišemo kot $\intoo{0}{1}$ in $\intcc{0}{1}$.

Če interval v katero smer gre v nedogled, preprosto zapišemo množico z ustreznim simbolom za urejenost in krajiščem v indeksu. Na primer, $\RR_{> 0}$ označuje množico pozitivnih realnih števil, $\RR_{\geq 0}$ pa množico nenegativnih realnih števil.

Primerjave med elementi, kot npr.~pravkar podani $>$ in $\geq$, imenujemo \df{relacije} (podrobneje jih bomo spoznali v poglavju~\ref{poglavje:relacije}). Zgornji zapis bomo uporabljali tudi za druge vrste relacij, ne samo za relacije urejenosti. Na primer, množico vseh neničelnih realnih števil zapišemo kot $\RR_{\neq 0}$.

\davorin{To bi vsaj bil moj predlog. Na ta način se izognemo dvoumnostim (kar je namen). Na primer, kaj pomeni $\forall\, a > 0$? Če zapišemo $\forall\, a \in \NN_{> 0}$ ali $\forall\, a \in \RR_{> 0}$, je jasno. Razlog, da matematiki ``goljufajo'' in pridejo skozi brez tega, je (napol dogovorjena in ponotranjena, ampak arbitrarna) izbira črk; vsak izkušen matematik ve, da $\forall\, \epsilon > 0$ pomeni $\forall\, \epsilon \in \RR_{> 0}$. Dodaten problem je, da kasneje uporabljamo urejene pare, ki jih vsi na naši fakulteti pišejo z okroglimi oklepaji. Poskusimo se izogniti zmedi, ali $(a, b)$ pomeni urejeni par ali odprti interval. Če se ne strinjate, popravite in pustite komentar.}

Če imamo dan neki element in neko množico, potem pripadnost tega elementa tej množici izrazimo s simbolom $\in$. Na primer, da je štiri naravno število, zapišemo $4 \in \NN$ (beri: ``štiri pripada množici naravnih števil'').

Elementi množic lahko zadoščajo raznim lastnostim. Na primer, recimo, da $\phi$ označuje lastnost ``biti manj od pet''; to potem zapišemo
\[\phi(x) \ = \ \ x < 5.\]
V tem primeru $x$ imenujemo \df{spremenljivka}, saj ne gre za točno določeno vrednost, pač pa predstavlja splošno število (recimo, da se dogovorimo, da s $\phi$ označujemo lastnost na realnih številih).

Tovrstne lastnosti nam omogočajo, da iz neke množice odberemo elemente z dano lastnostjo in na ta način dobimo novo množico, ki je podmnožica prejšnje. Množico vseh realnih števil, ki so manjša od pet, zapišemo na naslednji način.
\[\set{x \in \RR}{x < 5}\]
Seveda, ker je primerjava s števili zelo pogosta lastnost, je uporabno, če uvedemo krajše oznake, ki povejo isto; že prej smo se dogovorili, da tako množico označimo z $\RR_{< 5}$. Za povsem splošne lastnosti pa ne bomo imeli vnaprej dogovorjenih oznak, zato je dobro, da poznamo splošni zapis. Torej, če je $X$ poljubna množica in $\phi$ poljubna lastnost njenih elementov, tedaj podmnožico, ki vsebuje točno tiste elemente množice $X$, ki zadoščajo lastnosti $\phi$, označimo takole.
\[\set[1]{x \in X}{\phi(x)}\]

Pri tem se zavedajmo: ni pomembno, da spremenljivko označimo ravno z $x$. Zapis
\[\set[1]{y \in X}{\phi(y)}\]
še vedno označuje isto množico. V vsakem primeru gre za množico vseh elementov iz $X$ z lastnostjo $\phi$. Pravzaprav sploh ni nujno, da uporabimo črko; poslužimo se lahko kateregakoli simbola (ki mu nismo predtem že predpisali določenega pomena). Taisto množico lahko zapišemo tudi $\set{\heartsuit \in X}{\phi(\heartsuit)}$.

Kadar imamo spremenljivko, ki jo lahko preimenujemo, ne da bi spremenili pomen izraza, jo imenujemo \df{nema spremenljivka}. Takšne primere že dobro poznate; na primer, integral $\int_0^1 x^2 \,dx$ se ne spremeni, če preimenujete spremenljivko in zapišete $\int_0^1 y^2 \,dy$.

\begin{zgled}
Kako bi zapisali množico vseh sodih naravnih števil? Spomnimo se, da je število sodo, kadar je deljivo z $2$. Za $n \in \NN$ to zapišemo takole: $2 \divides n$ (beri: ``dve deli $n$''). Množica sodih naravnih števil se potem zapiše kot
\[\set[1]{n \in \NN}{2 \divides n}.\]
\end{zgled}


\section{Vaje}

\begin{vaja}
  Kaj veste povedati o množici $A$, če zanjo velja, da so vsi njeni elementi enaki?
\end{vaja}


%%% Local Variables:
%%% mode: latex
%%% TeX-master: "ucbenik-lmn"
%%% End:

\chapter{Logika}\label{POGLAVJE: Logika}

	\note{uvod}
	
	
	\section{Izjavni vezniki}
	
		V razdelku~\ref{RAZDELEK: Logični simboli} smo omenili nekaj izjavnih veznikov, podali oznake zanje in opisali njihov intuitivni pomen. Ampak če se hočemo zanašati na pravilnost naših sklepov, moramo tem oznakam dati \emph{formalni matematični pomen}.
		
		Če imamo neko izjavo, lahko določimo njeno resničnost, tj.~povemo, do kolikšne mere je resnična. Temu rečemo \df{resničnostna vrednost} izjave. Množico vseh možnih resničnostnih vrednosti označimo z $\tvs$. Seveda ni kaj dosti možnih resničnostnih vrednosti: to sta \df{resnica} (dogovorimo se, da bomo zanjo uporabljali oznako $\true$) in \df{neresnica} (oznaka $\false$). Se pravi, $\tvs = \set{\true, \false}$.
		
		\begin{opomba}
			Logiki, kjer sta edini resničnostni vrednosti resnica in neresnica, rečemo \df{dvovrednostna} oziroma \df{klasična logika}. Obstajajo splošnejše vrste logike, kjer je $\set{\true, \false}$ prava podmnožica $\tvs$, ampak v tej knjigi se bomo omejili na klasično logiko, na katero ste navajeni in ki se uporablja v večjem delu matematike.
		\end{opomba}
		
		\davorin{Kako izrecno bomo ločevali med izjavami in njihovimi logičnimi vrednostmi?}
		
		Izjavne veznike lahko potem formalno podamo kot preslikave. Na primer, negacija je preslikava $\lnot\colon \tvs \to \tvs$ (vsaki resničnostni vrednosti pripišemo njeno nasprotno vrednost). Preslikavo, definirano na majhni končni množici, lahko preprosto podamo s tabelo vseh njenih vrednosti. V primeru izjavnih veznikov takim tabelam rečemo \df{resničnostne tabele}. Resničnostna tabela za negacijo izgleda takole.
		\begin{center}
			\begin{tabular}{c|c}
				$p$ & $\lnot{p}$ \\
				\hline
				$\true$ & $\false$ \\
				$\false$ & $\true$
			\end{tabular}
		\end{center}
		Ta tabela povsem natančno definira negacijo kot preslikavo $\lnot\colon \tvs \to \tvs$. Seveda smo negacijo definirali tako, kot bi pričakovali: negacija resnice je neresnica, negacija neresnice je resnica.
		
		Podobno lahko naredimo z ostalimi izjavnimi vezniki, le da preostali vežejo dve izjavi. Se pravi, npr.~konjunkcija vzame dve resničnostni vrednosti in vrne resničnostno vrednost, ki pove, ali sta obe dani vrednosti resnični. Konjunkcijo lahko torej interpretiramo kot preslikavo $\land\colon \tvs \times \tvs \to \tvs$ (ali na kratko $\land\colon \tvs^2 \to \tvs$).
		
		V splošnem definiramo, da je \df{$n$-mestni izjavni veznik} preslikava oblike $\tvs^n \to \tvs$. Negacija je torej enomestni izjavni veznik, ostali vezniki, ki smo jih do zdaj omenili, pa so dvomestni.
		
		Definirajmo zdaj konjunkcijo natančno preko resničnostne tabele. Množica $\tvs \times \tvs$ ima štiri elemente --- vse možne pare, sestavljene iz $\true$ oz.~$\false$. Intuitivni pomen konjunkcije razumemo: konjunkcija dveh izjav je resnična natanko tedaj, ko sta obe izjavi resnični. To nas vodi do naslednje tabele.
		\begin{center}
			\begin{tabular}{cc|c}
				$p$ & $q$ & $p \land q$ \\
				\hline
				$\true$ & $\true$ & $\true$ \\
				$\true$ & $\false$ & $\false$ \\
				$\false$ & $\true$ & $\false$ \\
				$\false$ & $\false$ & $\false$
			\end{tabular}
		\end{center}
		
		Za disjunkcijo smo že rekli, da pride v dveh različicah: navadna pomeni, da vsaj ena od izjav velja, izključitvena pa pomeni, da velja natanko ena od izjav. Posledično je torej smiselno definirati funkciji $\lor, \xor\colon \tvs \times \tvs \to \tvs$ na sledeči način.
		\begin{center}
			\begin{tabular}{cc|cc}
				$p$ & $q$ & $p \lor q$ & $p \xor q$ \\
				\hline
				$\true$ & $\true$ & $\true$ & $\false$ \\
				$\true$ & $\false$ & $\true$ & $\true$ \\
				$\false$ & $\true$ & $\true$ & $\true$ \\
				$\false$ & $\false$ & $\false$ & $\false$
			\end{tabular}
		\end{center}
		Bodi pozoren na razliko med zadnjima dvema stolpcema!
		
		Obenem lahko še na hitro opravimo z veznikoma $\shf$ in $\luk$. Spomnimo se, da $p \shf q$ pomeni \qt{ne hkrati $p$ in $q$}, medtem ko $p \luk q$ pomeni \qt{niti $p$ niti $q$}.
		\begin{center}
			\begin{tabular}{cc|cc}
				$p$ & $q$ & $p \shf q$ & $p \luk q$ \\
				\hline
				$\true$ & $\true$ & $\false$ & $\false$ \\
				$\true$ & $\false$ & $\true$ & $\false$ \\
				$\false$ & $\true$ & $\true$ & $\false$ \\
				$\false$ & $\false$ & $\true$ & $\true$
			\end{tabular}
		\end{center}
		
		Implikacija je nekoliko bolj subtilna. Kaj točno trdimo z izjavo $p \impl q$, se pravi, kakor hitro velja $p$, mora veljati tudi $q$? No, če $p$ ne velja, potem sploh nismo postavili nobenega pogoja --- izjava je avtomatično izpolnjena. Če $p$ velja, pa zraven zahtevamo še $q$. Resničnostna tabela za implikacijo je potemtakem sledeča.
		\begin{center}
			\begin{tabular}{cc|c}
				$p$ & $q$ & $p \impl q$ \\
				\hline
				$\true$ & $\true$ & $\true$ \\
				$\true$ & $\false$ & $\false$ \\
				$\false$ & $\true$ & $\true$ \\
				$\false$ & $\false$ & $\true$
			\end{tabular}
		\end{center}
		
		Ekvivalenca je spet preprosta --- izjavi sta ekvivalentni, kadar imata isto resničnostno vrednost. Od tod dobimo sledečo resničnostno tabelo.
		\begin{center}
			\begin{tabular}{cc|c}
				$p$ & $q$ & $p \lequ q$ \\
				\hline
				$\true$ & $\true$ & $\true$ \\
				$\true$ & $\false$ & $\false$ \\
				$\false$ & $\true$ & $\false$ \\
				$\false$ & $\false$ & $\true$
			\end{tabular}
		\end{center}
		
		Za lažjo referenco zberimo resničnostne tabele vseh do zdaj omenjenih veznikov na eno mesto (tabela~\ref{TABELA: Resničnostna tabela osnovnih izjavnih veznikov}).
		
		\begin{table}[!ht]
			\centering
			\begin{tabular}{c|c}
				$p$ & $\lnot{p}$ \\
				\hline
				$\true$ & $\false$ \\
				$\false$ & $\true$
			\end{tabular}
			\qquad\quad
			\begin{tabular}{cc|ccccccc}
				$p$ & $q$ & $p \land q$ & $p \lor q$ & $p \xor q$ & $p \shf q$ & $p \luk q$ & $p \impl q$ & $p \lequ q$ \\
				\hline
				$\true$ & $\true$ & $\true$ & $\true$ & $\false$ & $\false$ & $\false$ & $\true$ & $\true$ \\
				$\true$ & $\false$ & $\false$ & $\true$ & $\true$ & $\true$ & $\false$ & $\false$ & $\false$ \\
				$\false$ & $\true$ & $\false$ & $\true$ & $\true$ & $\true$ & $\false$ & $\true$ & $\false$ \\
				$\false$ & $\false$ & $\false$ & $\false$ & $\false$ & $\true$ & $\true$ & $\true$ & $\true$
			\end{tabular}
			\caption{Resničnostna tabela osnovnih izjavnih veznikov}\label{TABELA: Resničnostna tabela osnovnih izjavnih veznikov}
		\end{table}
		
		Zdaj ko imamo natančno definicijo izjavnih veznikov, lahko trditve v zvezi z njimi tudi formalno utemeljimo. Na primer, spomnimo se, da smo že malo po omembi veznikov $\xor$, $\shf$, $\luk$ podali njihovo izražavo z vezniki $\lnot$, $\land$, $\lor$. Če na glas preberemo vse izjave, nam je intuitivno jasno, katere se ujemajo in zakaj, ampak zdaj lahko dejansko preverimo, da te izražave veljajo.
		
		Na primer, kaj pomeni, da se $p \luk q$ lahko izrazi kot $\lnot(p \lor q)$? To pomeni, da sta funkciji $\tvs \times \tvs \to \tvs$, dani s predpisoma $(p, q) \mapsto p \luk q$ in $(p, q) \mapsto \lnot(p \lor q)$, enaki. (Slednja funkcija je sestavljena, tj.~sklop dveh funkcij. Lahko bi tudi zapisali, da velja $\luk = \lnot \circ \lor$.) Funkciji z isto domeno in kodomeno sta enaki, kadar pri vsakem argumentu vrneta isti vrednosti, kar v našem primeru pomeni, da imata enaka stolpca v resničnostni tabeli. Poračunajmo torej vse izraze v danih izražavah. Ko dobimo enake rezultate, bomo vedeli, da izražave dejansko veljajo.
		
		\begin{center}
			\begin{tabular}{cc|cccccc}
				$p$ & $q$ & $p \shf q$ & $p \land q$ & $\lnot(p \land q)$ & $\lnot{p}$ & $\lnot{q}$ & $\lnot{p} \lor \lnot{q}$ \\
				\hline
				$\true$ & $\true$ & $\efalse$ & $\true$ & $\efalse$ & $\false$ & $\false$ & $\efalse$ \\
				$\true$ & $\false$ & $\etrue$ & $\false$ & $\etrue$ & $\false$ & $\true$ & $\etrue$ \\
				$\false$ & $\true$ & $\etrue$ & $\false$ & $\etrue$ & $\true$ & $\false$ & $\etrue$ \\
				$\false$ & $\false$ & $\etrue$ & $\false$ & $\etrue$ & $\true$ & $\true$ & $\etrue$
			\end{tabular}
		\end{center}
		
		\begin{center}
			\begin{tabular}{cc|cccccc}
				$p$ & $q$ & $p \luk q$ & $p \lor q$ & $\lnot(p \lor q)$ & $\lnot{p}$ & $\lnot{q}$ & $\lnot{p} \land \lnot{q}$ \\
				\hline
				$\true$ & $\true$ & $\efalse$ & $\true$ & $\efalse$ & $\false$ & $\false$ & $\efalse$ \\
				$\true$ & $\false$ & $\efalse$ & $\true$ & $\efalse$ & $\false$ & $\true$ & $\efalse$ \\
				$\false$ & $\true$ & $\efalse$ & $\true$ & $\efalse$ & $\true$ & $\false$ & $\efalse$ \\
				$\false$ & $\false$ & $\etrue$ & $\false$ & $\etrue$ & $\true$ & $\true$ & $\etrue$
			\end{tabular}
		\end{center}
		
		\begin{center}
			\begin{tabular}{cc|ccccc}
				$p$ & $q$ & $p \xor q$ & $p \lor q$ & $p \land q$ & $\lnot(p \land q)$ & $(p \lor q) \land \lnot(p \land q)$  \\
				\hline
				$\true$ & $\true$ & $\efalse$ & $\true$ & $\true$ & $\false$ & $\efalse$ \\
				$\true$ & $\false$ & $\etrue$ & $\true$ & $\false$ & $\true$ & $\etrue$ \\
				$\false$ & $\true$ & $\etrue$ & $\true$ & $\false$ & $\true$ & $\etrue$ \\
				$\false$ & $\false$ & $\efalse$ & $\false$ & $\false$ & $\true$ & $\efalse$
			\end{tabular}
		\end{center}
		
		\begin{center}
			\begin{tabular}{cc|cccccc}
				$p$ & $q$ & $p \xor q$ & $\lnot{q}$ & $p \land \lnot{q}$ & $\lnot{p}$ & $\lnot{p} \land q$ & $(p \land \lnot{q}) \lor (\lnot{p} \land q)$  \\
				\hline
				$\true$ & $\true$ & $\efalse$ & $\false$ & $\false$ & $\false$ & $\false$ & $\efalse$ \\
				$\true$ & $\false$ & $\etrue$ & $\true$ & $\true$ & $\false$ & $\false$ & $\etrue$ \\
				$\false$ & $\true$ & $\etrue$ & $\false$ & $\false$ & $\true$ & $\true$ & $\etrue$ \\
				$\false$ & $\false$ & $\efalse$ & $\true$ & $\false$ & $\true$ & $\false$ & $\efalse$
			\end{tabular}
		\end{center}
		
		Kako simbolno zapisati, da sta dve izražavi enaki? Lahko bi pisali
		\[\big((p, q) \mapsto p \shf q\big) = \big((p, q) \mapsto \lnot(p \land q)\big),\]
		ampak to je nekoliko nerodno in nepregledno. Kasneje (v razdelku~\note{o anonimnih funkcijah}) se bomo naučili $\lambda$-notacijo, s katero dobimo
		\[\xlam{(p, q)}[\tvs^2]{p \shf q} = \xlam{(p, q)}[\tvs^2]{\lnot(p \land q)},\]
		ampak to je še vedno nepregledno. Uveljavil se je običaj, da se izraze, ki so enakovredni v smislu, da dajo isti rezultat pri vsaki izbiri argumentov, poveže s simbolom $\equiv$, torej zapišemo
		\[p \shf q \equiv \lnot(p \land q).\]
		Konkretno za izraze v logiki se uporablja tudi $\sim$, se pravi, zapišemo lahko tudi
		\[p \shf q \sim \lnot(p \land q).\]
		V tej knjigi se bomo držali uporabe simbola $\equiv$. \davorin{Recimo. Po mojem je to boljše, ker lahko $\equiv$ uporabljamo še za druge funkcije (npr.~$f(x) \equiv 0$ pomeni, da je $f$ konstantno enaka $0$, medtem ko $f(x) = 0$ predstavlja enačbo, s katero iščemo ničle funkcije) in ker bomo kasneje $\sim$ uporabljali za ekvivalenčne relacije.}
		
		Med drugim smo s temi tabelami izpeljali tako imenovana \df{de Morganova zakona} za izjavno logiko \davorin{Verjetno je smiselno specificirati \qt{za izjavno logiko}. Imeli bomo namreč še zakona za predikatno logiko (za $\forall$ in $\exists$) ter za množice (za preseke in unije).}, ki povesta, kako negacija vpliva na konjunkcijo in disjunkcijo:
		\[\lnot(p \land q) \equiv \lnot{p} \lor \lnot{q},\]
		\[\lnot(p \lor q) \equiv \lnot{p} \land \lnot{q}.\]
		To je smiselno: kadar ni res, da veljata oba $p$ in $q$, vsaj eden od njiju ne velja. Kadar ni res, da velja vsaj eden od njiju, nobeden od njiju ne velja.
		
		Z resničnostnimi tabelami lahko preverimo še mnoge druge formule. \df{Zakon dvojne negacije} pravi $\lnot\lnot{p} \equiv p$, tj.~če dvakrat zanikamo izjavo, dobimo izjavo, enakovredno začetni. Poračunajmo tabelo.
		
		\begin{center}
			\begin{tabular}{c|ccc}
				$p$ & $\lnot{p}$ & $\lnot\lnot{p}$ & $p$ \\
				\hline
				$\true$ & $\false$ & $\etrue$ & $\etrue$ \\
				$\false$ & $\true$ & $\efalse$ & $\efalse$
			\end{tabular}
		\end{center}
		
		Spomnimo se: za poljubno dvomestno operacijo $\oper$ na neki množici $X$ rečemo, da je
		\begin{itemize}
			\item
				\df{izmenljiva} ali \df{komutativna}, kadar velja $a \oper b = b \oper a$ za vse $a, b \in X$ (na kratko: $a \oper b \equiv b \oper a$),
			\item
				\df{družitvena} \davorin{ne spomnim se --- kako se že temu reče po slovensko?} ali \df{asociativna}, kadar velja $(a \oper b) \oper c = a \oper (b \oper c)$ za vse $a, b, c \in X$ (na kratko: $(a \oper b) \oper c \equiv a \oper (b \oper c)$),
			\item
				\df{idempotentna} \davorin{a imamo slovenski izraz za to?}, kadar velja $a \oper a = a$ za vse $a \in X$ (torej $a \oper a \equiv a$).
		\end{itemize}
		
		Preverimo z resničnostno tabelo, da je konjunkcija komutativna, torej $p \land q \equiv q \land p$.
		
		\begin{center}
			\begin{tabular}{cc|ccccc}
				$p$ & $q$ & $p \land q$ & $q \land p$ \\
				\hline
				$\true$ & $\true$ & $\etrue$ & $\etrue$ \\
				$\true$ & $\false$ & $\efalse$ & $\efalse$ \\
				$\false$ & $\true$ & $\efalse$ & $\efalse$ \\
				$\false$ & $\false$ & $\efalse$ & $\efalse$
			\end{tabular}
		\end{center}
		
		Še hitreje lahko preverimo, da je konjunkcija idempotentna.
		
		\begin{center}
			\begin{tabular}{c|cc}
				$p$ & $p \land p$ & $p$ \\
				\hline
				$\true$ & $\etrue$ & $\etrue$ \\
				$\false$ & $\efalse$ & $\efalse$
			\end{tabular}
		\end{center}
		
		Kako pa preveriti, da je konjunkcija asociativna, torej $(p \land q) \land r \equiv p \land (q \land r)$? Vidimo, da v teh izrazih nastopajo tri spremenljivke in torej potrebujemo resničnostno tabelo, kjer upoštevamo vseh osem možnosti za izbiro $p$, $q$, $r$.
		
		\begin{center}
			\begin{tabular}{ccc|cccc}
				$p$ & $q$ & $r$ & $p \land q$ & $(p \land q) \land r$ & $q \land r$ & $p \land (q \land r)$ \\
				\hline
				$\true$ & $\true$ & $\true$ & $\true$ & $\etrue$ & $\true$ & $\etrue$ \\
				$\true$ & $\true$ & $\false$ & $\true$ & $\efalse$ & $\false$ & $\efalse$ \\
				$\true$ & $\false$ & $\true$ & $\false$ & $\efalse$ & $\false$ & $\efalse$ \\
				$\true$ & $\false$ & $\false$ & $\false$ & $\efalse$ & $\false$ & $\efalse$ \\
				$\false$ & $\true$ & $\true$ & $\false$ & $\efalse$ & $\true$ & $\efalse$ \\
				$\false$ & $\true$ & $\false$ & $\false$ & $\efalse$ & $\false$ & $\efalse$ \\
				$\false$ & $\false$ & $\true$ & $\false$ & $\efalse$ & $\false$ & $\efalse$ \\
				$\false$ & $\false$ & $\false$ & $\false$ & $\efalse$ & $\false$ & $\efalse$
			\end{tabular}
		\end{center}
		
		To pomeni, da lahko v izrazih, kjer nastopa več zaporednih konjunkcij, spuščamo oklepaje: namesto $p \land (\lnot{q} \land r)$ pišemo kar $p \land \lnot{q} \land r$.
		
		Enako velja tudi za disjunkcijo.
		
		\begin{vaja}
			Dokaži, da je disjunkcija komutativna, asociativna in idempotentna!
		\end{vaja}
		
		Preostali dvomestni vezniki, ki smo jih omenili, ne zadoščajo vsem trem lastnostim naenkrat.
		
		\begin{vaja}
			Preveri, kateri znani dvomestni izjavni vezniki so komutativni, asociativni oziroma idempotentni!
		\end{vaja}
		
		Ko rešite zgornjo vajo, boste med drugim opazili: implikacija ni komutativna. To pomeni, da lahko definiramo nov izjavni veznik $\revimpl$ na naslednji način: $p \revimpl q \dfeq q \impl p$ za vse $p, q \in \tvs$. Z drugimi besedami, $\revimpl$ je dan s sledečo resničnostno tabelo.
		\begin{center}
			\begin{tabular}{cc|c}
				$p$ & $q$ & $p \revimpl q$ \\
				\hline
				$\true$ & $\true$ & $\true$ \\
				$\true$ & $\false$ & $\true$ \\
				$\false$ & $\true$ & $\false$ \\
				$\false$ & $\false$ & $\true$
			\end{tabular}
		\end{center}
		
		\note{dokazi s pomočjo resničnostnih tabel še vseh ostalih formul, ki jih hočemo imeti, med drugim distributivnosti}
		
		Do zdaj smo omenili zgolj nekaj posamičnih izjavnih veznikov. Koliko pa je vseh skupaj? Spomnimo se, da je $n$-mestni izjavni veznik definiran kot preslikava $\tvs^n \to \tvs$. Množica $\tvs^n$ vsebuje vse urejene $n$-terice elementov $\true$ in $\false$; teh je $2^n$ (za vsako od $n$ mest v $n$-terici imamo dve možnosti in vse te izbire so neodvisne med sabo). Za vsako od teh $2^n$ večteric imamo dve možnosti, kam jo preslikamo: v $\true$ ali v $\false$. Vseh možnosti --- torej vseh $n$-mestnih veznikov --- je potemtakem $2^{2^n}$. (Vseh izjavnih veznikov, ko dopuščamo vse možne $n$, je seveda neskončno.)
		
		Za boljšo predstavo si oglejmo vse $n$-mestne veznike za majhne $n \in \NN$. Prva možnost je $n = 0$. Formula nam pravi, da je število ničmestnih izjavnih veznikov enako $2^{2^0} = 2^1 = 2$. Kaj pomeni, da pri nič vhodnih podatkih vrnemo $\true$ ali $\false$? To pomeni, da preprosto izberemo resničnostno vrednost --- z drugimi besedami, ničmestni izjavni vezniki so isto kot resničnostne vrednosti.
		
		Koliko je vseh enomestnih izjavnih veznikov? Formula pravi $2^{2^1} = 2^2 = 4$. Zapišimo vse možnosti.
		
		\begin{center}
			\begin{tabular}{c|cccc}
				$p$ &&&& \\
				\hline
				$\true$ & $\true$ & $\false$ & $\true$ & $\false$ \\
				$\false$ & $\true$ & $\false$ & $\false$ & $\true$
			\end{tabular}
		\end{center}
		
		Vidimo: vsi enomestni izjavni vezniki so obe konstantni funkciji v $\tvs$, identiteta na $\tvs$ in negacija.
		
		Kar se dvomestnih veznikov tiče, vidimo, da jih je $2^{2^2} = 2^4 = 16$.
		
		\begin{vaja}
			Preveri, da so vsi dvomestni vezniki natanko: konstanta z vrednostjo $\top$, projekcija na prvo komponento (tj.~$(p, q) \mapsto p$), projekcija na drugo komponento (tj.~$(p, q) \mapsto q$), konjunkcija $\land$, disjunkcija $\lor$, implikacija $\impl$, povratna implikacija $\revimpl$, ekvivalenca $\lequ$ in negacije vseh teh.
		\end{vaja}
		
		Tromestnih veznikov je že $2^{2^3} = 2^8 = 256$ in ne bomo vseh naštevali. Kako pa bi kakega dobili? Preprost način je, da vzamemo tri spremenljivke in jih združimo z večimi znanimi vezniki, na primer $(p, q, r) \mapsto p \land \lnot{q} \impl r$.\footnote{Načeloma sploh ni nujno, da vse tri spremenljivke dejansko uporabimo. Na primer, $(p, q, r) \mapsto p \land q$ še vedno podaja tromestni veznik, saj gre za preslikavo $\tvs^3 \to \tvs$.}
		
		Seveda se pojavi vprašanje, kako podati izjavne veznike, ki jih ne bi mogli sestaviti iz osnovnih. Izkaže se, da to ni problem: \emph{vsak veznik (ne glede na mestnost) je možno izraziti z osnovnimi}; pravzaprav zadostujejo že $\lnot$, $\land$ in $\lor$.
		
		Ideja je sledeča. Katerikoli izjavni veznik je oblike $V\colon \tvs^n \to \tvs$ in v celoti podan z resničnostno tabelo. Vzemimo konkreten primer; naj bo $V$ tromestni veznik, podan z naslednjo tabelo.
		
		\begin{center}
			\begin{tabular}{ccc|c}
				$p$ & $q$ & $r$ & $V(p, q, r)$ \\
				\hline
				$\true$ & $\true$ & $\true$ & $\false$ \\
				$\true$ & $\true$ & $\false$ & $\true$ \\
				$\true$ & $\false$ & $\true$ & $\true$ \\
				$\true$ & $\false$ & $\false$ & $\false$ \\
				$\false$ & $\true$ & $\true$ & $\true$ \\
				$\false$ & $\true$ & $\false$ & $\true$ \\
				$\false$ & $\false$ & $\true$ & $\false$ \\
				$\false$ & $\false$ & $\false$ & $\false$
			\end{tabular}
		\end{center}
		
		Tedaj lahko rečemo: $V$ je resničen tedaj, ko smo v 2., 3., 5.~ali 6.~vrstici. Kdaj smo v drugi vrstici? Točno tedaj, ko $p$ in $q$ veljata, $r$ pa ne, se pravi, ko velja $p \land q \land \lnot{r}$. Podobno naredimo še za preostale vrstice: tretja je določena s $p \land \lnot{q} \land r$, peta z $\lnot{p} \land q \land r$ in šesta z $\lnot{p} \land q \land \lnot{r}$. Potemtakem lahko zapišemo:
		\[V(p, q, r) \equiv (p \land q \land \lnot{r}) \lor (p \land \lnot{q} \land r) \lor (\lnot{p} \land q \land r) \lor (\lnot{p} \land q \land \lnot{r}).\]
		Temu rečemo \df{disjunktivna normalna oblika} (s kratico DNO) veznika $V$.
		
		Obstaja še dualna oblika take izražave. Lahko si rečemo tudi, da je $V$ resničen, kadar nismo v 1., 4., 7.~oz.~8.~vrstici. Kdaj nismo v prvi vrstici? Kadar niso vsi $p$, $q$, $r$ resnični, torej ko je vsaj eden od njih neresničen --- s formulo $\lnot{p} \lor \lnot{q} \lor \lnot{r}$. Kdaj nismo v četrti vrstici? Ko ni res, da je $p$ resničen, $q$ in $r$ pa ne, torej ko prekršimo vsaj enega teh pogojev, kar nam da formulo $\lnot{p} \lor q \lor r$. Podobno sklepamo, da nismo v sedmi vrstici, kadar velja $p \lor q \lor \lnot{r}$, in da nismo v osmi vrstici, kadar velja $p \lor q \lor r$. To nam da sledečo izražavo za $V$:
		\[V(p, q, r) \equiv (\lnot{p} \lor \lnot{q} \lor \lnot{r}) \land (\lnot{p} \lor q \lor r) \land (p \lor q \lor \lnot{r}) \land (p \lor q \lor r).\]
		Temu rečemo \df{konjunktivna normalna oblika} (s kratico KNO) veznika $V$.
		
		Spremenljivkam in njihovim negacijam z eno besedo rečemo \df{literali}. Disjunktivna normalna oblika je torej disjunkcija konjunkcij literalov, konjunktivna normalna oblika pa konjunkcija disjunkcij literalov.
		
		Iz tega primera je jasno, kako postopamo za poljuben izjavni veznik in zanj zapišemo DNO ali KNO. Opazimo: dolžina posamičnega člena, ki ga omejujejo oklepaji, je vedno enaka (vsebuje toliko literalov, kolikor je mestnost veznika), število teh členov pa razberemo iz stolpca, ki podaja vrednosti veznika v resničnostni tabeli. V primeru DNO je to število enako številu resnic $\true$, v primeru KNO pa številu neresnic $\false$. V zgornjem primeru sta bili DNO in KNO enako dolgi, ker smo imeli štiri $\true$ in $\false$, v splošnem pa se nam morda bolj splača uporabiti eno obliko kot drugo. Na primer, DNO implikacije se glasi $p \impl q \equiv (p \land q) \lor (\lnot{p} \land q) \lor (\lnot{p} \land \lnot{q})$, KNO pa je precej krajša: $p \impl q \equiv \lnot{p} \lor q$.
		
		Vidimo pa, da tu naletimo na problem: kaj se zgodi, če se katera resničnostna vrednost v stolpcu veznika sploh ne pojavi --- z drugimi besedami, kaj če je funkcija, ki podaja veznik, konstantna? Najprej dajmo takim ime: izjavni veznik, ki je pri vseh argumentih resničen, se imenuje \df{istorečje} ali \df{tavtologija}, izjavni veznik, ki je vedno neresničen, pa se imenuje \df{protislovje} ali \df{kontradikcija}.
		
		Za istorečje lahko vedno (ne glede na mestnost) zapišemo DNO (ki je sicer najdaljša možna), medtem ko bi KNO načeloma bila konjunkcija nič členov. Je to smiselno? V bistvu ja: če zahtevamo, da hkrati velja nič pogojev, je naša zahteva vedno izpolnjena. V tem smislu je konjunkcija nič členov enaka $\true$.
		
		Poglejmo podobne primere iz računstva. Kaj je vsota nič členov? Odgovor je seveda $0$. To je enota za seštevanje, kar je smiselno: če nič členom prištejemo en člen, moramo imeti zgolj ta člen. Podobno sklepamo: zmnožek nič členov je enota za množenje $1$ --- če nič faktorjem dodamo še en faktor, imamo skupaj zgolj ta faktor. Spomni se tudi: $a^0 = 1$ in $0! = 1$. To, da je ničkratna uporabe neke operacije enaka enoti za to operacijo, se izide tudi za konjunkcijo: dejansko velja $p \land \true \equiv p \equiv \true \land p$ (preveri z resničnostno tabelo!).
		
		Enak razmislek velja za protislovje. Zanj lahko zapišemo KNO na običajen način, medtem ko bi DNO bila disjunkcija nič členov. Smiselno je, da je disjunkcija nič členov enaka $\false$, tako zaradi tega, ker je $\false$ enota za disjunkcijo (preveri!), kot zaradi čisto intuitivnega razmisleka: kdaj je vsaj eden člen od nič členov resničen? Nikoli.
		
		Vseeno je nekoliko nerodno delati s konjunkcijo ali disjunkcijo nič členov --- kako točno bi to zapisali? Da velja $V(p_1, p_2, \ldots, p_n) \equiv $? Če nič ne zapišemo, kako sploh vemo, ali smo mislili na ničkratno konjunkcijo, disjunkcijo ali katerokoli drugo operacijo? Nekateri se zato preprosto dogovorijo, da ne dopuščajo ničkratnih operacij v DNO oz.~KNO in potem štejejo, da istorečja nimajo KNO, protislovja pa ne DNO.
		
		Tudi če ne dopuščamo ničkratnih operacij, pa še vedno velja: vsak izjavni veznik z mestnostjo vsaj $1$ ima vsaj eno od DNO oz.~KNO in ga torej lahko izrazimo samo z negacijo, konjunkcijo in disjunkcijo. Družini izjavnih veznikov, s katerimi lahko izrazimo vse veznike z mestnostjo vsaj $1$, rečemo \df{poln nabor}. Na kratko lahko torej rečemo, da je $\set{\lnot, \land, \lor}$ poln nabor.
		
		Jasno, če je neka množica veznikov poln nabor, je tudi vsaka njena nadmnožica poln nabor. Sledi, da je tudi na primer $\set{\lnot, \land, \lor, \impl}$ poln nabor.
		
		Spomnimo se zdaj de Morganovih zakonov in zakona o dvojni negaciji --- iz njih lahko izpeljemo $p \land q \equiv \lnot(\lnot{p} \lor \lnot{q})$ in $p \lor q \equiv \lnot(\lnot{p} \land \lnot{q})$. Se pravi, konjunkcijo lahko izrazimo z disjunkcijo in negacijo in prav tako lahko disjunkcijo izrazimo s konjunkcijo in negacijo. To pomeni, da sta že $\set{\lnot, \lor}$ in $\set{\lnot, \land}$ polna nabora! Se pravi, vse veznike s pozitivno mestnostjo je možno izraziti že samo z dvema.
		
		Je možno iti še dlje in najti en sam veznik, s katerim lahko izrazimo ostale? Odgovor je da: $\set{\shf}$ in $\set{\luk}$ sta polna nabora. (Izkaže se, da sta to edina taka veznika med dvomestniki vezniki.)
		
		\begin{vaja}\label{VAJA: polni nabori z enim veznikom}
			\
			\begin{enumerate}
				\item
					Izrazi negacijo samo z veznikom $\shf$. Izrazi še konjunkcijo ali disjunkcijo samo z veznikom $\shf$. Sklepaj, da je $\set{\shf}$ poln nabor.
				\item
					Izrazi negacijo samo z veznikom $\luk$. Izrazi še konjunkcijo ali disjunkcijo samo z veznikom $\luk$. Sklepaj, da je $\set{\luk}$ poln nabor.
			\end{enumerate}
		\end{vaja}
		
		\davorin{Bi na tem mestu predebatirali preklopna vezja?}
		
		\davorin{Mogoče lahko zavoljo celovitosti podamo karakterizacijo polnih naborov kot izrek (in se za dokaz skličemo na literaturo). Nabor je poln, če za vsako sledečih lastnosti obstaja veznik v njem, ki jo prekrši: ohranjanje resnice, ohranjanje neresnice, monotonost, sebi-dualnost, afinost (kot polinom Žegalkina).}
	
	
	\section{Predikati in kvantifikatorji}
	
		\note{\qt{Lastnostim} elementov množic, ki smo jih prej uporabljali za podajanje podmnožic in pri kvantifikatorjih, zdaj \qt{uradno} rečemo \df{predikati} in jih formalno definiramo: predikat na množici $X$ je preslikava $X \to \tvs$. Karakteristične preslikave podmnožic. Spomnimo se kvantifikatorjev. \davorin{Si jih drznemo definirati kot preslikave na eksponentih $\tvs^X$?} Povemo, da lahko imajo predikati več spremenljivk (lahko so definirani na produktu) in da lahko kvantificiramo po samo nekaterih. Vezane, nevezane spremenljivke. Pravila, ki veljajo za kvantifikatorje (de Morgan itd.).}
\chapter{Dokazovanje}\label{poglavje:dokazovanje}

        Matematične izsledke običajno podajamo preko jasno izraženih izjav. Med študijem matematike hitro opazite, da se takšne izjave podajajo pod imeni `izrek', `trditev', `lema', {posledica} in podobno. Kdaj uporabiti katerega teh imen ni natanko določeno, pač pa je prepuščeno presoji matematika. Približno vodilo je naslednje:
        \begin{itemize}
                \item
                        \df{izrek}: osrednji, bistven rezultat,
                \item
                        \df{trditev}: stranski rezultat,
                \item
                        \df{lema}: rezultat, ki sam po sebi nima toliko vsebine, se pa uporabi pri dokazovanju pomembnejšega rezultata,\footnote{Sicer ni nujno, da se resnična pomembnost izjav takoj pokaže. Mnogo je primerov, ko se kak matematični članek po določenem času začne ceniti ne toliko zaradi glavnega izreka, pač pa zaradi neke leme, ki se je za dokaz glavnega izreka uporabila.}
                \item
                        \df{posledica}: rezultat, ki je zanimiv sam po sebi, ki pa hitro sledi iz predhodne izjave.
        \end{itemize}

        Če skrbno analizirate izreke, trditve itd.~s predavanj (ali iz matematičnih člankov), opazite, da sestojijo iz treh delov: kontekst, predpostavke, sklepi.
        \begin{itemize}
                \item
                        \df{Kontekst} pove, katere objekte obravnavamo in kakšne vrste so.
                \item
                        \df{Predpostavke} so izjave, ki jih privzamemo.
                \item
                        \df{Sklepi} so izjave, ki jih (pri danih predpostavkah) dokazujemo.
        \end{itemize}

        Oglejmo si konkreten primer. Rolleov izrek je znan in uporaben izrek v analizi (če ga še niste spoznali, ga boste v kratkem).

        \begin{izrek}[Rolle]
                Naj bo $f$ realna funkcija, definirana na intervalu $\intcc{a}{b}$, kjer sta $a$ in $b$ realni števili in $a < b$. Če je $f$ zvezna na celem $\intcc{a}{b}$ in odvedljiva na odprtem intervalu $\intoo{a}{b}$ ter zavzame enaki vrednosti v krajiščih, tj.~$f(a) = f(b)$, tedaj ima $f$ stacionarno točko na $\intoo{a}{b}$.
        \end{izrek}

        Analizirajmo, kaj so kontekst, predpostavke in sklepi pri tem izreku.

        \begin{itemize}
                \item
                        Kontekst je sledeč:
                        \[a \in \RR, \qquad b \in \RR_{> a}, \qquad f \in \RR^{\intcc{a}{b}}.\]
                        To so objekti (in njihove vrste), o katerih govori izrek. Smiselno je, da jih zapišemo v tem vrstnem redu; na primer, $f$ zapišemo nazadnje, saj je njena domena odvisna od $a$ in $b$. Kadar imamo objekte, ki so neodvisni med sabo, jih lahko zapišemo v poljubnem vrstnem redu.
                \item
                        Predpostavke so tri. Vsako navedimo v običajnem jeziku in nato še s simbolnim matematičnim zapisom.
                        \begin{itemize}
                                \item
                                        $f$ je zvezna na $\intcc{a}{b}$.
                                        \[
                                                \hspace{-2em}
                                                \all{x \in \intcc{a}{b}}
                                                        \all{\epsilon \in \RR_{> 0}}
                                                                \some{\delta \in \RR_{> 0}}
                                                                        \all{y \in \intcc{a}{b}}
                                                                                (|x - y| < \delta \impl \big|f(x) - f(y)\big| < \epsilon
                                                                        )
                                        \]
                                \item
                                        $f$ je odvedljiva na $\intoo{a}{b}$.
                                        \begin{multline*}
                                                \all{x \in \intoo{a}{b}}
                                                        \some{v \in \RR}
                                                                \all{\epsilon \in \RR_{> 0}}
                                                                        \some{\delta \in \RR_{> 0}}
                                                \all{h \in \RR_{\neq 0}} \\
                                                        (|h| < \delta \implies \Big|\frac{f(x + h) - f(x)}{h} - v\Big| < \epsilon)
                                        \end{multline*}
                                \item
                                        $f$ na krajiščih intervala zavzame enaki vrednosti.
                                        \[f(a) = f(b)\]
                        \end{itemize}
                        Če se vam morda zdita formuli za zveznost in odvedljivost begajoči, imate dve tolažbi. Prva je ta, da se boste čez čas takšnih formul navadili. ;) Druga je, da so tudi drugi matematiki leni po naravi in zato uvedejo oznake za daljše izraze, ki se pogosto uporabljajo. Zgornja zveznost se na krajše zapiše $f \in \mathcal{C}(\intcc{a}{b})$ ($\mathcal{C}$ kot ``continuous'', tj.~zvezen), odvedljivost pa $f \in \mathcal{D}^1(\intoo{a}{b})$ ($\mathcal{D}$ kot ``differentiable'', tj.~odvedljiv, enka pa pomeni ``(vsaj) enkrat odvedljiv'').
                \item
                        Sklep je eden: $f$ ima stacionarno točko na $\intoo{a}{b}$, kar simbolno zapišemo takole.
                        \[\some{x \in \intoo{a}{b}} f'(x) = 0\]
        \end{itemize}

        V splošnem imamo določeno mero svobode, kako natančno razčleniti izrek. Na primer, za Rolleov izrek bi lahko kontekst zapisali tudi kot $a \in \RR, b \in \RR, f \in \RR^{\intcc{a}{b}}$ in pogoj $a < b$ dodali med predpostavke.

        Da ne bomo pisali dolgih seznamov, se dogovorimo za sledeče oznake. Izrek podamo tako, da najprej zapišemo kontekst, nato dvopičje, nato narišemo vodoravno črto, nad črto zapišemo predpostavke (ločene z vejicami), pod črto pa sklepe (ločene z vejicami). Rolleov izrek bi potemtakem povzeli takole.
        \[\claim{a \in \RR, b \in \RR_{> a}, f \in \RR^{\intcc{a}{b}}}{f \in \mathcal{C}(\intcc{a}{b}), f \in \mathcal{D}^1(\intoo{a}{b}), f(a) = f(b)}{\some{x \in \intoo{a}{b}} f'(x) = 0}\]

        V splošnem velja: vse proste spremenljivke, ki se pojavijo v predpostavkah ali sklepih, morajo biti navedene v kontekstu. Po domače povedano: če trdite, da za neko stvar nekaj velja, morate najprej povedati, o kateri stvari sploh govorite.

        Medtem ko je za težje matematične izreke potrebno obilo ustvarjalnosti, da se jih dokaže, pa lažje trditve pogosto lahko avtomatično dokažemo (dobesedno --- obstajajo avtomatični dokazovalniki \davorin{koliko povemo na to temo?}), pa tudi za težje je pomembno, da vemo, kako pristopiti k dokazu. Gre za to, da za vse logične veznike in kvantifikatorje obstajajo splošna pravila, kako ravnamo, če nastopajo kot predpostavke oziroma kot sklepi. To si bomo zdaj ogledali.

        \begin{itemize}
                \item\textbf{Konjunkcija}
                        \begin{itemize}
                                \item
                                        Če $p \land q$ nastopa kot \emph{predpostavka}:
                                        \begin{quote}
                                                predpostavko $p \land q$ nadomestimo s predpostavkama $p$, $q$ (to se pravi, pri dokazovanju lahko uporabimo tako predpostavko $p$ kot predpostavko $q$). S simboli, od trditve
                                                \[\claim{\Gamma}{\Pi', p \land q, \Pi''}{\Sigma}\]
                                                preidemo do trditve
                                                \[\claim{\Gamma}{\Pi', p, q, \Pi''}{\Sigma}\]
                                                (pri zapisih splošnih izrekov bomo kontekst označevali z $\Gamma$, predpostavke s $\Pi$ in sklepe s $\Sigma$).
                                        \end{quote}
                                \item
                                        Če $p \land q$ nastopa kot \emph{sklep}:
                                        \begin{quote}
                                                sklep $p \land q$ dokažemo tako, da dokažemo posebej $p$ in posebej $q$. S simboli:
                                                \[\claim{\Gamma}{\Pi}{\Sigma', p \land q, \Sigma''}\]
                                                preoblikujemo v
                                                \[\claim{\Gamma}{\Pi}{\Sigma', p, q, \Sigma''}\]
                                                (in se zavedamo, da je za dokaz izreka potrebno dokazati \emph{vse} sklepe).
                                        \end{quote}
                        \end{itemize}
                \item\textbf{Disjunkcija}
                        \begin{itemize}
                                \item
                                        Če $p \lor q$ nastopa kot \emph{predpostavka}:
                                        \begin{quote}
                                                ločimo primere: sklepe dokažemo posebej pri predpostavki $p$ (skupaj z ostalimi predpostavkami) in posebej pri predpostavki $q$ (skupaj z ostalimi). Torej, dokazati
                                                \[\claim{\Gamma}{\Pi', p \lor q, \Pi''}{\Sigma}\]
                                                pomeni isto, kot dokazati tako
                                                \[\claim{\Gamma}{\Pi', p, \Pi''}{\Sigma} \qquad \text{kot} \qquad \claim{\Gamma}{\Pi', q, \Pi''}{\Sigma}.\]
                                        \end{quote}
                                \item
                                        Če $p \lor q$ nastopa kot \emph{sklep}:
                                        \begin{quote}
                                                izberemo si enega od $p$, $q$ in ga dokažemo. Se pravi, če imamo
                                                \[\claim{\Gamma}{\Pi}{\Sigma', p \lor q, \Sigma''},\]
                                                si izberemo eno od trditev
                                                \[\claim{\Gamma}{\Pi}{\Sigma', p, \Sigma''} \qquad \text{oziroma} \qquad \claim{\Gamma}{\Pi}{\Sigma', q, \Sigma''}\]
                                                in jo izpeljemo.
                                        \end{quote}
                        \end{itemize}
                \item\textbf{Implikacija}
                        \begin{itemize}
                                \item
                                        Če $p \impl q$ nastopa kot \emph{predpostavka}:
                                        \begin{quote}
                                                če nam kadarkoli uspe izpeljati $p$, lahko dodamo $q$ med predpostavke. Torej, če znamo dokazati
                                                \[\claim{\Gamma}{\Pi', p \impl q, \Pi''}{q},\]
                                                potem za dokaz
                                                \[\claim{\Gamma}{\Pi', p \impl q, \Pi''}{\Sigma}\]
                                                zadostuje dokazati
                                                \[\claim{\Gamma}{\Pi', p \impl q, q, \Pi''}{\Sigma}\]
                                                (kar je lažje, ker imamo eno predpostavko več). To je smiselno: če vemo, da velja $p \impl q$ in dodatno ugotovimo, da velja $p$, potem vemo, da velja tudi $q$.
                                        \end{quote}
                                \item
                                        Če $p \impl q$ nastopa kot \emph{sklep}:
                                        \begin{quote}
                                                sklep $p \impl q$ nadomestimo s $q$, medtem ko $p$ dodamo med predpostavke. Pojasnimo. Trditev $p \impl q$ trdi nekaj samo v primeru, kadar $p$ velja --- v nasprotnem primeru je avtomatično resnična in ni ničesar za dokazati. Torej se lahko omejimo na primer, ko $p$ velja, se pravi, lahko predpostavimo $p$. Kadar $p$ velja, pa trditev $p \impl q$ pravi, da mora veljati tudi $q$. To pomeni, da pri predpostavki $p$ dokazujemo $q$. Simbolno, da dokažemo
                                                \[\claim{\Gamma}{\Pi}{\Sigma', p \impl q, \Sigma''},\]
                                                zadostuje dokazati
                                                \[\claim{\Gamma}{\Pi}{\Sigma', \Sigma''} \qquad \text{in} \qquad \claim{\Gamma}{\Pi, p}{q}.\]
                                        \end{quote}
                        \end{itemize}
                \item\textbf{Univerzalni kvantifikator}
                        \begin{itemize}
                                \item
                                        Če $\all{x \in X} \phi(x, y)$ nastopa kot \emph{predpostavka}:
                                        \begin{quote}
                                                če vemo za (ali med dokazom najdemo) katerikoli konkreten element $a \in X$, tedaj lahko med predpostavke dodamo $\phi(a, y)$. Namreč, če vemo, da lastnost $\phi$ (z morebitnimi nadaljnjimi parametri) velja za vse elemente množice $X$, potem ta lastnost velja za poljuben konkreten element. Simbolno, od
                                                \[\claim{\Gamma', a \in X, \Gamma''}{\Pi', \all{x \in X} \phi(x, y), \Pi''}{\Sigma}\]
                                                preidemo do
                                                \[\claim{\Gamma', a \in X, \Gamma''}{\Pi', \all{x \in X} \phi(x, y), \phi(a, y), \Pi''}{\Sigma}.\]
                                        \end{quote}
                                \item
                                        Če $\all{x \in X} \phi(x, y)$ nastopa kot \emph{sklep}:
                                        \begin{quote}
                                                v kontekst dodamo $x \in X$, sklep $\all{x \in X} \phi(x, y)$ pa nadomestimo s sklepom $\phi(x, y)$. S simboli, od
                                                \[\claim{\Gamma}{\Pi}{\Sigma', \all{x \in X} \phi(x, y), \Sigma''}\]
                                                preidemo do
                                                \[\claim{\Gamma, x \in X}{\Pi}{\Sigma', \phi(x, y), \Sigma''}\]
                                                Zakaj tako postopamo in kaj smo s tem pravzaprav naredili? Premislimo: želimo dokazati, da neka lastnost velja za vse elemente dane množice $X$. Če ima $X$ slučajno samo končno mnogo elementov, bi lahko lastnost preverili za vsakega posebej, ampak povečini delamo z neskončnimi množicami, kjer to ne deluje. Morda ima množica $X$ kakšno posebno lastnost, zaradi katere lahko univerzalni kvantifikator dokažemo na svojevrsten način (na primer, univerzalno kvantificirane izjave nad $\NN$ lahko dokazujemo z matematično indukcijo --- glej \note{razdelek o naravnih številih}), ampak to se zgodi v izjemnih primerih.

                                                V splošnem nimamo druge možnosti, kot da si izberemo simbol (tipično kar spremenljivko v kvantifikatorju), ki nam predstavlja poljuben, katerikoli element množice in zanj dokažemo želeno lastnost. Ideja je, da spremenljivka spet nastopa v vlogi ``škatlice'', kamor lahko vstavimo poljuben element množice $X$. Če nam je dokaz lastnosti uspel, ne da bi za spremenljivko predpostavili karkoli več, kot da predstavlja element množice $X$, tedaj dobimo dokaz lastnosti za katerikoli dejanski element množice $X$ tako, da v dobljeni dokaz namesto spremenljivke vstavimo ta element. Na ta način smo potem dejansko dobili dokaz lastnosti za vse elemente množice $X$.

                                                Besedni dokazi univerzalno kvantificirane izjave se zato tipično začnejo takole: ``Vzemimo poljuben $x \in X$. Dokažimo, da zanj velja dana lastnost.''
                                        \end{quote}
                        \end{itemize}
                \item\textbf{Eksistenčni kvantifikator}
                        \begin{itemize}
                                \item
                                        Če $\some{x \in X} \phi(x, y)$ nastopa kot \emph{predpostavka}:
                                        \begin{quote}
                                                v kontekst dodamo $x \in X$, eksistenčno predpostavko pa nadomestimo s $\phi(x, y)$. S simboli,
                                                \[\claim{\Gamma}{\Pi', \some{x \in X} \phi(x, y), \Pi''}{\Sigma}\]
                                                popravimo v
                                                \[\claim{\Gamma, x \in X}{\Pi', \phi(x, y), \Pi''}{\Sigma}.\]
                                                Zakaj to deluje? Naša predpostavka je, da v množici $X$ obstaja element z lastnostjo $\phi$ (z morebitnimi nadaljnjimi parametri). Torej si lahko vzamemo neki konkreten element množice $X$ s to lastnostjo, ki ga lahko uporabljamo kasneje v dokazu (za to ga moramo nekako označiti; v praksi ga tipično označimo kar z isto spremenljivko, kot v kvantifikatorju).
                                        \end{quote}
                                \item
                                        Če $\some{x \in X} \phi(x, y)$ nastopa kot \emph{sklep}:
                                        \begin{quote}
                                                da dokažemo eksistenčno izjavo, moramo podati neki konkreten element $x \in X$ in zanj dokazati dano lastnost $\phi(x, y)$. \davorin{Hm, kako točno to zapišemo simbolno v zgornji obliki?}
                                        \end{quote}
                        \end{itemize}
        \end{itemize}

        V zgornjem seznamu nismo omenili vseh veznikov in kvantifikatorjev. To je zato, ker jih pri dokazovanju nadomestimo z zgornjimi. Konkretno:
        \begin{itemize}
                \item
                        Za negacijo velja $\lnot{p} \equiv p \impl \false$. Med drugim to pomeni, da $\lnot{p}$ dokažemo na sledeči način: predpostavimo $p$ in iz tega izpeljemo neresnico.
                \item
                        Za ekvivalenco velja $p \lequ q \equiv (p \impl q) \land (p \revimpl q)$. To pomeni, da ekvivalenco dokažemo tako, da dokažemo implikacijo med $p$ in $q$ v obe smeri --- se pravi, enkrat predpostavimo $p$ in izpeljemo $q$, drugič pa predpostavimo $q$ in izpeljemo $p$.
                \item
                        Za veznike $\xor$, $\shf$, $\luk$ si preprosto izberemo eno od izražav z negacijo, konjunkcijo in disjunkcijo in nato delamo z njo.
                \item
                        Kvantifikator $\exactlyone{x \in X} \phi(x, y)$ ločimo na dva dela: na obstoj in enoličnost, in vsakega posebej dokažemo. Se pravi, skličemo se na izražavo
                        \[\exactlyone{x \in X} \phi(x, y) \equiv \some{x \in X} \phi(x, y) \land \all{a, b \in X} (\phi(a, y) \land \phi(b, y) \implies a = b).\]
                        Včasih je lažje, če najprej dokažemo obstoj elementa in ta element pri dokazu enoličnosti že uporabimo, torej dokazujemo izražavo
                        \[\exactlyone{x \in X} \phi(x, y) \equiv \some{x \in X} (\phi(x, y) \land \all{a \in X} (\phi(a, y) \implies a = x)).\]
        \end{itemize}

        Seveda ne bo možno dokazati vsakega izreka s slepim sledenjem zgornjim pravilom; včasih moramo uporabiti še kakšno dodatno strategijo. Spodnji dve sta zelo pogosti.
        \begin{itemize}
                \item
                        Med predpostavke dodamo trditev, za katero že vemo, da je resnična. Morda gre za trditev, ki smo jo že dokazali, morda pa gre kar za istorečje. Pogost primer tega je, da uporabimo zakon o izključenem tretjem in za dodatno predpostavko vzamemo $p \lor \lnot{p}$ (kjer je $p$ katerakoli konkretna izjava). Po zgornjih pravilih to potem pomeni, da ločimo primere in trditev dokažemo posebej pri predpostavki $p$ ter posebej pri predpostavki $\lnot{p}$.
                \item
                        Nekatere predpostavke ali sklepe nadomestimo z enakovrednimi izjavami. Na primer, velja
                        \[p \lor q \equiv \lnot(\lnot{p} \land \lnot{q}) \equiv \lnot{p} \impl q \equiv \lnot{q} \impl p.\]
                        To pomeni, da lahko disjunkcijo (poleg zgoraj omenjenega načina) dokažemo tudi tako, da predpostavimo, da nobena od možnosti ne velja, in od tod izpeljemo neresnico, ali pa predpostavimo, da ena od možnosti ne velja, in od tod izpeljemo drugo.

                        Zelo pogosta uporaba te ideje je \df{dokaz s protislovjem}, ki temelji na zakonu o dvojni negaciji $p \equiv \lnot\lnot{p}$. Izjavo torej lahko dokažemo tako, da predpostavimo njeno negacijo, in od tod izpeljemo neresnico. Tipičen besedni dokaz s protislovjem izgleda takole: ``Dokazujemo $p$. Pa recimo, da $p$ ne velja. Potem /neki sklepi/. To je v nasprotju s tem, kar smo dokazali prej, torej smo izpeljali protislovje. Se pravi, ni možno, da $p$ ne bi veljal, torej mora veljati.''
        \end{itemize}

        \note{mnogo zgovornih primerov dokazov, ki ponazorijo zgornje postopke}


\section{Vaje}


%%% Local Variables:
%%% mode: latex
%%% TeX-master: "ucbenik-lmn"
%%% End:

\chapter{Konstrukcije množic}

\section{Preprosti primeri}
\note{prazna množica, enojci}

\section{Podmnožice}
\davorin{Če ``embedding'' prevajamo kot ``vložitev'', kako potem prevedemo ``inclusion''? Imamo sicer tujko ``inkluzija'', ampak fino bi bilo imeti še slovenski izraz. Vključitev?}

\section{Potenčna množica}
\davorin{Verjetno je smiselno, da ta razdelek sledi razdelku o podmnožicah. Morda kar združimo ta dva razdelka?}

\section{Družine množic}

\section{Produkt množic}

\section{Vsota množic}

\section{Unija in presek}

\section{Eksponentna množica}

\davorin{Vrstni red teh razdelkov bomo najbrž še premešali.}


%%% Local Variables:
%%% mode: latex
%%% TeX-master: "ucbenik-lmn"
%%% End:

\chapter{Preslikave}



\section{Slike in praslike}

Preslikava kot taka nam pove za posamične elemente, kam se slikajo. Marsikdaj pa nas zanima več: kam se slikajo celotne množice elementov. Na primer, zanima nas lahko, v kaj se projicira neko prostorsko telo na ravnino.

\note{luštna slika projekcije nekega prostorskega objekta na neko ravnino}

Da dobimo sliko celotne množice, moramo zbrati skupaj slike vseh posamičnih elementov množice. Smiselna je torej naslednja definicija.

\begin{definicija}\label{definicija:slika}
Naj bo $f\colon X \to Y$ preslikava. \df{Slika} množice $A \subseteq X$ je označena in definirana kot
\[\img{f}{A} \dfeq \set[1]{f(x)}{x \in A} = \set[1]{y \in Y}{\some{x \in A} y = f(x)}.\]
Ta predpis definira preslikavo $\img{f}\colon \pst(X) \to \pst(Y)$.
\end{definicija}

\begin{opomba}
Kot običajno, obstajajo različne oznake v uporabi. Sliko $\img{f}{A}$ se označuje tudi kot $f[A]$ ali celo kar kot $f(A)$. V slednjem primeru se predpostavlja zadostna matematična zrelost bralca, da zna razbrati, kdaj $f$ označuje preslikavo $f\colon X \to Y$, kdaj pa preslikavo $f\colon \pst(X) \to \pst(Y)$.

V tej knjigi se bomo načrtno izogibali takšnim dvoumnostim in za sliko dosledno uporabljali oznako iz definicije~\ref{definicija:slika}.
\end{opomba}

\begin{naloga}
Prepričaj se, da za poljubno preslikavo $f\colon X \to Y$ velja sledeče:
\begin{itemize}
\item
$\img{f}{X} = \rn{f}$,
\item
$\img{f}{\emptyset} = \emptyset$,
\item
$\img[1]{f}{\set{x}} = \set[1]{f(x)}$ za vsak $x \in X$.
\end{itemize}
\end{naloga}

\note{primeri in lastnosti slik že tu ali kasneje skupaj s primeri/lastnostmi praslik?}

Včasih pa imamo obratno nalogo: iz dane slike ugotoviti, kaj vse se je z neko preslikavo vanjo preslikalo. Zato vpeljemo še sledečo definicijo.

\begin{definicija}\label{definicija:praslika}
Naj bo $f\colon X \to Y$ preslikava. \df{Praslika} množice $B \subseteq Y$ je označena in definirana kot
\[\pim{f}{B} \dfeq \set[1]{x \in X}{f(x) \in B}.\]
Ta predpis definira preslikavo $\pim{f}\colon \pst(Y) \to \pst(X)$.
\end{definicija}

\begin{opomba}
Tudi za prasliko obstajajo različne oznake. Praslika $\pim{f}{B}$ se označi tudi kot $f^{-1}[B]$ ali kar kot $f^{-1}(B)$. V slednjem primeru se spet zanašamo na izkušenost bralca, da praslike $f^{-1}\colon \pst(Y) \to \pst(X)$ ne zamenja z obratom $f^{-1}\colon Y \to X$. Slednji morda sploh ne obstaja! (Praslika seveda obstaja za vse funkcije.)

Če obrat funkcije obstaja, tedaj velja $\pim[1]{f}{\set{y}} = \set[1]{f^{-1}(y)}$ za vsak $y \in Y$ (premisli!), kar nekoliko pojasni oznako $f^{-1}$ tudi za prasliko. Kljub vsemu, z namenom izogibanja dvoumnostim se bomo v tej knjigi skrbno držali oznake iz definicije~\ref{definicija:praslika} za prasliko.

Ko smo že pri alternativnih, potencialno zavajajočih oznakah: pri prasliki enojca se tipično izpuščajo zaviti oklepaji, torej se namesto $\pim[1]{f}{\set{y}}$ piše $\pim{f}{y}$ (ali celo $f^{-1}(y)$).
\end{opomba}

\note{primeri, vaje}

\note{lastnosti: ohranjanje unij, presekov, komplementov}


\section{Injektivnost in surjektivnost}\label{razdelek:injektivnost-in-surjektivnost}

\note{Vključno z ekvivalenco z mono- in epimorfizmi.}


\section{Bijektivnost in obratne preslikave}\label{razdelek:bijektivnost-in-obratne-preslikave}

Kot dobro veste že iz srednje šole, nam injektivnost in surjektivnost omogočata definicijo bijektivnosti.

\begin{definicija}
Preslikava je \df{bijektivna}, kadar je injektivna in surjektivna.
\end{definicija}

To pomeni: če imamo bijektivno preslikavo (na kratko kar: \df{bijekcijo}) $f\colon X \to Y$, smo povezali elemente množice $X$ z elementi množice $Y$, in sicer tako, da vsakemu elementu v katerikoli od množic $X$ oz.~$Y$ pripišemo natanko en element druge množice.

\note{slika dveh množic s poparjenimi pikami}

Rečemo, da so elementi množice $X$ v \df{bijektivni korespondenci} (ali po slovensko \df{povratno enolični zvezi}) z elementi množice $Y$. Bijektivnost se na grafih kaže takole: preslikava je bijektivna, kadar vsaka vodoravnica seka njen graf natanko enkrat.

Bijektivne preslikave igrajo pomembno vlogo v matematiki. Oglejmo si tri primere.
\begin{itemize}
\item
Če imamo povratno enolično zvezo med elementi dveh množic, je jasno, da imata isto število elementov. To nam omogoča definicijo \df{kardinalnosti} množic --- glej poglavje~\note{o kardinalnosti}.
\item
Predstavljajmo si, da so elementi neke množice $X$ imena za določene objekte. Na bijektivno preslikavo $f\colon X \to Y$ lahko potem gledamo kot na preimenovanje teh objektov. Seveda preimenovanje ne spremeni narave (ali če hočete natančnejši izraz, matematične strukture) objektov --- z drugimi besedami, $X$ in $Y$ se razlikujeta zgolj po imenih svojih elementov. To nas privede do pojma \df{izomorfizma}. Za več podrobnosti glej poglavje~\note{o strukturiranih množicah}.
\item
Če imamo povratno enolično zvezo med elementi množic $X$ in $Y$, potem ta zveza ne podaja zgolj preslikave v smeri $X \to Y$, pač pa tudi v smeri $Y \to X$, ker za vsak element iz $Y$ obstaja enolično določen element iz $X$, ki se vanj preslika. Z drugimi besedami, bijektivne preslikave imajo \df{obrate}.
\end{itemize}

Povejmo več o obratih preslikav. Začnimo s formalno definicijo.

\begin{definicija}
Naj bo $f\colon X \to Y$ poljubna preslikava. Za preslikavo $g\colon Y \to X$ rečemo, da je \df{obrat} ali \df{inverz} preslikave $f$, kadar velja
\[g \circ f = \id[X] \qquad\qquad \text{in} \qquad\qquad f \circ g = \id[Y].\]
Z drugimi besedami, $g$ je obrat $f$, kadar slika v nasprotni smeri in za vsak $x \in X$ velja $g\big(f(x)\big) = x$ ter za vsak $y \in Y$ velja $f\big(g(y)\big) = y$. Kadar obrat preslikave $f$ obstaja, rečemo, da je $f$ \df{obrnljiva} (ali \df{invertibilna}) preslikava.
\end{definicija}

\begin{zgled}\label{zgled:logaritmiranje-je-obratno-od-eksponenciranja}
Kot veš že iz srednje šole, logaritmiranje je obratno od eksponenciranja. Če smo natančnejši: preslikavi $\lam{x \in \RR}  b^x$ in $\lam{x \in \RR_{>0}} \log_b x$ sta si obratni pri vsaki osnovi $b \in \RR_{> 0} \setminus \set{1}$.
\end{zgled}

\begin{naloga}\label{naloga:enolicnost-obrata-preslikave}
Dokaži: če sta $g$ in $h$ obrata iste preslikave $f$, tedaj $g = h$.
\end{naloga}

Vaja~\ref{naloga:enolicnost-obrata-preslikave} pove, da je obrat funkcije enolično določen, tj.~vsaka funkcija ima kvečjemu en obrat. Zato lahko uvedemo izrecno oznako: obrat preslikave $f$ (kadar obstaja) označimo z $f^{-1}$. Velja torej: kadar je preslikava $f\colon X \to Y$ obrnljiva, določa preslikavo $f^{-1}\colon Y \to X$.

Ta oznaka je nekoliko nerodna --- pomembno se je zavedati, da $f^{-1}(x)$ pomeni obrat preslikave $f$, izvrednoten na $x$, medtem kot $\big(f(x)\big)^{-1}$ pomeni obratna vrednost (v smislu deljenja) izvrednotenja preslikave $f$ na $x$. Za primerjavo, kot omenjeno v zgledu~\ref{zgled:logaritmiranje-je-obratno-od-eksponenciranja}, je obrat eksponenciranja logaritmiranje, medtem ko je obratna vrednost od $b^x$ enaka $(b^x)^{-1} = \frac{1}{b^x} = b^{-x}$.

\begin{naloga}
Premisli: če ima preslikava $f$ obrat $f^{-1}$, tedaj je tudi $f^{-1}$ obrnljiva preslikava in velja $(f^{-1})^{-1} = f$ (torej, obrat obrata je izvorna preslikava).
\end{naloga}

\begin{naloga}
Pogosto rečemo, da sta seštevanje in odštevanje obratni operaciji. Strogo vzeto, ti dve operaciji nista obratni kot preslikavi, saj obe slikata (recimo, da ju gledamo na realnih številih) $\RR \times \RR \to \RR$, tj.~ne slikata v nasprotnih smereh. Ugotovi, v kakšnem smislu točno sta seštevanje in odštevanje obratni, tj.~kateri dve preslikavi sta pravzaprav druga drugi obratni.
\end{naloga}

Zakaj se sploh ukvarjamo z obrati? Pogosto obravnavamo preslikavo, ki izhaja iz nekega konkretnega (na primer fizikalnega) problema, v smislu, da preslikava vzame začetne podatke in nam vrne, kaj se bo na koncu zgodilo. Marsikdaj pa hočemo rešiti obraten problem: želimo določene končne rezultate in se sprašujemo, kakšni morajo biti začetni pogoji, da jih bomo dosegli. V takem primeru pride prav obratna preslikava.

Kot omenjeno, je obrat preslikave enoličen. Ne velja pa, da za poljubne preslikave sploh obstaja. Na primer, naj bo $f$ edina možna preslikava $\set{0, 1} \to \set{\unit}$, torej tista, ki tako $0$ kot $1$ preslika v $\unit$. Nobena preslikava $g\colon \set{\unit} \to \set{0, 1}$ ne more biti obrat preslikave $f$, saj je $g \circ f$ gotovo konstantna in potemtakem ne more biti identiteta na $\set{0, 1}$.

Kdaj torej obstaja obrat preslikave?

\begin{trditev}
Za poljubno preslikavo $f\colon X \to Y$ sta ekvivalentni sledeči trditvi.
\begin{enumerate}
\item
Preslikava $f$ je obrnljiva.
\item
Preslikava $f$ je bijektivna.
\end{enumerate}
\end{trditev}

\begin{proof}
\begin{implproof}{1}{2}
Predpostavljamo, da obstaja obrat $f^{-1}$.

Dokažimo, da je $f$ injektivna. Vzemimo poljubna $x, y \in X$, za katera velja $f(x) = f(y)$. Tedaj $x = f^{-1}\big(f(x)\big) = f^{-1}\big(f(y)\big) = y$.

Dokažimo, da je $f$ surjektivna. Vzemimo poljuben $y \in Y$. Tedaj $y = f\big(f^{-1}(y)\big)$.
\end{implproof}
\begin{implproof}{2}{1}
Če je $f$ bijekcija, za vsak $y \in Y$ velja, da je $\pim[1]{f}{\set{y}}$ enojec (glej \note{ustrezne predhodne trditve v razdelku o injektivnosti in surjektivnosti}). Definirajmo $g\colon Y \to X$ na naslednji način: za vsak $y \in Y$ naj bo $g(y)$ tisti element $x \in X$, za katerega velja $\pim[1]{f}{\set{y}} = \set{x}$. \note{Iz lastnosti praslike sledi, da je $g$ obrat $f$.}
\end{implproof}
\end{proof}

Iz dokaza te trditve vidimo, da bi bilo koristno imeti oznako za ``tisti element'', če želimo podajati tovrstne preslikave s simboli. Naj bo $\phi$ lastnost elementov množice $X$ (torej predikat $\phi\colon X \to \tvs$), ki je resnična za natanko en element. Dogovorimo se, da
\[\that{x \in X} \phi(x)\]
pomeni ``tisti (edini) element množice $X$, ki ima lastnost $\phi$'' (simbolček na začetku je mala grška črka jota). Zdaj lahko izrecno zapišemo: če je $f\colon X \to Y$ bijekcija, tedaj je njen obrat $f^{-1}\colon Y \to X$ dan s predpisom
\[f^{-1}(y) = \that{x \in X} (f(x) = y).\]

\davorin{Andrej, omenjal si, da želiš imeti to oznako. Če sem kaj zgrešil, prosim popravi.}

Zaenkrat smo to joto uporabljali zgolj kot okrajšavo za stavek v običajnem jeziku, ampak če želimo $\iota$-izraze uporabljati v matematičnih dokazih, jim moramo dati natančen matematični pomen. Definirajmo torej joto formalno matematično.

Naj bo $X$ poljubna množica. Na njej imamo enakost; obravnavajmo jo na tem mestu kot lastnost dvojic elementov iz $X$, torej kot predikat $=_X\colon X \times X \to \tvs$ (za vsak par elementov vrnemo resničnostno vrednost izjave, da sta komponenti para enaki). Transponirajmo to preslikavo; dobimo $\transposed{=_X}\colon X \to \tvs^X$. Ta transponiranka je injektivna: če se za $a, b \in X$ preslikavi $\lam{x \in X} (a = x)$ in $\lam{x \in X} (b = x)$ ujemata, se ujemata tudi njuni vrednosti pri $b$. Ker drži $b = b$, potem drži tudi $a = b$.

Če zožimo kodomeno preslikave $\transposed{=_X}$ na njeno sliko, potemtakem dobimo bijekcijo. Naj bo jota njen obrat, torej $\iota \dfeq \big(\rstr{\transposed{=_X}}^{\rn{\transposed{=_X}}}\big)^{-1}$. V tem smislu je zgornja oznaka $\that{x \in X} \phi(x)$ okrajšava za $\iota (\lam{x \in X} \phi(x))$ (kar bi seveda lahko še skrajšali do $\iota(\phi)$, ampak v praksi je to običajno manj zgovorno).


\section{Vaje}


%%% Local Variables:
%%% mode: latex
%%% TeX-master: "ucbenik-lmn"
%%% End:

\chapter{Relacije}\label{POGLAVJE: Relacije}

        \section{Splošno o relacijah}

                V matematiki pogosto želimo izraziti, da so določeni objekti v nekem odnosu, npr.~eno število je večje od drugega; temu s tujko rečemo \df{relacija}. Kako to formalno izraziti? Ideja je, da relacijo podamo z množico vseh skupin elementov, ki so v relaciji. Na primer, relacijo $\leq$ na naravnih številih podamo kot podmnožico
                \[\set[1]{(a, b) \in \NN \times \NN}{\xsome{n}[\NN]{a + n = b}}.\]
                Torej, število $a$ je v relaciji $\leq$ s številom $b$ takrat, ko par $(a, b)$ pripada tej množici.

                Splošne relacije so lahko med poljubno mnogo elementi iz poljubnih (ne nujno istih) množic. Na primer, relacija komplanarnosti štirih točk v prostoru je podmnožica produkta $\RR^3 \times \RR^3 \times \RR^3 \times \RR^3$, relacija pripadnosti $\in$ med elementi neke množice $X$ in podmnožicami množice $X$ pa je podmnožica produkta $X \times \pst(X)$.

                Splošna definicija relacije je potemtakem naslednja.
                \begin{definicija}
                        \df{Relacija} na družini množic $\mathscr{D}$ je podmnožica produkta $\prod_{X \in \mathscr{D}} X$, skupaj s podatkom, za katero družino $\mathscr{D}$ gre.
                \end{definicija}

                \begin{opomba}\label{OPOMBA: definicija relacij}
                        Kaj mislimo tu z izrazom \qt{skupaj s podatkom}? Določena podmnožica ima mnogo nadmnožic in podatek, med elementi katerih množic opazujemo odnos, je za relacijo prav tako pomemben, saj so od tega odvisne lastnosti relacije. Lastnosti relacij obravnavamo kasneje v razdelku~\ref{RAZDELEK: Lastnosti relacij}, ampak če že zdaj damo primer: $\set{(a, a)}{a \in \NN}$ je refleksivna kot relacija na naravnih številih (tj.~kot podmnožica $\NN \times \NN$), ne pa tudi kot relacija na celih številih (tj.~kot podmnožica $\ZZ \times \ZZ$).

                        Kako \qt{priložiti} podatek o družini? Ena možnost je, da relacijo podamo kot urejeni par $\rel = (R, \mathscr{D})$, kjer $R \subseteq \prod_{X \in \mathscr{D}} X$. Še ena možnost je, da relacijo podamo kot družino preslikav $(\rel \to X)_{X \in \mathscr{D}}$, ki skupaj porodijo inkluzijo $\rel \hookrightarrow \prod_{X \in \mathscr{D}} X$. Ampak načeloma je povsem vseeno, ali vzamemo katero od teh dveh možnosti ali še kaj tretjega. V tej knjigi se ne bomo omejevali na posamičen formalen zapis za relacijo, bo pa seveda v vseh primerih jasno, za katero družino gre.
                \end{opomba}

                \davorin{Verjetno bi bilo smiselno omeniti še možnost podajanja relacije kot predikat $\prod_{X \in \mathscr{D}} X \to \tvs$.}

                V praksi se povečini uporabljajo relacije med dvema elementoma.
                \begin{definicija}
                        \df{Dvomestna} (ali \df{dvojiška} ali \df{binarna}) \df{relacija} $\rel$ med elementi množic $X$ in $Y$ je podmnožica produkta $X \times Y$, skupaj s podatkom o $X$ in $Y$. Za takšno relacijo definiramo:
                        \begin{itemize}
                                \item
                                        množica $X$ je \df{začetna množica} ali \df{domena} relacije $\rel$, kar označimo $\dom(\rel)$,
                                \item
                                        množica $Y$ je \df{ciljna množica} ali \df{kodomena} relacije $\rel$, kar označimo $\cod(\rel)$,
                                \item
                                        \df{definicijsko območje} ali \df{nosilec} relacije $\rel$ je množica $\dd{\rel} \dfeq \set{x \in X}{\xsome{y}[Y]{\rel[x][y]}}$ (torej $\dd{\rel} \subseteq \dom(\rel)$),
                                \item
                                        \df{zaloga vrednosti} ali \df{slika} \note{razpon?} relacije $\rel$ je množica $\rn{\rel} \dfeq \set{y \in Y}{\xsome{x}[X]{\rel[x][y]}}$ (torej $\dd{\rel} \subseteq \cod(\rel)$).
                        \end{itemize}
                \end{definicija}

                Skoraj vse relacije, ki nas zanimajo v tej knjigi, so dvomestne. Zato se dogovorimo, da z izrazom \qt{relacija} vselej mislimo dvomestno relacijo, razen če je izrecno rečeno drugače.

                Če je $\rel \subseteq X \times Y$ relacija, potemtakem lahko zapišemo, da sta $x \in X$ in $y \in Y$ v relaciji $\rel$ takole: $(x, y) \in \rel$. Ampak to vodi do čudnih zapisov v primeru običajnih relacij, npr.~$(2, 3) \in \mathnormal{<}$. To je bolje zapisati $2 < 3$ in posledično se dogovorimo, da v primeru dvojiške relacije raje uporabljamo zapis $\rel[x][y]$.

                Povečini se še dodatno omejimo na relacije z isto domeno in kodomeno.
                \begin{definicija}
                        \df{Dvomestna} (\df{dvojiška}, \df{binarna}) \df{relacija} na množici $X$ je podmnožica produkta $X \times X$, skupaj s podatkom o $X$.
                \end{definicija}

                Takšne relacije lahko lepo ponazorimo z usmerjenimi grafi. Graf relacije $\rel \subseteq X \times X$ je definiran takole: vozlišča grafa so elementi množice $X$ in za vsaka dva elementa $a, b \in X$, za katera velja $\rel[a][b]$, narišemo puščico od $a$ do $b$.

                \GraphInit[vstyle = Normal]
                \tikzset
                {
                        EdgeStyle/.append style = {->, bend left}
                }

                \begin{zgled}\label{ZGLED: graf relacije}
                        Naj bo $X = \set{A, B, C, D, E, F}$ in naj bo
                        \[\rel \dfeq \set{...}\]
                        relacija na $X$. Njen graf izgleda takole.

                        \note{graf relacije $\rel$}
                        %\begin{center}
                                %\begin{tikzpicture}
                                        %\SetGraphUnit{3}
                                        %\Vertex[Math=true, x=0, y=0]{A}
                                        %\Vertex[Math=true, x=3, y=2]{B}
                                        %\Vertex[Math=true, x=2, y=-3]{C}
                                        %\Vertex[Math=true, x=6, y=1]{D}
                                        %\Vertex[Math=true, x=8, y=-1]{E}
                                        %\Vertex[Math=true, x=10, y=2]{F}
                                        %
                                        %\Edge(A)(B)
                                        %\Loop[dist = 5em, dir = EA](B)
                                %\end{tikzpicture}
                        %\end{center}
                        %\davorin{izgled grafa je še treba popraviti}
                \end{zgled}


        \section{Operacije z relacijami}\label{RAZDELEK: Operacije z relacijami}

                Običajno je, da iz že danih matematičnih objektov lahko skonstruiramo nove preko določenih operacij. Z relacijami ni nič drugače; v tem razdelku si bomo ogledali običajne operacije na relacijah.

                Ker so relacije podmnožice, imamo za začetek vse operacije na podmnožicah. Torej, za poljubno družino $(\rel_i)_{i \in I}$ podmnožic produkta $X \times Y$ sta tudi unija $\bigcup_{i \in I} \rel_i$ in presek $\bigcap_{i \in I} \rel_i$ relaciji. Če je $\rel \subseteq X \times Y$ relacija, je njena komplementarna relacija $\complement{\rel} = X \times Y \setminus \rel \ \subseteq \ X \times Y$.

                Posebej imamo \df{prazno relacijo} $\emptyset \subseteq X \times Y$ (nobena dva elementa nista v relaciji) in \df{polno relacijo} $X \times Y \subseteq X \times Y$ (vsaka dva elementa sta v relaciji), ki sta si medsebojno komplementarni.

                Poleg operacij, ki jih relacije podedujejo od podmnožic, imamo še operacije, ki upoštevajo produktno strukturo.

                Če so $X$, $Y$, $Z$ množice in $\rel \subseteq X \times Y$, $\srel \subseteq Y \times Z$ relaciji, tedaj je \df{sklop} (\df{kompozicija}, \df{kompozitum}) \df{relacij} definiran kot
                \[\srel \circ \rel \dfeq \set[1]{(x, z) \in X \times Z}{\some{y}[Y]{\rel[x][y] \land \srel[y][z]}}\]
                (po vzoru preslikav tudi sklop relacij pišemo v obratnem vrstnem redu; glej razdelek~\ref{RAZDELEK: Izpeljava preslikav iz relacij}). Opazimo: domena $\srel \circ \rel$ je domena $\rel$, kodomena $\srel \circ \rel$ je kodomena $\srel$. Sklapljanje je asociativna operacija, torej pri sklopu večih relacij oklepaji niso pomembni.

                \begin{vaja}
                        Dokaži, da je sklapljanje relacij asociativno!
                \end{vaja}

                Večkraten sklop relacije $\rel \subseteq X \times X$ same s sabo označimo
                \[\rel^n \dfeq \underbrace{\rel \circ \rel \circ \ldots \circ \rel}_{\text{$n$ $\rel$-jev}}\]
                za $n \in \NN_{\geq 2}$. Seveda je smiselno definirati, da je $\rel^1$ enak $\rel$ in da je $\rel^0$ relacija enakosti na množici $X$, saj je to enota za sklapljanje relacij na $X$, tj.~$=_X \circ \rel = \rel = \rel \circ =_X$ (premisli, da je to res!).

                \begin{zgled}
                        Naj bo $\rel \subseteq X \times X$ relacija. Tedaj iz grafa relacije zlahka razberemo, kaj je $\rel^n$: elementa $a, b \in X$ sta v relaciji $\rel^n$ natanko tedaj, ko imamo pot dolžine $n$ od $a$ do $b$ (to deluje tudi za $n = 1$ in $n = 0$). Naj primer, če je $\rel$ relacija iz zgleda~\ref{ZGLED: graf relacije}, tedaj graf relacije $\rel^3$ izgleda takole.

                        \note{graf $\rel^3$}
                \end{zgled}

                Za poljubno relacijo $\rel \subseteq X \times Y$ definiramo \df{obratno} (\df{inverzno}) \df{relacijo} kot
                \[\rel^{-1} \dfeq \set{(y, x) \in Y \times X}{\rel[x][y]}\]
                (torej ima obratna relacija glede na izvorno zamenjano domeno in kodomeno). Posledično lahko za poljubno relacijo $\rel \subseteq X \times X$ definiramo njeno potenco s poljubno celo stopnjo: $\rel^{-n} \dfeq (\rel^{-1})^n = (\rel^n)^{-1}$.

                \begin{vaja}
                        Preveri, da velja $(\srel \circ \rel)^{-1} = \rel^{-1} \circ \srel^{-1}$!
                \end{vaja}

                \begin{zgled}
                        Graf relacije, ki je obratna relaciji $\rel \subseteq X \times X$, dobimo tako, da v grafu relacije $\rel$ obrnemo puščice. Na primer, če je $\rel$ relacija iz zgleda~\ref{ZGLED: graf relacije}, tedaj graf relacije $\rel^{-1}$ izgleda takole.

                        \note{graf $\rel^{-1}$}
                \end{zgled}

                \begin{zgled}
                        Naj bo $L$ množica ljudi. Vpeljimo oznake za naslednje relacije na $L$:
                        \begin{itemize}
                                \item
                                        $\texttt{St}$ je relacija \qt{je starš od},
                                \item
                                        $\texttt{Oč}$ je relacija \qt{je oče od},
                                \item
                                        $\texttt{Ma}$ je relacija \qt{je mati od},
                                \item
                                        $\texttt{Si}$ je relacija \qt{je sin od},
                                \item
                                        $\texttt{Hč}$ je relacija \qt{je hči od},
                                \item
                                        $\texttt{Br}$ je relacija \qt{je brat od},
                                \item
                                        $\texttt{Se}$ je relacija \qt{je sestra od}
                        \end{itemize}

                        Na primer: Marko $\texttt{Br}$ Metka pomeni \qt{Marko je brat od Metke.} (oz.~v lepši slovenščini \qt{Marko je Metkin brat.}).

                        Velja med drugim:

                        \begin{tabular}{l}
                                $\texttt{Oč} \cup \texttt{Ma} = \texttt{St}$, \\
                                $\texttt{St} \circ \texttt{St} = \texttt{St}^2 = \text{\qt{je stari starš od}}$, \\
                                $\texttt{St} \circ \texttt{Br} = \text{\qt{je stric od}}$, \\
                                $\texttt{Br} \cup \texttt{Se} = \text{\qt{je sorojenec od}}$, \\
                                $\texttt{St}^{-1} = \text{\qt{je otrok od}}$, \\
                                $\bigcup_{n \in \NN_{\geq 1}} \texttt{St}^n = \text{\qt{je prednik od}}$, \\
                                $\bigcup_{n \in \NN_{\geq 1}} \texttt{St}^{-n} = \text{\qt{je potomec od}}$, \\
                                $\texttt{St} \circ (\texttt{Br} \cup \texttt{Se}) \circ \texttt{Hč} = \text{\qt{je sestrična od}}$.
                        \end{tabular}

                        Sklapljanje relacij ni komutativno; na primer $\texttt{Ma} \circ \texttt{Oč}$ je stari oče po materini strani, $\texttt{Oč} \circ \texttt{Ma}$ pa stara mama po očetovi strani.

                        \davorin{V tem zgledu sicer predpostavljamo, da je vsaka oseba bodisi moškega bodisi ženskega spola, kar ni čisto res. Ima kdo kakšno idejo, kako se temu izogniti (in še vedno imeti lahko razumljiv zgled)?}
                \end{zgled}

                \note{Na smiselnem mestu omenimo še zožitve relacij (tako domene kot kodomene).}


        \section{Lastnosti relacij}\label{RAZDELEK: Lastnosti relacij}

                Vemo, da so na primer racionalna števila uporabnejša od celih, saj lahko v okviru njih dodatno delimo --- z drugimi besedami, racionalna števila imajo več uporabne \emph{strukture} oz.~več uporabnih \emph{lastnosti}. Podobno za relacije obstajajo lastnosti, ki so se skozi prakso izkazale za zelo uporabne. Nekatere izmed njih si bomo ogledali v tem razdelku.

                Vse sledeče lastnosti se nanašajo na dvomestno relacijo z isto domeno in kodomeno.

                \begin{definicija}
                        Naj bo $\rel \subseteq X \times X$ relacija.
                        \begin{itemize}
                                \item
                                        Relacija $\rel$ je \df{povratna} (ali \df{refleksivna}), kadar velja
                                        \[\xall{x}[X]{\rel[x][x]},\]
                                        tj.~vsak element je v relaciji s samim sabo.
                                \item
                                        Relacija $\rel$ je \df{nepovratna} (ali \df{irefleksivna}), kadar velja
                                        \[\xall{x}[X]{\lnot(\rel[x][x])},\]
                                        tj.~noben element ni v relaciji s samim sabo.
                                \item
                                        Relacija $\rel$ je \df{somerna} (ali \df{simetrična}), kadar velja
                                        \[\all{x, y}[X]{\rel[x][y] \implies \rel[y][x]},\]
                                        tj.~če je en element v relaciji z drugim, je tudi drugi s prvim.
                                \item
                                        Relacija $\rel$ je \df{protisomerna} (ali \df{antisimetrična}), kadar velja
                                        \[\all{x, y}[X]{\rel[x][y] \land \rel[y][x] \implies x = y},\]
                                        tj.~dva elementa sta obojestransko v relaciji samo v primeru, če gre za en in isti element.

                                        \davorin{Mogoče pretiravam s slovenskimi imeni\ldots}
                                \item
                                        Relacija $\rel$ je \df{nesomerna} (ali \df{asimetrična}), kadar velja
                                        \[\xall{x, y}[X]{\lnot(\rel[x][y] \land \rel[y][x])},\]
                                        tj.~nobena dva elementa nista obojestransko v relaciji.
                                \item
                                        Relacija $\rel$ je \df{prehodna} (ali \df{tranzitivna}), kadar velja
                                        \[\all{x, y, z}[X]{\rel[x][y] \land \rel[y][z] \implies \rel[x][z]},\]
                                        tj.~če je en element v relaciji z drugim in drugi s tretjim, je tudi prvi v relaciji s tretjim.
                                \item
                                        Relacija $\rel$ je \df{neprehodna} (ali \df{intranzitivna}), kadar velja
                                        \[\xall{x, y, z}[X]{\lnot(\rel[x][y] \land \rel[y][z] \land \rel[x][z])},\]
                                        tj.~če je en element v relaciji z drugim in drugi s tretjim, prvi ne more tudi biti v relaciji s tretjim.
                                \item
                                        Relacija $\rel$ je \df{enolična}, kadar velja
                                        \[\all{x, y, z}[X]{\rel[x][y] \land \rel[x][z] \implies y = z},\]
                                        tj.~vsak element je v relaciji s kvečjemu enim elementom.
                                \item
                                        Relacija $\rel$ je \df{celovita}, kadar velja
                                        \[\xall{x}[X]{\xsome{y}[Y]{\rel[x][y]}},\]
                                        tj.~vsak element je v relaciji z vsaj enim elementom, se pravi $\dd{f} = \dom(f)$.
                                \item
                                        Relacija $\rel$ je \df{sovisna}, kadar velja
                                        \[\all{x, y}[X]{x \neq y \implies \rel[x][y] \lor \rel[y][x]},\]
                                        tj.~za vsaka dva različna elementa velja, da je eden od njiju v relaciji z drugim.
                                \item
                                        Relacija $\rel$ je \df{strogo sovisna}, kadar velja
                                        \[\all{x, y}[X]{\rel[x][y] \lor \rel[y][x]},\]
                                        tj.~za vsaka dva elementa velja, da je eden od njiju v relaciji z drugim.
                        \end{itemize}
                \end{definicija}

                \begin{zgled}
                        Za nekaj običajnih relacij si oglejmo njihove lastnosti.
                        \begin{itemize}
                                \item
                                        Relacija $\leq$ na $\NN$, $\ZZ$, $\QQ$, $\RR$ je refleksivna, antisimetrična, tranzitivna in strogo sovisna.
                                \item
                                        Relacija $<$ na $\NN$, $\ZZ$, $\QQ$, $\RR$ je irefleksivna, asimetrična, tranzitivna in sovisna.
                                \item
                                        Relacija deljivosti $|$ na $\NN_{\geq 1}$ je refleksivna, antisimetrična in tranzitivna.
                                \item
                                        Relacija $\subseteq$ na $\pst(X)$ je refleksivna, antisimetrična in tranzitivna.
                                \item
                                        Relacija enakosti $=_X$ na katerikoli množici $X$ je refleksivna, simetrična, antisimetrična, tranzitivna in enolična.
                        \end{itemize}
                \end{zgled}

                Lastnosti operacij smo podali z izjavami, ampak lahko jih na ekvivalenten način podamo z operacijami ali lastnostmi grafa --- glej tabelo~\ref{TABELA: Lastnosti relacije}.

                \davorin{Ko \LaTeX\ hoče biti neumen, zna biti precej neumen. Tabelo~\ref{TABELA: Lastnosti relacije} vrže na konec celotnega poglavja, čeprav mu je zapovedano, da jo naj da \emph{prav sem}.}

                \begin{table}[!ht]
                        \centering
                        \newcommand{\opis}[1]{\begin{minipage}{0.45\textwidth}\begin{center}{#1}\end{center}\end{minipage}}
                        \def\arraystretch{3}
                        \begin{tabular}{|ccc|}
                                \hline
                                \textbf{Lastnost relacije} & \textbf{Izražava z operacijami} & \textbf{Lastnost grafa} \\
                                \hline
                                refleksivnost & $=_X \subseteq \rel$ & \opis{Vsako vozlišče ima zanko.} \\
                                irefleksivnost & $=_X \cap \rel = \emptyset$ & \opis{Nobeno vozlišče nima zanke.} \\
                                simetričnost & $\rel = \rel^{-1}$ & \opis{Vsaka puščica ima nasprotno puščico.} \\
                                antisimetričnost & $\rel \cap \rel^{-1} \subseteq =_X$ & \opis{Edine puščice z nasprotnimi puščicami so zanke.} \\
                                asimetričnost & $\rel \cap \rel^{-1} = \emptyset$ & \opis{Nobena puščica nima nasprotne puščice.} \\
                                tranzitivnost & $\rel^2 \subseteq \rel$ & \opis{Za vsako pot pozitivne dolžine obstaja puščica, ki gre od začetka do konca poti.} \\
                                intranzitivnost & $\rel^2 \cap \rel = \emptyset$ & \opis{Za nobeno pot pozitivne dolžine ne obstaja puščica, ki gre od začetka do konca poti.} \\
                                enoličnost & $\rel \circ \rel^{-1} \subseteq =_X$ & \opis{Iz vsakega vozlišča gre kvečjemu ena puščica.} \\
                                celovitost & $=_X \subseteq \rel^{-1} \circ \rel$ & \opis{Iz vsakega vozlišča gre vsaj ena puščica.} \\
                                sovisnost & $=_X \cup \rel \cup \rel^{-1} = X$ & \opis{Vsaki dve različni vozlišči sta povezani s puščico.} \\
                                stroga sovisnost & $\rel \cup \rel^{-1} = X$ & \opis{Vsaki dve vozlišči sta povezani s puščico.} \\
                                \hline
                        \end{tabular}
                        \caption{Lastnosti relacije $\rel \subseteq X \times X$ in njihove karakterizacije}\label{TABELA: Lastnosti relacije}
                \end{table}

                \begin{vaja}
                        Dokaži, da so vse karakterizacije v vsaki vrstici tabele~\ref{TABELA: Lastnosti relacije} res ekvivalentne!
                \end{vaja}

                Marsikdaj imamo sledeči problem: za določene pare elementov $(x_i, y_i)_{i \in I}$ hočemo, da so v neki relaciji in relacija mora zadoščati predpisani lastnosti. Kako definirati takšno relacijo? Smiselna izbira je vzeti najmanjšo relacijo s predpisano lastnostjo, ki vsebuje vse $(x_i, y_i)$. V ta namen definiramo pojem ogrinjače relacij.

                \begin{definicija}
                        Naj bo $\rel \subseteq X \times X$ relacija in $\mathscr{L}$ lastnost relacij na $X$. Najmanjša relacija na $X$, ki vsebuje $\rel$ in ima lastnost $\mathscr{L}$, se imenuje \df{$\mathscr{L}$-ogrinjača} ali \df{$\mathscr{L}$-ovojnica} relacije $\rel$.
                \end{definicija}

                Ogrinjača relacije je dobro definirana (v smislu, da je enolično določena): če imamo dve relaciji $\rel$ in $\srel$, ki obe vsebujeta dano relacijo in imata lastnost $\mathscr{L}$ ter sta najmanjši taki, mora potem veljati, da sta vsebovani ena v drugi, tj.~$\rel \subseteq \srel$ in $\srel \subseteq \rel$, kar pomeni, da sta enaki.

                Ni pa nujno, da ogrinjača dane relacije za dano lastnost sploh obstaja --- na primer, irefleksivna ogrinjača ne obstaja za nobeno relacijo, ki ni že sama po sebi irefleksivna (premisli, zakaj). Seveda, če relacija je irefleksivna, tedaj je svoja lastna irefleksivna ogrinjača. To očitno velja v splošnem: če ima relacija lastnost $\mathscr{L}$, je enaka svoji $\mathscr{L}$-ogrinjači.

                Premislimo, kdaj smo lahko gotovi, da ogrinjača obstaja.

                \begin{definicija}
                        Naj bo $X$ množica in $\mathscr{L}$ lastnost relacij na $X$. Rečemo, da je $\mathscr{L}$ \df{presečno dedna}, kadar velja: poljuben presek relacij na $X$ z lastnostjo $\mathscr{L}$ prav tako ima lastnost $\mathscr{L}$.
                \end{definicija}

                \begin{vaja}\label{VAJA: presečna dednost zaprta za konjunkcije}
                        Dokaži: konjunkcija končno mnogo presečno dednih lastnosti relacij na dani množici je presečno dedna.
                \end{vaja}

                \begin{trditev}\label{TRDITEV: obstoj ogrinjače iz presečne dednosti}
                        Če je $\mathscr{L}$ presečno dedna lastnost relacij na $X$, tedaj za vsako relacijo $\rel$ na $X$ obstaja njena $\mathscr{L}$-ogrinjača, in sicer je enaka preseku vseh relacij na $X$, ki vsebujejo $\rel$ in imajo lastnost $\mathscr{L}$.
                \end{trditev}

                \begin{dokaz}
                        Naj bo $\srel$ presek vseh relacij na $X$, ki vsebujejo $\rel$ in imajo lastnost $\mathscr{L}$. Posledično je $\srel$ vsebovana v vseh relacijah na $X$ z lastnostjo $\mathscr{L}$, ki vsebujejo $\rel$. Ker je $\mathscr{L}$ presečno dedna lastnost, jo ima tudi $\srel$.
                \end{dokaz}

                Kako pa vemo, kdaj je lastnost presečno dedna? Včasih lahko to razberemo kar iz oblike logične formule, s katero je lastnost podana.

                \begin{izrek}\label{IZREK: presečna dednost iz logične oblike}
                        Naj bo $\mathscr{L}$ lastnost relacij na množici $X$, ki jo lahko za poljubno relacijo $\rel$ podamo z zapisom oblike
                        \[\all[1]{x_1, x_2, \ldots, x_n}[X]{\phi(\rel, x_1, x_2, \ldots, x_n) \implies \psi(\rel, x_1, x_2, \ldots, x_n)},\]
                        kjer sta $\phi(\rel, x_1, x_2, \ldots, x_n)$ in $\psi(\rel, x_1, x_2, \ldots, x_n)$ konjunkciji končno mnogo členov oblike $\rel[x_i][x_j]$ --- v posebnem primeru je lahko $\phi(\rel, x_1, x_2, \ldots, x_n)$ konjunkcija nič členov in potem je $\mathscr{L}$ podana z zapisom oblike
                        \[\xall{x_1, x_2, \ldots, x_n}[X]{\psi(\rel, x_1, x_2, \ldots, x_n)}.\]
                        Tedaj je $\mathscr{L}$ presečno dedna lastnost in torej ima vsaka relacija na $X$ $\mathscr{L}$-ogrinjačo.
                \end{izrek}

                \begin{dokaz}
                        Naj bo $(\rel_i)_{i \in I}$ poljubna družina relacij na $X$ z lastnostjo $\mathscr{L}$ in naj bo $\rel \dfeq \bigcap_{i \in I} \rel_I$ njen presek. Dokazujemo, da $\mathscr{L}$ velja za $\rel$.

                        Vzemimo poljubne $x_1, x_2, \ldots, x_n \in X$, za katere velja $\phi(\rel, x_1, x_2, \ldots, x_n)$. Ker je $\phi(\rel, x_1, x_2, \ldots, x_n)$ konjunkcija členov oblike $\rel[x_i][x_j]$, velja tudi $\phi(\rel_i, x_1, x_2, \ldots, x_n)$ za vsak $i \in I$. Po predpostavki torej velja $\psi(\rel_i, x_1, x_2, \ldots, x_n)$ za vsak $i \in I$.

                        Vzemimo poljuben člen $\rel[x_a][x_b]$ iz $\psi(\rel, x_1, x_2, \ldots, x_n)$. Videli smo, da velja $x_a \mathrel{\rel_i} x_b$ za vsak $i \in I$, torej velja $\rel[x_a][x_b]$.

                        Vidimo, da pod našimi predpostavkami velja $\psi(\rel, x_1, x_2, \ldots, x_n)$. Sklenemo, da velja lastnost $\mathscr{L}$ za relacijo $\rel$.
                \end{dokaz}

                \begin{posledica}\label{POSLEDICA: obstoj ogrinjač}
                        Za naslednje lastnosti relacij (in njihovo poljubno konjunkcijo) vselej obstaja ogrinjača: refleksivnost, simetričnost, tranzitivnost.
                \end{posledica}

                \begin{dokaz}
                        Vse izmed naštetih lastnosti se po definiciji dajo zapisati v obliki iz izreka~\ref{IZREK: presečna dednost iz logične oblike}. Za njihovo konjunkcijo glej še vajo~\ref{VAJA: presečna dednost zaprta za konjunkcije} in trditev~\ref{TRDITEV: obstoj ogrinjače iz presečne dednosti}.
                \end{dokaz}

                \begin{vaja}
                        Dokaži, da za poljubno relacijo $\rel$ na množici $X$ velja spodnja tabela!
                        \begin{center}
                                \begin{tabular}{|c|c|}
                                        \hline
                                        \textbf{Lastnost} & \textbf{Ogrinjača relacije $\rel$} \\
                                        \hline
                                        refleksivnost & $\rel \cup =_X$ \\
                                        simetričnost & $\rel \cup \rel^{-1}$ \\
                                        tranzitivnost & $\bigcup_{n \in \NN_{\geq 1}} \rel^n$ \\
                                        \hline
                                \end{tabular}
                        \end{center}
                \end{vaja}

                \note{ena izmed nalog: Za relacijo $\rel[n][(n+1)]$ na $\NN$ (ali $\ZZ$) preveri, da je njena tranzitivna ogrinjača $<$.}


        \section{Izpeljava preslikav iz relacij}\label{RAZDELEK: Izpeljava preslikav iz relacij}

                Ko definiramo temeljne matematične pojme, imamo določeno mero izbire, kaj vzamemo za izvoren pojem, kaj pa definiramo preko drugih pojmov. V tej knjigi smo od začetka vzeli preslikave za bolj osnoven pojem in relacije lahko definiramo s pomočjo preslikav (kot omenjeno v opombi~\ref{OPOMBA: definicija relacij}, relacijo lahko definiramo kot družino preslikav), lahko pa postopamo tudi obratno --- pojem preslikave izpeljemo iz pojma relacije. Kako to gre, si bomo pogledali v tem razdelku.

                \begin{definicija}
                        \df{Delna preslikava} (ali \df{delna funkcija} ali \df{parcialna funkcija}) je enolična dvomestna relacija.
                \end{definicija}

                Kot dvomestna relacija ima vsaka delna preslikava določeno domeno, kodomeno, definicijsko območje in zalogo vrednosti. Če je $f$ delna preslikava z domeno $X$ in kodomeno $Y$, to zapišemo kot $f\colon X \parto Y$.

                V primeru delne preslikave podmnožico produkta domene in kodomene, ki določa relacijo, označimo z $\graph{f}$ in imenujemo \df{graf} delne preslikave $f$ (ne zamešaj tega s prej definiranim pojmom grafa relacije --- prejšnji pojem je pomenil graf v smislu teorije grafov, sedanji pojem pa graf v smislu preslikav). Delna preslikava je torej v celoti podana z informacijo o domeni, kodomeni in grafu.

                Ideja je, da za delno preslikavo $f\colon X \parto Y$ za vsak $x \in \dd{f}$ obstaja natanko en $y \in Y$, s katerim je $x$ v relaciji. To potem zapišemo $f(x) = y$. Torej, če je $x$ v definicijskem območju, rečemo, da je $f(x)$ definiran, kar zapišemo $\isdefined{f(x)}$, in v tem primeru je $f(x)$ enak vrednosti, s katero je $x$ v relaciji. V nasprotnem primeru rečemo, da $f(x)$ ni definiran.

                Če imamo dve vrednosti, ki morda nista definirani, ni posebej smiselno pisati enakosti med njima. Smiselna relacija med njima je \df{Kleenejeva enakost}, kar pišemo $f(x) \kleq g(y)$, kar pomeni naslednje: leva stran $f(x)$ je definirana natanko tedaj, ko je definirana desna stran $g(y)$, in če sta obe definirani, sta enaki.

                \begin{zgled}
                        Deljenje na realnih številih lahko obravnavamo kot delno preslikavo $/\colon \RR \times \RR \parto \RR$; njeno definicijsko območje je $\dd{/} = \RR \times \RR_{\neq 0}$. Za vsak $x \in \RR$ velja $\frac{x}{x^2} \kleq \frac{1}{x}$, ne pa tudi $\frac{x^2}{x} \kleq x$ (premisli, zakaj).
                \end{zgled}

                \begin{zgled}
                        Delne preslikave so zelo uporabne v računalništvu. Za algoritme pričakujemo, da jim podamo vhodne podatke in bodo potem vrnili željene izhodne podatke. Zgodi se pa lahko, da se algoritem pri nekaterih vhodnih podatkih nikoli ne ustavi (ali javi napako), se pravi, ne dobimo rezultata. Če je $P$ množica možnih podatkov, lahko poljuben algoritem obravnavamo kot delno preslikavo $P \parto P$.\footnote{Natančneje, to velja za deterministične algoritme (takšne, ki se pri enakih vhodnih podatkih vedno enako obnašajo). V primeru nedeterminističnih algoritmov dobimo splošno relacijo na $P$.}

                        Izkaže se, da za nekatere probleme ne obstaja računski postopek, ki bi pri vseh možnih vnosih vrnil pravilen odgovor. Primer tega je \df{problem zaustavitve}: želimo algoritem, ki kot vhodna podatka sprejme poljuben algoritem in poljuben vnos ter se odloči, ali se dani algoritem pri danem vnosu ustavi. Kakršenkoli program, ki sprejme takšna podatka in nikoli ne vrne napačnega rezultata, nujno določa delno preslikavo, ki ni povsod definirana. \davorin{Verjetno bomo nekje hoteli imeti razdelek o diagonalizaciji; morda lahko tja dodamo dokaz te trditve.}
                \end{zgled}

                \begin{definicija}
                        \df{Preslikava} (ali \df{funkcija}) je celovita (z drugimi besedami, povsod definirana) delna preslikava. Če je domena preslikave $f$ množica $X$ in kodomena množica $Y$, to zapišemo kot $f\colon X \to Y$.
                \end{definicija}

                Seveda lahko vsako delno preslikavo zožimo do preslikave: delna preslikava $f\colon X \parto Y$ porodi preslikavo $\rstr{f}_{\dd{f}}\colon \dd{f} \to Y$.

                \begin{vaja}
                        Operacijo sklapljanja $\circ$ smo definirali za splošne relacije (razdelek~\ref{RAZDELEK: Operacije z relacijami}). Preveri, da se ta definicija ujema z običajno definicijo sklapljanja preslikav. Premisli še, kaj je sklop delnih preslikav.
                \end{vaja}


        \section{Relacije urejenosti}\label{RAZDELEK: Relacije urejenosti}

                Že od začetka tega poglavja kot klasične primere relacij podajamo razne urejenosti, kot $\leq$ in $<$. V tem razdelku si bomo ogledali, kakšne lastnosti morajo imeti relacije, da na določen način \qt{urejajo} množico.

                Sledeča definicija povzame štiri tipične primere relacij urejenosti.

                \begin{definicija}
                        Naj bo $X$ množica in $\preceq$ relacija na $X$. Tedaj:
                        \begin{itemize}
                                \item
                                        relacija $\preceq$ je \df{šibka urejenost}, kadar je refleksivna in tranzitivna,
                                \item
                                        relacija $\preceq$ je \df{delna urejenost}, kadar je antisimetrična šibka urejenost (tj.~refleksivna, tranzitivna, antisimetrična),
                                \item
                                        relacija $\preceq$ je \df{linearna urejenost}, kadar je strogo sovisna delna urejenost (tj.~refleksivna, tranzitivna, antisimetrična, strogo sovisna),
                                \item
                                        relacija $\preceq$ je \df{stroga linearna urejenost}, kadar je irefleksivna, tranzitivna in sovisna.
                        \end{itemize}
                        \davorin{Poimenovanja v zvezi s sovisnostjo in strogostjo sem povzel po trenutnih predavanjih iz Logike in množic, ampak mislim, da bi se strogost lahko naredila bolj konsistentna.}
                \end{definicija}

                V tej definiciji smo uporabili znak $\preceq$ za relacijo. Pogosto uporabimo kakšen takšen znak, če hočemo sugerirati, da gre za relacijo urejenosti.

                Tipična primera linearne oz.~stroge linearne urejenosti sta relaciji $\leq$ in $<$ na številskih množicah $\NN$, $\ZZ$, $\QQ$, $\RR$. Tipičen primer delne urejenosti je relacija inkluzije $\subseteq$ na katerikoli potenčni množici $\pst(X)$ (če ima $X$ vsaj dva elementa, ta relacija ne bo linearna).

                Primere šibkih urejenosti pogosto dobimo na sledeči način. Naj bo $f\colon X \to Y$ preslikava in $\preceq_Y$ neka relacija urejenosti na $Y$. Za poljubna $a, b \in X$ definirajmo
                \[a \preceq_X b \dfeq f(a) \preceq_Y f(b).\]
                Tudi če je $\preceq_Y$ močnejše vrste relacija --- delna ali linearna urejenost --- je relacija $\preceq_X$ v splošnem zgolj šibka urejenost na $X$.

                \note{še več primerov}

                \note{razlaga imen relacij}

                \note{najmanjši/največji, minimalni/maksimalni elementi, natančne meje}


        \section{Ekvivalenčne relacije in kvocientne množice}

                Ena temeljnih matematičnih dejavnosti je \df{abstrakcija} \davorin{pojmovanje?}, tj.~iz posamičnih primerov izluščimo njihovo temeljno, bistveno lastnost in potem delamo s to lastnostjo. \davorin{To je pomembna stvar. Dajmo to razlago čimbolj izboljšati.} Na primer, vemo, kaj pomeni \qt{pet rac}, \qt{pet avtov}, \qt{pet sekund}, ampak kaj pomeni \qt{pet}?

                V tem razdelku si bomo ogledali, kako lahko formalno abstrahiramo pojme s posamičnih primerov s pomočjo ekvivalenčnih relacij.

                \begin{definicija}
                        \df{Ekvivalenčna relacija} je relacija, ki je refleksivna, simetrična in tranzitivna.
                \end{definicija}

                Ekvivalenčne relacije tipično označimo z $\equ$ (obstaja več načinov, kako to preberemo: vijuga, tilda, kača\ldots) ali čim podobnih.

                \begin{zgled}
                        Vsaka množica $X$ ima najmanjšo ekvivalenčno relacijo --- enakost $=_X$ --- in največjo --- polno relacijo $X \times X$.
                \end{zgled}

                \begin{zgled}
                        Za poljubni celi števili $a, b \in \ZZ$ definiramo: $a$ je v relaciji z $b$, kadar sta $a$ in $b$ iste parnosti. To določa ekvivalenčno relacijo na $\ZZ$.
                \end{zgled}

                Za poljubno relacijo $\rel \subseteq X \times X$ in poljuben $a \in X$ lahko definiramo
                \[\ec[\rel]{a} \dfeq \set{x \in X}{\rel[a][x]}.\]
                Torej, $\ec[\rel]{a}$ sestoji iz vseh elementov, s katerimi je $a$ v relaciji. V primeru, da imamo ekvivalenčno relacijo $\equ$, imenujemo množico $\ec[\equ]{a}$ \df{ekvivalenčni razred} elementa $a$. Kadar je jasno, za katero ekvivalenčno relacijo gre, pogosto ekvivalenčne razrede krajše označujemo kar z $\ec{a}$.

                Bistvo ekvivalenčne relacije je, da ekvivalenčni razredi tvorijo razbitje množice.

                \davorin{Razbitje množice bomo verjetno definirali že prej, najbrž pri vsotah množic. Če ne, potem na tem mestu pride še definicija razbitja.}

                \davorin{Marko raje uporablja izraz `razdelitev množice', ker se mu `razbitje množice' zdi preveč \qt{nasilno}. ;) Kakšna so mnenja drugih? Kateri izraz bi uporabljali?}

                \begin{izrek}[ekvivalenčne relacije natanko ustrezajo razbitjem]
                        Naj bo $X$ poljubna množica.
                        \begin{enumerate}
                                \item
                                        Naslednji trditvi sta ekvivalentni za vsako relacijo $\rel$ na $X$.
                                        \begin{enumerate}
                                                \item
                                                        $\rel$ je ekvivalenčna relacija.
                                                \item
                                                        $\set[1]{\ec[\rel]{a}}{a \in X}$ je razbitje množice $X$.
                                        \end{enumerate}
                                \item
                                        Za vsako razbitje množice $X$ obstaja enolično določena ekvivalenčna relacija $\equ$ na $X$, tako da je razbitje enako $\set[1]{\ec[\equ]{a}}{a \in X}$.
                        \end{enumerate}
                \end{izrek}

                \begin{dokaz}
                        \begin{enumerate}
                                \item
                                        \begin{itemize}
                                                \item\proven{$(\text{\textit{a}} \impl \text{\textit{b}})$}
                                                \item\proven{$(\text{\textit{b}} \impl \text{\textit{a}})$}
                                        \end{itemize}
                                \item
                        \end{enumerate}
                        \note{dokončaj dokaz}
                \end{dokaz}

                Če je $\equ$ ekvivalenčna relacija na množici $X$, tedaj množico vseh njenih ekvivalenčnih razredov označimo z
                \[X/_\equ \dfeq \set[1]{\ec{a}}{a \in X}\]
                in imenujemo \df{kvocientna množica} množice $X$ po relaciji $\equ$.

                \note{kvocientna množica kot množica abstrahiranih pojmov}

                \begin{vaja}
                        Iz posledice~\ref{POSLEDICA: obstoj ogrinjač} sklepamo, da za vsako relacijo na katerikoli množici obstaja njena ekvivalenčna ogrinjača. Dokaži: če je $\rel$ relacija na množici $X$, tedaj je njena ekvivalenčna ogrinjača enaka
                        \[\bigcup_{n \in \NN} (\rel \cup \rel^{-1})^n.\]
                \end{vaja}

                \begin{vaja}
                        Naj bo $(X, \preceq)$ šibka urejenost. Za poljubna $a, b \in X$ definiramo
                        \[a \approx b \dfeq a \preceq b \land b \preceq a.\]
                        \begin{enumerate}
                                \item
                                        Preveri, da je $\approx$ ekvivalenčna relacija na množici $X$.
                                \item
                                        Na kvocientni množici $X/_\approx$ definiramo relacijo $\leq$ na sledeči način: za poljubna $a, b \in X$ naj velja
                                        \[\ec{a} \leq \ec{b} \dfeq a \preceq b.\]
                                        Dokaži, da ta predpis podaja dobro definirano relacijo na $X/_\approx$.
                                \item
                                        Dokaži: $(X/_\approx, \leq)$ je delna urejenost.
                        \end{enumerate}
                        To je kanoničen način, kako šibko urejenost okrepimo do delne urejenosti.
                \end{vaja}

                \note{Kakšen zanimiv zgled uporabe te vaje?}

                Ko smo obravnavali bijekcije v razdelku~\ref{RAZDELEK: Bijektivnost in obratne preslikave}, smo omenili, zakaj je uporabno imeti obrate preslikav. Težava je seveda, da imajo samo bijekcije obrate (v smislu, da so tudi obrati preslikave --- kot relacije seveda imajo obrate), medtem ko včasih želimo obrniti tudi druge preslikave.

                Vzemimo na primer eksponentno funkcijo $\xlam{x}{e^x}$. Če jo obravnavamo kot preslikavo $\RR \to \RR$, seveda nima obrata, saj ni surjektivna. Ideja je, da zožimo kodomeno do zaloge vrednosti --- preslikava $\xlam{x}{e^x}\colon \RR \to \RR_{> 0}$ je bijektivna in posledično lahko definiramo njen obrat (naravni logaritem) $\ln\colon \RR_{> 0} \to \RR$.

                To je standarden trik, če preslikava ni surjektivna. Kaj pa, če ni injektivna? Pogosto v tem primeru zožimo še domeno na območje, na katerem je preslikava injektivna. Na primer, preslikavo $\xlam{x}{x^2}\colon \RR \to \RR$ zožimo do bijekcije $\xlam{x}{x^2}\colon \RR_{\geq 0} \to \RR_{\geq 0}$, kjer imamo obrat $\xlam{x}{\sqrt{x}}\colon \RR_{\geq 0} \to \RR_{\geq 0}$.

                Ima pa ta prostop težave. Prvič, v nasprotju z zožanje kodomene pri zožitvi domene izgubimo določeno količino informacije o preslikavi (kam so se preslikale vrednosti, ki so prej bile v domeni, zdaj pa niso več?). Drugič, izbira zožene domene ni kanonična. Preslikavo $\xlam{x}{x^2}$ bi ravno tako lahko zožili na $\RR_{\leq 0} \to \RR_{\geq 0}$ ali na $\QQ_{\geq 0} \cup (\RR \setminus \QQ)_{\leq 0} \to \RR_{\geq 0}$ ali celo do $\emptyset \to \emptyset$ ali še neskončno drugih možnosti, pri katerih dobimo bijekcijo.

                S pomočjo kvocientov lahko rešimo te probleme in najdemo kanoničen način, kako preslikavo popraviti do injektivne (in če dodamo še zožitev kodomene, do surjektivne in torej v celoti do bijektivne). Vemo že, da sta injektivnost in surjektivnost dualni (razdelek~\ref{RAZDELEK: Injektivnost in surjektivnost}). Kaj je dualno zožitvi kodomene? Odgovor: kvocient domene. Namreč, če zožimo množico, je tako, kot da jo zdaj gledamo od precej bliže --- vidimo samo manjše območje okoli sebe. Kvocienti počnejo obratno --- tako je, kot če bi množico pogledali od precej daleč. Ne vidimo več posamičnih potez, pač pa se te združijo v bolj splošne oblike. (Seveda se ta dualnost, tako kot pri injektivnosti in surjektivnosti, da utemeljiti tudi formalno matematično. \davorin{Bomo govorili o zožkih in kozožkih?})

                \begin{izrek}[naravna razčlenitev preslikave]
                        Za vsako preslikavo $f\colon X \to Y$ obstaja (kanonična) razčlenitev
                        \[f = i \circ \tilde{f} \circ q,\]
                        kjer je $q$ surjekcija, $\tilde{f}$ bijekcija in $i$ injekcija. Konkretno, $q\colon X \to X/_\equ$ je naravna kvocientna preslikava $q(x) = \ec{x}$, pri čemer je ekvivalenčna relacija $\equ$ na $X$ definirana kot
                        \[a \equ b \dfeq f(a) = f(b),\]
                        preslikava $i\colon \rn{f} \hookrightarrow Y$ je vključitev zaloge vrednosti v kodomeno, preslikava $\tilde{f}\colon X/_\equ \to \rn{f}$ pa je v celoti določena s pogojem
                        \[\tilde{f}([x]) = f(x)\]
                        (med drugim to pomeni, da sta množici $X/_\equ$ in $\im(f)$ v bijektivni korespondenci). \davorin{To je vir raznih izrekov o izomorfizmih v algebri. A povemo kaj na to temo?}

                        Za ponazoritev, imamo spodnji diagram.

                        \note{diagram s tikz}
                \end{izrek}

                \begin{dokaz}
                        \note{napiši dokaz}
                \end{dokaz}
\chapter{Strukture}

\note{Struktura na množici. Morfizmi, ki to strukturo ohranjajo. Izomorfnost. Definicija strukturirane množice preko njene karakterizacije --- potrebna obstoj in enoličnost (do izomorfizma). Primeri. Posebej primeri struktur urejenosti (izhaja iz razdelka o strukturah urejenosti v poglavju o relacijah) in osnovnih algebrskih struktur (pride prav kasneje pri konstrukciji številskih množic). Urejenostna in algebrska struktura se združita v pojmu mreže. Definicija (polnih) Boolovih mrež/kolobarjev in povezava z logiko. Širša slika strukturiranih množic --- kategorije.}


%%% Local Variables:
%%% mode: latex
%%% TeX-master: "ucbenik-lmn"
%%% End:

\chapter{Številske množice}

Številske množice (naravna števila, cela števila, \ldots) poznate že od nekdaj. O njih imate zadosti občutka oz.~intuitivne predstave, da jih lahko uporabljate in pridete do pravilnih rezultatov. Tudi v tej knjigi smo jih že kar naprej izkoriščali za razne primere.

Ampak intuitivna predstava je tudi vse, kar zaenkrat imamo o številskih množicah. Nismo še podali natančne matematične definicije zanje, na osnovi katere bi lahko neizpodbitno dokazovali izreke o njih.

Za vajo lahko sami premislite, ali bi znali na tem mestu podati natančno definicijo, kaj pomeni biti naravno, celo, racionalno oz.~realno število. Definicija seveda mora biti natančna --- npr.~reči, da so realna števila tista, ki ležijo na številski premici, ni zadovoljiva definicija (vsaj ne, če ne pojasnite nedvoumno, kaj pomeni ``številska premica'' in kaj pomeni ``ležati'' na njej).

V tem poglavju se bomo sistematično lotili obravnave najpogosteje uporabljanih številskih množic. Podali bomo njihove konstrukcije, karakterizacije in temeljne lastnosti.


\section{Naravna števila}

\subsection{Peanovi aksiomi}

Če vas kdo vpraša, kako dobiti vsa naravna števila, verjetno odgovorite nekaj v naslednjem smislu: naravna števila so $0$ in vsa tista števila, ki jih dobite s prištevanjem enice, tj.~jemanjem naslednika. Torej, začnemo z $0$, vzamemo naslednika in dobimo $1$, nato še enkrat vzamemo naslednika in dobimo $2$ itd.

Prvi, ki je znal to intuitivno predstavo preliti v natančno matematično definicijo, je bil Peano\footnote{Giuseppe Peano (1858 -- 1932) je bil italijanski matematik.} komaj dobro stoletje nazaj. Pogoje, ki jih zahtevamo za neko množico, da jo lahko imenujemo ``množica naravnih števil'', po njem imenujemo \df{Peanovi aksiomi}. \davorin{Nekje bomo predebatirali, kaj je aksiom in zakaj jih uporabljamo. Peanove aksiome povežimo s tem.}

Če boste brskali po literaturi, boste naleteli na mnogo različnih inačic Peanovih aksiomov. Mi bomo izbrali sledečo jedrnato različico.

\begin{definicija}[Peano]\label{definicija:naravna-stevila}
\df{Množica naravnih števil} je množica (običajno označena z $\NN$), skupaj z izbranim njenim elementom (običajno označenim z $0$, kar beremo ``ničla'' ali ``nič'') in preslikavo na tej množici (običajno označeno z $\suc\colon \NN \to \NN$, ki jo imenujemo ``naslednik''), kadar veljajo naslednje lastnosti:
\begin{itemize}
\item
$\suc$ je injektivna preslikava,
\item
$0 \notin \rn{\suc}$,
\item
velja načelo \df{matematične indukcije}: če je $\phi$ predikat na $\NN$, za katerega velja
\[\phi(0) \qquad\qquad \text{in} \qquad\qquad \all{n \in \NN} (\phi(n) \implies \phi\big(\suc(n)\big)),\]
tedaj $\phi$ velja za vse elemente $\NN$.
\end{itemize}
\end{definicija}

Poskusimo si zdaj natančno pojasniti pomen teh pogojev.

S pomočjo elementa $0$ in preslikave $\suc$ lahko v nedogled generiramo elemente množice $\NN$. Začnemo z $0$, nato vzamemo naslednika in dobimo $\suc(0)$, nato vzamemo naslednika tega elementa in dobimo $\suc(\suc(0))$, nato naslednika $\suc(\suc(\suc(0)))$ itd. Takšen zapis je sicer precej nepraktičen --- si predstavljate, da rečete ``dobimo se čez naslednika od naslednika od naslednika od naslednika od naslednika ničle ur'' (namesto ``dobimo se čez pet ur'')? Zato sprejmemo dogovor: $\suc(0)$ označimo krajše z $1$ in preberemo ``ena'', $\suc(\suc(0))$ označimo z $2$ in preberemo ``dve'' in tako naprej.\footnote{Trenutno dogovorjena sistematična imena za števila gredo do \df{centiljona}, ki ga zapišemo z enico, ki ji sledi 600 ničel (vsaj pri nas; ponekod po svetu centiljon pomeni enica s 303 ničlami). To pomeni, da lahko sistematično izrazimo števila do $10^{606}-1$ (= devetsto devetindevetdeset centiljard devetsto devetindevetdeset centiljonov devetsto devetindevetdeset novemnonagintiljard\ldots). Nekateri razširijo to lestvico še z nadaljnjimi latinskimi izpeljankami, obstajajo pa tudi posebna imena za nekatera posamična velika števila, na primer \df{gugol} za $10^{100}$ (od tod izhaja ime spletnega brskalnika Google).}

Smo na ta način dobili neskončno različnih elementov $\NN$? Če ne bi zahtevali zgornjih pogojev, to ne bi bilo nujno. Lahko bi se namreč zaciklali (v smislu, da je naslednik nekega elementa element, ki smo ga že prej navedli).

Včasih je takšno zaciklanje nekaj, kar dejansko hočemo. Na primer, pri algebri boste spoznali tako imenovane \df{ciklične grupe}. Ciklično grupo z $n$ elementi označimo $\ZZ_n$, njene elemente pa kar z $0, 1, \ldots, n-1$. Spodaj je slika ciklične grupe $\ZZ_5$.

\note{slika usmerjenega grafa, ki predstavlja $\ZZ_5$}

Puščice označujejo, kako slika naslednik v tej grupi: naslednik $0$ je $1$, naslednik $1$ je $2$, naslednik $2$ je $3$, naslednik $3$ je $4$, nato pa se zacikla in naslednik $4$ je $0$.

Pogoj $0 \notin \rn{\suc}$ reče, da nič ni naslednik nobenega naravnega števila. Na ta način se izognemo, da bi naravna števila tvorila ciklično grupo.

Obstaja pa še en način, kako se lahko jemanje naslednika zacikla. Vzemimo spodnji primer.

\note{slike polgrupe $\set{0, \ldots, 4}$, ki se zacikla $4 \to 2$}

Nasledniki se lahko zaciklajo tudi pri elementu, ki ni $0$. V danem primeru je naslednik $0$ element $1$, naslednik $1$ je $2$, naslednik $2$ je $3$, naslednik $3$ je $4$, naslednik $4$ pa je $2$.

Zakaj naravna števila niso taka? Ker v danem primeru $\suc$ ni injektivna preslikava. Pogoj o injektivnosti nam v bistvu pove sledeče: naravna števila se ne morejo zaciklati pri nobenem nasledniku.

Vidimo, da se naravna števila ne morejo zaciklati niti na začetku (pri $0$) niti nekje vmes v verigi naslednikov --- torej gredo v nedogled, kot želimo. Z drugimi besedami, $0$, $\suc(0)$, $\suc(\suc(0))$, $\suc(\suc(\suc(0)))$,\ldots so medsebojno različni elementi množice $\NN$ in naravnih števil je posledično neskončno.

Čemu pa služi zadnji pogoj iz definicije~\ref{definicija:naravna-stevila}, tj.~načelo o indukciji? Že brez tega pogoja vemo, da so $0$, $\suc(0)$, $\suc(\suc(0))$, $\suc(\suc(\suc(0)))$,\ldots naravna števila, česar pa ne vemo, je, da so to \emph{vsa} naravna števila --- da torej ni nobenih drugih.

\begin{vaja}
Premisli, da množica $\RR_{> -1}$ z naslednikom $\suc(x) \dfeq x+1$ zadošča vsem pogojem iz definicije~\ref{definicija:naravna-stevila}, razen načelu indukcije.
\end{vaja}

Vidimo, da bi brez načela indukcije lahko imeli v množici $\NN$ odvečna števila (takšna, ki jih ne štejemo kot naravna). S predpostavko o indukciji se to ne more zgoditi. Ta namreč pravi: če neka lastnost velja za začetni element verige $0$, $\suc(0)$, $\suc(\suc(0))$, $\suc(\suc(\suc(0)))$,\ldots in če lahko sklepamo, da kakor hitro ta lastnost velja za določen element verige, velja tudi za naslednjega, potem ta lastnost velja za vsa naravna števila. Če za lastnost vzamemo ``biti element te verige'', iz načela o indukciji sklenemo, da se vsako naravno število nahaja nekje v tej verigi. Peanovi aksiomi torej podajajo strukturo, ki ustreza naši intuitivni predstavi množice naravnih števil.

Glede na to, da je načelo o matematični indukciji eden od osnovnih aksiomov, s katerimi so naravna števila podana, ne preseneča, da je indukcija eden najpogostejših načinov, kako dokazujemo izjave na naravnih številih. Natančneje rečeno, z matematično indukcijo dokazujemo univerzalno kvantificirane izjave na naravnih številih, torej izjave oblike
\[xall{n \in \NN} \phi(n).\]
Po načelu indukcije za dokaz take izjave zadostuje narediti naslednje. Najprej dokažemo
\[\phi(0)\]
(da torej lastnost $\phi$ velja za začetno naravno število). To imenujemo \df{temelj} ali \df{osnova} ali \df{baza} indukcije. Nato dokažemo izjavo
\[\all{n \in \NN} \phi(n) \implies \phi(\suc(n));\]
to imenujemo \df{indukcijski korak}. Z besedami, dokažemo, da kakor hitro velja lastnost $\phi$ za neko naravno število, mora veljati ta lastnost tudi za naslednje.

Intuitivno je jasno, da to mora delovati. Temelj indukcije nam pove, da dana lastnost velja za $0$. Ker zdaj vemo, da velja za $0$, mora po indukcijskem koraku veljati za naslednika ničle, torej za $1$. Zdaj vemo, da velja za $1$, torej mora po indukcijskem koraku veljati tudi za $2$. Tako nadaljujemo: sklepamo, da lastnost velja za $3$, nato za $4$ in tako naprej. Ker se vsa naravna števila pojavijo v verigi naslednikov ničle, mora z indukcijo dokazana lastnost dejansko veljati za vsa naravna števila.

V poglavju~\ref{poglavje:indukcija} se bomo vrnili k indukciji, jo natančneje preučili in si ogledali primere dokazovanja z njo. Na tem mestu pa jo bomo uporabili za izpeljavo \emph{rekurzije}, ki nam bo služila za definicijo nadaljnje strukture na naravnih številih.

\subsection{Rekurzija}

Poenostavljeno povedano, rekurzija pomeni, da določimo vrednost preslikave pri nekem argumentu iz (že prej naračunanih) vrednosti pri manjših argumentih. Tipičen primer rekurzivno podane preslikave je faktoriela: če zapišemo $0! \dfeq 1$ in $n! \dfeq (n+1) \cdot n!$ za vse $n \in \NN$, smo s tem enolično podali preslikavo $!\colon \NN \to \NN$.

Naračunajmo nekaj vrednosti te preslikave. Neposredno iz definicije dobimo $0! = 1$ --- to je \df{temelj} oz.~\df{osnova} oz.~\df{baza} rekurzije. Od tod s pomočjo \df{rekurzijskega koraka} izpeljemo
\[1! = 1 \cdot 0! = 1 \cdot 1 = 1.\]
S pomočjo te vrednosti in z rekurzijskim korakom lahko naračunamo vrednost faktoriele pri naslednjem naravnem številu.
\[2! = 2 \cdot 1! = 2 \cdot 1 = 2\]
In tako naprej.
\[3! = 3 \cdot 2! = 3 \cdot 2 = 6\]
\[4! = 4 \cdot 3! = 4 \cdot 6 = 24\]
\[5! = 5 \cdot 4! = 5 \cdot 24 = 120\]
\[\vdots\]
Vidimo, da lahko po tem postopku prej ali slej naračunamo $n!$ za poljuben $n \in \NN$.

V primeru faktoriele smo neko vrednost naračunali iz predhodne, uporabljajo se pa tudi splošnejše rekurzivne definicije, kjer vrednost naračunamo iz večih prejšnjih. Slovit primer je \df{Fibonaccijevo zaporedje} $F\colon \NN \to \NN$, podano kot $F_0 \dfeq 0$, $F_1 \dfeq 1$ in $F_{n+2} \dfeq F_{n+1} + F_n$ za vse $n \in \NN$. Od tod lahko naračunamo:
\begin{align*}
F_0 &= 0, \\
F_1 &= 1, \\
F_2 &= F_1 + F_0 = 1 + 0 = 1, \\
F_3 &= F_2 + F_1 = 1 + 1 = 2, \\
F_4 &= F_3 + F_2 = 2 + 1 = 3, \\
F_5 &= F_4 + F_3 = 3 + 2 = 5, \\
F_6 &= F_5 + F_4 = 5 + 3 = 8, \\
&\vdots
\end{align*}
Bo pa za naše potrebe zaenkrat zadostovala oblika rekurzije, kjer se skličemo samo na en predhodni člen, in na tako se bomo v tej knjigi tudi omejili. \davorin{Lahko pa vseeno v kakšni vaji zahtevamo od študentov, da zapišejo in dokažejo splošnejše načelo rekurzije.}

Zakaj bi pa sploh podajali preslikave rekurzivno namesto z izrecnim (eksplicitnim) predpisom? Včasih to sledi iz narave problema. Na primer, imamo stanje, ki se razvija korak za korakom, kjer je trenutno stanje odvisno od prejšnjega. Zanima nas, kako se naš sistem razvija, in v tem primeru je naravno podati trenutno stanje sistema kot rekurzivno preslikavo. \note{ponazorimo s primerom}

Včasih preslikavo podamo rekurzivno, ker je rekurzivni predpis mnogo enostavnejši kot izrecni. Na primer, izrecna predpisa za faktorielo in Fibonaccijevo zaporedje sta
\[n! = \int_0^\infty x^n e^{-x} \; dx\]
in
\[F_n = \frac{\Big(\frac{1+\sqrt{5}}{2}\Big)^n - \Big(\frac{1-\sqrt{5}}{2}\Big)^n}{\sqrt{5}}.\]
Odvisno od tega, katera vrednost vas zanima, utegneta biti ta dva predpisa mnogo bolj okorna za računanje, kot pa rekurzivna. Pravzaprav nekaj časa traja, da sploh dokažete, da so rezultati teh predpisov naravna števila!

Včasih pa preslikavo podamo rekurzivno preprosto zato, ker nimamo druge možnosti. Zgornja predpisa sicer podajata preslikavi izrecno, ampak cena za to je uporaba zapletenih operacij na realnih številih, kot so integral, eksponentna funkcija z naravno osnovo in korenjenje. Strogo vzeto smo zaenkrat od številskih množic definirali samo naravna števila, pa še zanje znamo povedati zgolj, kaj je $0$ in kaj je naslednik. V bistvu še ne ``znamo'' niti seštevati!

S pomočjo rekurzije bomo lahko definirali ostalo strukturo, ki jo poznamo na naravnih številih: seštevanje, množenje in tako naprej. Za začetek pa natančno izoblikujmo in dokažimo načelo o rekurziji na naravnih številih. Iz zgornje razprave je jasno, da je rekurzija tesno povezana z indukcijo, od koder jo bomo tudi izpeljali.

\begin{izrek}[Načelo rekurzije]\label{izrek:rekurzija}
Imejmo poljubni množici $X$ in $Y$ ter preslikavi $b\colon X \to Y$ in $r\colon X \times Y \times \NN \to Y$. Tedaj obstaja natanko ena preslikava $f\colon X \times \NN \to Y$, za katero velja
\[f(x, 0) = b(x)\]
in
\[f\big(x, \suc(n)\big) = r\big(x, f(n, x), n\big)\]
za vse $x \in X$ in $n \in \NN$.

Temu natančneje rečemo \df{načelo parametrizirane rekurzije}, ker pri preslikavi $f$ na naravnih številih dopuščamo še poljuben parameter iz množice $X$. Če za $X$ vzamemo enojec, se zgornja izjava reducira na sledeče \df{načelo neparametrizirane rekurzije}.

Če imamo množico $Y$, element $b \in Y$ in preslikavo $r\colon Y \times \NN \to Y$, tedaj obstaja natanko ena preslikava $f\colon \NN \to Y$, za katero velja
\[f(0) = b\]
in
\[f\big(\suc(n)\big) = r\big(f(n), n\big)\]
za vse $n \in \NN$.
\end{izrek}

\begin{dokaz}
\end{dokaz}

Rekurzijo smo na ta način izpeljali iz indukcije, poudarimo pa, da je možen tudi obraten pristop: načelo o rekurziji vzamemo kot osnoven aksiom naravnih števil \emph{namesto} indukcije, nato pa od tod izpeljemo načelo o indukciji. Poglejmo, kako to storimo.

Vzemimo poljuben predikat $\phi\colon \NN \to \tvs$, za katerega velja $\phi(0)$ in $\all{n \in \NN} (\phi(n) \implies \phi(\suc(n)))$. Po načelu rekurzije obstaja natanko ena preslikava $f\colon \NN \to \tvs$, za katero velja $f(0) = \true$ in $f\big(\suc(n)\big) = f(n) \lor \phi\big(\suc(n)\big)$ za vse $n \in \NN$. Ampak predikat $\phi$ sam zadošča tema pogojema, saj lahko izjavo $\phi(n) \implies \phi\big(\suc(n)\big)$ enakovredno zapišemo kot $\phi\big(\suc(n)\big) = \phi(n) \lor \phi\big(\suc(n)\big)$. Očitno pa tudi povsod resničen predikat zadošča danima pogojema, od koder zaključimo $\phi = \lam{n \in \NN}{\true}$.

V tem smislu sta načeli rekurzije in indukcije enakovredni. Kot vidimo, lahko pravzaprav na indukcijo gledamo kot na poseben primer rekurzije, konkretno za preslikave oblike $\NN \to \tvs$. To nam pove, da je ta primer tako generičen, da je iz njega možno dobiti načelo za poljubne preslikave oblike $X \times \NN \to Y$.

\note{Rekurzor kot preslikava. Morda pripomba, ki zgornjo diskusijo poveže s primitivno rekurzijo iz teorije izračunljivosti.}

\subsection{Računske operacije}

Uporabimo zdaj izpeljano rekurzijo za natančno matematično definicijo strukture na naravnih številih, ki jo neformalno poznate že od malih nog. Začnimo z osnovnimi računskimi operacijami.

Seštevanje želimo definirati kot preslikavo $\NN \times \NN \to \NN$. Da ga definiramo rekurzivno, moramo povedati, kaj pomeni prišteti ničlo in kaj pomeni prišteti naslednika nekega števila (izraženo z vsoto, ki jo dobimo iz prištetja tega števila samega). Smiselno je podati naslednje.
\begin{align*}
m + 0 &\dfeq m \\
m + \suc(n) &\dfeq \suc(m + n)
\end{align*}
V tej definiciji $m$ nastopa kot parameter --- se pravi, uporabili bomo načelo parametrizirane rekurzije. Glede na oznake iz izreka~\ref{izrek:rekurzija} smo vzeli $X = \NN$, $Y = \NN$, $b(m) = m$ (torej je $b$ identiteta na $\NN$) in $r(m, v, n) = \suc(v)$ (se pravi, $r$ je kompozicija projekcije na drugo komponento in preslikave naslednika). Po tem izreku dobimo enolično določeno preslikavo $+\colon \NN \times \NN \to \NN$ (ki igra vlogo preslikave $f$ iz izreka).

Dokažimo, da pravkar definirano seštevanje zadošča zakonom, na katere smo navajeni. Začnimo s tem, da preverimo, da je $0$ enota za seštevanje.

Seveda velja $a + 0 = a$ za vse $a \in \NN$ --- to je del definicije seštevanja. Od tod pa ne smemo takoj sklepati na $0 + a = a$, saj še nismo dokazali izmenljivosti seštevanja. Lahko bi na tem mestu začeli z dokazom izmenljivosti, ampak kot bomo videli, bomo za to že potrebovali dejstvo, da je $0$ enota. Dokažimo torej $0 + a = a$ za vse $a \in \NN$ neposredno.

Trditev dokazujemo z indukcijo. Najprej dokažemo trditev za $a = 0$, torej $0 + 0 = 0$. To je res po definiciji.

Privzemimo, da velja $0 + a = a$ za neki $a \in \NN$. Dokazujemo $0 + \suc(a) = \suc(a)$. Preverimo:
\[0 + \suc(a) = \suc(0 + a) = \suc(a).\]

Kaj pa, če namesto $0$ prištejemo $1$? Takrat seveda pričakujemo, da dobimo naslednika. Preverimo.

Za poljuben $a \in \NN$ dobimo $a + 1 = a + \suc(0) = \suc(a + 0) = \suc(a)$. Tukaj sploh nismo potrebovali indukcije. Jo pa potrebujemo za dokaz, da za vsak $a \in \NN$ velja $1 + a = \suc(a)$. Za $a = 0$ je to definicija oznake $1$. Recimo, da za neki $a \in \NN$ velja $1 + a = \suc(a)$. Tedaj $1 + \suc(a) = \suc(1 + a) = \suc(\suc(a))$.

Prepričajmo se zdaj o družilnosti (asociativnosti) seštevanja. Dokazati želimo izjavo
\[\all{a \in \NN}\all{b \in \NN}\all{c \in \NN}{(a + b) + c = a + (b + c)}.\]
Vzemimo poljubna $a, b \in \NN$, notranjo univerzalno kvantificirano izjavo pa dokažimo z indukcijo (po spremenljivki $c$). Če vzamemo $c = 0$, izjava velja: $(a + b) + 0 = a + b = a + (b + 0)$. Privzemimo zdaj, da pri nekem $c \in \NN$ velja $(a + b) + c = a + (b + c)$. Poračunamo
\[(a + b) + \suc(c) = \suc\big((a + b) + c\big) = \suc\big(a + (b + c)\big) = a + \suc(b + c) = a + \big(b + \suc(c)\big).\]

Zdaj lahko dokažemo izmenljivost (komutativnost) seštevanja. Dokazati želimo izjavo
\[\all{a \in \NN}\all{b \in \NN}{a + b = b + a}.\]
Vzemimo poljuben $a \in \NN$, nato pa nadaljujmo z indukcijo (po $b$). Za $b = 0$ trdimo $a + 0 = 0 + a$. To smo že dokazali --- obe strani enakosti sta enaki $a$, saj vemo, da je $0$ enota za seštevanje.

Predpostavimo zdaj, da velja $a + b = b + a$ za neki $b \in \NN$. Izpeljati želimo $a + \suc(b) = \suc(b) + a$. Preverimo:
\[a + \suc(b) = \suc(a + b) = \suc(b + a) = 1 + (b + a) = (1 + b) + a = \suc(b) + a.\]

Na podoben način lahko definiramo množenje in dokažemo njegove lastnosti. Smiselna rekurzivna definicija množenja je sledeča.
\begin{align*}
m \cdot 0 &\dfeq 0 \\
m \cdot \suc(n) &\dfeq m \cdot n + m
\end{align*}
Če primerjamo z izrekom~\ref{izrek:rekurzija}, smo vzeli $X = Y = \NN$, $b(m) = 0$ (torej je $b$ konstantna ničelna preslikava) in $r(m, v, n) = v + m$ (to preslikavo lahko definiramo s pomočjo pravkar definiranega seštevanja). Izrek nam porodi enolično določeno preslikavo $\cdot\colon \NN \times \NN \to \NN$.

Podobno kot prej pri seštevanju za začetek ugotovimo, kaj se zgodi, ko množimo z $0$ oziroma $1$. Po definiciji vemo $a \cdot 0 = 0$ za vse $a \in \NN$. Dokažimo še $0 \cdot a = 0$ za vse $a \in \NN$. Za $a = 0$ velja $0 \cdot 0 = 0$ po definiciji. Vzemimo, da velja $0 \cdot a = 0$ za neki $a \in \NN$. Tedaj $0 \cdot \suc(a) = 0 \cdot a + 0 = 0 + 0 = 0$.

Število $1$ bi morala biti enota za množenje. Preverimo. Najprej $a \cdot 1 = a \cdot s(0) = a \cdot 0 + a = 0 + a = a$. Po drugi strani trditev, da za vse $a \in \NN$ velja $1 \cdot a = a$, dokažemo z indukcijo. Enakost $1 \cdot 0 = 0$ je jasna. Recimo, da trditev velja za neki $a \in \NN$. Tedaj $1 \cdot \suc(a) = 1 \cdot a + 1 = a + 1 = \suc(a)$.

Preden se lotimo družilnosti in izmenljivosti množenja, dokažimo, da je množenje razčlenitveno (distributivno) čez seštevanje. Se pravi, dokazati želimo izjavi
\[\all{a \in \NN}\all{b \in \NN}\all{c \in \NN}{(a + b) \cdot c = a \cdot c  + b \cdot c}\]
in
\[\all{a \in \NN}\all{b \in \NN}\all{c \in \NN}{a \cdot (b + c) = a \cdot b + a \cdot c}.\]
Pri prvi od izjav (desni razčlenitvi) vzemimo poljubna $a, b \in \NN$, nato pa se lotimo indukcije po $c$. Dobimo $(a + b) \cdot 0 = 0 = 0 + 0 = a \cdot 0 + b \cdot 0$. Če velja $(a + b) \cdot c = a \cdot c  + b \cdot c$ za neki $c$, tedaj
\[(a + b) \cdot \suc(c) = (a + b) \cdot c + (a + b) = a \cdot c + b \cdot c + a + b = a \cdot c + a + b \cdot c + b = a \cdot \suc(c) + b \cdot \suc(c).\]
Pri drugi izjavi (levi razčlenitvi) sklepamo podobno: $a \cdot (b + 0) = a \cdot b = a \cdot b + 0 = a \cdot b + a \cdot 0$. Nato privzamemo izjavo za neki $c$ in poračunamo
\[a \cdot \big(b + \suc(c)\big) = a \cdot \suc(b + c) = a \cdot (b + c) + a = a \cdot b + a \cdot c + a = a \cdot b + a \cdot \suc(c).\]

Preverimo zdaj družilnost množenja, torej izjavo
\[\all{a \in \NN}\all{b \in \NN}\all{c \in \NN}{(a \cdot b) \cdot c = a \cdot (b \cdot c)}.\]
Vzemimo poljubna $a, b \in \NN$ in se lotimo indukcije po $c$. Za $c = 0$ dobimo $(a \cdot b) \cdot 0 = 0 = a \cdot 0 = a \cdot (b \cdot 0)$. Predpostavimo izjavo za neki $c$ in poračunamo
\[(a \cdot b) \cdot \suc(c) = (a \cdot b) \cdot c + a \cdot b = a \cdot (b \cdot c) + a \cdot b = a \cdot (b \cdot c + b) = a \cdot \big(b \cdot \suc(c)\big).\]

Naposled preverimo še izmenljivost množenja na naravnih številih, torej izjavo
\[\all{a \in \NN}\all{b \in \NN}{a \cdot b = b \cdot a}.\]
Vzemimo poljuben $a \in \NN$. Za $b = 0$ dobimo $a \cdot 0 = 0 = 0 \cdot a$. Vzemimo, da izjava velja za neki $b$. Tedaj
\[a \cdot \suc(b) = a \cdot b + a = b \cdot a + a = b \cdot a + 1 \cdot a = (b + 1) \cdot a = \suc(b) \cdot a.\]

Na kratko lahko to celotno razpravo povzamemo: množica naravnih števil $\NN$ tvori izmenljiv polkolobar z enico. \davorin{ta pojem bo pojasnjen že v prejšnjem poglavju o strukturah} Seveda pa ne tvori kolobarja; vemo, da naravnih števil ne moremo poljubno odštevati. Še vedno pa lahko odštevanje na naravnih številih podamo kot \emph{delno} operacijo, torej kot delno preslikavo $-\colon \NN \times \NN \parto \NN$. Spomnimo se namreč \note{od polkolobarjev v prejšnjem poglavju}, da je odštevanje delna preslikava natanko tedaj, ko je polkolobar krajšalen.

Dokažimo krajšalnost polkolobarja naravnih števil, torej izjavo
\[\all{a \in \NN}\all{b \in \NN}\all{x \in \NN}\big(a + x = b + x \implies a = b\big).\]
Vzemimo poljubna $a, b \in \NN$, nato pa se kot običajno poslužimo indukcije. Pri $x = 0$ smo takoj na koncu. Privzemimo izjavo $a + x = b + x \implies a = b$ za neki $x$ in naj velja $a + \suc(x) = b + \suc(x)$. Tedaj $\suc(a + x) = \suc(b + x)$ in ker je $\suc$ injektivna preslikava (eden od Peanovih aksiomov!), sklepamo $a + x = b + x$, od tod pa $a = b$.\footnote{Injektivnost preslikave $\suc$ je točno to, kar potrebujemo za krajšalnost. Velja namreč tudi obrat: če imamo $\suc(a) = \suc(b)$, tj.~$a + 1 = b + 1$, in lahko krajšamo, potem $a = b$.}

S pomočjo (delnega) odštevanja lahko definiramo \df{predhodnika} na naravnih številih, in sicer kot $\prd(n) \dfeq n - 1$. Tudi to je zgolj delna preslikava $\prd\colon \NN \parto \NN$; ničla je edino naravno število, ki ni v njenem definicijskem območju.

\begin{vaja}
Dokaži $\all{n \in \NN}{\prd\big(\suc(n)\big) \kleq n}$.
\end{vaja}

Včasih je pa uporabno imeti obliko predhodnika in odštevanja, ki sta celoviti preslikavi. Pri predhodniku se dogovorimo, da se pomaknemo za eno nazaj, če se le da (pri ničli torej ostanemo, kjer smo). To različico predhodnika lahko definiramo z rekurzijo na naslednji način.
\begin{align*}
\tilde{\prd}(0) &\dfeq 0 \\
\tilde{\prd}\big(\suc(n)\big) &\dfeq n
\end{align*}
Po načelu o neparametrizirani rekurziji dobimo enolično določeno preslikavo $\tilde{\prd}\colon \NN \to \NN$ (konkretno, v izreku~\ref{izrek:rekurzija} vzamemo $Y = \NN$, $b = 0$ in $r(v, n) = n$).\footnote{Morda se vam zdi vprašljivo, če bi to definicijo sploh imenovali ``rekurzivna'', saj $\tilde{\prd}\big(\suc(n)\big)$ nismo izrazili s $\tilde{\prd}(n)$ (ali z drugimi besedami, preslikava $r$ ni odvisna od svojega prvega argumenta). Ampak izrek~\ref{izrek:rekurzija} za ta primer še vedno velja in zgornja definicija torej podaja dobro definirano preslikavo $\tilde{\prd}\colon \NN \to \NN$.}

Od tod lahko definiramo tako imenovano \df{prisekano odštevanje} na naravnih številih. Simbol za to operacijo je $\monus$, kar se prebere ``monus'' (torej: $1 + 2$ se bere ``ena plus dve'', $1 - 2$ se bere ``ena minus dve'' in $1 \monus 2$ se bere ``ena monus dve'').

Ideja prisekanega odštevanja je, da zmanjševanec zmanjšamo za tolikšen del odštevanca, kolikor le lahko (tako da še ostanemo v okviru naravnih števil). Z drugimi besedami: če se običajno odštevanje izide v naravnih številih, velja $a \monus b = a - b$, sicer pa velja $a \monus b = 0$. Natančna rekurzivna definicija je sledeča.
\begin{align*}
m \monus 0 &\dfeq n \\
m \monus \suc(n) &\dfeq \tilde{\prd}(m \monus n)
\end{align*}

\note{lastnosti prisekanega odštevanja}

\note{Ena od vaj: rekurzivna definicija in dokaz lastnosti potenciranja}

\subsection{Urejenost}
\subsection{Nadaljnje karakterizacije}


\section{Cela števila}
\section{Racionalna števila}
\section{Realna števila}
\section{Kompleksna števila}

\davorin{Se ustavimo že pri realnih številih? Gremo še dlje do kvaternionov?}


%%% Local Variables:
%%% mode: latex
%%% TeX-master: "ucbenik-lmn"
%%% End:

\chapter{Indukcija}

\note{Dobro osnovano urejene in dobro urejene množice. Indukcija na dobro osnovano urejenih množicah. Strukturna indukcija.}


%%% Local Variables:
%%% mode: latex
%%% TeX-master: "ucbenik-lmn"
%%% End:

\chapter{Kumulativna hierarhija}

\section{Aksiomi teorije množic}
\label{sec:aksiomi-teorije-mnozic}

\andrej{Zagotovo pa ne ZFC, ampak neka aksiomatizacija, ki ima razrede, recimo BGN ali MK.}

% Aksiom izbire ne paše sem, obravnavan bo že dosti prej.

%%% Local Variables:
%%% mode: latex
%%% TeX-master: "ucbenik-lmn"
%%% End:

\chapter{Kardinalna števila}

\section{Končnost in neskončnost}
\section{Števnost}
\section{Kardinalnost množice}

%%% Local Variables:
%%% mode: latex
%%% TeX-master: "ucbenik-lmn"
%%% End:

\chapter{Ordinalna števila}

\davorin{Mogoče združimo kardinalna in ordinalna števila v eno poglavje?}

%%% Local Variables:
%%% mode: latex
%%% TeX-master: "ucbenik-lmn"
%%% End:


\end{document}