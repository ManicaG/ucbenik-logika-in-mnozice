\chapter{Dokazovanje}\label{poglavje:dokazovanje}

        Matematične izsledke običajno podajamo preko jasno izraženih izjav. Med študijem matematike hitro opazite, da se takšne izjave podajajo pod imeni `izrek', `trditev', `lema', {posledica} in podobno. Kdaj uporabiti katerega teh imen ni natanko določeno, pač pa je prepuščeno presoji matematika. Približno vodilo je naslednje:
        \begin{itemize}
                \item
                        \df{izrek}: osrednji, bistven rezultat,
                \item
                        \df{trditev}: stranski rezultat,
                \item
                        \df{lema}: rezultat, ki sam po sebi nima toliko vsebine, se pa uporabi pri dokazovanju pomembnejšega rezultata,\footnote{Sicer ni nujno, da se resnična pomembnost izjav takoj pokaže. Mnogo je primerov, ko se kak matematični članek po določenem času začne ceniti ne toliko zaradi glavnega izreka, pač pa zaradi neke leme, ki se je za dokaz glavnega izreka uporabila.}
                \item
                        \df{posledica}: rezultat, ki je zanimiv sam po sebi, ki pa hitro sledi iz predhodne izjave.
        \end{itemize}

        Če skrbno analizirate izreke, trditve itd.~s predavanj (ali iz matematičnih člankov), opazite, da sestojijo iz treh delov: kontekst, predpostavke, sklepi.
        \begin{itemize}
                \item
                        \df{Kontekst} pove, katere objekte obravnavamo in kakšne vrste so.
                \item
                        \df{Predpostavke} so izjave, ki jih privzamemo.
                \item
                        \df{Sklepi} so izjave, ki jih (pri danih predpostavkah) dokazujemo.
        \end{itemize}

        Oglejmo si konkreten primer. Rolleov izrek je znan in uporaben izrek v analizi (če ga še niste spoznali, ga boste v kratkem).

        \begin{izrek}[Rolle]
                Naj bo $f$ realna funkcija, definirana na intervalu $\intcc{a}{b}$, kjer sta $a$ in $b$ realni števili in $a < b$. Če je $f$ zvezna na celem $\intcc{a}{b}$ in odvedljiva na odprtem intervalu $\intoo{a}{b}$ ter zavzame enaki vrednosti v krajiščih, tj.~$f(a) = f(b)$, tedaj ima $f$ stacionarno točko na $\intoo{a}{b}$.
        \end{izrek}

        Analizirajmo, kaj so kontekst, predpostavke in sklepi pri tem izreku.

        \begin{itemize}
                \item
                        Kontekst je sledeč:
                        \[a \in \RR, \qquad b \in \RR_{> a}, \qquad f \in \RR^{\intcc{a}{b}}.\]
                        To so objekti (in njihove vrste), o katerih govori izrek. Smiselno je, da jih zapišemo v tem vrstnem redu; na primer, $f$ zapišemo nazadnje, saj je njena domena odvisna od $a$ in $b$. Kadar imamo objekte, ki so neodvisni med sabo, jih lahko zapišemo v poljubnem vrstnem redu.
                \item
                        Predpostavke so tri. Vsako navedimo v običajnem jeziku in nato še s simbolnim matematičnim zapisom.
                        \begin{itemize}
                                \item
                                        $f$ je zvezna na $\intcc{a}{b}$.
                                        \[
                                                \hspace{-2em}
                                                \all{x \in \intcc{a}{b}}
                                                        \all{\epsilon \in \RR_{> 0}}
                                                                \some{\delta \in \RR_{> 0}}
                                                                        \all{y \in \intcc{a}{b}}
                                                                                (|x - y| < \delta \impl \big|f(x) - f(y)\big| < \epsilon
                                                                        )
                                        \]
                                \item
                                        $f$ je odvedljiva na $\intoo{a}{b}$.
                                        \begin{multline*}
                                                \all{x \in \intoo{a}{b}}
                                                        \some{v \in \RR}
                                                                \all{\epsilon \in \RR_{> 0}}
                                                                        \some{\delta \in \RR_{> 0}}
                                                \all{h \in \RR_{\neq 0}} \\
                                                        (|h| < \delta \implies \Big|\frac{f(x + h) - f(x)}{h} - v\Big| < \epsilon)
                                        \end{multline*}
                                \item
                                        $f$ na krajiščih intervala zavzame enaki vrednosti.
                                        \[f(a) = f(b)\]
                        \end{itemize}
                        Če se vam morda zdita formuli za zveznost in odvedljivost begajoči, imate dve tolažbi. Prva je ta, da se boste čez čas takšnih formul navadili. ;) Druga je, da so tudi drugi matematiki leni po naravi in zato uvedejo oznake za daljše izraze, ki se pogosto uporabljajo. Zgornja zveznost se na krajše zapiše $f \in \mathcal{C}(\intcc{a}{b})$ ($\mathcal{C}$ kot ``continuous'', tj.~zvezen), odvedljivost pa $f \in \mathcal{D}^1(\intoo{a}{b})$ ($\mathcal{D}$ kot ``differentiable'', tj.~odvedljiv, enka pa pomeni ``(vsaj) enkrat odvedljiv'').
                \item
                        Sklep je eden: $f$ ima stacionarno točko na $\intoo{a}{b}$, kar simbolno zapišemo takole.
                        \[\some{x \in \intoo{a}{b}} f'(x) = 0\]
        \end{itemize}

        V splošnem imamo določeno mero svobode, kako natančno razčleniti izrek. Na primer, za Rolleov izrek bi lahko kontekst zapisali tudi kot $a \in \RR, b \in \RR, f \in \RR^{\intcc{a}{b}}$ in pogoj $a < b$ dodali med predpostavke.

        Da ne bomo pisali dolgih seznamov, se dogovorimo za sledeče oznake. Izrek podamo tako, da najprej zapišemo kontekst, nato dvopičje, nato narišemo vodoravno črto, nad črto zapišemo predpostavke (ločene z vejicami), pod črto pa sklepe (ločene z vejicami). Rolleov izrek bi potemtakem povzeli takole.
        \[\claim{a \in \RR, b \in \RR_{> a}, f \in \RR^{\intcc{a}{b}}}{f \in \mathcal{C}(\intcc{a}{b}), f \in \mathcal{D}^1(\intoo{a}{b}), f(a) = f(b)}{\some{x \in \intoo{a}{b}} f'(x) = 0}\]

        V splošnem velja: vse proste spremenljivke, ki se pojavijo v predpostavkah ali sklepih, morajo biti navedene v kontekstu. Po domače povedano: če trdite, da za neko stvar nekaj velja, morate najprej povedati, o kateri stvari sploh govorite.

        Medtem ko je za težje matematične izreke potrebno obilo ustvarjalnosti, da se jih dokaže, pa lažje trditve pogosto lahko avtomatično dokažemo (dobesedno --- obstajajo avtomatični dokazovalniki \davorin{koliko povemo na to temo?}), pa tudi za težje je pomembno, da vemo, kako pristopiti k dokazu. Gre za to, da za vse logične veznike in kvantifikatorje obstajajo splošna pravila, kako ravnamo, če nastopajo kot predpostavke oziroma kot sklepi. To si bomo zdaj ogledali.

        \begin{itemize}
                \item\textbf{Konjunkcija}
                        \begin{itemize}
                                \item
                                        Če $p \land q$ nastopa kot \emph{predpostavka}:
                                        \begin{quote}
                                                predpostavko $p \land q$ nadomestimo s predpostavkama $p$, $q$ (to se pravi, pri dokazovanju lahko uporabimo tako predpostavko $p$ kot predpostavko $q$). S simboli, od trditve
                                                \[\claim{\Gamma}{\Pi', p \land q, \Pi''}{\Sigma}\]
                                                preidemo do trditve
                                                \[\claim{\Gamma}{\Pi', p, q, \Pi''}{\Sigma}\]
                                                (pri zapisih splošnih izrekov bomo kontekst označevali z $\Gamma$, predpostavke s $\Pi$ in sklepe s $\Sigma$).
                                        \end{quote}
                                \item
                                        Če $p \land q$ nastopa kot \emph{sklep}:
                                        \begin{quote}
                                                sklep $p \land q$ dokažemo tako, da dokažemo posebej $p$ in posebej $q$. S simboli:
                                                \[\claim{\Gamma}{\Pi}{\Sigma', p \land q, \Sigma''}\]
                                                preoblikujemo v
                                                \[\claim{\Gamma}{\Pi}{\Sigma', p, q, \Sigma''}\]
                                                (in se zavedamo, da je za dokaz izreka potrebno dokazati \emph{vse} sklepe).
                                        \end{quote}
                        \end{itemize}
                \item\textbf{Disjunkcija}
                        \begin{itemize}
                                \item
                                        Če $p \lor q$ nastopa kot \emph{predpostavka}:
                                        \begin{quote}
                                                ločimo primere: sklepe dokažemo posebej pri predpostavki $p$ (skupaj z ostalimi predpostavkami) in posebej pri predpostavki $q$ (skupaj z ostalimi). Torej, dokazati
                                                \[\claim{\Gamma}{\Pi', p \lor q, \Pi''}{\Sigma}\]
                                                pomeni isto, kot dokazati tako
                                                \[\claim{\Gamma}{\Pi', p, \Pi''}{\Sigma} \qquad \text{kot} \qquad \claim{\Gamma}{\Pi', q, \Pi''}{\Sigma}.\]
                                        \end{quote}
                                \item
                                        Če $p \lor q$ nastopa kot \emph{sklep}:
                                        \begin{quote}
                                                izberemo si enega od $p$, $q$ in ga dokažemo. Se pravi, če imamo
                                                \[\claim{\Gamma}{\Pi}{\Sigma', p \lor q, \Sigma''},\]
                                                si izberemo eno od trditev
                                                \[\claim{\Gamma}{\Pi}{\Sigma', p, \Sigma''} \qquad \text{oziroma} \qquad \claim{\Gamma}{\Pi}{\Sigma', q, \Sigma''}\]
                                                in jo izpeljemo.
                                        \end{quote}
                        \end{itemize}
                \item\textbf{Implikacija}
                        \begin{itemize}
                                \item
                                        Če $p \impl q$ nastopa kot \emph{predpostavka}:
                                        \begin{quote}
                                                če nam kadarkoli uspe izpeljati $p$, lahko dodamo $q$ med predpostavke. Torej, če znamo dokazati
                                                \[\claim{\Gamma}{\Pi', p \impl q, \Pi''}{q},\]
                                                potem za dokaz
                                                \[\claim{\Gamma}{\Pi', p \impl q, \Pi''}{\Sigma}\]
                                                zadostuje dokazati
                                                \[\claim{\Gamma}{\Pi', p \impl q, q, \Pi''}{\Sigma}\]
                                                (kar je lažje, ker imamo eno predpostavko več). To je smiselno: če vemo, da velja $p \impl q$ in dodatno ugotovimo, da velja $p$, potem vemo, da velja tudi $q$.
                                        \end{quote}
                                \item
                                        Če $p \impl q$ nastopa kot \emph{sklep}:
                                        \begin{quote}
                                                sklep $p \impl q$ nadomestimo s $q$, medtem ko $p$ dodamo med predpostavke. Pojasnimo. Trditev $p \impl q$ trdi nekaj samo v primeru, kadar $p$ velja --- v nasprotnem primeru je avtomatično resnična in ni ničesar za dokazati. Torej se lahko omejimo na primer, ko $p$ velja, se pravi, lahko predpostavimo $p$. Kadar $p$ velja, pa trditev $p \impl q$ pravi, da mora veljati tudi $q$. To pomeni, da pri predpostavki $p$ dokazujemo $q$. Simbolno, da dokažemo
                                                \[\claim{\Gamma}{\Pi}{\Sigma', p \impl q, \Sigma''},\]
                                                zadostuje dokazati
                                                \[\claim{\Gamma}{\Pi}{\Sigma', \Sigma''} \qquad \text{in} \qquad \claim{\Gamma}{\Pi, p}{q}.\]
                                        \end{quote}
                        \end{itemize}
                \item\textbf{Univerzalni kvantifikator}
                        \begin{itemize}
                                \item
                                        Če $\all{x \in X} \phi(x, y)$ nastopa kot \emph{predpostavka}:
                                        \begin{quote}
                                                če vemo za (ali med dokazom najdemo) katerikoli konkreten element $a \in X$, tedaj lahko med predpostavke dodamo $\phi(a, y)$. Namreč, če vemo, da lastnost $\phi$ (z morebitnimi nadaljnjimi parametri) velja za vse elemente množice $X$, potem ta lastnost velja za poljuben konkreten element. Simbolno, od
                                                \[\claim{\Gamma', a \in X, \Gamma''}{\Pi', \all{x \in X} \phi(x, y), \Pi''}{\Sigma}\]
                                                preidemo do
                                                \[\claim{\Gamma', a \in X, \Gamma''}{\Pi', \all{x \in X} \phi(x, y), \phi(a, y), \Pi''}{\Sigma}.\]
                                        \end{quote}
                                \item
                                        Če $\all{x \in X} \phi(x, y)$ nastopa kot \emph{sklep}:
                                        \begin{quote}
                                                v kontekst dodamo $x \in X$, sklep $\all{x \in X} \phi(x, y)$ pa nadomestimo s sklepom $\phi(x, y)$. S simboli, od
                                                \[\claim{\Gamma}{\Pi}{\Sigma', \all{x \in X} \phi(x, y), \Sigma''}\]
                                                preidemo do
                                                \[\claim{\Gamma, x \in X}{\Pi}{\Sigma', \phi(x, y), \Sigma''}\]
                                                Zakaj tako postopamo in kaj smo s tem pravzaprav naredili? Premislimo: želimo dokazati, da neka lastnost velja za vse elemente dane množice $X$. Če ima $X$ slučajno samo končno mnogo elementov, bi lahko lastnost preverili za vsakega posebej, ampak povečini delamo z neskončnimi množicami, kjer to ne deluje. Morda ima množica $X$ kakšno posebno lastnost, zaradi katere lahko univerzalni kvantifikator dokažemo na svojevrsten način (na primer, univerzalno kvantificirane izjave nad $\NN$ lahko dokazujemo z matematično indukcijo --- glej \note{razdelek o naravnih številih}), ampak to se zgodi v izjemnih primerih.

                                                V splošnem nimamo druge možnosti, kot da si izberemo simbol (tipično kar spremenljivko v kvantifikatorju), ki nam predstavlja poljuben, katerikoli element množice in zanj dokažemo želeno lastnost. Ideja je, da spremenljivka spet nastopa v vlogi ``škatlice'', kamor lahko vstavimo poljuben element množice $X$. Če nam je dokaz lastnosti uspel, ne da bi za spremenljivko predpostavili karkoli več, kot da predstavlja element množice $X$, tedaj dobimo dokaz lastnosti za katerikoli dejanski element množice $X$ tako, da v dobljeni dokaz namesto spremenljivke vstavimo ta element. Na ta način smo potem dejansko dobili dokaz lastnosti za vse elemente množice $X$.

                                                Besedni dokazi univerzalno kvantificirane izjave se zato tipično začnejo takole: ``Vzemimo poljuben $x \in X$. Dokažimo, da zanj velja dana lastnost.''
                                        \end{quote}
                        \end{itemize}
                \item\textbf{Eksistenčni kvantifikator}
                        \begin{itemize}
                                \item
                                        Če $\some{x \in X} \phi(x, y)$ nastopa kot \emph{predpostavka}:
                                        \begin{quote}
                                                v kontekst dodamo $x \in X$, eksistenčno predpostavko pa nadomestimo s $\phi(x, y)$. S simboli,
                                                \[\claim{\Gamma}{\Pi', \some{x \in X} \phi(x, y), \Pi''}{\Sigma}\]
                                                popravimo v
                                                \[\claim{\Gamma, x \in X}{\Pi', \phi(x, y), \Pi''}{\Sigma}.\]
                                                Zakaj to deluje? Naša predpostavka je, da v množici $X$ obstaja element z lastnostjo $\phi$ (z morebitnimi nadaljnjimi parametri). Torej si lahko vzamemo neki konkreten element množice $X$ s to lastnostjo, ki ga lahko uporabljamo kasneje v dokazu (za to ga moramo nekako označiti; v praksi ga tipično označimo kar z isto spremenljivko, kot v kvantifikatorju).
                                        \end{quote}
                                \item
                                        Če $\some{x \in X} \phi(x, y)$ nastopa kot \emph{sklep}:
                                        \begin{quote}
                                                da dokažemo eksistenčno izjavo, moramo podati neki konkreten element $x \in X$ in zanj dokazati dano lastnost $\phi(x, y)$. \davorin{Hm, kako točno to zapišemo simbolno v zgornji obliki?}
                                        \end{quote}
                        \end{itemize}
        \end{itemize}

        V zgornjem seznamu nismo omenili vseh veznikov in kvantifikatorjev. To je zato, ker jih pri dokazovanju nadomestimo z zgornjimi. Konkretno:
        \begin{itemize}
                \item
                        Za negacijo velja $\lnot{p} \equiv p \impl \false$. Med drugim to pomeni, da $\lnot{p}$ dokažemo na sledeči način: predpostavimo $p$ in iz tega izpeljemo neresnico.
                \item
                        Za ekvivalenco velja $p \lequ q \equiv (p \impl q) \land (p \revimpl q)$. To pomeni, da ekvivalenco dokažemo tako, da dokažemo implikacijo med $p$ in $q$ v obe smeri --- se pravi, enkrat predpostavimo $p$ in izpeljemo $q$, drugič pa predpostavimo $q$ in izpeljemo $p$.
                \item
                        Za veznike $\xor$, $\shf$, $\luk$ si preprosto izberemo eno od izražav z negacijo, konjunkcijo in disjunkcijo in nato delamo z njo.
                \item
                        Kvantifikator $\exactlyone{x \in X} \phi(x, y)$ ločimo na dva dela: na obstoj in enoličnost, in vsakega posebej dokažemo. Se pravi, skličemo se na izražavo
                        \[\exactlyone{x \in X} \phi(x, y) \equiv \some{x \in X} \phi(x, y) \land \all{a, b \in X} (\phi(a, y) \land \phi(b, y) \implies a = b).\]
                        Včasih je lažje, če najprej dokažemo obstoj elementa in ta element pri dokazu enoličnosti že uporabimo, torej dokazujemo izražavo
                        \[\exactlyone{x \in X} \phi(x, y) \equiv \some{x \in X} (\phi(x, y) \land \all{a \in X} (\phi(a, y) \implies a = x)).\]
        \end{itemize}

        Seveda ne bo možno dokazati vsakega izreka s slepim sledenjem zgornjim pravilom; včasih moramo uporabiti še kakšno dodatno strategijo. Spodnji dve sta zelo pogosti.
        \begin{itemize}
                \item
                        Med predpostavke dodamo trditev, za katero že vemo, da je resnična. Morda gre za trditev, ki smo jo že dokazali, morda pa gre kar za istorečje. Pogost primer tega je, da uporabimo zakon o izključenem tretjem in za dodatno predpostavko vzamemo $p \lor \lnot{p}$ (kjer je $p$ katerakoli konkretna izjava). Po zgornjih pravilih to potem pomeni, da ločimo primere in trditev dokažemo posebej pri predpostavki $p$ ter posebej pri predpostavki $\lnot{p}$.
                \item
                        Nekatere predpostavke ali sklepe nadomestimo z enakovrednimi izjavami. Na primer, velja
                        \[p \lor q \equiv \lnot(\lnot{p} \land \lnot{q}) \equiv \lnot{p} \impl q \equiv \lnot{q} \impl p.\]
                        To pomeni, da lahko disjunkcijo (poleg zgoraj omenjenega načina) dokažemo tudi tako, da predpostavimo, da nobena od možnosti ne velja, in od tod izpeljemo neresnico, ali pa predpostavimo, da ena od možnosti ne velja, in od tod izpeljemo drugo.

                        Zelo pogosta uporaba te ideje je \df{dokaz s protislovjem}, ki temelji na zakonu o dvojni negaciji $p \equiv \lnot\lnot{p}$. Izjavo torej lahko dokažemo tako, da predpostavimo njeno negacijo, in od tod izpeljemo neresnico. Tipičen besedni dokaz s protislovjem izgleda takole: ``Dokazujemo $p$. Pa recimo, da $p$ ne velja. Potem /neki sklepi/. To je v nasprotju s tem, kar smo dokazali prej, torej smo izpeljali protislovje. Se pravi, ni možno, da $p$ ne bi veljal, torej mora veljati.''
        \end{itemize}

        \note{mnogo zgovornih primerov dokazov, ki ponazorijo zgornje postopke}


\section{Vaje}

\anja{Kako bomo zapisovali rešitve nalog z navodilom \nls{dokaži}? Ali jih bomo pisali tako kot na predavanjih in bomo posodobili to poglavje v učbeniku?}

\begin{vaja}
  Dokažite naslednje izjave:
  %
  \begin{enumerate}
  \item $p \land q \impl p \lor q$
  \item $p \impl (q \impl p)$
  \item $p \land q \impl \lnot (\lnot p \lor \lnot q)$
  \item $(p \land q) \lor r \equiv (p \lor r) \land (q \lor r)$
  \item $(p \impl (q  \lor r) \land q \impl \lnot p) \impl (p \impl r)$
  \end{enumerate}
\end{vaja}

\begin{vaja}
Preveri veljavnost naslednjih sklepov.
\anja{Ali to nalogo nekoliko preoblikujemo, ali pa povemo, da imamo lahko več predpostavk?}

\begin{enumerate}
  \item Študent se je s trolejbusom peljal na izpit. Rekel si je: \nls{Če bo semafor pri Drami zelen, bom naredil izpit.} No, ko je avtobus pripeljal na križišče, je semafor svetil rdečo, študent pa si je dejal: \nls{Presneto, spet bom padel.}
  
  \item Če preveč pijem ali kadim, slabo spim. Če slabo spim ali premalo jem, sem utrujen.
  Če sem utrujen, ne telovadim in ne študiram. Preveč kadim. Torej ne študiram.

  \item  Če ima Biblija prav, potem obstajata Bog in hudič. 
             Če obstaja Bog, je na svetu veselje.
             Če obstaja hudič, je na svetu žalost.
             Na svetu sta veselje in žalost.
             Sklep: Biblija ima prav.

  \item Razglednik Vid vsako soboto obišče Ljubljanski grad. Na grič se vzpne po Študentovski poti (s tržnice) ali Mačji stezi (skozi gozd), včasih pa kar po cesti. Če gre po prvi poti, s seboj nosi svežo zelenjavo s tržnice. Kadar se vzpne po Mačji stezi, spotoma nabere za pest odpadlega listja. To soboto Vid na grad ni šel po cesti in s sabo ni nosil zelenjave. Torej je na grad dospel z listjem v rokah.
\end{enumerate}
\end{vaja}

\begin{vaja}
  Dokažite naslednje izjave:
  %
  \begin{enumerate}
  \item $(\all{x \in A} p(x)) \lor (\all{y \in A} q(y)) \impl \all{z \in A} p(z) \lor q(z)$
  \item $(\some{x \in A} p(x)) \lor (\some{y \in A} q(y)) \impl \some{z \in A} p(z) \lor q(z)$
  \item $(\some{x \in A} \all{y \in B} p(x, y)) \impl \all{b \in B} \some{a \in A} p(a, b)$
  \end{enumerate}
\end{vaja}

\begin{vaja}
  Dokažite naslednje izjave:
  %
  \begin{enumerate}
  \item $(\some{n \in \NN} 24 \cdot n = a) \impl \some{k \in \NN} 3 \cdot k = a$.
  \item $\all{m \in \NN} \some{\ell \in \NN} (m^2 = 4 \ell \lor m^2 = 4 \ell + 1).$\\
    Namig:  brez dokaza smemo uporabiti dejstvo
    $\all{n \in \NN} \some{k \in \NN} (n = 2 k \lor n = 2 k + 1).$
  \end{enumerate}
\end{vaja}

\begin{vaja}
  Dana je funkcija $f : \RR \to \RR$. Dokažite naslednje izjave:
  %
  \begin{enumerate}
  \item  $\all{x \in \RR} \some{y \in \RR} f(x) \leq y.$
  \item$(\some{M \in \RR} \all{x \in \RR} f(x) \leq M) \impl \all{x \in \RR} \some{y \in \RR} f(x) \leq y.$
  \item $(\some{M \in \RR} \all{x \in \RR} f(x) \leq M) \impl \some{N \in \RR} \all {x \in \RR} 2 \cdot f(x) + 7 \leq N.$
  \item $(\some{M \in \RR} \all{x \in \RR} x^2 + 7 \leq M) \impl \all{x \in \RR} \some{y \in \RR} x^2 + 7 \leq y$.
  \end{enumerate}
\end{vaja}


%%% Local Variables:
%%% mode: latex
%%% TeX-master: "ucbenik-lmn"
%%% End:
