\chapter{Strukture}

\note{Struktura na množici. Morfizmi, ki to strukturo ohranjajo. Izomorfnost. Definicija strukturirane množice preko njene karakterizacije --- potrebna obstoj in enoličnost (do izomorfizma). Primeri. Posebej primeri struktur urejenosti (izhaja iz razdelka o strukturah urejenosti v poglavju o relacijah) in osnovnih algebrskih struktur (pride prav kasneje pri konstrukciji številskih množic). Urejenostna in algebrska struktura se združita v pojmu mreže. Definicija (polnih) Boolovih mrež/kolobarjev in povezava z logiko. Širša slika strukturiranih množic --- kategorije.}


%%% Local Variables:
%%% mode: latex
%%% TeX-master: "ucbenik-lmn"
%%% End:
